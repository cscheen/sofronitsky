\chapter[%
Chronologie des récitals et concerts][%
Chronologie des récitals et concerts]{%
Chronologie des récitals et concerts}
\label{chap:Recitals}

\section{Références bibliographiques}

Ce chapitre s'efforce de présenter une liste des récitals de piano donnés
par \VSofronitsky{} au long de sa carrière.
Les sources -- toujours en russe, en dehors de la thèse d'\EWhite{} -- sont
souvent fragmentaires et disséminées dans différents ouvrages~; on peut
néanmoins citer, par ordre alphabétique, \citet[p.~354]{Badeyan08},
\citet[p.~427, p.~435, p.~437, p.~440-452 et p.~455-461]{Milshteyn82a},
\citet{Nekrasova08}, \citet[p.~393-395]{Nikonovich08}, \citet{Panarine},
\citet{Shiryaeva}, \citet{Sofronitsky82a}, \citet[p.~\hbox{40-71}]{White},
\citet[p.~301]{Zhukova82} et \citet[p.~216]{Zhukova08}.
L'énumération la plus détaillée et exhaustive a été publiée en~2013 par
\citet[p.~393-452]{Scriabine}, à partir de \citet{Milshteyn82a}, de
\citet{Nikonovich08} et de nombreux autres documents et publications rédigés
en russe \citep[voir][p.~444-452]{Scriabine}.
Seuls \citeauthor{Nekrasova08}, \citeauthor{Scriabine} et \citeauthor{White}
décrivent toute la période en question, du~27 mars~1910 (a.s.) au~9
janvier~1961 (n.s.).
Ce chapitre transcrit les informations de \citet[p.~393-444]{Scriabine} et
de \citet{Nekrasova08}.
\citeauthor{Nekrasova08} a reçu les archives de son amie \AVizel{}, lors du
décès de celle-ci en~1974~: il s'agit d'une collection de documents relatifs
à la vie musicale de \VSofronitsky{}, de son enfance à sa dernière saison
en~1960-1961 \citep[voir][p.~132-133]{Nekrasova08}.
Plusieurs dates, précédées de la lettre~B, ont été ajoutées à la liste des
récitals afin de mentionner des repères biographiques importants.

\section{Année~1901}

\begin{description}
 \item[B\DateWithWeekDay{1901-05-08}]
 Naissance de Vladimir Vladimirovič \Sofronitsky{} (\Dates{1901}{1961}) et
 de sa sœur jumelle, Vera Vladimirovna (\Dates{1901}{1948}), à
 Sankt-Peterburg.
 Leur père, Vladimir Nikolaevič (\Dates{1869}{1942}), est professeur de
 mathématiques et de physique.
 Leur mère, Vera Aleksandrovna (\Dates{1875}{1935}), est issue de la famille
 Borovikovskij, célèbre dans le domaine des arts, en poésie et peinture.
 Leur frère et leurs trois sœurs, tous plus âgés qu'eux, sont~:
 Nikolaj Vladimirovič,
 Ol'ga Vladimirovna (\Dates{1893}{1933}),
 Natal'ja Vladimirovna (\Dates{1897}{1979}),
 Elena Vladimirovna (\Dates{1898}{1986}).

 La date de naissance est donnée ici dans le calendrier grégorien (n.s.),
 \cad{}~1901-04-25 dans le calendrier julien (a.s.)~: voir note en bas de
 page~\pageref{fn:ASNS}.
 \VSofronitsky{} lui-même se donnait parfois un an plus jeune, avec une date
 de naissance en~1902, par exemple dans une lettre à sa famille le~11
 mai~1938 \citep[p.~23]{Kogan08} et dans un fragment autobiographique du~25
 juin~1938 \citep[p.~368]{Nikonovich08}.
\end{description}

\section{Année~1903}

\begin{description}
 \item[B1903]
 Installation de la famille \Sofronitsky{} à Warszawa, où le père, Vladimir
 Nikolaevič, a été nommé inspecteur pour un district scolaire
 \citep[voir][p.~84]{Artese}.
\end{description}

\section{Année~1908}

\begin{description}
 \item[B1908 (date imprécise)]
 \VSofronitsky{} étudie, pendant quelques mois, avec \Quote{le compositeur
 Ruzicki, père du pianiste polonais bien connu} \citep{Vitsinsky}.
 Selon \citet{Voskobojnikov16}, il s'agit en réalité d'\hbox{Aleksander}
 Różycki, pianiste et professeur au conservatoire de Warszawa, père du
 compositeur polonais Ludomir Różycki (\Dates{1883}{1953}).
 \citet[p.~134, 138 et~148]{Nekrasova08} confirme qu'il s'agit en effet
 d'\hbox{Aleksander} Różycki, dont la méthode publiée accorde une grande
 importance au rythme et aux combinaisons rythmiques, sur la base de ses
 propres arrangements d'œuvres populaires et du folklore polonais.
 \item[B\DateWithWeekDay{1908-08-29}]
 Commencement des études de piano avec \ALebedevaGetsevich{} à Warszawa.
 \ALebedevaGetsevich{} est une pianiste russe qui a étudié avec Nikolaj
 Grigor'evič Rubinštejn au conservatoire de Moskva~; elle est la mère du
 pianiste Vsevolod Ivanovič Bujukli.
 La date exacte du~29 août est mentionnée par \citet[p.~135]{Nekrasova08}
 pour le début des cours.
 Un cahier avec la retranscription des leçons des dix-huit premiers mois de
 ses cours auprès d'\ALebedevaGetsevich{} a été offert, en témoignage
 d'amitié, par \VSofronitsky{} à \EDaugovet{} (\Dates{1882}{1942}), et
 retrouvé en~1978, bien après le décès de celle-ci durant le siège de
 Leningrad \citep[voir][p.~135-136]{Nekrasova08}.
\end{description}

\section{Année~1909}

\begin{description}
 \item[B1909]
 Découverte de la musique du compositeur \AScriabine{} au travers de la
 partition de son Prélude pour la main gauche en \kC \Sharp mineur, \Opus{9}
 \Number{1}~; le jeune \Sofronitsky{} l'appelle \emph{Skrjabinov} [sic] dans
 une lettre à sa sœur aînée Natal'ja \citep[voir][p.~55]{Juban}.
 \item[1909-1910 (début de l'année scolaire)]
 Lors d'un concert de jeunes élèves d'\ALebedevaGetsevich{}, au domicile de
 leur professeur, \VSofronitsky{} joue une Chanson de \Mendelssohn{} et un
 Air irlandais de Hertz \citep[voir][p.~136]{Nekrasova08}.
 \item[B1909-12-15 (jour incertain)]
 Lors d'une soirée dans l'entourage (\Gercog{}) de la famille du poète
 \ABlok{}, rue de la Montagne, à Warszawa, le jeune \Sofronitsky{} joue
 devant \Blok{}, qu'il ne rencontrera pourtant, selon sa propre expression,
 qu'\Quote{après [l]a mort [de \Blok{}]}
 \citep[voir][p.~392-393]{Voskobojnikov09a}.
 \Blok{} était venu assister aux funérailles de son père, Aleksandr L'vovič,
 à Warszawa, et y avait séjourné du~1\fup{er} au~18 décembre~1909~: voir
 \citet[p.~432-433]{Milshteyn82a} et \citet[p.~136-139]{Nekrasova08}.
 La poétesse \EGercog{} (\Dates{1884}{>1955}) a été le témoin de cette
 rencontre du jeune \VSofronitsky{} avec \ABlok{}.
 Les carnets de \Blok{} font mention d'une soirée chez les \Gercog{} le~15
 décembre~1909, mais il est possible que le poète n'ait pas pris de notes à
 propos de ses autres visites éventuelles chez les \Gercog{}, raison pour
 laquelle cette date reste incertaine \citep[voir][p.~137]{Nekrasova08}.
 Ce n'est que bien plus tard, à Petrograd, qu'\EGercog{} a relié, pour
 \VSofronitsky{}, le poète qu'il avait découvert et qu'il aimait et l'homme
 pour qui il avait joué du piano jadis à Warszawa, lorsqu'il était enfant.
 Selon \citet[p.~14]{Evans08}, \ABlok{} a écrit en partie son poème
 \foreignlanguage{russian}{\emph{Весенний день прошел без дела}} tandis que
 \VSofronitsky{} jouait~; en tout cas, \EGercog{} a donné à celui-ci
 l'autographe de ce poème qu'\ABlok{} lui avait écrit en souvenir sur papier
 timbré \citep[voir][p.~137]{Nekrasova08}.
\end{description}

\section{Année~1910}

\begin{description}
 \item[1910-03-27%
 \footnote{Dans le calendrier julien ou \Quote{ancien style} (abrégé~a.s.),
 \cad{}~\DateWithWeekDay{1910-04-09} dans le calendrier grégorien ou
 \Quote{nouveau style} (abrégé~n.s.) qui n'a été adopté en Rossija que lors
 de la révolution bolchevique~: par décret, le~31 janvier~1918 a alors été
 suivi du~14 février~1918.\label{fn:ASNS}}]
 Warszawa~: salle du gouvernement régional (ou de l'hôtel de ville).
 Première apparition en public lors d'une soirée donnée par les élèves et
 les étudiants de la pianiste \ALebedevaGetsevich{} et de la cantatrice
 \VSviatlovskaya{}.
 Prestation critiquée dans la presse%
 \footnote{Zataevič, Aleksandr Viktorovič.
 In~: \emph{\foreignlanguage{russian}{Варшавского дневника} [Varšavskogo
 dnevnika]}.
 Deuxième critique, signée \Quote{S.~O.}, in~:
 \emph{\foreignlanguage{russian}{Варшавском вестнике} [Varšavskom
 vestnike]}.
 Troisième critique, non signée, citée, de même que les deux premières, par
 \citet[p.~21]{Sofronitsky82a}.}.

 \textsc{\Clementi{}}~: Finale d'une Grande Sonate.
 \textsc{\Mozart{}}~: Rondo (peut-être en \kA mineur, K~511).
 \textsc{\Beethoven{}}~: Menuet de la Sonate en \kG majeur (peut-être
 \Opus{79}).
 \textsc{\Schubert{}}~: Moment musical en \kF mineur, D~780 \Number{3} ou
 \Number{5}.
 Improvisation finale sur un thème proposé par le public présent lors du
 récital.
 \item[B1910 (printemps)]
 Bref voyage de la famille \Sofronitsky{} à Sankt-Peterburg, au cours duquel
 Vladimir est présenté à \AGlazounov{}, au chanteur (basse) et professeur
 russe Stanislav Ivanovič Gabel' (\Dates{1849}{1924}) et au compositeur et
 critique musical russe Cezar' Antonovič Kjui (Cui) (\Dates{1835}{1918}), au
 conservatoire de la ville.
 Au terme de cette entrevue, \Glazounov{} conseille aux époux \Sofronitsky{}
 de confier la suite des études musicales de leur fils cadet au pianiste et
 professeur \AMichalowski{}, à Warszawa.
 \VSofronitsky{} était pourtant admissible sans délai au conservatoire de
 Sankt-Peterburg, après cette entrevue, mais puisque son père ne pouvait pas
 déménager sa famille dans l'immédiat, il a été décidé de suivre entre-temps
 la seconde option d'\AGlazounov{} et de choisir ce grand professeur de
 piano à Warszawa.
 Voir \citet[p.~21-22]{Sofronitsky82a} et \citet[p.~140]{Nekrasova08}.
 \item[B1910 (automne)]
 Commencement des cours de piano avec \AMichalowski{} à Warszawa, à l'École
 de la Société de musique de la ville.
 \AMichalowski{} est un ancien élève d'\hbox{Ignaz} Moscheles, Carl Reinecke
 et surtout Theodor Coccius au conservatoire de Leipzig, dans les lignées de
 \Chopin{} et \Beethoven{}.
 Il est aussi devenu un ami de Karol Mikuli, élève de \FChopin{}, et il a pu
 rencontrer la princesse Marceline Czartoryska (née Radziwiłł), élève de
 Carl Czerny et de \FChopin{}.
 La première réaction d'\AMichalowski{}, devant \VSofronitsky{} et sa mère,
 a été d'indiquer qu'il ne travaillait pas avec des enfants, mais il est
 devenu très enthousiaste après avoir écouté le jeu du jeune garçon de neuf
 ans \citep[voir][p.~140]{Nekrasova08}.
 Trois années de cours commencent, durant lesquelles \VSofronitsky{}, sur le
 conseil de son professeur, peut aussi écouter en concert, à la philharmonie
 de Warszawa, des musiciens tels que \SRachmaninov{}, Vsevolod Bujukli,
 \KIgumnov{}, Nikolaj Andreevič Orlov, \AMichalowski{} lui-même, Irina Èneri
 (Gorjainova) et les Polonais Stanislav Barcevič et Iosif Adamovič
 Turčinskij, de même que des chefs d'orchestre comme Vasilij Il'ič Safonov
 et Gžegož (Grigorij) Fitel'berg.
 Il rencontre aussi le ténor de l'opéra de Milan, G.~Genzardi.
 Voir \citet[p.~23]{Sofronitsky82a}.
\end{description}

\section{Année~1913}

\begin{description}
 \item[B1913-09]
 Retour de la famille \Sofronitsky{} à Sankt-Peterburg, mais le jeune
 Vladimir poursuit néanmoins ses leçons auprès de \Michalowski{} à Warszawa,
 une fois par mois, jusqu'au début de la Première Guerre mondiale en~1914.
 Selon \citet[p.~140]{Nekrasova08}, les séjours pour les leçons duraient une
 semaine.
\end{description}

\section{Année~1914}

\begin{description}
 \item[1914-08-09 (a.s.)]
 Warszawa.
 Apparition lors d'un concert de charité.
 Concert évoqué par \citet[p.~25]{Sofronitsky82a}, qui ajoute une Toccata
 d'\ARubinstein{} en début de programme.

 \textsc{\ARubinstein{}}~: Toccata.
 \textsc{\Liszt{}}~: \emph{Gondoliera}.
 \textsc{\Mendelssohn{}}~: Prélude.
 \item[B1914-1915 (hiver)]
 \VSofronitsky{} se produit deux fois lors de concerts de bienfaisance, dont
 une représentation organisée par le chef d'orchestre anglais \ACoates{}
 \citep[voir][p.~142]{Nekrasova08}~; la presse prédit au jeune musicien un
 brillant avenir.
 \item[1914-12-14 (a.s.)]
 Petrograd.
 Première apparition lors d'un concert en plein air.
 Selon \citet[p.~25]{Sofronitsky82a}, il s'agit ici du concert de charité
 organisé par \ACoates{} et annoncé ci-dessus.
 Concert critiqué dans le journal \foreignlanguage{russian}{Новое время}
 [Novoe vremja]~: voir \citet[p.~25]{Sofronitsky82a}, qui confirme le
 programme.

 \textsc{\Rachmaninov{}}~: Prélude.
 \textsc{\Tchaikovski{}/\Pabst{}}~: Fantaisie sur des thèmes du ballet
 \emph{La Belle au bois dormant}.
\end{description}

\section{Année~1915}

\begin{description}
 \item[B1915]
 \VSofronitsky{} poursuit ses études sous la supervision du pianiste et
 compositeur Leonid Aleksandrovič Ščedrin~: la mère de \Sofronitsky{}, qui
 n'a jamais manqué une leçon avec \AMichalowski{}, sollicite l'assistance de
 Ščedrin pour les cours de son fils après la période de Warszawa.
 Voir \citet[p.~422]{Milshteyn82a}, \citet[p.~6]{White} et
 \citet[p.~55]{Juban}.
 \item[B1915-04-02 (a.s.)]
 En raison d'un refroidissement -- ou plutôt à cause du mauvais temps --,
 \VSofronitsky{} perd l'occasion d'aller écouter le dernier récital de
 \Scriabine{}, donné à Petrograd.
 Il pleuvait, ce soir-là, et le jeune Vladimir n'a pas été autorisé à sortir
 de chez lui par ses parents~: \Quote{Tu pourrais prendre froid~; il vaut
 mieux attendre la prochaine fois.} \citep[p.~88.]{Nikonovich08a}
 Le compositeur meurt deux semaines plus tard, le~1915-04-27 (n.s.).
 \item[B1915 (automne)]
 Inscription au conservatoire de Petrograd, dans la classe de piano de
 \LNikolaiev{} et dans celle de composition de Maksimilian Štejnberg, ancien
 disciple de Nikolaj Rimskij-Korsakov.
 \citet[p.~55]{Juban} mentionne plutôt la date de~1916~; \VSofronitsky{}
 lui-même indique d'ailleurs~1916 pour le début de ses études régulières au
 conservatoire de Petrograd \citep[voir][p.~422]{Milshteyn82a}.
 \LNikolaiev{} est un ancien élève de \STaneiev{} et Michail
 Ippolitov-Ivanov (classe de composition) et de Vasilij Il'ič Safonov
 (classe de piano), au conservatoire de Moskva, dans la lignée de Teodor
 Leszetycki.
 À propos de \LNikolaiev{} et de sa classe de piano au conservatoire de
 Petrograd, voir~: \citet{Alshvang39}~; \citet[p.~70-71]{Bogdanov67b}~;
 \citet{Delson35}~; \citet[p.~140-142]{Nekrasova08}~; \citet{Nikolaiev13}~;
 \citet[p.~\hbox{5-9} et~85-86]{Artese}~; \citet{Chkourak10}~;
 \citet[p.~55-56]{Juban}~; \citet[p.~6 et~108-111]{White}.
\end{description}

\section{Année~1916}

\begin{description}
 \item[1916-10-28 (a.s.)]
 Petrograd~: petite salle du conservatoire.
 Apparition lors d'une soirée musicale donnée par les étudiants des cours
 supérieurs du conservatoire.
 \citet[p.~7]{Artese} indique la première représentation de \VSofronitsky{}
 dans une œuvre avec orchestre.

 \textsc{\Rachmaninov{}}~: Concerto \Number{1} en \kF \Sharp mineur,
 \Opus{1}.
 \item[1916-12-06 (a.s.)]
 Petrograd~: petite salle du conservatoire.
 Apparition lors d'un concert en matinée donné par les étudiants de la
 classe de \LNikolaiev{}.
 \citet[p.~88-89]{Savshinsky61} se souvient de cette \emph{master class}%
 \footnote{\foreignlanguage{russian}{\emph{Советская музыка}}, vol.~277,
 \Number{12} (1961), p.~88-91.}.

 \textsc{\Liszt{}}~: Tarentelle de \emph{Venezia e Napoli}, S~162
 \Number{3}.
\end{description}

\section{Année~1917}

\begin{description}
 \item[\DateWithWeekDay{1917-01-29}]
 Petrograd~: Institut Kseninskij.
 Apparition lors d'un concert de charité.

 \textsc{\Rachmaninov{}}~: Concerto \Number{1} en \kF \Sharp mineur,
 \Opus{1} (partie~1, avec \LNikolaiev{}).
 \item[B1917]
 Inscription d'\EScriabina{} (\Dates{1900}{1990}) au conservatoire de
 Petrograd, dans la classe de piano de \LNikolaiev{}.
 \EScriabina{} a ensuite poursuivi sa formation musicale sous la direction
 de \KIgumnov{} \citep[voir][p.~142]{Nekrasova08}.
\end{description}

\section{Année~1918}

\begin{description}
 \item[\DateWithWeekDay{1918-02-02}]
 Petrograd~: grande salle du conservatoire.
 Participation à un concert de charité (\Medtner{}, \Scriabine{}) auquel
 \VScriabina{}, la veuve du compositeur, prend part.
 \item[\DateWithWeekDay{1918-02-10}]
 Petrograd.
 Apparition lors d'une soirée des étudiants du conservatoire.

 \textsc{\Scriabine{}}~: Nuances, \Opus{56} \Number{3}~; Poème satanique,
 \Opus{36}.
 \item[\DateWithWeekDay{1918-12-10}]
 Petrograd~: petite salle du conservatoire.
 Apparition lors d'une soirée musicale des étudiants.

 \textsc{\Scriabine{}}~: Nuances, \Opus{56} \Number{3}~; Poème satanique,
 \Opus{36}.
\end{description}

\section{Année~1919}

\begin{description}
 \item[\DateWithWeekDay{1919-01-11}]
 Petrograd~: salle de conférence du conservatoire.
 Participation à la première partie de la \Quote{Journée de travail
 musicale}.
 \item[\DateWithWeekDay{1919-02-02}]
 Petrograd~: grande salle du conservatoire.
 Participation à un concert de charité des professeurs et des étudiants du
 conservatoire auquel \VScriabina{} prend part.

 \textsc{\Medtner{}}~: \emph{Skazka}.
 \textsc{\Scriabine{}}~: Valse en \kA \Flat majeur, \Opus{38}.
 \item[\DateWithWeekDay{1919-02-08}]
 Petrograd~: \Quote{Théâtre intime}.
 Participation à la première partie du concert spectacle en faveur du fonds
 de l'école juive orthodoxe de Petrograd.
 \item[1919-03]
 Petrograd~: Club des artistes.
 Concert avec des œuvres de \Liszt{}, \Schumann{} et \Scriabine{}.
 \item[\DateWithWeekDay{1919-05-11}]
 Petrograd.
 Participation à un concert supplémentaire du \foreignlanguage{russian}
 {Пролеткульт} (organisation culturelle et éducative prolétarienne).
 \item[\DateWithWeekDay{1919-05-26}]
 Petrograd~: salles du sixième district de l'école de musique.
 Premier concert en soliste.
 Voir \citet[p.~435]{Milshteyn82a} et \citet[p.~394]{Scriabine}.

 \textsc{\Liszt{}}~: Funérailles, S~173 \Number{7}.
 \textsc{\Schumann{}}~: Études symphoniques, \Opus{13}.
 \textsc{\Scriabine{}}~: Deux Poèmes, \Opus{32}~; Préludes des \Opus{37, 11
 et~27}~; Valse en \kA \Flat majeur, \Opus{38}~; Trois Études de
 l'\Opus{8}~; Feuillet d'album en \kE \Flat majeur, \Opus{45} \Number{1}~;
 Poème satanique, \Opus{36}.
\end{description}

\section{Année~1920}

\begin{description}
 \item[B1920 (début des années)]
 \VSofronitsky{} fait la connaissance de la famille de l'artiste \EVizel{}
 \citep[voir][p.~144]{Nekrasova08}.
 À propos de \Sofronitsky{} et de la famille \Vizel{}, voir par exemple~:
 \citet{Gorohovsky}~; \citet{Lobacheva03}~; \citet{Vizel03}~;
 \citet{Vizel13a}~; \citet{Vizel13b}~; \citet{Vizel13c}.
 \item[1920]
 En~1920, la maison des arts de Petrograd a organisé deux autres concerts de
 \VSofronitsky{}, en plus de ceux indiqués~; le premier était consacré à des
 œuvres de \Liszt{}, le second à des œuvres de \Schumann{}.
 \item[\DateWithWeekDay{1920-01-08}]
 Petrograd.
 Participation à un concert collectif organisé par le bureau de district des
 usines de mécanique de précision.
 \item[\DateWithWeekDay{1920-04-08}]
 Petrograd~: maison des arts.
 Concert avec une ou des œuvres de \Schumann{}.
 \item[\DateWithWeekDay{1920-04-14}]
 Petrograd~: petite salle du conservatoire.
 Premier concert avec des œuvres de \Scriabine{}, pour commémorer le
 cinquième anniversaire de la mort du compositeur
 \citep[voir][p.~142]{Nekrasova08}.
 Ce concert et celui du~24 avril ont un grand retentissement et contribuent
 beaucoup à la renommée naissante de \VSofronitsky{}, encore étudiant à
 l'époque \citep[voir][]{Bogdanov67a}.
 \citet[p.~8]{Artese} mentionne le programme suivant.

 \textsc{\Scriabine{}}~: Vingt-quatre Préludes, \Opus{11}~; Deux Poèmes,
 \Opus{32}~; Valse~; Sonate \Number{4} en \kF \Sharp majeur, \Opus{30}~;
 Études, \Opus{42}.
 \item[\DateWithWeekDay{1920-04-24}]
 Petrograd~: salle de l'académie chorale du peuple (actuelle Chapelle
 \Glinka{}).
 Reprise du concert du~14 avril~1920 avec des œuvres de \Scriabine{}.
 \item[\DateWithWeekDay{1920-05-02}]
 Petrograd~: petite salle du conservatoire.
 Concert.
 \item[B\DateWithWeekDay{1920-05-20}]
 Mariage de \VSofronitsky{} et \EScriabina{} à Petrograd.
 Le témoin était \AGlazounov{}.
 Jour exact du mariage indiqué par \RKoganSofronitskaya{}
 \citep[voir][p.~87]{Artese}.
 \item[\DateWithWeekDay{1920-08-17}]
 Petrograd~: Station de chemin de fer Pavlovskij.
 Une note a été conservée \citep[voir][p.~142]{Nekrasova08}, écrite par
 \VSofronitsky{} à \LNikolaiev{} le jour même du concert.

 \textsc{\Scriabine{}}~: Sonate en \kF \Sharp mineur, \Opus{23}~; Préludes
 des \Opus{27 et~37}~; Désir, \Opus{57} \Number{1}~; Ironies en \kC majeur,
 \Opus{56} \Number{2}~; Nuances, \Opus{56} \Number{3}~; Vers la flamme,
 \Opus{72}~; Prélude, \Opus{59} \Number{2}~; Prélude de l'\Opus{74}~;
 Masque, \Opus{63} \Number{1}~; Poème satanique, \Opus{36}.
 \item[\DateWithWeekDay{1920-12-19}]
 Petrograd~: salle de l'académie chorale du peuple.
 Apparition au début de la seconde partie d'un concert supplémentaire.

 \textsc{\Liszt{}}~: Méphisto-valse.
 \item[\DateWithWeekDay{1920-12-31}]
 Petrograd~: maison des arts.

 \textsc{\Liszt{}}~: Funérailles, S~173 \Number{7}~; \emph{La Campanella},
 S~141 \Number{3}~; \emph{Sonetto~123 del Petrarca}, S~161 \Number{6}~;
 Méphisto-valse.
 \textsc{\Scriabine{}}~: Deux Poèmes, \Opus{32}~; Prélude de l'\Opus{37}~;
 Deux Préludes de l'\Opus{11}~; Feuillet d'album en \kE \Flat majeur,
 \Opus{45} \Number{1}~; Désir, \Opus{57} \Number{1}~; Poème satanique,
 \Opus{36}~; Poème, \Opus{59} \Number{1}~; Masque, \Opus{63} \Number{1}~;
 Prélude de l'\Opus{74}~; Vers la flamme, \Opus{72}.
\end{description}

\section{Année~1921}

\begin{description}
 \item[\DateWithWeekDay{1921-01-22}]
 Petrograd~: Département de protection de la maternité et de l'enfance.
 Concert avec des œuvres de \Liszt{} et de \Schumann{}.
 \item[B\DateWithWeekDay{1921-03-31}]
 Naissance d'un fils, \ASofronitsky{}, à Petrograd.
 \item[\DateWithWeekDay{1921-04-08}]
 Petrograd~: maison des arts.
 Concert (reporté du~11 mars~1921) avec des œuvres de \Schumann{}.
 \item[\DateWithWeekDay{1921-05-13}]
 Petrograd~: conservatoire.
 Fin d'études au conservatoire de Petrograd, épreuve finale d'\hbox{État}
 (première partie).

 \textsc{\JBach{}}~: Prélude et fugue en \kB \Flat mineur du livre~II du
 \emph{Clavier bien tempéré}.
 \textsc{\Schumann{}}~: Études symphoniques, \Opus{13}.
 \textsc{\Liszt{}}~: Sonate en \kB mineur, S~178.

 \VSofronitsky{} lui-même parle d'une épreuve avec la Sonate en \kC mineur,
 \Opus{111}, de \Beethoven{}, la Fantaisie en \kC majeur, \Opus{17}, de
 \Schumann{}, la Sonate en \kB mineur, S~178, de \Liszt{}, et le Prélude en
 \kD mineur, \Opus{28} \Number{24}, de \Chopin{} \citep[voir][]{Vitsinsky}.
 \item[\DateWithWeekDay{1921-06-02}]
 Petrograd~: Épreuve finale d'\hbox{État} (seconde partie).
 \citet[p.~88]{Bogdanov65} se souvient de ce concert%
 \footnote{\foreignlanguage{russian}{\emph{Советская музыка}}, vol.~325,
 \Number{12} (1965), p.~83-91.}.

 \textsc{\Scriabine{}}~: Concerto en \kF \Sharp mineur, \Opus{20}.
 \item[B1921-06]
 Cinquante-sixième remise des diplômes du conservatoire de Petrograd.
 Attribution du Prix \ARubinstein{} à \MYudina{} et à \VSofronitsky{}.
 Selon la tradition depuis \Rubinstein{}, \Yudina{} et \Sofronitsky{}
 auraient dû recevoir chacun un piano à queue de la firme Schröder, mais, en
 raison des difficultés financières durant ces années, les instruments sont
 restés \Quote{sur le papier}.
 À propos de cette double épreuve d'\hbox{État}, voir \citet{Strelnikov}.
 \item[B1921 (début de l'été)]
 \VSofronitsky{} rencontre le compositeur, professeur, musicologue et
 critique musical \VBogdanovBerezovsky{} \citep[voir][]{Bogdanov67a}, mais
 ce dernier avait déjà pu écouter le pianiste lors des concerts \Scriabine{}
 en avril~1920 à Petrograd.
 \item[\DateWithWeekDay{1921-07-03}]
 Petrograd~: grande salle du conservatoire.
 Apparition lors d'une cérémonie publique pour la~56\ieme{} promotion du
 conservatoire.

 \textsc{\Liszt{}}~: Sonate en \kB mineur, S~178.
 \item[\DateWithWeekDay{1921-07-31}]
 Petrograd~: grande salle de la société philharmonique.
 Première apparition avec orchestre, sous la direction d'\ECooper{}, dans le
 cadre d'un cycle \Scriabine{}%
 \footnote{Voir \href{https://www.philharmonia.spb.ru/afisha/159512/}%
 {https://www.philharmonia.spb.ru/afisha/159512/}.}.

 \textsc{\Scriabine{}}~: Concerto en \kF \Sharp mineur, \Opus{20}.
 \item[\DateWithWeekDay{1921-09-10}]
 Moskva~: petite salle du \foreignlanguage{russian}{МУЗО Наркомпрос}
 (département de musique du commissariat pour l'éducation publique).
 Première apparition à Moskva \citep[voir][p.~30]{Sofronitsky82a}.

 \textsc{\Schumann{}}~: Carnaval, \Opus{9}.
 \textsc{\Liszt{}}~: Sonate en \kB mineur, S~178~; Méphisto-valse.
 \textsc{\Scriabine{}}~: Poème en \kF \Sharp majeur, \Opus{32} \Number{1}~;
 Prélude en \kB \Flat mineur~; Désir, \Opus{57} \Number{1}~; Vers la flamme,
 \Opus{72}~; Poème satanique, \Opus{36}~; Étude en \kD \Sharp mineur,
 \Opus{8} \Number{12} (Étude attribuée).

 La Méphisto-valse et le Poème \emph{Vers la flamme} ne figurent pas au
 programme indiqué par \citet[p.~395]{Scriabine}.
\end{description}

\section{Année~1922}

\begin{description}
 \item[1922-01]
 Petrograd~: maison des arts.
 Concert.
 \item[\DateWithWeekDay{1922-03-23} ou~25]
 Petrograd~: petite salle du conservatoire.
 Concert symphonique placé sous la direction d'\ECooper{}.
 \citet[p.~28]{Sofronitsky82a} donne la date du~25 mars.

 \textsc{\Scriabine{}}~: Concerto en \kF \Sharp mineur, \Opus{20}.
 \item[\DateWithWeekDay{1922-04-23} ou~25]
 Petrograd~: petite salle du conservatoire.
 Reprise du concert symphonique du~23 ou~25 mars~1922, sous la direction
 d'\ECooper{}%
 \footnote{Voir \href{https://www.philharmonia.spb.ru/afisha/160495/}%
 {https://www.philharmonia.spb.ru/afisha/160495/}, qui indique la date du~23
 avril~1922 et la grande salle du conservatoire de Petrograd.}.
 \citet[p.~28]{Sofronitsky82a} donne la date du~23 avril.
 \item[\DateWithWeekDay{1922-05-04}]
 Petrograd~: petite salle de la société philharmonique.

 \textsc{\Scriabine{}}~: Sonate en \kF \Sharp mineur, \Opus{23}~; Prélude de
 l'\Opus{37}~; Deux Préludes, \Opus{27}~; Nuances, \Opus{56} \Number{3}~;
 Ironies en \kC majeur, \Opus{56} \Number{2}~; Désir, \Opus{57} \Number{1}~;
 Poème satanique, \Opus{36}~; Poème, \Opus{59} \Number{1}~; Masque,
 \Opus{63} \Number{1}~; Vers la flamme, \Opus{72}~; Prélude, \Opus{57}
 [sic]~; Poème de l'\Opus{71}~; Prélude, \Opus{74} \Number{2}~; Sonate,
 \Opus{53}.
 \item[\DateWithWeekDay{1922-05-24}]
 Petrograd.
 Premier concert en soliste dans la grande salle de la société
 philharmonique%
 \footnote{Voir \href{https://www.philharmonia.spb.ru/afisha/160510/}%
 {https://www.philharmonia.spb.ru/afisha/160510/}.}.

 \textsc{\Schumann{}}~: Études symphoniques, \Opus{13}~; Carnaval, \Opus{9}.
 \textsc{\Liszt{}}~: Sonate en \kB mineur, S~178~; Funérailles, S~173
 \Number{7}~; Feux follets, S~139 \Number{5}~; \emph{La Campanella}, S~141
 \Number{3}~; Méphisto-valse \Number{1}, S~514.
 \item[\DateWithWeekDay{1922-07-05}]
 Petrograd~: École de musique \NRimskiKorsakov{}.
 Participation à un concert collectif.
\end{description}

\section{Année~1923}

\begin{description}
 \item[B1923]
 \VSofronitsky{} évite, dans la mesure du possible, les activités de concert
 et les charges pédagogiques, en dépit des offres nombreuses qu'il reçoit~;
 il se consacre en revanche à l'étude d'œuvres nouvelles pour son répertoire
 et à certains aspects techniques de sa sonorité et de son jeu
 \citep[voir][p.~28]{Sofronitsky82a}.
 \item[\DateWithWeekDay{1923-02-14}]
 Petrograd~: grande salle de la société philharmonique.
 Participation à un concert symphonique sous la direction d'\ECooper{}%
 \footnote{Voir \href{https://www.philharmonia.spb.ru/afisha/177794/}%
 {https://www.philharmonia.spb.ru/afisha/177794/}.}
 où \VSofronitsky{} joue néanmoins en solo~; le chef d'orchestre de ce
 concert monographique dirige la Symphonie \Number{2} et le Poème de
 l'\hbox{Extase}.

 \textsc{\Scriabine{}}~: Sonate, \Opus{53}.
 \item[\DateWithWeekDay{1923-05-23}]
 Petrograd~: grande salle de la société philharmonique.
 Participation à un concert symphonique sous la direction d'\ECooper{}.
 Concert annoncé dans les journaux \foreignlanguage{russian}{Музыка и театр}
 [Muzyka i teatr], \Number{20} (21~mai~1923), p.~19~;
 \foreignlanguage{russian}{Жизнь искусства} [Žizn' iskusstva], vol.~20,
 \Number{895} (22~mai~1923), p.~28.

 \textsc{\Scriabine{}}~: Concerto en \kF \Sharp mineur, \Opus{20}.
\end{description}

\section{Année~1924}

\begin{description}
 \item[\DateWithWeekDay{1924-02-12}]
 Leningrad~: conservatoire.
 Concert ou \foreignlanguage{german}{\emph{Klavierabend}} annoncé par le
 journal \foreignlanguage{russian}{Жизнь искусства} [Žizn' iskusstva],
 vol.~6, \Number{980} (5~février~1924), p.~29.
 L'annonce confirme que \VSofronitsky{} \Quote{ne s'est pas produit depuis
 longtemps}.
 Concert avec des œuvres de \Scriabine{}.
 \item[\DateWithWeekDay{1924-03-08}]
 Moskva~: petite salle du conservatoire.

 \textsc{\Schumann{}}~: Études symphoniques, \Opus{13}~; Fantaisie en \kC
 majeur, \Opus{17}.
 \textsc{\Chopin{}}~: Vingt-quatre Préludes, \Opus{28}.
 \textsc{\Scriabine{}}~: Sonate en \kF \Sharp majeur, \Opus{30}~; Étude en
 \kD \Sharp mineur, \Opus{8} \Number{12} (Étude attribuée).
 \item[\DateWithWeekDay{1924-04-05}]
 Moskva~: petite salle du conservatoire.
 Concert absent de la chronologie établie par \citet[p.~396]{Scriabine}.
 Une affiche annonce néanmoins le \Quote{deuxième concert} de \Sofronitsky{}
 avec des œuvres de \Scriabine{}.
 \item[\DateWithWeekDay{1924-04-22}]
 Moskva~: petite salle du conservatoire.
 Concert avec des œuvres de \Scriabine{}.
 \item[\DateWithWeekDay{1924-04-30}]
 Moskva~: petite salle du conservatoire.
 \citet{Lazarev20} présente l'affiche d'un concert collectif auquel a
 participé \VSofronitsky{} et qui était consacré au neuvième anniversaire de
 la mort de \Scriabine{}.
 Parmi les autres musiciens, figuraient aussi \KIgumnov{} et \SFeinberg{}.
 La recette du concert a été versée à l'entretien du musée \Scriabine{}.
 \item[\DateWithWeekDay{1924-05-09}]
 Leningrad~: grande salle de la société philharmonique.
 Concert avec des œuvres de \Scriabine{}.
 \item[\DateWithWeekDay{1924-05-17}]
 Leningrad~: grande salle de la société philharmonique.
 Concert avec des œuvres de \JBach{}/\Busoni{}, \Liszt{} et \Chopin{}.
 \item[B1924-08]
 Recension des concerts suivants (août~1924) par le critique et traducteur
 Viktor Pavlovič Kolomijcev%
 \footnote{\foreignlanguage{russian}{\emph{Вечерняя красная газета}},
 \Number{189} (22~août~1924) et \Number{260} (14~novembre~1924)~; extraits
 cités par \citet[p.~28-29]{Sofronitsky82a} et
 \citet[p.~422-423]{Milshteyn82a}.}.
 Critique par le compositeur, professeur et musicologue \VKaratigine{}%
 \footnote{\foreignlanguage{russian}{\emph{Жизнь искусства}}, vol.~35,
 \Number{1009} (26~août~1924), p.~14~; extraits cités par
 \citet[p.~29-30]{Sofronitsky82a}.}.
 \item[\DateWithWeekDay{1924-08-14}]
 Leningrad~: Cercle des amis de la musique de chambre.

 \textsc{\Scriabine{}}~: Sonate en \kF \Sharp majeur, \Opus{30}~; Sonate,
 \Opus{53}~; Préludes des \Opus{31, 37, 56 et~74}~; Désir, \Opus{57}
 \Number{1}~; Ironies en \kC majeur, \Opus{56} \Number{2}~; Nuances,
 \Opus{56} \Number{3}~; Poèmes des \Opus{52, 69, 63, 72 et~51}.
 \item[\DateWithWeekDay{1924-08-22}]
 Leningrad~: Cercle des amis de la musique de chambre.

 \textsc{\Scriabine{}}~: Pièces brèves~; Sonate en \kF \Sharp mineur,
 \Opus{23}~; Sonate en \kF \Sharp majeur, \Opus{30}.
 \item[\DateWithWeekDay{1924-08-26}]
 Leningrad~: Cercle des amis de la musique de chambre.
 Concert.
 \item[\DateWithWeekDay{1924-09-21}]
 Leningrad~: Cercle des amis de la musique de chambre.

 \textsc{\Scriabine{}}~: Sonate en \kF \Sharp mineur, \Opus{23}~; Préludes
 des \Opus{27, 31, 33, 35, 39, 48 et~49}~; Deux Poèmes, \Opus{32}~; Valse~;
 Poème tragique, \Opus{34}~; Études de l'\Opus{8}.
\end{description}

\section{Année~1925}

\begin{description}
 \item[\DateWithWeekDay{1925-02-13}]
 Leningrad~: grande salle de la société philharmonique%
 \footnote{Voir \href{https://www.philharmonia.spb.ru/afisha/181105/}%
 {https://www.philharmonia.spb.ru/afisha/181105/}.}.

 \textsc{\Schumann{}}~: Fantaisie en \kC majeur, \Opus{17}~; Carnaval,
 \Opus{9}.
 \textsc{\Liszt{}}~: Sonate en \kB mineur, S~178.
 \item[\DateWithWeekDay{1925-02-17}]
 Leningrad~: petite salle de la société philharmonique.

 \textsc{\Beethoven{}}~: Sonate en \kC \Sharp mineur, \Opus{27} \Number{2}.
 \textsc{\Schumann{}}~: \emph{Novelette} en \kE majeur, \Opus{21}
 \Number{7}~; \emph{Novelette} en \kF \Sharp mineur, \Opus{21} \Number{8}.
 \textsc{\Medtner{}}~: Deux \emph{Skazki}.
 \textsc{\Chopin{}}~: Vingt-quatre Préludes, \Opus{28}.
 \textsc{\Scriabine{}}~: Six Préludes~; Sonate.
 \item[\DateWithWeekDay{1925-04-11}]
 Leningrad~: grande salle de la société philharmonique.
 Concert pour commémorer le dixième anniversaire de la mort de \Scriabine{}%
 \footnote{Voir \href{https://www.philharmonia.spb.ru/afisha/181617/}%
 {https://www.philharmonia.spb.ru/afisha/181617/}.}.

 \textsc{\Scriabine{}}~: \emph{Allegro} de concert en \kB \Flat mineur,
 \Opus{18} (œuvre incertaine~: voir \citet[p.~396]{Scriabine})~; Trois
 Préludes de l'\Opus{37}~; Sonate en \kF \Sharp mineur, \Opus{23}~; Deux
 Poèmes, \Opus{32}~; Deux Préludes, \Opus{39} \Number{2} et \Number{3}~;
 Prélude en \kF majeur, \Opus{49} \Number{2}~; Nuances, \Opus{56}
 \Number{3}~; Danse languide, \Opus{51} \Number{4}~; Prélude en \kE majeur,
 \Opus{56} \Number{1}~; Prélude de l'\Opus{48}~; Poème satanique,
 \Opus{36}~; Sonate en \kF \Sharp majeur, \Opus{30}~; Vers la flamme,
 \Opus{72}~; Poème, \Opus{59} \Number{1}~; Énigme, \Opus{52} \Number{2}~;
 Prélude, \Opus{74} \Number{2}~; Sonate, \Opus{53}.
 \item[\DateWithWeekDay{1925-04-21}]
 Leningrad~: grande salle de la société philharmonique.
 Deuxième concert en mémoire de \Scriabine{}%
 \footnote{Voir \href{https://www.philharmonia.spb.ru/afisha/181623/}%
 {https://www.philharmonia.spb.ru/afisha/181623/}.}.

 \textsc{\Scriabine{}}~: Six Préludes~; Sonate en \kG \Sharp mineur,
 \Opus{19}~; Prélude en \kB \Flat majeur, \Opus{35} \Number{2}~; Prélude en
 \kF \Sharp majeur, \Opus{48} \Number{1}~; Mazurka extraite de l'\Opus{40}~;
 Poème tragique, \Opus{34}~; Fragilité, \Opus{51} \Number{1}~; Valse~;
 Poèmes des \Opus{52, 59 et~69}~; Prélude, \Opus{74} \Number{3}~;
 Guirlandes, \Opus{73} \Number{1}~; Flammes sombres, \Opus{73} \Number{2}~;
 Sonate, \Opus{70}~; Poèmes, \Opus{71}.
 \item[\DateWithWeekDay{1925-05-14}]
 Leningrad~: grande salle de la société philharmonique.
 \foreignlanguage{german}{\emph{Klavierabend}}%
 \footnote{Voir \href{https://www.philharmonia.spb.ru/afisha/181808/}%
 {https://www.philharmonia.spb.ru/afisha/181808/}.}.

 \textsc{\Pachelbel{}/\Nikolaiev{}}~: Toccata pour orgue (première
 représentation de cette œuvre).
 \textsc{\JBach{}}~: Prélude et fugue en \kB \Flat mineur.
 \textsc{\Schumann{}}~: Fantaisie en \kC majeur, \Opus{17}~; Études
 symphoniques, \Opus{13}.
 \textsc{\Liszt{}}~: \emph{Sposalizio}, S~161 \Number{1}~; Valse oubliée~;
 Méphisto-valse.
 \emph{Bis} -- \textsc{\Schumann{}}~: \emph{Intermezzo}.
 \item[\DateWithWeekDay{1925-05-23}]
 Leningrad~: grande salle de la société philharmonique.
 Concert avec des œuvres de \Chopin{} et de \Scriabine{}.
 \item[B1925-06]
 Rencontre du couple \Sofronitsky{} avec Vladimir Samojlovič Gorovic
 (Vladimir Horowitz) à Odésa \citep[voir][p.~56]{Juban}.
 \item[\DateWithWeekDay{1925-06-12}]
 Odésa~: société philharmonique.
 Première apparition dans la ville, lors d'un concert symphonique dirigé par
 \VBerdiaiev{}, pour commémorer le dixième anniversaire de la mort de
 \Scriabine{} \citep[voir][p.~152-153]{Nekrasova08}.

 \textsc{\Scriabine{}}~: Concerto en \kF \Sharp mineur, \Opus{20}.
 \item[\DateWithWeekDay{1925-06-16}]
 Odésa~: salle du Conseil municipal à l'hôtel de ville.
 Deuxième concert de \VSofronitsky{} à Odésa, pour commémorer le dixième
 anniversaire de la mort de \Scriabine{}.
 Voir \citet[p.~424]{Milshteyn82a}.

 \textsc{\Scriabine{}}~: Prélude de l'\Opus{13}~; Trois Préludes de
 l'\Opus{37}~; Sonate en \kF \Sharp mineur, \Opus{23}~; Deux Poèmes,
 \Opus{32}~; Deux Préludes de l'\Opus{39}~; Trois Préludes des \Opus{49, 35
 et~31}~; Nuances, \Opus{56} \Number{3}~; Deux Préludes des \Opus{56
 et~48}~; Poème satanique, \Opus{36}~; Sonate en \kF \Sharp majeur,
 \Opus{30}~; Sonate, \Opus{53}~; Vers la flamme, \Opus{72}.
 \item[\DateWithWeekDay{1925-07-31}]
 Leningrad~: grande salle de la société philharmonique.
 Participation à un concert symphonique dirigé par \ECooper{}.

 \textsc{\Scriabine{}}~: Concerto en \kF \Sharp mineur, \Opus{20}.
 \item[\DateWithWeekDay{1925-09-01}]
 Leningrad~: Cercle des amis de la musique de chambre.
 Concert avec des œuvres de \Schumann{} et de \Chopin{}.
 \item[\DateWithWeekDay{1925-09-25}]
 Leningrad~: Cercle des amis de la musique de chambre.
 Concert.
 \item[\DateWithWeekDay{1925-09-29}]
 Leningrad~: Cercle des amis de la musique de chambre.
 Concert à deux pianos avec \NGolubovskaya{}.
 Voir \citet[p.~435]{Milshteyn82a} et \citet[p.~397]{Scriabine}.

 \textsc{\Schumann{}}~: Variations en \kB \Flat majeur, \Opus{46}.
 \textsc{\JBach{}}~: Concertos en \kC mineur et en \kC majeur.
 \textsc{\Mozart{}}~: Sonate en \kD majeur, K~448.
 \item[B1925-09]
 \VSofronitsky{} exempté de conscription, à la demande en particulier
 d'\AGlazounov{}.
 \item[\DateWithWeekDay{1925-11-22}]
 Leningrad~: grande salle de la société philharmonique.
 Participation à un concert chorégraphique de chambre%
 \footnote{Voir \href{https://www.philharmonia.spb.ru/afisha/182649/}%
 {https://www.philharmonia.spb.ru/afisha/182649/} pour plus de détails à
 propos des participants et du programme général.}.

 \textsc{\Rachmaninov{}}~: Deux Préludes (en \kG majeur et en \kG mineur).
\end{description}

\section{Année~1926}

\begin{description}
 \item[\DateWithWeekDay{1926-01-20}]
 Leningrad~: petite salle de la société philharmonique.
 Apparition lors d'une soirée commémorative pour \VKaratigine{}.

 \textsc{\Medtner{}}~: Marche funèbre, \Opus{31} \Number{2}.
 \textsc{\Scriabine{}}~: Vers la flamme, \Opus{72}.
 \item[\DateWithWeekDay{1926-01-27}]
 Leningrad~: petite salle de la société philharmonique.

 \textsc{\JBach{}}~: Prélude et fugue.
 \textsc{\JBach{}/\Busoni{}}~: Deux Chorals pour orgue.
 \textsc{\Beethoven{}}~: Sonate en \kC mineur, \Opus{111}.
 \textsc{\Chopin{}}~: Ballade~; Nocturne~; Barcarolle en \kF \Sharp majeur,
 \Opus{60}.
 \textsc{\Liszt{}}~: Églogue, S~160 \Number{7}~; \emph{Sposalizio}, S~161
 \Number{1}~; \emph{Canzonetta del Salvator Rosa}, S~161 \Number{3}.
 \textsc{\Schubert{}/\Liszt{}}~: Deux Valses-caprices~; \emph{Erlkönig},
 S~558 \Number{4}.
 \item[\DateWithWeekDay{1926-01-30}]
 Leningrad~: grande salle de la société philharmonique.
 Apparition lors d'un concert symphonique dirigé par \NMalko{}%
 \footnote{Voir \href{https://www.philharmonia.spb.ru/afisha/183038/}%
 {https://www.philharmonia.spb.ru/afisha/183038/}.}.

 \textsc{\Scriabine{}}~: Prométhée (partie de piano), \Opus{60}.
 \item[\DateWithWeekDay{1926-02-14}]
 Leningrad~: Cercle des amis de la musique de chambre.
 Participation au premier concert de l'\hbox{Association} de musique
 contemporaine.
 Concert en matinée évoqué par \citet{Bogdanov67a}~: le compositeur
 mentionne plutôt des Variations, une série de dix pièces brèves avec des
 titres de programmes.

 \textsc{\BogdanovBerezovsky{}}~: Préludes.
 \item[\DateWithWeekDay{1926-02-16}]
 Leningrad.

 \textsc{\Beethoven{}}~: Sonate en \kC mineur, \Opus{111}.
 \textsc{\Chopin{}}.
 \textsc{\Liszt{}}.
 \textsc{\Schubert{}/\Liszt{}}.
 \item[\DateWithWeekDay{1926-02-19}]
 Leningrad~: petite salle de la société philharmonique.
 Participation au deuxième concert de l'\hbox{Association} de musique
 contemporaine.

 \textsc{\Scriabine{}}~: Sonate, \Opus{53}~; Trois Morceaux, \Opus{52}.
 \item[\DateWithWeekDay{1926-03-19}]
 Leningrad~: petite salle de la société philharmonique.
 Participation au troisième concert de l'\hbox{Association} de musique
 contemporaine.

 \textsc{\Scriabine{}}~: Sonate, \Opus{70}~; Quatre Préludes de l'\Opus{74}.
 \item[\DateWithWeekDay{1926-04-07}]
 Leningrad~: société philharmonique.
 Concert avec des œuvres de \Chopin{}.
 \item[\DateWithWeekDay{1926-04-08}]
 Leningrad~: grande salle de la société philharmonique.
 \foreignlanguage{german}{\emph{Klavierabend}}%
 \footnote{Voir \href{https://www.philharmonia.spb.ru/afisha/183834/}%
 {https://www.philharmonia.spb.ru/afisha/183834/}.}.

 \textsc{\Liszt{}}~: Funérailles, S~173 \Number{7}~; Sonate en \kB mineur,
 S~178~; \emph{Il penseroso}, S~161 \Number{2}~; \emph{Canzonetta del
 Salvator Rosa}, S~161 \Number{3}~; Valse en \kF \Sharp majeur~; Valse en
 \kA majeur~; Feux follets, S~139 \Number{5}~; \emph{Sonetto~123 del
 Petrarca}, S~161 \Number{6}.
 \textsc{\Schubert{}/\Liszt{}}~: Deux Valses-caprices~; \emph{Erlkönig},
 S~558 \Number{4}.
 \textsc{\Liszt{}}~: \emph{Sposalizio}, S~161 \Number{1}~; Méphisto-valse
 \Number{1}, S~514.
 \item[\DateWithWeekDay{1926-04-15}]
 Leningrad~: société philharmonique.
 Concert incertain \citep[voir][p.~397]{Scriabine}.
 \item[\DateWithWeekDay{1926-04-19}]
 Moskva~: grande salle du conservatoire.
 Première apparition avec l'orchestre \foreignlanguage{russian}{Персимфанс}
 [Persimfans\index[ndxnames]{Persimfans}], abréviation de
 \foreignlanguage{russian}{Первый симфонический ансамбль Моссовета без
 дирижёра} ou \Quote{Premier Ensemble symphonique du Conseil municipal de
 Moskva sans chef d'orchestre}.

 \textsc{\Scriabine{}}~: Concerto en \kF \Sharp mineur, \Opus{20}.
 \item[\DateWithWeekDay{1926-04-25}]
 145\ieme{}~concert du \foreignlanguage{russian}{Персимфанс}
 [Persimfans\index[ndxnames]{Persimfans}].
 Reprise du concert du~19 avril~1926.
 \item[\DateWithWeekDay{1926-04-28}]
 Moskva~: petite salle du conservatoire.

 \textsc{\Scriabine{}}~: Quatre Préludes, \Opus{22}~; Sonate, \Opus{23}~;
 Sonate, \Opus{53}~; Sonate, \Opus{70}~; Poème languide en \kB majeur,
 \Opus{52} \Number{3}~; Masque, \Opus{63} \Number{1}~; Étrangeté, \Opus{63}
 \Number{2}~; Deux Poèmes, \Opus{69}~; Vers la flamme, \Opus{72}~;
 Guirlandes, \Opus{73} \Number{1}~; Flammes sombres, \Opus{73} \Number{2}.
 \item[\DateWithWeekDay{1926-05-05}]
 Moskva~: petite salle du conservatoire.

 \textsc{\Liszt{}}~: Funérailles, S~173 \Number{7}~; Sonate en \kB mineur,
 S~178~; \emph{Sposalizio}, S~161 \Number{1}~; \emph{Il penseroso}, S~161
 \Number{2}~; Deux Valses~; Feux follets, S~139 \Number{5}~;
 \emph{Sonetto~123 del Petrarca}, S~161 \Number{6}~; Méphisto-valse.
 \textsc{\Schubert{}/\Liszt{}}.
 \item[\DateWithWeekDay{1926-05-08}]
 Moskva.
 Apparition au musée \Scriabine{} lors d'un concert collectif donné en
 raison du départ de \HNeuhaus{} \citep[voir][]{Lazarev20}.
 Les autres musiciens étaient Elena Aleksandrovna Bekman-Ščerbina, Vasilij
 Vasil'evič Nečaev et Samuil Evgen'evič \Feinberg{}%
 \footnote{\foreignlanguage{russian}{\emph{Вечерняя Москва}}, 5~mai~1926.}.

 \textsc{\Scriabine{}}~: Poème satanique, \Opus{36}~; Préludes de
 l'\Opus{74}~; Vers la flamme, \Opus{72}.
 \item[\DateWithWeekDay{1926-10-31}]
 Moskva~: petite salle du conservatoire.

 \textsc{\JBach{}}~: Prélude et fugue en \kB \Flat mineur.
 \textsc{\Schumann{}}~: Fantaisie en \kC majeur, \Opus{17}~; Huit
 \emph{Albumblätter}, \Opus{124}~; \emph{Novelette} en \kE majeur, \Opus{21}
 \Number{7}~; \emph{Novelette} en \kF \Sharp mineur, \Opus{21} \Number{8}~;
 Carnaval, \Opus{9}.
 \textsc{\Chopin{}}~: Nocturne en \kF majeur, \Opus{15} \Number{1}~;
 Nocturne en \kF \Sharp majeur, \Opus{15} \Number{2}~; Barcarolle en \kF
 \Sharp majeur, \Opus{60}.
 \item[\DateWithWeekDay{1926-11-01}]
 Moskva~: grande salle du conservatoire.
 Concert avec le \foreignlanguage{russian}{Персимфанс}
 [Persimfans\index[ndxnames]{Persimfans}].

 \textsc{\Glazounov{}}~: Concerto en \kF mineur, \Opus{92}.
 \item[\DateWithWeekDay{1926-11-06}]
 Moskva~: petite salle du conservatoire.
 \citet{Lazarev20} présente une affiche du concert.

 \textsc{\Medtner{}}~: \emph{Skazka} en \kF mineur~; Sonate en \kA \Flat
 majeur, \Opus{11} \Number{1}~; Sonate en \kD mineur, \Opus{11} \Number{2}~;
 Sonate en \kC majeur, \Opus{11} \Number{3}~; Marche funèbre en \kB mineur,
 \Opus{31} \Number{2}~; \emph{Skazka} en \kB mineur.
 \textsc{\Prokofiev{}}~: Sonate en \kA mineur, \Opus{28}.
 \textsc{\Scriabine{}}~: Sonate, \Opus{70}~; Préludes~; Valse~; Poème~;
 Sonate en \kF mineur, \Opus{6}.
 \item[\DateWithWeekDay{1926-11-08}]
 Moskva~: grande salle du conservatoire.
 Concert avec le \foreignlanguage{russian}{Персимфанс}
 [Persimfans\index[ndxnames]{Persimfans}].

 \textsc{\Glazounov{}}~: Concerto en \kF mineur, \Opus{92}.
 \item[\DateWithWeekDay{1926-11-17}]
 Moskva~: maison de la science.
 Concert.
 \item[\DateWithWeekDay{1926-11-19}]
 Moskva~: grande salle du conservatoire.
 Concert privé en matinée~; premier concert en soliste de \VSofronitsky{} à
 la grande salle du conservatoire de Moskva
 \citep[voir][p.~423-424]{Milshteyn82a}.

 \textsc{\Liszt{}}~: \emph{Sposalizio}, S~161 \Number{1}~; Sonate en \kB
 mineur, S~178.
 \textsc{\Schumann{}}~: Carnaval, \Opus{9}~; Études symphoniques, \Opus{13}.
 \textsc{\Scriabine{}}~: Sonate en \kF \Sharp mineur, \Opus{23}~; Poème
 satanique, \Opus{36}.
 \item[\DateWithWeekDay{1926-12-19}]
 Leningrad~: grande salle de la société philharmonique.
 Participation à un concert symphonique dirigé par \NMalko{}%
 \footnote{Voir \href{https://www.philharmonia.spb.ru/afisha/193125/}%
 {https://www.philharmonia.spb.ru/afisha/193125/}.}.

 \textsc{\Scriabine{}}~: Concerto en \kF \Sharp mineur, \Opus{20}.
\end{description}

\section{Année~1927}

\begin{description}
 \item[\DateWithWeekDay{1927-01-14}]
 Leningrad~: grande salle de la société philharmonique.
 \foreignlanguage{german}{\emph{Klavierabend}}%
 \footnote{Voir \href{https://www.philharmonia.spb.ru/afisha/193175/}%
 {https://www.philharmonia.spb.ru/afisha/193175/}.}.

 \textsc{\Schumann{}}~: \emph{Kreisleriana}, \Opus{16}~; Carnaval,
 \Opus{9}~; Sonate en \kF \Sharp mineur, \Opus{11}.
 \textsc{\Chopin{}}~: Vingt-quatre Préludes, \Opus{28}.
 \item[\DateWithWeekDay{1927-01-20}]
 Leningrad~: grande salle de la société philharmonique.
 \foreignlanguage{german}{\emph{Klavierabend}}%
 \footnote{Voir \href{https://www.philharmonia.spb.ru/afisha/193185/}%
 {https://www.philharmonia.spb.ru/afisha/193185/}.}.

 \textsc{\Medtner{}}~: Un \emph{Skazka} de l'\Opus{26}~; \emph{Skazka} en
 \kB mineur, \Opus{20} \Number{2}~; Sonate en \kA \Flat majeur, \Opus{11}
 \Number{1}~; Sonate en \kD mineur, \Opus{11} \Number{2}~; Sonate en \kC
 majeur, \Opus{11} \Number{3}~; Marche funèbre en \kB mineur, \Opus{31}
 \Number{2}.
 \textsc{\Scriabine{}}~: Sonate, \Opus{53}~; Poèmes (incertains).
 \textsc{\Balakirev{}}~: \emph{Islamey}, \Opus{18}.
 \textsc{\Prokofiev{}}~: Menuet, \Opus{32} \Number{2}~; Gavotte, \Opus{32}
 \Number{3}~; Sonate en \kA mineur, \Opus{28}.
 \item[\DateWithWeekDay{1927-02-13}]
 Leningrad~: Cercle des amis de la musique de chambre.
 Concert avec des œuvres de \Schumann{}, \Liszt{}, \Medtner{} et
 \Prokofiev{}.
 \item[\DateWithWeekDay{1927-02-25}]
 Leningrad~: Cercle des amis de la musique de chambre.
 Concert avec des œuvres de \Schumann{}, \Beethoven{} et \Chopin{}.
 \item[\DateWithWeekDay{1927-03-11}]
 Leningrad~: grande salle de la société philharmonique.
 Apparition lors d'un concert de pianistes issus de la classe de
 \LNikolaiev{}%
 \footnote{Voir \href{https://www.philharmonia.spb.ru/afisha/193552/}%
 {https://www.philharmonia.spb.ru/afisha/193552/}.}.
 Les six autres musiciens qui ont pris part à ce concert en commun étaient
 Aleksandr Danilovič \Kamensky{}, Isaj Mixajlovič Renzin, Iosif Zaxarovič
 Švarc, Konstantin Grigor'evič Šmidt, Dmitrij Dmitrievič \Chostakovitch{} et
 Marija Veniaminovna \Yudina{}.
 Concert évoqué par \citet{TADGO19270311}.
 Le programme ci-dessous a été joué par \VSofronitsky{}.

 \textsc{\Prokofiev{}}~: Six Visions fugitives, \Opus{22}~; Sonate en \kA
 mineur, \Opus{28}.
 \item[\DateWithWeekDay{1927-03-18}]
 Moskva~: grande salle du conservatoire.
 Apparition lors d'un concert de pianistes issus de la classe de
 \LNikolaiev{}.
 Même programme que le~11 mars~1927.
 \item[1927-05]
 Leningrad~: Cercle des amis de la musique de chambre.

 \textsc{\Scriabine{}}~: Sonate en \kF \Sharp mineur, \Opus{23}~; Sonate,
 \Opus{53}.
 \textsc{\Medtner{}}.
 \item[\DateWithWeekDay{1927-05-07}]
 Leningrad~: Cercle des amis de la musique de chambre.
 Concert avec des œuvres de \Schumann{}, \Medtner{} et \Scriabine{}.
 \item[\DateWithWeekDay{1927-05-20}]
 Leningrad~: société philharmonique.
 Concert.
 \item[\DateWithWeekDay{1927-06-17}]
 Leningrad~: Cercle des amis de la musique de chambre.
 Concert avec des œuvres de \Scriabine{}.
 \item[\DateWithWeekDay{1927-09-15}]
 Saratov~: Collège de musique.
 Selon \Vizel{}, il s'agit plutôt du~17 septembre~1927.
 Voir \citet[p.~424-425]{Milshteyn82a}.

 \textsc{\Schumann{}}~: Fantaisie en \kC majeur, \Opus{17}~; Carnaval,
 \Opus{9}.
 \textsc{\Liszt{}}~: Valse oubliée~; Méphisto-valse~; \emph{Sposalizio},
 S~161 \Number{1}.
 \textsc{\Scriabine{}}~: Poème satanique, \Opus{36}.
 \item[\DateWithWeekDay{1927-09-22}]
 Saratov~: Collège de musique.

 \textsc{\Schumann{}}~: Études symphoniques, \Opus{13}.
 \textsc{\Liszt{}}~: Sonate en \kB mineur, S~178.
 \textsc{\Medtner{}}~: Deux \emph{Skazki}.
 \textsc{\Scriabine{}}~: Deux Poèmes, \Opus{32}~; Étude en \kD \Sharp
 mineur, \Opus{8} \Number{12}~; Sonate en \kF \Sharp majeur, \Opus{30}.
 \item[1927-09]
 Astrakan.
 Concert (incertain).
 \item[\DateWithWeekDay{1927-10-26}]
 Odésa~: salle du Conseil municipal (ancienne Bourse).
 Premier concert en soliste lors d'un séjour à Odésa.
 Il est possible que ce concert ait été reporté au~29 octobre~1927.

 \textsc{\Beethoven{}}~: Sonate en \kC \Sharp mineur, \Opus{27} \Number{2}.
 \textsc{\Liszt{}}~: Sonate en \kB mineur, S~178.
 \textsc{\Medtner{}}~: Nouvelle en \kC mineur, \Opus{17} \Number{2}~;
 \emph{Skazka} en \kB mineur.
 \textsc{\Prokofiev{}}~: Gavotte~; Sonate en \kA mineur, \Opus{28}.
 \textsc{\Scriabine{}}~: Sonate, \Opus{53}.
 \item[\DateWithWeekDay{1927-11-02}]
 Odésa~: salle du Conseil municipal à l'hôtel de ville (anciennement, la
 Bourse), société philharmonique d'\hbox{Odésa}.
 Deuxième concert en soliste lors d'un séjour à Odésa.

 \textsc{\Schumann{}}~: \emph{Kreisleriana}, \Opus{16}~; Carnaval, \Opus{9}.
 \textsc{\Chopin{}}~: Nocturne en \kF majeur, \Opus{15} \Number{1}~;
 Barcarolle en \kF \Sharp majeur, \Opus{60}.
 \textsc{\Liszt{}}~: Funérailles, S~173 \Number{7}~; Valse oubliée~;
 \emph{Sposalizio}, S~161 \Number{1}~; \emph{Il penseroso}, S~161
 \Number{2}~; \emph{Canzonetta del Salvator Rosa}, S~161 \Number{3}~;
 Méphisto-valse.
 \item[\DateWithWeekDay{1927-11-05}]
 Odésa~: maison de la science.
 Troisième et dernier concert en soliste lors d'un séjour à Odésa.

 \textsc{\Schumann{}}~: Fantaisie en \kC majeur, \Opus{17}.
 \textsc{\Chopin{}}~: Nocturne en \kF \Sharp majeur, \Opus{15} \Number{2}~;
 Six Préludes.
 \textsc{\Liszt{}}~: \emph{Sonetto~123 del Petrarca}, S~161 \Number{6}~;
 Étude en \kF mineur, S~139 \Number{10}.
 \textsc{\Medtner{}}~: \emph{Skazka} en \kF mineur, \Opus{26} \Number{3}~;
 \emph{Skazka} en \kE mineur, \Opus{34} \Number{2}.
 \textsc{\Scriabine{}}~: Valse~; Une Mazurka de l'\Opus{40}~; Deux Poèmes
 des \Opus{32 et~52}~; Poème satanique, \Opus{36}~; Vers la flamme,
 \Opus{72}~; Sonate en \kF \Sharp majeur, \Opus{30}.
 \item[\DateWithWeekDay{1927-11-18}]
 Leningrad~: Cercle des amis de la musique de chambre.

 \textsc{\Medtner{}}~: Trois Sonates~; Deux \emph{Skazki}~; Marche funèbre
 en \kB mineur, \Opus{31} \Number{2}.
 \textsc{\Scriabine{}}~: Sonate, \Opus{53}.
 \textsc{\Liszt{}}.
 \textsc{\Prokofiev{}}~: Sonate en \kA mineur, \Opus{28}.
 \textsc{\Balakirev{}}~: \emph{Islamey}, \Opus{18}.
 \emph{Bis} -- \textsc{\Schumann{}}~: Carnaval, \Opus{9}.
 \item[\DateWithWeekDay{1927-12-02}]
 Leningrad~: Cercle des amis de la musique de chambre.
 Concert mentionné dans un autographe de \VMeyerhold{}, daté du~7
 décembre~1927 et retranscrit par \citet[p.~31]{Sofronitsky82a}~:
 \Quote{[...] [\VSofronitsky{}] est en musique à la fois un penseur et un
 poète}.
 Plus tard, en septembre~1934, \VMeyerhold{} dédiera à \VSofronitsky{} sa
 production de \emph{La Dame de pique} de \Tchaikovski{} (voir
 page~\pageref{bio:LDDP}).

 \textsc{\Schumann{}}~: Arabesque en \kC majeur, \Opus{18}~; \emph{Bunte
 Blätter}, \Opus{99}~; autres œuvres.
 \textsc{\Chopin{}}.
\end{description}

\section{Année~1928}

\begin{description}
 \item[\DateWithWeekDay{1928-01-15}]
 Leningrad~: grande salle de la société philharmonique.
 Concert en matinée%
 \footnote{Voir \href{https://www.philharmonia.spb.ru/afisha/199701/}%
 {https://www.philharmonia.spb.ru/afisha/199701/}.}
 (concert \Number{4}, abonnement \Number{7}).

 \textsc{\Prokofiev{}}~: Légende, \Opus{12} \Number{6}~; Cinq Visions
 fugitives de l'\Opus{22}~; Sarcasmes, \Opus{17}~; Danse, \Opus{32}
 \Number{1}~; Menuet, \Opus{32} \Number{2}~; Gavotte, \Opus{32} \Number{3}~;
 Scherzo et Marche pour piano extraits de l'opéra L'Amour des trois oranges,
 \Opus{33ter}~; Sonate en \kA mineur, \Opus{28}.
 \textsc{\Scriabine{}}~: Trois Études de l'\Opus{8}~; Trois Préludes de
 l'\Opus{37}~; Sonate en \kF \Sharp majeur, \Opus{30}~; Vers la flamme,
 \Opus{72}~; Deux Préludes de l'\Opus{74}~; Sonate, \Opus{68}~; Sonate,
 \Opus{70}.
 \item[\DateWithWeekDay{1928-01-19}]
 Moskva~: petite salle du conservatoire.
 Selon \Vizel{}, il s'agit plutôt du~18 janvier~1928.
 \citet{Lazarev20} présente une affiche qui porte la date du~19 janvier.

 \textsc{\Prokofiev{}}~: Légende, \Opus{12} \Number{6}~; Cinq Visions
 fugitives de l'\Opus{22}~; Sarcasmes, \Opus{17}~; Danse, \Opus{32}
 \Number{1}~; Menuet, \Opus{32} \Number{2}~; Gavotte, \Opus{32} \Number{3}~;
 Scherzo et Marche pour piano extraits de l'opéra L'Amour des trois oranges,
 \Opus{33ter}.
 \textsc{\Balakirev{}}~: \emph{Islamey}, \Opus{18}.
 \textsc{\Scriabine{}}~: Trois Études de l'\Opus{8}~; Trois Préludes de
 l'\Opus{37}~; Sonate en \kF \Sharp majeur, \Opus{30}~; Vers la flamme,
 \Opus{72}~; Deux Préludes de l'\Opus{74}~; Sonate, \Opus{70}~; Sonate,
 \Opus{68}.
 \item[\DateWithWeekDay{1928-01-25}]
 Moskva~: petite salle du conservatoire.

 \textsc{\Beethoven{}}~: Sonate en \kC \Sharp mineur, \Opus{27} \Number{2}~;
 Sonate en \kF mineur, \Opus{57}.
 \textsc{\Liszt{}}~: Années de pèlerinage~; Méphisto-valse~;
 \emph{Tarantella} de \emph{Venezia e Napoli}, S~162 \Number{3}.
 \item[\DateWithWeekDay{1928-02-01}]
 Moskva~: petite salle du conservatoire.

 \textsc{\Schumann{}}~: Fantaisie en \kC majeur, \Opus{17}.
 \textsc{\Medtner{}}~: \emph{Novelette} en \kC mineur, \Opus{17}
 \Number{2}~; \emph{Skazka} en \kE mineur, \Opus{34} \Number{2}~;
 \emph{Skazka} en \kF mineur, \Opus{26} \Number{3}~; \emph{Skazka} en \kB
 mineur, \Opus{20} \Number{2}.
 \textsc{\Prokofiev{}}~: Trois Visions fugitives de l'\Opus{22}~; Sarcasme,
 \Opus{17} \Number{5}~; Sonate en \kA mineur, \Opus{28}.
 \textsc{\Scriabine{}}~: Cinq Préludes~; Étude en \kD \Sharp mineur,
 \Opus{8} \Number{12}~; Vers la flamme, \Opus{72}~; Nuances, \Opus{56}
 \Number{3}~; Danse languide, \Opus{51} \Number{4}~; Prélude~; Désir,
 \Opus{57} \Number{1}~; Sonate, \Opus{53}.
 \emph{Bis} -- \textsc{\Rachmaninov{}}~: Prélude en \kG majeur, \Opus{32}
 \Number{5}~; Prélude en \kG \Sharp mineur, \Opus{32} \Number{12}.
 \item[\DateWithWeekDay{1928-02-26}]
 Leningrad~: petite salle du conservatoire.

 \textsc{\Glazounov{}}~: Prélude et fugue.
 \textsc{\Miaskovski{}}~: \foreignlanguage{russian}{Причуды} [Fantaisies],
 six sketches, \Opus{25}.
 \textsc{\Liadov{}}~: Quatre Pièces, \Opus{64}.
 \textsc{\Prokofiev{}}~: Trois Visions fugitives de l'\Opus{22}~; Sarcasmes,
 \Opus{17}.
 \textsc{\Scriabine{}}~: Sonate en \kF \Sharp majeur, \Opus{30}~; Sonate,
 \Opus{53}~; Sonate, \Opus{70}~; Vers la flamme, \Opus{72}.
 \item[\DateWithWeekDay{1928-03-04}]
 Leningrad~: grande salle de la société philharmonique.
 Participation à un concert symphonique dirigé par \AGauk{}%
 \footnote{Voir \href{https://www.philharmonia.spb.ru/afisha/200522/}%
 {https://www.philharmonia.spb.ru/afisha/200522/}.}
 (concert \Number{7}, abonnement \Number{6}), à la veille du départ de
 \Glazounov{}.

 \textsc{\Glazounov{}}~: Concerto en \kF mineur, \Opus{92}.
 \item[\DateWithWeekDay{1928-03-06}]
 Leningrad~: petite salle du conservatoire.
 Concert supplémentaire.
 Concert d'adieu avant le départ de \VSofronitsky{} en Pologne puis en
 France~; grand succès \citep[voir][p.~148]{Nekrasova08}.

 \textsc{\Beethoven{}}~: Sonate en \kC mineur, \Opus{111}~; Sonate en \kF
 mineur, \Opus{57}.
 \textsc{\Liszt{}}~: Sonate en \kB mineur, S~178.
 \textsc{\Schumann{}}~: Carnaval, \Opus{9}.
 \item[\DateWithWeekDay{1928-03-07}]
 Moskva~: petite salle du conservatoire (lieu de concert incertain).
 Concert supplémentaire.
 Concert d'adieu avant le départ de \VSofronitsky{} en Pologne puis en
 France~; selon \citet[p.~148]{Nekrasova08}, il s'agit d'une reprise du
 concert de la veille, donné à Leningrad.
 À la fin de ce concert très réussi, \Sofronitsky{} reçoit un message signé
 par les professeurs du conservatoire de Moskva et retranscrit par
 \citet[p.~31]{Sofronitsky82a}.
 \item[B\DateWithWeekDay{1928-03-08}]
 Voyage à l'étranger de \VSofronitsky{} et de son épouse~; le voyage les
 conduira tout d'abord à Warszawa en mars~1928, et ensuite beaucoup plus
 longuement à Paris de mars~1928 à janvier~1930.
 \item[B\DateWithWeekDay{1928-03-13}]
 Concert annulé en raison du voyage à l'étranger entrepris par
 \VSofronitsky{} et son épouse le~8 mars~1928.
 Le programme était annoncé avec les œuvres suivantes.

 \textsc{\Beethoven{}}~: Sonate en \kC mineur, \Opus{111}.
 \textsc{\Liszt{}}~: Sonate en \kB mineur, S~178.
 \textsc{\Chopin{}}~: Ballade en \kG mineur, \Opus{23}~; Nocturne en \kF
 majeur, \Opus{15} \Number{1}~; Prélude en \kB \Flat mineur, \Opus{28}
 \Number{16}~; Prélude en \kE \Flat majeur, \Opus{28} \Number{19}~; Prélude
 en \kC mineur, \Opus{28} \Number{20}~; Prélude en \kF majeur, \Opus{28}
 \Number{23}~; Prélude en \kD mineur, \Opus{28} \Number{24}~; Barcarolle en
 \kF \Sharp majeur, \Opus{60}.
 \textsc{\Schumann{}}~: Carnaval, \Opus{9}.
 \textsc{\Scriabine{}}~: Étude en \kD \Flat majeur, \Opus{8} \Number{10}~;
 Étude en \kD \Sharp mineur, \Opus{8} \Number{12}.
 \item[\DateWithWeekDay{1928-03-20}]
 Warszawa~: salle philharmonique.
 Voir en particulier \citet[p.~\hbox{32-33}]{Sofronitsky82a}, qui mentionne
 le programme détaillé du concert.
 Annoncé dans les journaux et sur les affiches de la ville, le concert est
 un grand succès, diffusé \via{} les stations de radio polonaises
 \citep[voir][p.~148]{Nekrasova08}.

 \textsc{\Schumann{}}~: Fantaisie en \kC majeur, \Opus{17}.
 \textsc{\Chopin{}}~: Nocturne en \kF majeur, \Opus{15} \Number{1}~;
 Barcarolle en \kF \Sharp majeur, \Opus{60}.
 \textsc{\Liszt{}}~: \emph{Sposalizio}, S~161 \Number{1}~; Valse oubliée
 \Number{1}, S~215 \Number{1}~; Une Méphisto-valse.
 \textsc{\Prokofiev{}}~: Légende, \Opus{12} \Number{6}.
 \textsc{\Scriabine{}}~: Poèmes, \Opus{32}~; Une Étude extraite de
 l'\Opus{8}~; Sonate en \kF \Sharp majeur, \Opus{30}~; Vers la flamme,
 \Opus{72}~; Poème satanique, \Opus{36}.
 \item[B1928-03 (à Warszawa)]
 Rencontre de \VSofronitsky{} avec son ancien professeur, \AMichalowski{},
 et le directeur de la société philharmonique de Warszawa, Roman Xojnackij
 \citep[voir][p.~\hbox{32-33}]{Sofronitsky82a}.
 \item[B1928-03 (fin)]
 Selon \citet[p.~148]{Nekrasova08}, c'est vers la fin du mois de mars que
 les époux \Sofronitsky{} arrivent à Paris, juste après ce bref séjour à
 Warszawa~; \citet[p.~112]{White} indique plutôt la fin du mois d'avril.
 \item[B1928-1930 (à Paris)]
 \VSofronitsky{} rencontre et fréquente Nikolaj Karlovič \Medtner{} et
 Sergej Sergeevič \Prokofiev{}.
 \Sofronitsky{} deviendra un proche ami de \Prokofiev{} et bénéficiera des
 conseils de \Medtner{} au sujet de la production sonore au piano et de
 l'interprétation de la Sonate \emph{Appassionata} de \Beethoven{} en \kF
 mineur, \Opus{57}
 \citep[voir \INikonovich{}, traduit en anglais par][p.~\hbox{7-9}]{White}.
 En outre, \VSofronitsky{} a l'occasion de rencontrer ou d'écouter des
 musiciens qui se produisent alors sur la scène parisienne~: Fëdor Ivanovič
 Šaljapin (Fédor Chaliapine), Titta Ruffo, Iosif (ou Jaša) Ruvimovič Xejfec
 (Jascha Heifetz), Mixail (ou Miša) Saulovič Èl'man (Mischa Elman), Arthur
 Rubinstein, Egon Petri, Willem Mengelberg, Bruno Walter, \SRachmaninov{},
 Walter Gieseking, Vladimir Samojlovič Gorovic (Vladimir Horowitz),
 Aleksandr Kirillovič Borovskij, Ignacy Jan Paderewski, Leopold Godowsky,
 Robert Casadesus, Wanda Landowska et de nombreux autres.

 À Paris, l'imprésario de \VSofronitsky{} était un certain Lublinsky
 \citep[voir][p.~56]{Juban}, qui lui avait peut-être été conseillé en
 Pologne au mois de mars~1928 \citep[voir][p.~89]{Artese}.
 Bien que les concerts de \Sofronitsky{} à Paris aient été des succès sur
 le plan musical, cet imprésario a échoué à en faire la promotion et à leur
 assurer un plus grand rayonnement auprès du public -- voir à ce sujet les
 propos de \RKoganSofronitskaya{} rapportés par \citet[p.~112-113]{White}
 puis par \citet[p.~89]{Artese}, les souvenirs de
 \citet[p.~131-132]{Nikonovich08a} et l'article de \citet[p.~56]{Juban}.
 Lublinsky organisait une poignée de concerts, puis disparaissait.
 Il portait une barbe imposante et, en guise d'humour lorsqu'il avait besoin
 des services de son imprésario, \VSofronitsky{} appelait ainsi l'éternel
 absent~: \foreignlanguage{russian}{\emph{Где моя \Quote{борода}~?!}}
 \Quote{Où est ma \Quote{barbe}~?!}
 \item[\DateWithWeekDay{1928-05-31}]
 Paris~: maison \Pleyel{}, salle \Debussy{}.
 Concert privé%
 \footnote{L'invitation précise que \VSofronitsky{} joue \Quote{pour ses
 invités, en matinée}.}
 et première représentation à Paris.
 Étaient présents dans la salle~: \SProkofiev{}, Aleksandr Kirillovič
 Borovskij, Boris de Schlœzer (Boris Fëdorovič Šlëcer) et Boris Stepanovič
 Zaxarov.
 Voir page~\pageref{rec:Paris0}.

 \textsc{\Beethoven{}}~: Sonate en \kC mineur, \Opus{111}.
 \textsc{\Schumann{}}~: Fantaisie en \kC majeur, \Opus{17}.
 \textsc{\Scriabine{}}~: Sonate, \Opus{53}.
 \emph{Bis} -- Plusieurs œuvres brèves de \Chopin{} et de \Scriabine{}.
 \item[B\DateWithWeekDay{1928-06-15}]
 Lettre adressée par \VSofronitsky{} aux sœurs \Vizel{}, où le musicien
 indique qu'il va jouer à Paris et à Berlin à l'automne, et qu'un engagement
 à Rome et à Amsterdam est en cours d'organisation, sans compter une
 invitation à jouer le Concerto de \Scriabine{} à Paris
 \citep[voir][p.~150]{Nekrasova08}~; il est possible que l'on doive à
 l'incompétence de l'imprésario Lublinsky le fait que \Sofronitsky{} ne
 jouera pas en dehors de Paris, et qu'il n'y jouera pas avec un orchestre.
 \item[\DateWithWeekDay{1928-06-18}]
 Paris~: salle des agriculteurs.
 Selon \citet[p.~400]{Scriabine}~: salle \Chopin{}.
 Premier concert public à Paris.
 Concert critiqué par \citeauthor{Messager}%
 \footnote{\emph{Comœdia}, 26~juin~1928.},
 \citeauthor{Golestan28a}%
 \footnote{\emph{Le Figaro}, vol.~103, \Number{178} (26~juin~1928), p.~5.},
 Boris de Schlœzer%
 \footnote{\emph{Les Dernières Nouvelles}, 26~juin~1928.},
 \citeauthor{Tromp}%
 \footnote{\emph{Excelsior}, 28~juin~1928.},
 \citeauthor{Ple}%
 \footnote{\emph{Courrier musical}, 1\ier{}~juillet~1928.}
 et \citeauthor{Wolf28}%
 \footnote{\emph{Courrier musical}, 1\ier{}~août~1928.}.
 Voir page~\pageref{rec:Paris1}.

 \textsc{\Liszt{}}~: Sonate en \kB mineur, S~178.
 \textsc{\Chopin{}}~: Ballade en \kG mineur, \Opus{23}~; Nocturne en \kF
 majeur, \Opus{15} \Number{1}~; Trois Préludes~; Barcarolle en \kF \Sharp
 majeur, \Opus{60}~; Deux Mazurkas~; Scherzo en \kB \Flat mineur, \Opus{31}.
 \textsc{\Scriabine{}}~: Étude en \kD \Sharp mineur, \Opus{8} \Number{12}~;
 Étude en \kD \Flat majeur, \Opus{8} \Number{10}~; Un Poème de l'\Opus{32}~;
 Préludes, \Opus{74}~; Sonate, \Opus{53}.
 \item[\DateWithWeekDay{1928-06-25}]
 Paris~: salle des agriculteurs.
 Selon le fils du pianiste, cité par \citet[p.~400]{Scriabine}~: salle
 \Chopin{}.
 Deuxième concert.
 Concert critiqué par \citeauthor{Golestan28a}%
 \footnote{\emph{Le Figaro}, vol.~103, \Number{178} (26~juin~1928), p.~5.}.
 Voir page~\pageref{rec:Paris2}.

 \textsc{\Beethoven{}}~: Sonate en \kC \Sharp mineur, \Opus{27} \Number{2}~;
 Sonate en \kF mineur, \Opus{57}.
 \textsc{\Schumann{}}~: Carnaval, \Opus{9}.
 \textsc{\Poulenc{}}~: Trois Promenades, FP~24 \Number{2}, \Number{4} et
 \Number{8}.
 \textsc{\Prokofiev{}}~: Trois Sarcasmes extraits de l'\Opus{17}.
 \textsc{\Liszt{}}~: Méphisto-valse.
 \item[B1928 (été)]
 Voyage de \VSofronitsky{} et de son épouse, avec \VDukelsky{} et
 \SProkofiev{}, dans le massif du Mont-Blanc \citep[voir][p.~400]{Scriabine}
 et en Suisse \citep[voir][p.~58]{Juban}, en passant par les rives du Léman
 \citep[voir][p.~150]{Nekrasova08}~; \Prokofiev{} conduit l'automobile
 (\Quote{110~verstes par heure}, soit~117\,km/h).
 Les époux \Sofronitsky{} avaient quitté Paris le~7 juillet~1928
 \citep[voir][p.~366]{NikonovichScriabine08}.
 Une lettre de \Sofronitsky{} aux sœurs \Vizel{}, datée du~1\ier{}~août, a
 été envoyée de Saint-Palais-sur-Mer et évoque à nouveau de paisibles
 vacances \citep[voir][p.~150]{Nekrasova08}~: l'océan, une forêt de pins et
 des photographies prises sur la plage.
 \item[1928 (été ou automne)]
 Concert avec des œuvres de \Prokofiev{} à la résidence privée d'un riche
 Français, Orcel.
 \item[\DateWithWeekDay{1928-10-26}]
 Paris.
 Radiodiffusion de l'émission \emph{Radio Liberté} à~12\up{h}\,30\up{m} dont
 le programme est mentionné par \citet{liberte1928-10-26}.

 \textsc{\Chopin{}}~: Trois Préludes~; Barcarolle en \kF \Sharp majeur,
 \Opus{60}.
 \item[\DateWithWeekDay{1928-10-29}]
 Paris~: maison \Pleyel{}, salle \Chopin{}.
 Selon \citet[p.~401]{Scriabine}~: salle \Debussy{}.
 Le fils du pianiste indique la salle \Chopin{}.
 Troisième concert.
 Concert annoncé par \citeauthor{CarolBerard}%
 \footnote{\emph{La Semaine à Paris}, \Number{335} (26~octobre~1928),
 p.~49.}
 et critiqué par \citeauthor{Golestan28b}%
 \footnote{\emph{Le Figaro}, vol.~103, \Number{307} (2~novembre~1928),
 p.~4.}.
 Voir page~\pageref{rec:Paris3}.

 \textsc{\Schumann{}}~: Études symphoniques, \Opus{13}.
 \textsc{\Liszt{}}~: \emph{Sposalizio}, S~161 \Number{1}~; \emph{Canzonetta
 del Salvator Rosa}, S~161 \Number{3}~; \emph{Il penseroso}, S~161
 \Number{2}~; Méphisto-valse.
 \textsc{\Medtner{}}~: \emph{Skazka} en \kF mineur, \Opus{26} \Number{3}~;
 \emph{Skazka} en \kB \Flat mineur, \Opus{20} \Number{1}~; \emph{Skazka} en
 \kB mineur, \Opus{20} \Number{2}.
 \textsc{\Prokofiev{}}~: Trois Visions fugitives de l'\Opus{22}~; Sonate en
 \kA mineur, \Opus{28}.
 \textsc{\Ravel{}}~: Sonatine.
 \textsc{\Scriabine{}}~: Poème et Énigme de l'\Opus{52}~; Poème satanique,
 \Opus{36}.
 \item[\DateWithWeekDay{1928-11-05}]
 Paris~: maison \Pleyel{}, salle \Chopin{}.
 Quatrième concert.
 Concert annoncé par \citeauthor{CarolBerard}%
 \footnote{\emph{La Semaine à Paris}, \Number{335} (26~octobre~1928),
 p.~49.}.
 Voir page~\pageref{rec:Paris4}.

 \textsc{\Chopin{}}~: Fantaisie en \kF mineur, \Opus{49}~; Ballade en \kG
 mineur, \Opus{23}~; Ballade en \kF mineur, \Opus{52}~; Huit Préludes~;
 Sonate en \kB mineur, \Opus{58}~; Nocturne en \kF majeur, \Opus{15}
 \Number{1}~; Quatre Études~; Barcarolle en \kF \Sharp majeur, \Opus{60}~;
 Tarentelle en \kA \Flat majeur, \Opus{43}~; Deux Mazurkas~; Scherzo en \kB
 mineur, \Opus{20}.
 \item[B\DateWithWeekDay{1928-11-08}]
 Lettre de \VSofronitsky{} à \AVizel{} (Ada), où le musicien annonce son
 interprétation de la Barcarolle de \Chopin{} en \kF \Sharp majeur,
 \Opus{60}, qu'il a jouée pour la radio, \Quote{jeudi à~12\up{h}}~; voir
 \citet[p.~150]{Nekrasova08}.
\end{description}

\section{Année~1929}

\begin{description}
 \item[\DateWithWeekDay{1929-01-11}]
 Paris~: maison \Pleyel{}, salle \Chopin{}.
 Cinquième concert.
 Concert critiqué de façon détaillée par \citeauthor{Baruzi}%
 \footnote{\emph{Le Ménestrel}, vol.~4838, \Number{3} (18~janvier~1929),
 p.~27-28.}.
 Voir page~\pageref{rec:Paris5}.
 \citet[p.~151]{Nekrasova08} mentionne Vers la flamme, \Opus{72}, à la place
 du Poème, \Opus{32} \Number{1}, mais c'est infirmé par le compte-rendu
 détaillé de \citeauthor{Baruzi}, qui a assisté au concert.

 \textsc{\Schumann{}}~: \emph{Bunte Blätter}, \Opus{99}~; \emph{Novelette}
 en \kE majeur, \Opus{21} \Number{7}~; \emph{Novelette} en \kF \Sharp
 mineur, \Opus{21} \Number{8}.
 \textsc{\Liszt{}}~: Après une lecture de Dante, S~161 \Number{7}.
 \textsc{\Ravel{}}~: Sonatine.
 \textsc{\Debussy{}}~: \emph{Doctor Gradus ad Parnassum}, L~113 \Number{I}~;
 \emph{General Lavine -- eccentric}, L~123 \Number{VI}.
 \textsc{\Prokofiev{}}~: Cinq Sarcasmes, \Opus{17}.
 \textsc{\Scriabine{}}~: Poème en \kF \Sharp majeur, \Opus{32} \Number{1}~;
 Étude en \kD \Sharp mineur, \Opus{8} \Number{12}~; Sonate en \kF \Sharp
 majeur, \Opus{30}.
 \item[\DateWithWeekDay{1929-01-21}]
 Paris.
 Concert chez Dubost.
 \item[\DateWithWeekDay{1929-02-07}]
 Paris~: Ambassade de l'\hbox{URSS}.
 Concert avec \SProkofiev{}~: les deux musiciens ont joué, sur deux pianos,
 la Suite de Valses de \Schubert{} dans la transcription de \Prokofiev{}.
 Hôte des musiciens à l'\hbox{Ambassade}~: Valerian Savel'evič Dovgalevskij
 \citep[voir][p.~36]{Sofronitsky82a}.
 \item[\DateWithWeekDay{1929-03-05}]
 Paris~: Ambassade de l'\hbox{URSS}.
 Concert avec \SProkofiev{}~: les deux musiciens ont joué, sur deux pianos,
 la Suite de Valses de \Schubert{} dans la transcription de \Prokofiev{}.
 Voir \citet[p.~359-360]{Prokofiev08}.
 Hôte des musiciens à l'\hbox{Ambassade}~: Valerian Savel'evič Dovgalevskij
 \citep[voir][p.~36]{Sofronitsky82a}.
 \item[\DateWithWeekDay{1929-05-17}]
 Paris~: maison \Pleyel{}, salle \Chopin{}.
 Sixième concert.
 Concert critiqué par \citeauthor{oeuvre1929-05-24}%
 \footnote{\emph{L'Œuvre}, \Number{4984} (24~mai~1929), p.~6.}.
 Voir page~\pageref{rec:Paris6}.
 Concert évoqué, de même que les deux précédents avec \SProkofiev{}, dans
 une lettre de \VSofronitsky{} à \AVizel{} (Ada) datée du~7 avril
 \citep[voir][p.~151]{Nekrasova08}.

 \textsc{\Beethoven{}}~: Sonate en \kC mineur, \Opus{111}.
 \textsc{\Schumann{}}~: Carnaval, \Opus{9}.
 \textsc{\Chopin{}}~: Nocturne en \kF \Sharp majeur, \Opus{15} \Number{2}~;
 Mazurka en \kC \Sharp mineur~; Scherzo en \kB mineur, \Opus{20}.
 \textsc{\Liszt{}}~: Valse oubliée.
 \textsc{\Schubert{}/\Liszt{}}~: Valse-caprice~; \emph{Erlkönig}, S~558
 \Number{4}.
 \item[\DateWithWeekDay{1929-05-25}]
 Concert avec des œuvres de \Prokofiev{} à la résidence privée d'un riche
 Français, Orcel.
 Le programme comportait en particulier la Sonate en \kA mineur, \Opus{28},
 et des Visions fugitives, \Opus{22}.
 \citet[p.~360]{Prokofiev08} mentionne en outre l'\hbox{Ouverture} sur des
 thèmes juifs en \kC mineur pour clarinette, piano et quatuor à cordes,
 \Opus{34}, des œuvres pour violon, des Romances, la Ballade en \kC mineur
 pour violoncelle et piano, \Opus{15},~etc., où \VSofronitsky{} tenait la
 partie de piano.
 \item[\DateWithWeekDay{1929-12-20}]
 Paris~: maison \Pleyel{}, salle \Chopin{}.
 Selon \citet[p.~401]{Scriabine}~: salle \Debussy{}.
 Le fils du pianiste indique la salle \Chopin{}.
 Septième et dernier concert à Paris~; d'après les personnes présentes, un
 concert particulièrement bon \citep[p.~151]{Nekrasova08}.
 Voir page~\pageref{rec:Paris7}.

 \textsc{\Chopin{}}~: Sonate en \kB mineur, \Opus{58}~; Sonate en \kB \Flat
 mineur, \Opus{35}.
 \textsc{\Scriabine{}}~: Sonate, \Opus{70}~; Sonate, \Opus{53}~; plusieurs
 autres œuvres de forme brève.
 \item[\DateWithWeekDay{1929-12-21}]
 Paris.
 Reprise du concert du~20 décembre~1929 (date incertaine).
 \item[B\DateWithWeekDay{1929-12-21}]
 Extrait d'une lettre de \VSofronitsky{} à la famille \Vizel{}, datée du~21
 décembre~: \foreignlanguage{russian}{\emph{С каждым днем мы все больше и
 мучительнее тоскуем по родине.
 Франция надоела, но просидим здесь до конца января.}}
 \Quote{Chaque jour, nous nous languissons de plus en plus douloureusement
 de notre patrie.
 La France est ennuyeuse, mais nous resterons ici jusqu'à la fin du mois de
 janvier.}
 \citep[Voir][p.~151.]{Nekrasova08}
 \citet[p.~132]{Nikonovich08a} ajoute que pour \VSofronitsky{}, un Russe ne
 peut pas vivre longtemps sans la Rossija.
 Il ne pourrait rien y avoir de plus triste que l'émigration.
\end{description}

\section{Année~1930}

\begin{description}
 \item[B1930-01 (dix derniers jours)]
 \VSofronitsky{} quitte Paris et la France.
 \item[B\DateWithWeekDay{1930-01-27}]
 Retour de \VSofronitsky{} en Union soviétique, à Leningrad.
 \EScriabina{}, quant à elle, reste en France et ne revient qu'en~1934 en
 Union soviétique~: elle arrivera à Moskva à la mi-décembre~1934, puis à
 Leningrad entre la fin janvier et le début février~1935.
 \citet[p.~90]{Bogdanov65} évoque le changement de style de jeu du pianiste,
 après son séjour en Pologne et en France%
 \footnote{\foreignlanguage{russian}{\emph{Советская музыка}}, vol.~325,
 \Number{12} (1965), p.~83-91.}.
 \item[\DateWithWeekDay{1930-03-17}]
 Moskva~: grande salle du conservatoire.
 Premier concert après le retour de France.
 Concert donné par le \foreignlanguage{russian}{Персимфанс}
 [Persimfans\index[ndxnames]{Persimfans}] avec \VSofronitsky{} en tant que
 soliste.
 \citet[p.~152]{Nekrasova08} indique une interprétation du Concerto pour
 piano de \Scriabine{} en \kF \Sharp mineur, \Opus{20}, avec l'orchestre
 Persimfans\index[ndxnames]{Persimfans}.

 \textsc{\Liszt{}}~: Sonate en \kB mineur, S~178.
 \item[\DateWithWeekDay{1930-04-27}]
 Moskva~: grande salle du conservatoire.
 Concert avec des œuvres de \Scriabine{} pour commémorer le quinzième
 anniversaire de la mort du compositeur.

 \textsc{\Scriabine{}}~: Deux Poèmes, \Opus{32}~; Trois Préludes extraits de
 l'\Opus{37}~; Sonate \Number{3} en \kF \Sharp mineur, \Opus{23}~; Sonate
 \Number{5}, \Opus{53}~; Sonate \Number{4} en \kF \Sharp majeur, \Opus{30}~;
 Valse~; Poème, \Opus{59} \Number{1}~; Énigme, \Opus{52} \Number{2}~; Sonate
 \Number{10}, \Opus{70}~; Poème satanique, \Opus{36}~; Un Prélude extrait de
 l'\Opus{74}~; Vers la flamme, \Opus{72}.
 \item[\DateWithWeekDay{1930-05-09}]
 Leningrad~: grande salle de la société philharmonique.

 \textsc{\Scriabine{}}~: Deux Poèmes, \Opus{32}~; Trois Préludes extraits de
 l'\Opus{37}~; Sonate \Number{3} en \kF \Sharp mineur, \Opus{23}~; Sonate
 \Number{5}, \Opus{53}~; Sonate \Number{4} en \kF \Sharp majeur, \Opus{30}~;
 Valse~; Poème, \Opus{52} \Number{1}~; Énigme, \Opus{52} \Number{2}~; Sonate
 \Number{10}, \Opus{70}~; Poème satanique, \Opus{36}~; Un Prélude extrait de
 l'\Opus{74}~; Vers la flamme, \Opus{72}.
 \item[\DateWithWeekDay{1930-05-17}]
 Leningrad~: grande salle de la société philharmonique.

 \textsc{\JBach{}/\Busoni{}}~: Trois Préludes de chorals.
 \textsc{\Liszt{}}~: Après une lecture de Dante, S~161 \Number{7}.
 \textsc{\Chopin{}}~: Sonate \Number{3} en \kB mineur, \Opus{58}~; Ballade~;
 Nocturne~; Barcarolle en \kF \Sharp majeur, \Opus{60}~; Deux Études~;
 Scherzo \Number{1} en \kB mineur, \Opus{20}.
 \item[\DateWithWeekDay{1930-05-31}]
 Leningrad~: grande salle de la société philharmonique.
 Troisième concert (ou \foreignlanguage{german}{\emph{Klavierabend}}) de
 \VSofronitsky{}%
 \footnote{Voir
 \href{https://100philharmonia.spb.ru/historical-poster/7165/}%
 {https://100philharmonia.spb.ru/historical-poster/7165/}.}
 à Leningrad après le retour de France.

 \textsc{\Schumann{}}~: Études symphoniques, \Opus{13}.
 \textsc{\Chopin{}}~: Fantaisie en \kF mineur, \Opus{49}~; Deux Mazurkas~;
 Scherzo \Number{2} en \kB \Flat mineur, \Opus{31}~; Sonate \Number{2} en
 \kB \Flat mineur, \Opus{35}.
 \textsc{\Liszt{}}~: \emph{Il penseroso}, S~161 \Number{2}~; Églogue, S~160
 \Number{7} (œuvre incertaine~: \Quote{Chant d'un berger})~; Deux Valses~;
 Méphisto-valse~; \emph{Sonetto del Petrarca}~; Deux Études.
 \item[\DateWithWeekDay{1930-06-08}]
 Moskva~: grande salle du conservatoire.
 Concert avec des œuvres de \Chopin{} pour les étudiants du conservatoire.

 \textsc{\Chopin{}}~: Sonate \Number{2} en \kB \Flat mineur, \Opus{35}~;
 Nocturne~; Préludes~; Quatre Mazurkas~; Études~; Valses~; Fantaisie~;
 Scherzo \Number{1} en \kB mineur, \Opus{20}.
 \item[\DateWithWeekDay{1930-06-15}]
 Leningrad~: petite salle du conservatoire (lieu incertain).
 Concert avec des œuvres de \Chopin{}.
 Tout en donnant la date du~15 mai -- et non celle du~15 juin --,
 \citet[p.~152]{Nekrasova08} indique le programme \Chopin{} suivant.

 \textsc{\Chopin{}}~: Fantaisie en \kF mineur, \Opus{49}~; Sonate en \kB
 mineur, \Opus{58}~; Ballade en \kG mineur, \Opus{23}~; Nocturne en \kF
 \Sharp majeur, \Opus{15} \Number{2}~; Nocturne en \kF majeur, \Opus{15}
 \Number{1}~; Deux Mazurkas en \kC mineur~; Barcarolle en \kF \Sharp majeur,
 \Opus{60}~; Scherzo en \kB mineur, \Opus{20}~; Sonate en \kB \Flat mineur,
 \Opus{35}.
 \item[\DateWithWeekDay{1930-07-05}]
 Leningrad~: petite salle du conservatoire.
 Concert pour les étudiants.
 Concert inachevé en raison de problèmes cardiaques.

 \textsc{\JBach{}/\Busoni{}}~: Deux Préludes de chorals.
 \textsc{\Schumann{}}~: \emph{Kreisleriana}, \Opus{16}.
 \textsc{\Chopin{}}~: Fantaisie.
 \textsc{\Liszt{}}~: Méphisto-valse.
 \textsc{\Scriabine{}}~: œuvres de forme brève~; Sonate.
 \item[\DateWithWeekDay{1930-10-08}]
 Moskva~: grande salle du conservatoire.

 \textsc{\JBach{}}~: Suite française \Number{5} en \kG majeur.
 \textsc{\Haydn{}}~: Sonate en \kD majeur, Hob.XVI:37.
 \textsc{\Schumann{}}~: Deux Novelettes, \Opus{21} (\Number{7} en \kE majeur
 et une autre).
 \textsc{\Chopin{}}~: Ballade \Number{4} en \kF mineur, \Opus{52}~;
 Mazurka~; Quatre Préludes.
 \textsc{\Liszt{}}~: Un \emph{Sonetto del Petrarca}, S~161~;
 \emph{Tarantella} de \emph{Venezia e Napoli}, S~162 \Number{3}.
 \item[\DateWithWeekDay{1930-10-11}]
 Leningrad~: grande salle de la société philharmonique.
 Apparition lors d'un concert symphonique dirigé par \AGauk{}.

 \textsc{\Scriabine{}}~: Concerto en \kF \Sharp mineur, \Opus{20}.
 \item[\DateWithWeekDay{1930-10-17}]
 Leningrad~: grande salle de la société philharmonique.

 \textsc{\JBach{}}~: Suite française \Number{5} en \kG majeur.
 \textsc{\Haydn{}}~: Sonate en \kD majeur, Hob.XVI:37.
 \textsc{\Schumann{}}~: Carnaval, \Opus{9}.
 \textsc{\Chopin{}}~: Ballade \Number{4} en \kF mineur, \Opus{52}~;
 Mazurka~; Quatre Préludes~; Valse en \kG \Flat majeur, \Opus{70}
 \Number{1}.
 \textsc{\Liszt{}}~: \emph{Gnomenreigen}, S~145 \Number{2}~; Un
 \emph{Sonetto del Petrarca}, S~161~; \emph{Tarantella} de \emph{Venezia e
 Napoli}, S~162 \Number{3}.
 \item[\DateWithWeekDay{1930-10-21}]
 Leningrad~: grande salle de la société philharmonique.

 \textsc{\Schumann{}}~: Fantaisie en \kC majeur, \Opus{17}.
 \textsc{\Chopin{}}~: Deux Mazurkas~; Deux Études~; Ballade~; Nocturne~;
 Sonate \Number{2} en \kB \Flat mineur, \Opus{35}.
 \textsc{\Liszt{}}~: Deux Études.
 \item[\DateWithWeekDay{1930-10-24}]
 Moskva~: grande salle du conservatoire.
 Même programme que lors du concert du~21 octobre~1930 à Leningrad.
 \item[\DateWithWeekDay{1930-11-21}]
 Leningrad~: grande salle de la société philharmonique.
 Concert avec des œuvres de \JBach{}, \Scarlatti{}, \Chopin{}, \Prokofiev{},
 \Debussy{} et \Scriabine{}.
 Selon \AVizel{}, il s'agit plutôt du~22 novembre~1930.
 \item[\DateWithWeekDay{1930-12-04}]
 Leningrad~: grande salle de la société philharmonique.

 \textsc{\Schumann{}}~: Sonate en \kF \Sharp mineur, \Opus{11}.
 \textsc{\Liszt{}}~: Sonate en \kB mineur, S~178.
 \textsc{\Chopin{}}~: Polonaise en \kF \Sharp mineur, \Opus{44}~; Mazurka~;
 Une Valse en \kD \Flat majeur~; Valse~; Scherzo \Number{1} en \kB mineur,
 \Opus{20}.
 \item[\DateWithWeekDay{1930-12-08}]
 Leningrad~: grande salle de la société philharmonique.

 \textsc{\Schubert{}}~: Un Impromptu, D~935~; Trois Impromptus, D~899.
 \textsc{\Beethoven{}}~: Sonate \Number{23} en \kF mineur, \Opus{57}.
 \textsc{\Ravel{}}~: Sonatine.
 \textsc{\Prokofiev{}}~: Cinq Pièces.
 \textsc{\Scriabine{}}~: Valse~; Sonate \Number{5}, \Opus{53}.
 \item[\DateWithWeekDay{1930-12-14}]
 Moskva~: grande salle du conservatoire.

 \textsc{\Schubert{}}~: Un Impromptu, D~935~; Trois Impromptus, D~899.
 \textsc{\Beethoven{}}~: Sonate \Number{23} en \kF mineur, \Opus{57}.
 \textsc{\Ravel{}}~: Sonatine.
 \textsc{\Prokofiev{}}~: Six Pièces.
 \textsc{\Scriabine{}}~: Deux Pièces~; Sonate \Number{5}, \Opus{53}.
 \item[\DateWithWeekDay{1930-12-28}]
 Leningrad~: grande salle de la société philharmonique.
 Apparition lors d'un concert de pianistes issus de la classe de
 \LNikolaiev{}.

 \textsc{\Schumann{}}~: Carnaval, \Opus{9}.
\end{description}

\section{Année~1931}

\begin{description}
 \item[\DateWithWeekDay{1931-01-10}]
 Leningrad~: grande salle de la société philharmonique.
 Voir en particulier \citet[p.~444]{Milshteyn82a}.
 \citet[p.~402]{Scriabine} parlent plutôt d'un concert à la grande salle du
 conservatoire de Moskva, qui aurait été reporté le~15 janvier~1931~; selon
 \citeauthor{Scriabine}, la Ballade jouée serait plutôt celle en \kF mineur,
 \Opus{52}.

 \textsc{\Chopin{}}~: Fantaisie en \kF mineur, \Opus{49}~; Ballades en \kG
 mineur, \Opus{23}, en \kF majeur, \Opus{38}, et en \kA \Flat majeur,
 \Opus{47}~; Barcarolle en \kF \Sharp majeur, \Opus{60}~; Deux Mazurkas (en
 \kC majeur et en \kA mineur)~; Deux Valses (en \kA \Flat majeur et en \kD
 \Flat majeur)~; Polonaise en \kF \Sharp mineur, \Opus{44}~; Polonaise en
 \kA \Flat majeur, \Opus{53}~; Scherzo en \kB mineur, \Opus{20}.
 \item[\DateWithWeekDay{1931-01-22}]
 Moskva~: conservatoire.
 Concert incertain \citep[p.~402]{Scriabine}.
 \item[\DateWithWeekDay{1931-02-05}]
 Moskva~: petite salle du conservatoire.
 Apparition lors d'un concert de musique de chambre avec le violoniste
 \DTziganov{} et le violoncelliste \SShirinsky{}, fondateurs et membres du
 quatuor à cordes \Beethoven{}.

 \textsc{\Schubert{}}~: Trio en \kB \Flat majeur, D~898.
 \item[\DateWithWeekDay{1931-03-06}]
 Leningrad~: grande salle de la société philharmonique.

 \textsc{\Mozart{}}~: Sonate en \kD majeur, K~576.
 \textsc{\Schumann{}}~: \emph{Kreisleriana}, \Opus{16}.
 \textsc{\Chopin{}}~: Ballade \Number{3} en \kA \Flat majeur, \Opus{47}.
 \textsc{\Liszt{}}~: Funérailles, S~173 \Number{7}~; Étude en \kB \Flat
 majeur~; Valse oubliée~; \emph{Sposalizio}, S~161 \Number{1}.
 \textsc{\Schubert{}/\Liszt{}}~: \emph{Auf dem Wasser zu singen}, S~558
 \Number{2}~; \emph{Der Doppelgänger}, S~560 \Number{12}~; \emph{Erlkönig},
 S~558 \Number{4}.
 \textsc{\Liszt{}}~: \emph{Sonetto~104 del Petrarca}, S~161 \Number{5}~;
 Étude en \kF mineur.
 \item[\DateWithWeekDay{1931-03-29}]
 Moskva~: grande salle du conservatoire.

 \textsc{\Mozart{}}~: Sonate en \kD majeur, K~576.
 \textsc{\Schumann{}}~: \emph{Kreisleriana}, \Opus{16}.
 \textsc{\Liszt{}}~: Funérailles, S~173 \Number{7}~; Feux follets, S~139
 \Number{5}~; \emph{Sonetto~104 del Petrarca}, S~161 \Number{5}~; Valse
 oubliée.
 \textsc{\Schubert{}/\Liszt{}}~: \emph{Der Doppelgänger}, S~560
 \Number{12}~; \emph{Erlkönig}, S~558 \Number{4}.
 \item[\DateWithWeekDay{1931-04-06}]
 Leningrad~: grande salle de la société philharmonique.

 \textsc{\JBach{}/\Busoni{}}~: Deux Préludes de chorals en \kC majeur et en
 \kG majeur.
 \textsc{\Beethoven{}}~: Sonate en \kC mineur, \Opus{111}.
 \textsc{\Liszt{}}~: \emph{Sonetto~104 del Petrarca}, S~161 \Number{5}~;
 Étude de concert en \kF mineur \Number{2}~; \emph{Gnomenreigen}, S~145
 \Number{2}.
 \textsc{\Wagner{}/\Brassin{}}~: \emph{Feuerzauber} extrait de \emph{Die
 Walküre}.
 \textsc{\Liszt{}}~: Rhapsodie hongroise, S~244 \Number{12}.
 \item[\DateWithWeekDay{1931-04-14}]
 Moskva.
 Concert.
 \item[\DateWithWeekDay{1931-04-18}]
 Moskva.
 Concert.
 \item[\DateWithWeekDay{1931-04-21}]
 Leningrad~: grande salle de la société philharmonique.
 Concert avec des œuvres de \Chopin{}.
 \item[1931-04]
 Concert en commun de \VSofronitsky{} et de \MYudina{} (date précise
 inconnue et lieu non spécifié~; voir \citet[p.~403]{Scriabine}).
 \item[\DateWithWeekDay{1931-05-02}]
 Moskva~: grande salle du conservatoire.
 Concert en commun de \VSofronitsky{} et de \HNeuhaus{}.

 \textsc{\Mozart{}}~: Sonate pour deux pianos en \kD majeur, K~448.
 \textsc{\Liszt{}}~: Concerto pathétique pour deux pianos, S~258.
 \textsc{\Chopin{}}~: Nocturne en \kD \Flat majeur, \Opus{27} \Number{2}~;
 Sonate en \kB \Flat mineur, \Opus{35}.
 \item[\DateWithWeekDay{1931-05-03}]
 Leningrad~: grande salle de la société philharmonique.

 \textsc{\Chopin{}}~: Fantaisie en \kF mineur, \Opus{49}~; Deux Ballades~;
 Sonate~; Nocturne~; Quatre Préludes~; Deux Études~; Mazurka~; Deux Valses~;
 Scherzo.
 \item[B1931-05]
 Série de concerts à Tbilissi, à l'initiative de \MYudina{} qui, à l'époque,
 se produisait souvent en Géorgie.
 Certains concerts en solo, d'autres en commun avec \MYudina{}.
 Voyage avec \EVizel{} et sa fille aînée \AVizel{} (Ada)~; pendant le voyage
 en train et durant les répétitions des concerts à Tbilissi, \EVizel{} a
 réalisé de nombreux croquis \Quote{en direct} de \VSofronitsky{}, dont
 certains ont été préservés jusqu'à nos jours.
 Voir \citet[p.~152 et~153]{Nekrasova08} et \citet[p.~427]{Milshteyn82a}.
 \item[\DateWithWeekDay{1931-05-14}]
 Tbilissi~: salle Rustaveli.

 \textsc{\Chopin{}}~: Fantaisie en \kF mineur, \Opus{49}~; Ballade en \kF
 mineur, \Opus{52}~; Ballade en \kA \Flat majeur, \Opus{47}~; Ballade en \kG
 mineur, \Opus{23}~; Sonate en \kB \Flat mineur, \Opus{35}~; Nocturne en \kF
 mineur, \Opus{55} \Number{1}~; Deux Études~; Deux Valses~; Scherzo en \kB
 mineur, \Opus{20}.
 \item[\DateWithWeekDay{1931-05-16}]
 Tbilissi~: salle Rustaveli.

 \textsc{\JBach{}/\Busoni{}}~: Deux Préludes de chorals.
 \textsc{\Beethoven{}}~: Sonate en \kC mineur, \Opus{111}~; Sonate en \kC
 \Sharp mineur, \Opus{27} \Number{2}.
 \textsc{\Liszt{}}~: \emph{Il penseroso}, S~161 \Number{2}~;
 \emph{Gnomenreigen}, S~145 \Number{2}~; Feux follets, S~139 \Number{5}~;
 Valse oubliée~; Méphisto-valse.
 \item[\DateWithWeekDay{1931-05-21}]
 Tbilissi~: salle Rustaveli.

 \textsc{\Mozart{}}~: Fantaisie.
 \textsc{\Schumann{}}~: Carnaval, \Opus{9}.
 \textsc{\Medtner{}}~: Deux \emph{Skazki}.
 \textsc{\Prokofiev{}}~: Deux Sarcasmes extraits de l'\Opus{17}.
 \textsc{\Scriabine{}}~: Deux Poèmes~; Poème satanique, \Opus{36}.
 \item[\DateWithWeekDay{1931-05-22}]
 Tbilissi~: salle Rustaveli.

 \textsc{\Schumann{}}~: Fantaisie en \kC majeur, \Opus{17}.
 \textsc{\Chopin{}}~: Quatre Mazurkas~; Barcarolle en \kF \Sharp majeur,
 \Opus{60}.
 \textsc{\Liszt{}}~: Sonate en \kB mineur, S~178.
 \textsc{\Scriabine{}}~: Sonate, \Opus{53}.
 \item[\DateWithWeekDay{1931-05-24}]
 Tbilissi.
 Concert pour deux pianos avec \MYudina{}.
 Voir \citet{Yudina02}.

 \textsc{\JBach{}}~: Deux Fugues de l'\hbox{Art} de la fugue.
 \textsc{\Mozart{}}~: Sonate en \kD majeur, K~448.
 \textsc{\Schumann{}}~: Variations en \kB \Flat majeur, \Opus{46}.
 \textsc{\Taneiev{}}~: Prélude et fugue en \kG \Sharp mineur.
 \textsc{\Busoni{}}~: Duo concertant dans le style de \Mozart{}.
 \textsc{\Debussy{}}~: En blanc et noir.
 \item[\DateWithWeekDay{1931-06-20}]
 Leningrad~: salle de la chapelle académique.
 Concert pour deux pianos avec \MYudina{}.
 Même programme que pour le concert à Tbilissi le~24 mai~1931.
 Voir \citet[p.~46 et note~4]{White} et \citet[p.~403]{Scriabine}.
 \item[\DateWithWeekDay{1931-10-08}]
 Leningrad~: grande salle de la société philharmonique.
 Concert.

 \textsc{\Haendel{}}~: Variations en \kE majeur.
 \textsc{\Mozart{}}~: Fantaisie en \kC mineur.
 \textsc{\Chopin{}}~: Vingt-quatre Préludes, \Opus{28}~; Deux Études
 extraites de l'\Opus{10} (\Number{6} en \kE \Flat mineur et \Number{5} en
 \kG \Flat majeur)~; Scherzo \Number{3} en \kC \Sharp mineur, \Opus{39}~;
 Deux Mazurkas extraites de l'\Opus{7}~; Polonaise en \kA \Flat majeur,
 \Opus{53}.
 \item[\DateWithWeekDay{1931-10-11}]
 Moskva~: salle des colonnes de la maison des syndicats.
 Concert avec des œuvres de \Schumann{}, pour le~75\ieme{} anniversaire de
 la mort du compositeur.
 \item[1931-10]
 Leningrad~: petite salle du conservatoire.
 Concert avec des œuvres de \Schumann{}, pour le~75\ieme{} anniversaire de
 la mort du compositeur.
 \item[1931-12]
 Moskva~: grande salle de la société philharmonique.
 Concert avec des œuvres de \Schumann{}, pour le~75\ieme{} anniversaire de
 la mort du compositeur.

 \textsc{\Schumann{}}~: Arabesque en \kC majeur, \Opus{18}~; Fantaisie en
 \kC majeur, \Opus{17}~; Études symphoniques, \Opus{13}~; Papillons,
 \Opus{2}~; Carnaval, \Opus{9}.
 \item[\DateWithWeekDay{1931-12-25}]
 Moskva~: musée polytechnique.
 Concert avec des œuvres de \Schumann{}, pour le~75\ieme{} anniversaire de
 la mort du compositeur.

 \textsc{\Schumann{}}~: \emph{Phantasiestücke}, \Opus{12}~; autres œuvres.
\end{description}

\section{Année~1932}

\begin{description}
 \item[\DateWithWeekDay{1932-01-25}]
 Leningrad~: grande salle de la société philharmonique.
 Concert avec des œuvres de \Chopin{}.

 \textsc{\Chopin{}}~: Fantaisie en \kF mineur, \Opus{49}~; Ballade en \kF
 mineur, \Opus{52}~; Ballade en \kA \Flat majeur, \Opus{47}~; Ballade en \kG
 mineur, \Opus{23}~; Quatre Mazurkas~; Nocturne en \kF \Sharp majeur,
 \Opus{15} \Number{2}~; Barcarolle en \kF \Sharp majeur, \Opus{60}~; Deux
 Valses~; Scherzo en \kB mineur, \Opus{20}.
 \item[\DateWithWeekDay{1932-01-31}]
 Moskva.
 Concert incertain.
 \item[\DateWithWeekDay{1932-02-04}]
 Moskva.
 Concert incertain.
 \item[\DateWithWeekDay{1932-02-18}]
 Moskva~: grande salle du conservatoire.

 \textsc{\Chopin{}}~: Variations, \Opus{12}~; Nocturne~; Ballade en \kG
 mineur, \Opus{23}~; Ballade en \kA \Flat majeur, \Opus{47}~; Ballade en \kF
 mineur, \Opus{52}~; Deux Préludes, \Opus{28} \Number{15} et \Number{16}~;
 Deux Études, \Opus{10} \Number{5} et \Opus{25} \Number{3}~; Deux Valses,
 \Opus{69} \Number{1} et \Opus{70} \Number{1}~; Barcarolle en \kF \Sharp
 majeur, \Opus{60}~; Polonaise en \kA \Flat majeur, \Opus{53}.
 \item[\DateWithWeekDay{1932-02-22}]
 Moskva.
 Concert incertain, mentionné dans la correspondance.
 \item[\DateWithWeekDay{1932-03-06}]
 Leningrad~: grande salle de la société philharmonique.

 \textsc{\Haydn{}}~: Variations en \kF mineur, Hob.XVII:6.
 \textsc{\Chopin{}}~: Sonate en \kB \Flat mineur, \Opus{35}~; Quatre
 Études~; Trois Valses~; Scherzo.
 \textsc{\Liszt{}}~: \emph{Sonetto del Petrarca}~; Valse oubliée \Number{2},
 S~215 \Number{2}~; Deux Études de concert.
 \item[\DateWithWeekDay{1932-03-18}]
 Moskva~: grande salle du conservatoire.

 \textsc{\Beethoven{}}~: Rondo en \kG majeur, \Opus{51} \Number{2}.
 \textsc{\Liszt{}}~: Sonate en \kB mineur, S~178.
 \textsc{\Chopin{}}~: Sonate en \kB \Flat mineur, \Opus{35}~; Deux Mazurkas
 extraites des \Opus{30 et~63}~; Scherzo en \kB \Flat mineur, \Opus{31}.
 \item[B\DateWithWeekDay{1932-04-02}]
 Moskva.
 Participation à un débat sur la musique, mentionnée dans la correspondance.
 \item[\DateWithWeekDay{1932-04-06}]
 Moskva~: maison de la presse.
 Concert mentionné dans la correspondance.
 \item[\DateWithWeekDay{1932-04-12}]
 Moskva~: maison de la presse.
 Concert mentionné dans la correspondance.
 \item[\DateWithWeekDay{1932-04-19}]
 Leningrad~: grande salle de la société philharmonique.

 \textsc{\Liszt{}}~: Funérailles, S~173 \Number{7}~; Feux follets, S~139
 \Number{5}~; Valse oubliée~; \emph{Sposalizio}, S~161 \Number{1}~;
 \emph{Canzonetta del Salvator Rosa}, S~161 \Number{3}~; \emph{Il
 penseroso}, S~161 \Number{2}~; Méphisto-valse~; Après une lecture de Dante,
 S~161 \Number{7}~; \emph{Gondoliera}, \emph{Canzone} et \emph{Tarantella}
 de \emph{Venezia e Napoli}, S~162~; \emph{Sonetto~104 del Petrarca}, S~161
 \Number{5}~; Étude en \kF mineur~; \emph{Gnomenreigen}, S~145 \Number{2}~;
 Deux Études.
 \item[\DateWithWeekDay{1932-04-20}]
 Moskva~: grande salle du conservatoire.

 \textsc{\Liszt{}}~: Funérailles, S~173 \Number{7}~; Feux follets, S~139
 \Number{5}~; Étude d'exécution transcendante \Number{10} en \kF mineur~;
 \emph{Sposalizio}, S~161 \Number{1}~; \emph{Il penseroso}, S~161
 \Number{2}~; \emph{Canzonetta del Salvator Rosa}, S~161 \Number{3}~;
 \emph{Sonetto~104 del Petrarca}, S~161 \Number{5}~; \emph{Sonetto~123 del
 Petrarca}, S~161 \Number{6}~; Méphisto-valse \Number{1}, S~514~; Valse
 oubliée~; \emph{Gnomenreigen}, S~145 \Number{2}~; \emph{Gondoliera},
 \emph{Canzone} et \emph{Tarantella} de \emph{Venezia e Napoli}, S~162~;
 \emph{La Campanella}, S~141 \Number{3}.
 \item[\DateWithWeekDay{1932-04-25}]
 Leningrad~: grande salle de la société philharmonique.

 Concert remis en question par \Vizel{}, avec des œuvres de \Schumann{},
 \Brahms{} et \Chopin{}.
 \item[\DateWithWeekDay{1932-05-12}]
 petite salle de l'\hbox{École} supérieure de musique.
 Concert mentionné dans la correspondance.

 \textsc{\Liszt{}}~: Funérailles, S~173 \Number{7}~; Feux follets, S~139
 \Number{5}~; Valse oubliée~; \emph{Sposalizio}, S~161 \Number{1}~;
 \emph{Canzonetta del Salvator Rosa}, S~161 \Number{3}~; \emph{Il
 penseroso}, S~161 \Number{2}~; \emph{Canzone}, \emph{Gondoliera} et
 \emph{Tarantella} de \emph{Venezia e Napoli}, S~162~; \emph{Sonetto~104
 del Petrarca}, S~161 \Number{5}~; Étude en \kF mineur~;
 \emph{Gnomenreigen}, S~145 \Number{2}~; \emph{La Campanella}, S~141
 \Number{3}~; Variations sur un thème de \Paganini{}.
 \item[\DateWithWeekDay{1932-05-14}]
 petite salle de l'\hbox{École} supérieure de musique.
 Reprise du concert du~12~mai, mentionnée dans la correspondance, peut-être
 le~16~mai.
 \item[\DateWithWeekDay{1932-06-05}]
 Moskva~: maison centrale des artistes.
 Concert mentionné dans la correspondance, peut-être le~6~juin.
 \item[\DateWithWeekDay{1932-06-09}]
 Leningrad~: grande salle de la société philharmonique.
 Concert pour deux pianos avec \LOborine{}.
 Voir en particulier \citet[p.~437]{Milshteyn82a}.

 \textsc{\Mozart{}/\Busoni{}}~: Fantaisie pour orgue en \kF mineur.
 \textsc{\Busoni{}}~: Duo concertant dans le style de \Mozart{}.
 \textsc{\Mozart{}}~: Fugue~; Sonate en \kD majeur, K~448.
 \textsc{\Schumann{}}~: Variations en \kB \Flat majeur, \Opus{46}.
 \textsc{\SaintSaens{}}~: Scherzo, \Opus{87}.
 \textsc{\Debussy{}}~: En blanc et noir.
 \textsc{\Taneiev{}}~: Prélude et fugue en \kG \Sharp mineur.
 \textsc{\Liszt{}}~: Concerto pathétique, S~258.
 \item[\DateWithWeekDay{1932-06-24}]
 Moskva~: petite salle du conservatoire.

 \textsc{\Beethoven{}}~: Sonate en \kD majeur, \Opus{28}.
 \textsc{\Schubert{}}~: Deux Impromptus, D~899 \Number{2} et D~899
 \Number{4}~; Un Impromptu extrait du cahier D~935~; Moment musical en \kA
 \Flat majeur, D~780 \Number{6}.
 \textsc{\Schumann{}}~: Papillons, \Opus{2}.
 \textsc{\Scriabine{}}~: Poème satanique, \Opus{36}~; quelques autres
 œuvres.
 \item[\DateWithWeekDay{1932-07-17}]
 Moskva~: petite salle du conservatoire.
 Concert incertain, mentionné dans la correspondance.
 \item[1932-09 et 1932-10]
 Toškent.
 Plusieurs concerts -- cinq, selon \citet[p.~155]{Nekrasova08}.
 Dans une lettre à son fils \citet[p.~44]{Sofronitsky82b}, \VSofronitsky{}
 indique qu'il a été invité à l'observatoire de Toškent~: il a pu observer
 la Lune (\Quote{grossissement de~800 fois}), Saturne, la [galaxie]
 d'Andromède (Messier~31), Véga (α~Lyrae) et d'autres corps célestes.
 Il en est sorti \Quote{absolument abasourdi}.
 \citet[p.~44]{Sofronitsky82b} donne la date du~6 octobre~1930 pour cette
 lettre de Toškent, mais il s'agit sans doute de~1932.
 \item[\DateWithWeekDay{1932-09-24}]
 Toškent~: Université d'\hbox{État} d'\hbox{Asie} centrale.
 Premier concert.
 \citet[p.~155]{Nekrasova08} confirme ce programme, mais pour le concert
 du~27 septembre~1932.

 \textsc{\JBach{}/\Busoni{}}~: Deux Préludes de chorals en \kG mineur et en
 \kG majeur.
 \textsc{\Mozart{}}~: Fantaisie en \kC mineur.
 \textsc{\Chopin{}}~: Ballade en \kF mineur, \Opus{52}~; Ballade en \kA
 \Flat majeur, \Opus{47}~; Ballade en \kG mineur, \Opus{23}.
 \textsc{\Liszt{}}~: Funérailles, S~173 \Number{7}~; Étude en \kF mineur,
 S~139 \Number{10}~; \emph{Gnomenreigen}, S~145 \Number{2}~;
 Méphisto-valse~; Feux follets, S~139 \Number{5}.
 \item[\DateWithWeekDay{1932-09-27}]
 Toškent~: Université d'\hbox{État} d'\hbox{Asie} centrale.
 Soirée \Chopin{}.
 \citet[p.~155]{Nekrasova08} confirme ce programme comme deuxième concert à
 Toškent, mais indique qu'aucune date n'est précisée.

 \textsc{\Chopin{}}~: Fantaisie en \kF mineur, \Opus{49}~; Nocturne~; Quatre
 Mazurkas~; Sonate en \kB \Flat mineur, \Opus{35}~; Vingt-quatre Préludes,
 \Opus{28}~; Études~; Valses~; Scherzo en \kB mineur, \Opus{20}.
 \emph{Bis} -- \textsc{\Chopin{}}~: Ballade en \kA \Flat majeur, \Opus{47}.
 \item[\DateWithWeekDay{1932-09-30}]
 Toškent~: Université d'\hbox{État} d'\hbox{Asie} centrale.
 Troisième concert.

 \textsc{\Schumann{}}~: Études symphoniques, \Opus{13}~; Carnaval, \Opus{9}.
 \textsc{\Scriabine{}}~: Sonate \Number{4} en \kF \Sharp majeur, \Opus{30}~;
 Poème satanique, \Opus{36}~; Un Poème extrait de l'\Opus{32}~; Préludes~;
 Mazurka.
 \textsc{\Prokofiev{}}~: Contes de la vieille grand-mère, \Opus{31}~; Deux
 Sarcasmes extraits de l'\Opus{17}.
 \item[\DateWithWeekDay{1932-10-06}]
 Toškent~: Université d'\hbox{État} d'\hbox{Asie} centrale.
 Les sources divergent quant au programme de ce quatrième concert à Toškent.

 Programme selon \citet[p.~405]{Scriabine}.--
 \textsc{\Mendelssohn{}}.
 \textsc{\Haydn{}}.
 \textsc{\Schumann{}}.
 \textsc{\Prokofiev{}}.
 \textsc{\Scriabine{}}.

 Programme selon \citet[p.~155]{Nekrasova08}.--
 \textsc{\Beethoven{}}~: Sonate \emph{quasi una fantasia}, \Opus{27}.
 \textsc{\Liszt{}}~: Sonate en \kB mineur, S~178.
 \textsc{\Chopin{}}~: Barcarolle en \kF \Sharp majeur, \Opus{60}~; Scherzo
 en \kB \Flat mineur, \Opus{31}~; Scherzo en \kC \Sharp mineur, \Opus{39}~;
 Mazurka en \kC \Sharp mineur~; Polonaise en \kA \Flat majeur, \Opus{53}.
 \item[\DateWithWeekDay{1932-10-10}]
 Toškent ou Tbilissi.
 Voir en particulier \citet[p.~427]{Milshteyn82a}, qui indique Tbilissi.
 Selon \citet[p.~405]{Scriabine}, il s'agit plutôt d'un concert à Toškent,
 mais les programmes indiqués par ces deux groupes d'auteurs coïncident.
 \citet[p.~155]{Nekrasova08} confirme à la fois la date et le programme de
 ce cinquième et dernier concert à Toškent.

 \textsc{\Haendel{}}~: Variations en \kE majeur.
 \textsc{\Scarlatti{}}~: Deux Sonates, en \kA majeur et en \kD majeur.
 \textsc{\Haydn{}}~: Sonate en \kD majeur, Hob.XVI:37.
 \textsc{\Chopin{}}~: Sonate en \kB mineur, \Opus{58}.
 \textsc{\Liszt{}}~: Après une lecture de Dante, S~161 \Number{7}~;
 \emph{Gondoliera}, \emph{Canzone} et \emph{Tarantella} de \emph{Venezia e
 Napoli}, S~162.
 \item[\DateWithWeekDay{1932-10-24}]
 Leningrad~: société philharmonique.

 \textsc{\Haydn{}}~: Variations en \kF mineur, Hob.XVII:6.
 \textsc{\Schumann{}}~: \emph{Kreisleriana}, \Opus{16}.
 \textsc{\Chopin{}}~: Quatre Mazurkas~; Ballade en \kA \Flat majeur,
 \Opus{47}~; Nocturne en \kF majeur, \Opus{15} \Number{1}~; Scherzo en \kC
 \Sharp mineur, \Opus{39}.
 \textsc{\Schubert{}/\Liszt{}}~: Valse-caprice en \kA majeur~; Quatre
 lieder~: \emph{Gretchen am Spinnrade}, \emph{Die Post}, \emph{Auf dem
 Wasser zu singen} et \emph{Erlkönig}.
 \textsc{\SaintSaens{}/\Liszt{}}~: Danse macabre.
 \item[\DateWithWeekDay{1932-11-24}]
 Leningrad~: société philharmonique.

 \textsc{\Haydn{}}~: Variations en \kF mineur, Hob.XVII:6.
 \textsc{\Schumann{}}~: \emph{Kreisleriana}, \Opus{16}.
 \textsc{\Chopin{}}~: Nocturne en \kF majeur, \Opus{15} \Number{1}~; Quatre
 Mazurkas~; Ballade en \kA \Flat majeur, \Opus{47}~; Scherzo en \kC \Sharp
 mineur, \Opus{39}.
 \textsc{\Schubert{}/\Liszt{}}~: Valse-caprice \Number{7}~; Quatre lieder
 (dont \emph{Gretchen am Spinnrade}, \emph{Auf dem Wasser zu singen} et
 \emph{Erlkönig}).
 \textsc{\SaintSaens{}/\Liszt{}}~: Danse macabre.
 \item[B1932-12]
 Leningrad.
 \VSofronitsky{} gravement malade avec des complications cardiaques.
 Les concerts du~30 novembre et du~6 décembre ont été annulés en raison de
 la maladie.
 Néanmoins, \citet[p.~155]{Nekrasova08} mentionne le concert suivant, le~15
 décembre.
 \item[\DateWithWeekDay{1932-12-15}]
 Moskva~: grande salle du conservatoire.
 Concert mentionné par \citet[p.~155]{Nekrasova08} avec son programme, mais
 absent de la chronologie due à \citet[p.~405]{Scriabine}.

 \textsc{\JBach{}/\Busoni{}}~: Toccata et fugue pour orgue en \kC majeur.
 \textsc{\Chopin{}}~: Douze Études~; Barcarolle en \kF \Sharp majeur,
 \Opus{60}~; Deux Mazurkas en \kD majeur et en \kB majeur.
 \textsc{\Prokofiev{}}~: Trois Pièces.
 \textsc{\Debussy{}}~: Les Collines d'\hbox{Anacapri}, L~117 \Number{V}~;
 \emph{Minstrels}, L~117 \Number{XII}~; \emph{General Lavine -- eccentric},
 L~123 \Number{VI}~; Feux d'artifice, L~123 \Number{XII}.
\end{description}

\section{Année~1933}

\begin{description}
 \item[\DateWithWeekDay{1933-01-03}]
 Leningrad~: société philharmonique.

 \textsc{\Mendelssohn{}}~: Variations sérieuses en \kD mineur, \Opus{54}.
 \textsc{\Chopin{}}~: Sonate en \kB mineur, \Opus{58}.
 \textsc{\Scriabine{}}~: Dix Études extraites de l'\Opus{8}~; Deux Poèmes,
 \Opus{32}~; Mazurka extraite de l'\Opus{40}~; Quatre Préludes~; Morceau
 extrait de l'\Opus{59}~; Sonate en \kF \Sharp majeur, \Opus{30}~; Sonate,
 \Opus{53}.
 \item[\DateWithWeekDay{1933-01-19}]
 Moskva~: grande salle du conservatoire.

 \textsc{\Mendelssohn{}}~: Variations sérieuses en \kD majeur, \Opus{54}.
 \textsc{\Chopin{}}~: Sonate en \kB mineur, \Opus{58}.
 \textsc{\Scriabine{}}~: Études, \Opus{8} (à l'exception du \Number{3} et du
 \Number{9})~; Deux Poèmes, \Opus{32}~; Sonate en \kF \Sharp majeur,
 \Opus{30}.
 \item[\DateWithWeekDay{1933-01-31}]
 Moskva~: grande salle du conservatoire.
 Concert évoqué par \citet[p.~247]{Lobanov08a}.

 \textsc{\Haydn{}}~: Variations \kF mineur, Hob.XVII:6.
 \textsc{\Schumann{}}~: \emph{Kreisleriana}, \Opus{16}.
 \textsc{\Chopin{}}~: Ballade en \kA \Flat majeur, \Opus{47}~; Scherzo en
 \kC \Sharp mineur, \Opus{39}~; Nocturne~; Mazurkas.
 \textsc{\Schubert{}/\Liszt{}}~: Deux lieder.
 \textsc{\SaintSaens{}/\Liszt{}}~: Danse macabre.
 \item[\DateWithWeekDay{1933-03-04}]
 Leningrad~: grande salle de la société philharmonique.

 \textsc{\JBach{}}~: Deuxième Suite française en \kD mineur.
 \textsc{\Mozart{}}~: Fantaisie.
 \textsc{\Schubert{}}~: Fantaisie.
 \textsc{\Chopin{}}~: Fantaisie en \kF mineur, \Opus{49}~; Ballade en \kG
 mineur, \Opus{23}~; Deux Valses en \kA \Flat majeur et en \kE \Flat majeur.
 \textsc{\Liszt{}}~: \emph{Sonetto~104 del Petrarca}, S~161 \Number{5}~;
 Rhapsodie hongroise \Number{12}.
 \item[\DateWithWeekDay{1933-04-06}]
 Moskva~: maison de la presse.
 Concert incertain.
 \item[\DateWithWeekDay{1933-04-11}]
 Leningrad~: grande salle de la société philharmonique.

 \textsc{\Schumann{}}~: Sonate en \kG mineur, \Opus{22}.
 \textsc{\Brahms{}}~: Ballade en \kG mineur, \Opus{118} \Number{3}~;
 Rhapsodie extraite de l'\Opus{79}~; Rhapsodie en \kE \Flat majeur,
 \Opus{119} \Number{4}.
 \textsc{\Medtner{}}~: Deux \emph{Skazki} extraits de l'\Opus{26}.
 \textsc{\Rachmaninov{}}~: Deux Préludes extraits de l'\Opus{32}~; Deux
 Études-tableaux extraites de l'\Opus{33}.
 \textsc{\Prokofiev{}}~: Trois Visions fugitives extraites de l'\Opus{22}~;
 Marche~; Rigaudon~; Prélude~; Suggestion diabolique, \Opus{4} \Number{4}.
 \textsc{\Scriabine{}}~: Un Poème extrait de l'\Opus{32}~; Sonate,
 \Opus{70}~; Sonate, \Opus{53}.
 \item[\DateWithWeekDay{1933-04-19}]
 Leningrad.
 Voir en particulier \citet[p.~427]{Milshteyn82a}.

 \textsc{\Liszt{}}~: Funérailles, S~173 \Number{7}~; Feux follets, S~139
 \Number{5}~; Valse oubliée \Number{1}, S~215 \Number{1}~;
 \emph{Sposalizio}, S~161 \Number{1}~; \emph{Canzonetta del Salvator Rosa},
 S~161 \Number{3}~; \emph{Il penseroso}, S~161 \Number{2}~; Méphisto-valse~;
 Après une lecture de Dante, S~161 \Number{7}~; \emph{Gondoliera},
 \emph{Canzone} et \emph{Tarantella} de \emph{Venezia e Napoli}, S~162~;
 \emph{Sonetto~104 del Petrarca}, S~161 \Number{5}~; Étude en \kF mineur~;
 \emph{Gnomenreigen}, S~145 \Number{2}~; Deux Études.
 \item[\DateWithWeekDay{1933-04-20}]
 Moskva.
 Mention de l'interprétation de \Quote{certaines œuvres de \Brahms{}} par
 \VSofronitsky{} en~1933 à Moskva.
 \item[\DateWithWeekDay{1933-04-25}]
 Leningrad~: grande salle de la société philharmonique.

 \textsc{\Schumann{}}~: Arabesque en \kC majeur, \Opus{18}~; Sonate en \kG
 mineur, \Opus{22}.
 \textsc{\Brahms{}}~: Ballade en \kG mineur, \Opus{118} \Number{3}~;
 Rhapsodie extraite de l'\Opus{79}~; Rhapsodie en \kE \Flat majeur,
 \Opus{119} \Number{4}.
 \textsc{\Chopin{}}~: Ballade~; Sonate en \kB \Flat mineur, \Opus{35}.
 \textsc{\Rachmaninov{}}~: Deux Préludes extraits de l'\Opus{32}~; Deux
 Études-tableaux.
 \textsc{\Scriabine{}}~: Trois Pièces~; Sonate, \Opus{53}.
 \item[\DateWithWeekDay{1933-05-27}]
 Leningrad~: grande salle de la société philharmonique.

 \textsc{\Schumann{}}~: Fantaisie en \kC majeur, \Opus{17}~; Études
 symphoniques, \Opus{13}.
 \textsc{\Chopin{}}~: Vingt-quatre Préludes, \Opus{28}~; Six Études~;
 Mazurka~; Polonaise en \kA \Flat majeur, \Opus{53}.
 \item[\DateWithWeekDay{1933-06-04}]
 Moskva~: petite salle du conservatoire.

 \textsc{\Schumann{}}~: Arabesque en \kC majeur, \Opus{18}~; Études
 symphoniques, \Opus{13}.
 \textsc{\Brahms{}}~: Ballade en \kG mineur, \Opus{118} \Number{3}~; Deux
 Rhapsodies.
 \textsc{\Chopin{}}~: Sonate.
 \textsc{\Rachmaninov{}}~: Préludes~; Études-tableaux.
 \item[\DateWithWeekDay{1933-10-23}]
 Leningrad~: grande salle de la société philharmonique.

 \textsc{\Beethoven{}}~: Rondo en \kG majeur, \Opus{51} \Number{2}~; Sonate
 en \kC mineur, \Opus{13}.
 \textsc{\Chopin{}}~: Douze Études~; Scherzo en \kB \Flat mineur,
 \Opus{31}~; Deux Mazurkas (en \kC majeur et en \kB majeur).
 \textsc{\Prokofiev{}}~: Contes de la vieille grand-mère, \Opus{31}.
 \textsc{\Debussy{}}~: \emph{Serenade of the Doll}, L~113 \Number{III}~;
 \emph{Minstrels}, L~117 \Number{XII}~; \emph{General Lavine -- eccentric},
 L~123 \Number{VI}~; Feux d'artifice, L~123 \Number{XII}.
 \item[\DateWithWeekDay{1933-11-09}]
 Leningrad~: maison de l'\hbox{Armée} rouge.
 Concert mixte avec des œuvres de \Chopin{}.
 \item[\DateWithWeekDay{1933-12-11}]
 Leningrad~: grande salle de la société philharmonique.

 \textsc{\JBach{}/\Busoni{}}~: Trois Préludes de chorals~; Toccata et fugue
 pour orgue.
 \textsc{\Schumann{}}~: Études symphoniques, \Opus{13}.
 \textsc{\Beethoven{}}~: Sonate en \kC \Sharp mineur, \Opus{27} \Number{2}.
 \textsc{\Liszt{}}~: Sonate en \kB mineur, S~178.
 \item[B1933]
 Décès de la sœur aînée de \VSofronitsky{}, Ol'ga Vladimirovna, à Leningrad
 à l'âge de~40~ans.
 \item[\DateWithWeekDay{1933-12-23}]
 Leningrad~: petite salle du conservatoire.

 \textsc{\Haendel{}}~: Variations en \kE majeur.
 \textsc{\Mozart{}}~: Fantaisie en \kC mineur.
 \textsc{\Chopin{}}~: Vingt-quatre Préludes, \Opus{28}~; Deux Études
 extraites de l'\Opus{10}~; Scherzo en \kC \Sharp mineur, \Opus{39}~; Deux
 Mazurkas~; Polonaise en \kA \Flat majeur, \Opus{53}.
 \item[\DateWithWeekDay{1933-12-25}]
 Moskva~: grande salle du conservatoire.
 Concert philharmonique.
 Voir en particulier \citet[p.~427]{Milshteyn82a}.

 \textsc{\JBach{}/\Busoni{}}~: Toccata et fugue pour orgue en \kC majeur.
 \textsc{\Beethoven{}}~: Sonate en \kC \Sharp mineur, \Opus{27} \Number{2}.
 \textsc{\Chopin{}}~: Douze Études (\Opus{10}~: en \kC majeur, \kA mineur,
 \kE majeur, \kC \Sharp mineur, \kG \Flat majeur, \kE \Flat mineur et \kF
 majeur~; \Opus{25}~: en \kF mineur, \kF majeur et \kC mineur)~; Barcarolle
 en \kF \Sharp majeur, \Opus{60}~; Mazurkas en \kD majeur, \Opus{33}
 \Number{2}, et en \kB majeur, \Opus{63} \Number{1}~; Scherzo en \kB \Flat
 mineur, \Opus{31}.
 \textsc{\Prokofiev{}}~: Trois pièces extraites des Contes de la vieille
 grand-mère, \Opus{31}.
 \textsc{\Debussy{}}~: Les collines d'\hbox{Anacapri}, L~117 \Number{V}~;
 \emph{Minstrels}, L~117 \Number{XII}~; \emph{General Lavine -- eccentric},
 L~123 \Number{VI}~; Feux d'artifice, L~123 \Number{XII}.
 \textsc{\Rachmaninov{}}~: Prélude en \kG majeur.
 \textsc{\Liszt{}}~: \emph{Gnomenreigen}, S~145 \Number{2}.
 \textsc{\JBach{}/\Busoni{}}~: Prélude de choral.
 \textsc{\Chopin{}}~: Étude~; Prélude en \kD mineur, \Opus{28} \Number{24}.
\end{description}

\section{Année~1934}

\begin{description}
 \item[\DateWithWeekDay{1934-02-18}]
 Leningrad~: grande salle de la société philharmonique.

 \textsc{\JBach{}}~: Deux Préludes et fugues.
 \textsc{\Mendelssohn{}}~: Variations sérieuses en \kD mineur, \Opus{54}.
 \textsc{\Schumann{}}~: Carnaval, \Opus{9}.
 \textsc{\Scriabine{}}~: Deux Études.
 \textsc{\Medtner{}}~: Marche funèbre en \kB mineur, \Opus{31} \Number{2}~;
 Quatre \emph{Skazki}.
 \textsc{\Prokofiev{}}~: Cinq Visions fugitives extraites de l'\Opus{22}.
 \textsc{\Balakirev{}}~: \emph{Islamey}, \Opus{18}.
 \item[\DateWithWeekDay{1934-03-11}]
 Leningrad~: société philharmonique ou conservatoire.
 Concert.
 \item[\DateWithWeekDay{1934-03-19}]
 Leningrad~: grande salle de la société philharmonique.
 Concert mentionné par \citet[p.~155]{Nekrasova08}, mais sans programme.
 \item[\DateWithWeekDay{1934-03-29}]
 Leningrad~: grande salle de la société philharmonique.
 Voir en particulier \citet[p.~427]{Milshteyn82a}.

 \textsc{\Scarlatti{}}~: Sonates en \kG majeur, en \kB mineur, en \kE
 mineur, en \kG majeur et en \kD majeur.
 \textsc{\Beethoven{}}~: Sonate en \kC mineur, \Opus{111}.
 \textsc{\Schumann{}}~: Carnaval, \Opus{9}~; Papillons, \Opus{2}.
 \textsc{\Liszt{}}~: Funérailles, S~173 \Number{7}~; \emph{La leggierezza},
 S~144 \Number{2}~; Feux follets, S~139 \Number{5}~; \emph{Liebesträume}
 \Number{3}.
 \textsc{\SaintSaens{}/\Liszt{}}~: Danse macabre.
 \item[\DateWithWeekDay{1934-04-09}]
 Leningrad ou Moskva.
 Concert.
 \item[\DateWithWeekDay{1934-04-19}]
 Leningrad~: grande salle de la société philharmonique.
 Concert mentionné par \citet[p.~155]{Nekrasova08}, mais sans programme.
 \item[\DateWithWeekDay{1934-05-25}]
 Leningrad~: société philharmonique ou conservatoire.
 Concert.
 \citet[p.~155]{Nekrasova08} indique la grande salle de la société
 philharmonique, mais ne précise pas le programme.
 \item[\DateWithWeekDay{1934-05-31}]
 Leningrad ou Moskva.
 Concert.
 \item[\DateWithWeekDay{1934-06-05}]
 Leningrad ou Moskva.
 Concert.
 \citet[p.~155]{Nekrasova08} indique la grande salle de la société
 philharmonique de Leningrad, mais ne précise pas le programme.
 \item[B1934-08]
 \VSofronitsky{} vit avec son fils, Aleksandr, dans les environs de
 Leningrad.
 Dans des lettres à \ESofronitskaya{} en France, il lui demande de revenir
 bientôt.
 \item[B1934-09]\phantomsection\label{bio:LDDP}
 \VMeyerhold{} met en scène \emph{La Dame de pique} et dédie cette
 production à \VSofronitsky{}.
 \item[\DateWithWeekDay{1934-09-18}]
 Leningrad ou Moskva.
 Concert.
 \citet[p.~155]{Nekrasova08} indique la grande salle de la société
 philharmonique de Leningrad, mais ne précise pas le programme.
 \item[\DateWithWeekDay{1934-10-21}]
 Leningrad~: société philharmonique ou conservatoire.
 Concert.
 \citet[p.~155]{Nekrasova08} indique la grande salle de la société
 philharmonique, mais ne précise pas le programme.
 \item[\DateWithWeekDay{1934-11-02}]
 Leningrad~: grande salle de la société philharmonique.

 \textsc{\JBach{}/\Busoni{}}~: Toccata et fugue pour orgue en \kE mineur.
 \textsc{\Schumann{}}~: Fantaisie en \kC majeur, \Opus{17}.
 \textsc{\Liszt{}}~: Ballade \Number{2} en \kB mineur, S~171~;
 \emph{Sonetto~104 del Petrarca}, S~161 \Number{5}~; \emph{Venezia e
 Napoli}, S~162~; Cinq Études de \Paganini{}.
 \item[\DateWithWeekDay{1934-11-15}]
 Leningrad~: société philharmonique ou conservatoire.
 Concert.
 \citet[p.~156]{Nekrasova08} indique la grande salle de la société
 philharmonique, mais ne précise pas le programme.
 \item[\DateWithWeekDay{1934-11-20}]
 Leningrad~: grande salle de la société philharmonique.
 Concert mentionné par \citet[p.~156]{Nekrasova08}, mais sans programme.
 \item[\DateWithWeekDay{1934-11-26}]
 Leningrad~: société philharmonique ou conservatoire.
 Concert.
 \item[\DateWithWeekDay{1934-12-08}]
 Leningrad~: grande salle de la société philharmonique.
 Selon une lettre de \VSofronitsky{} à son épouse en décembre~1934, ce
 concert a été un échec.

 \textsc{\Schumann{}}~: \emph{Kreisleriana}, \Opus{16}.
 \textsc{\Chopin{}}~: Ballade~; Nocturne~; Scherzo.
 \textsc{\Scriabine{}}~: Sonate en \kF \Sharp mineur, \Opus{23}~; Deux
 Poèmes~; Sonate, \Opus{53}~; Poème satanique, \Opus{36}.
 \item[B\DateWithWeekDay{1934-12-15} ou~16]
 \ESofronitskaya{} rentre à Moskva après son départ de Paris.
 \item[\DateWithWeekDay{1934-12-24}]
 Leningrad~: grande salle de la société philharmonique.
 Concert annoncé peu de jours à l'avance.
 Selon \VSofronitsky{}, un concert infructueux et malheureux.
 Concert radiodiffusé.
 \item[\DateWithWeekDay{1934-12-30}]
 Leningrad.
 Concert mentionné dans la correspondance.
\end{description}

\section{Année~1935}

\begin{description}
 \item[B1935-01 (fin) -- 1935-02 (début)]
 Arrivée d'\ESofronitskaya{} à Leningrad.
 \item[B1935-02 -- 1935-03]
 Maladie de la mère de \VSofronitsky{}.
 \item[\DateWithWeekDay{1935-02-02}]
 Leningrad~: société philharmonique académique d'\hbox{État}.

 \foreignlanguage{german}{\emph{Klavierabend}} avec des œuvres de
 \GHaendel{}, \JBach{}, \RSchumann{} et \FChopin{} (les œuvres précises ne
 sont pas spécifiées).
 \item[\DateWithWeekDay{1935-02-07}]
 Leningrad.

 \foreignlanguage{german}{\emph{Klavierabend}} avec des œuvres de
 \GHaendel{}, \JBach{}, \RSchumann{} et \FChopin{} (les œuvres précises ne
 sont pas spécifiées).
 \item[\DateWithWeekDay{1935-03-31}]
 Leningrad.
 Concert privé à la maison de la presse.
 \item[\DateWithWeekDay{1935-04-09}]
 Leningrad~: grande salle de la société philharmonique.
 Concert qui a peut-être été reporté de quelques jours, au plus tard le~17
 avril.

 \textsc{\CBach{}}~: \emph{Rondo espressivo}.
 \textsc{\JBach{}/\Busoni{}}~: Deux Préludes pour orgue.
 \textsc{\Beethoven{}}~: Sonate en \kD majeur, \Opus{10} \Number{3}~; Sonate
 en \kC mineur, \Opus{13}.
 \textsc{\Liszt{}}~: \emph{Sposalizio}, S~161 \Number{1}~; \emph{Canzonetta
 del Salvator Rosa}, S~161 \Number{3}~; \emph{Il penseroso}, S~161
 \Number{2}~; Méphisto-valse.
 \item[\DateWithWeekDay{1935-05-31}]
 Leningrad~: société philharmonique.
 Voir en particulier \citet[p.~427]{Milshteyn82a}.

 \textsc{\Beethoven{}}~: Sonate en \kD majeur, \Opus{28}~; Sonate en \kF
 \Sharp majeur, \Opus{78}~; Sonate en \kC \Sharp mineur, \Opus{27}
 \Number{2}.
 \textsc{\Liszt{}}~: Après une lecture de Dante, S~161 \Number{7}.
 \textsc{\Schubert{}/\Liszt{}}~: \emph{Der Doppelgänger}~;
 \emph{Aufenthalt}~; \emph{Der Müller und der Bach}~; \emph{Am Meer}~;
 \emph{Erstarrung}~; \emph{Die Post}~; \emph{Gretchen am Spinnrade}~;
 \emph{Erlkönig}.
 \item[\DateWithWeekDay{1935-06-05}]
 Moskva~: grande salle du conservatoire.

 \textsc{\Beethoven{}}~: Sonate en \kD majeur, \Opus{28}~; Sonate en \kF
 \Sharp majeur, \Opus{78}~; Sonate en \kC \Sharp mineur, \Opus{27}
 \Number{2}.
 \textsc{\Liszt{}}~: Après une lecture de Dante, S~161 \Number{7}.
 \textsc{\Schubert{}/\Liszt{}}~: Quatre lieder.
 \textsc{\Liszt{}}~: Deux Études~; Tarentelle de \emph{Venezia e Napoli},
 S~162 \Number{3}.
 \item[B1935-06]
 La mère de \VSofronitsky{} subit une opération.
 \item[B\DateWithWeekDay{1935-07-07}]
 Lettre de \VSofronitsky{} à \AVizel{} où le musicien évoque l'extension de
 son répertoire de concert et les modifications de son style de jeu, dans la
 première moitié des années~1930 et à la suite de son séjour à Paris~; il y
 indique aussi les réactions pas toujours positives du public, des musiciens
 et de son propre entourage face à cette nouvelle étape de son développement
 artistique.
 \citet[p.~156]{Nekrasova08} publie un extrait de cette lettre.
 \item[\DateWithWeekDay{1935-09-17}]
 Moskva.

 \textsc{\Mozart{}}~: Fantaisie en \kC mineur, K~475.
 \textsc{\Schubert{}}~: Fantaisie \Quote{Wanderer} en \kC majeur, D~760.
 \textsc{\Schumann{}}~: Fantaisie en \kC majeur, \Opus{17}.
 \textsc{\Chopin{}}~: Fantaisie en \kF mineur, \Opus{49}.
 \item[\DateWithWeekDay{1935-09-18}]
 Moskva~: grande salle du conservatoire.
 Récital évoqué et critiqué par \citet{Drozdov35}.

 \textsc{\Mozart{}}~: Fantaisie.
 \textsc{\Schumann{}}~: Fantaisie en \kC majeur, \Opus{17}.
 \textsc{\Chopin{}}~: Fantaisie en \kF mineur, \Opus{49}~; Nocturne~; Deux
 Mazurkas~; Scherzo.
 \textsc{\Debussy{}}~: \emph{Serenade of the Doll}, L~113 \Number{III}~;
 Deux Préludes~; Feux d'artifice, L~123 \Number{XII}.
 \textsc{\Scriabine{}}~: Vers la flamme, \Opus{72}.
 \item[B1935-10]
 Tbilissi (plusieurs concerts).
 Il s'agit de la deuxième tournée de concerts de \VSofronitsky{} à Tbilissi,
 après celle de mai~1931~; il s'y rend avec son épouse, \ESofronitskaya{}.
 Voir \citet[p.~157]{Nekrasova08}.
 \item[\DateWithWeekDay{1935-10-19}]
 Concert en un lieu inconnu.

 \textsc{\JBach{}/\Ziloti{}}~: Deux Préludes pour orgue.
 \textsc{\JBach{}/\Busoni{}}~: Toccata et fugue pour orgue.
 \textsc{\Mozart{}}~: Fantaisie.
 \textsc{\Schumann{}}~: Toccata, \Opus{7}.
 \textsc{\Chopin{}}~: Fantaisie en \kF mineur, \Opus{49}~; Deux Mazurkas~;
 Scherzo.
 \textsc{\Debussy{}}~: \emph{Serenade of the Doll}, L~113 \Number{III}~;
 \emph{General Lavine -- eccentric}, L~123 \Number{VI}~; Canope, L~123
 \Number{X}~; Feux d'artifice, L~123 \Number{XII}.
 \textsc{\Scriabine{}}~: Vers la flamme, \Opus{72}.
 \item[\DateWithWeekDay{1935-10-27}]
 Tbilissi.

 \textsc{\Mozart{}}~: Fantaisie.
 \textsc{\Schumann{}}~: Fantaisie en \kC majeur, \Opus{17}.
 \textsc{\Chopin{}}~: Fantaisie en \kF mineur, \Opus{49}~; Nocturne~; Deux
 Mazurkas~; Scherzo en \kB mineur, \Opus{20}.
 \textsc{\Debussy{}}~: \emph{Serenade of the Doll}, L~113 \Number{III}~;
 \emph{General Lavine -- eccentric}, L~123 \Number{VI}~; Canope, L~123
 \Number{X}~; Feux d'artifice, L~123 \Number{XII}.
 \textsc{\Scriabine{}}~: Vers la flamme, \Opus{72}.
 \item[\DateWithWeekDay{1935-10-30}]
 Tbilissi.

 \textsc{\JBach{}/\Busoni{}}~: Toccata et fugue pour orgue.
 \textsc{\Schumann{}}~: Études symphoniques, \Opus{13}.
 \textsc{\Chopin{}}~: Ballade.
 \textsc{\Liszt{}}~: Deux Études de \Paganini{}~; \emph{Canzone},
 \emph{Gondoliera} et \emph{Tarantella} de \emph{Venezia e Napoli}, S~162~;
 \emph{Sonetto del Petrarca}~; Méphisto-valse.
 \item[B\DateWithWeekDay{1935-10-31}]
 Décès de la mère de \VSofronitsky{} à Leningrad à l'âge de~60~ans.
 \item[B1935-11]
 \VSofronitsky{} malade.
 \item[\DateWithWeekDay{1935-12-22}]
 Leningrad~: grande salle de la société philharmonique.
 Concert symphonique sous la direction du chef d'orchestre anglais
 \ACoates{}.
 Soliste~: \VSofronitsky{}.
 Concert dédié à \ESofronitskaya{}.

 \textsc{\Scriabine{}}~: Concerto en \kF \Sharp mineur, \Opus{20}.
 \item[\DateWithWeekDay{1935-12-24}]
 Leningrad~: grande salle de la société philharmonique.
 Concert symphonique.
 Reprise du programme du~22 décembre.
\end{description}

\section{Année~1936}

\begin{description}
 \item[\DateWithWeekDay{1936-01-07}]
 Moskva~: grande salle du conservatoire.
 Concert avec des œuvres de \JBach{}, \Schumann{}, \Chopin{} et \Liszt{}.
 \item[\DateWithWeekDay{1936-01-11}]
 Bakou~: salle de concerts~DKAF.
 Concert avec des œuvres de \JBach{}, \Haydn{}, \Schumann{}, \Chopin{},
 \Liszt{}, \Debussy{}, \Prokofiev{} et \Scriabine{}.
 \item[\DateWithWeekDay{1936-01-13}]
 Bakou~: salle de concerts~DKAF.
 Concert avec des œuvres de \JBach{}, \Haydn{}, \Schumann{}, \Chopin{},
 \Liszt{}, \Debussy{}, \Prokofiev{} et \Scriabine{}.
 \item[\DateWithWeekDay{1936-01-17}]
 Rostov-na-Donu~: maison de la santé.
 Concert.

 \textsc{\JBach{}}~: Rondo.
 \textsc{\JBach{}/\Ziloti{}}~: Deux Préludes en \kE mineur et en \kB mineur.
 \textsc{\Haydn{}}~: Sonate.
 \textsc{\Schumann{}}~: Carnaval, \Opus{9}.
 \textsc{\Chopin{}}~: Fantaisie en \kF mineur, \Opus{49}~; Ballade.
 \textsc{\Liszt{}}~: Deux Études~; Feux follets, S~139 \Number{5}~;
 \emph{Venezia e Napoli}, S~162.
 \item[\DateWithWeekDay{1936-01-18}]
 Rostov-na-Donu~: maison de la santé.
 Concert.
 Ce même jour, lettre de \VSofronitsky{} à l'épouse d'\EVizel{}, confirmant
 qu'il s'agit en effet du deuxième concert à Rostov-na-Donu, et évoquant un
 départ, le lendemain matin, pour Odésa \citep[voir][p.~157]{Nekrasova08}.

 \textsc{\JBach{}/\Busoni{}}~: Trois Préludes de chorals.
 \textsc{\Schumann{}}~: Fantaisie en \kC majeur, \Opus{17}.
 \textsc{\Prokofiev{}}~: Contes de la vieille grand-mère, \Opus{31}.
 \textsc{\Debussy{}}~: \emph{Serenade of the Doll}, L~113 \Number{III}~;
 \emph{General Lavine -- eccentric}, L~123 \Number{VI}~; Canope, L~123
 \Number{X}~; Feux d'artifice, L~123 \Number{XII}.
 \textsc{\Scriabine{}}~: Vers la flamme, \Opus{72}.
 \textsc{\Liszt{}}~: Méphisto-valse.
 \item[\DateWithWeekDay{1936-01-20}]
 Kiïv.
 Concert.
 \item[\DateWithWeekDay{1936-01-22}]
 Kiïv.
 Concert.
 \item[1936-02 (début)]
 Moskva~: grande salle du conservatoire.
 Concert.
 Selon une lettre à \ESofronitskaya{}, un mauvais concert.
 \item[\DateWithWeekDay{1936-03-03}]
 Leningrad~: société philharmonique.

 \textsc{\Schumann{}}~: Sonate en \kF \Sharp mineur, \Opus{11}.
 \textsc{\Chopin{}}~: Scherzo en \kB \Flat mineur, \Opus{31}~; Ballade en
 \kA \Flat majeur, \Opus{47}~; Impromptu en \kG \Flat majeur, \Opus{51}~;
 Barcarolle en \kF \Sharp majeur, \Opus{60}.
 \textsc{\Scriabine{}}~: Sonate en \kF \Sharp mineur, \Opus{23}~; Deux
 Préludes, \Opus{27}~; Deux Études extraites de l'\Opus{42}~; Vers la
 flamme, \Opus{72}.
 \item[\DateWithWeekDay{1936-03-11}]
 Leningrad~: grande salle de la société philharmonique.
 Concert de \IBraudo{} et \VSofronitsky{}.

 \textsc{\Liszt{}}~: Années de pèlerinage~; Sonate en \kB mineur, S~178.
 \item[B1936 (printemps)]
 Rencontre avec \ACortot{}, à l'occasion de la tournée de celui-ci à Moskva
 \citep[p.~101]{Nikonovich08a}.
 \item[\DateWithWeekDay{1936-03-27}]
 Leningrad~: grande salle de la société philharmonique.
 Concert%
 \footnote{Pour le programme de ce récital, voir
 \href{https://100philharmonia.spb.ru/historical-poster/11113/}%
 {https://100philharmonia.spb.ru/historical-poster/11113/}.}.

 \textsc{\JBach{}}~: Prélude et fugue.
 \textsc{\Schumann{}}~: \emph{Kreisleriana}, \Opus{16}.
 \textsc{\Chopin{}}~: Douze Préludes~; Quatre Mazurkas (en \kA mineur, \kA
 mineur, \kF mineur et \kC majeur).
 \textsc{\Schumann{}}~: Carnaval, \Opus{9}.
 \textsc{\Liszt{}}~: \emph{Tarantella} de \emph{Venezia e Napoli}, S~162
 \Number{3}.
 \item[1936-04]
 Moskva~: Club~MGU.
 Concert (date incertaine).

 \textsc{\Beethoven{}}~: Sonate en \kF mineur, \Opus{57}.
 \textsc{\Schubert{}}~: Deux Impromptus, D~899 \Number{2} et D~899
 \Number{4}~; Un Impromptu extrait du cahier D~935.
 \textsc{\Chostakovitch{}}~: Préludes (dont celui en \kC \Sharp mineur).
 \textsc{\Prokofiev{}}~: Contes de la vieille grand-mère, \Opus{31}~; Cinq
 Sarcasmes, \Opus{17}.
 \item[\DateWithWeekDay{1936-04-09}]
 Leningrad~: grande salle de la société philharmonique.

 \textsc{\CBach{}}~: Rondo (œuvre jouée incertaine).
 \textsc{\JBach{}/\Ziloti{}}~: Deux Préludes pour orgue.
 \textsc{\Beethoven{}}~: Sonate en \kD majeur, \Opus{10} \Number{3}~; Sonate
 en \kC mineur, \Opus{13}.
 \textsc{\Chopin{}}~: Fantaisie en \kF mineur, \Opus{49}.
 \item[B1936-05]
 \VSofronitsky{} travaille sur les Sonates de \Beethoven{} \Opus{28 et~57}.
 \item[\DateWithWeekDay{1936-05-25}]
 Leningrad~: grande salle de la société philharmonique.

 \textsc{\Beethoven{}}~: Sonate en \kD majeur, \Opus{28}~; Sonate en \kF
 mineur, \Opus{57}.
 \textsc{\Liszt{}}~: Sonate en \kB mineur, S~178.
 \textsc{\Debussy{}}~: Trois œuvres.
 \item[B1936-06 (début)]
 \VSofronitsky{} malade.
 \item[\DateWithWeekDay{1936-06-15}]
 Leningrad~: petite salle du conservatoire.
 Participation à une soirée en faveur de l'\hbox{Espagne}.
 \item[\DateWithWeekDay{1936-06-26}]
 Leningrad~: salle Kapelly.

 \textsc{\Beethoven{}}~: \emph{Andante favori} en \kF majeur, WoO~57.
 \textsc{\Schumann{}}~: Phantasiestücke, \Opus{12}.
 \textsc{\Chopin{}}~: Ballade en \kF mineur, \Opus{52}~; Études~; Mazurkas~;
 Polonaise en \kA \Flat majeur, \Opus{53}.
 \item[B1936-06]
 \VSofronitsky{} présent à une réunion des pianistes de Leningrad.
 \item[\DateWithWeekDay{1936-10-15}]
 Leningrad~: conservatoire.
 Concert.
 \item[B\DateWithWeekDay{1936-10-15} ou~20]
 Début du travail pédagogique de \VSofronitsky{} au conservatoire de
 Leningrad, en charge d'un nombre réduit d'étudiants, compatible avec le
 projet pour la saison~1937-1938~: un cycle de douze concerts anthologiques
 avec des programmes différents.
 Voir \citet[p.~90-91]{Savshinsky61}%
 \footnote{\foreignlanguage{russian}{\emph{Советская музыка}}, vol.~277,
 \Number{12} (1961), p.~88-91.}
 et \citet[p.~157]{Nekrasova08}.
 \item[\DateWithWeekDay{1936-10-21}]
 Leningrad~: salle Kapelly.

 \textsc{\Mozart{}}~: Sonate en \kE \Flat majeur, K~282.
 \textsc{\Schumann{}}~: \emph{Humoreske} en \kB \Flat majeur, \Opus{20}~;
 \textsc{\Chopin{}}~: Quatre Mazurkas~; Scherzo.
 \textsc{\Liszt{}} (date d'anniversaire)~: Funérailles, S~173 \Number{7}~;
 Au Lac de Wallenstadt, S~160 \Number{2}~; Étude de \Paganini{}~;
 \emph{Sonetto~104 del Petrarca}, S~161 \Number{5}~; Méphisto-valse.
 \item[B1936-10 (fin)]
 Leningrad.
 Le dernier récital de \VSofronitsky{} a été discuté lors d'une réunion de
 l'\hbox{Union} des compositeurs soviétiques.
 Présents lors de la réunion~: \LNikolaiev{}, \NGolubovskaya{}, \IBraudo{}
 et d'autres, appréciant beaucoup le travail de \VSofronitsky{}.
 Ce dernier évoque son \foreignlanguage{german}{\emph{Klavierabend}} et la
 réunion dans une lettre du~1\ier{} novembre~1936 à sa famille
 \citep[p.~16]{Kogan08}.
 \item[\DateWithWeekDay{1936-11-03}]
 Leningrad.
 Concert pour la radio (diffusion en direct).
 \item[\DateWithWeekDay{1936-11-15}]
 Leningrad~: petite salle du conservatoire.

 \textsc{\Schumann{}}~: Études symphoniques, \Opus{13}.
 \textsc{\Chopin{}}~: Six Études~; Fantaisie en \kF mineur, \Opus{49}~;
 Vingt-quatre Préludes, \Opus{28}.
 \item[\DateWithWeekDay{1936-11-20}]
 Leningrad.
 Concert de \VSofronitsky{} à la radio (à~22\up{h}~45\up{m}).
 \item[\DateWithWeekDay{1936-11-26}]
 Leningrad~: grande salle de la société philharmonique.

 \textsc{\Schumann{}}~: Fantaisie en \kC majeur, \Opus{17}.
 \textsc{\Chopin{}}~: Dix Études.
 \textsc{\Scriabine{}}~: Deux Préludes extraits de l'\Opus{13}~; Deux
 Préludes extraits de l'\Opus{11}~; Sonate (œuvre incertaine)~; Dix Études
 extraites de l'\Opus{8}.
\end{description}

\section{Année~1937}

\begin{description}
 \item[\DateWithWeekDay{1937-01-09}]
 Leningrad~: grande salle de la société philharmonique%
 \footnote{Pour certains détails du programme de ce récital, voir
 \href{https://100philharmonia.spb.ru/historical-poster/11646/}%
 {https://100philharmonia.spb.ru/historical-poster/11646/}.}.

 \textsc{\Buxtehude{}/\Nikolaiev{}}~: Prélude pour orgue en \kF \Sharp
 mineur.
 \textsc{\Pachelbel{}/\Nikolaiev{}}~: Toccata pour orgue en \kF majeur.
 \textsc{\Beethoven{}}~: Sonate en \kF mineur, \Opus{57}.
 \textsc{\Chopin{}}~: Ballade~; Dix Mazurkas.
 \textsc{\Schumann{}}~: Carnaval, \Opus{9}.
 \item[\DateWithWeekDay{1937-01-13}]
 Leningrad.
 Concert et conférence.

 \textsc{\Liszt{}}~: Sonate en \kB mineur, S~178 (œuvre incertaine).
 \item[\DateWithWeekDay{1937-01-15}]
 Leningrad.
 Concert incertain.
 \item[\DateWithWeekDay{1937-02-11}]
 Leningrad~: grande salle de la société philharmonique.

 \textsc{\Schumann{}}~: Arabesque en \kC majeur, \Opus{18}~; Prélude en \kB
 \Flat mineur~; Romances (œuvres incertaines)~; \emph{Intermezzo} en \kF
 mineur~; Scherzo en \kG mineur, \Opus{32} \Number{1}~; \emph{Träumerei}~;
 \emph{Aufschwung}~; Études symphoniques, \Opus{13}.
 \textsc{\Chopin{}}~: Nocturne en \kC mineur, \Opus{48} \Number{1}~; Ballade
 en \kF mineur, \Opus{52}~; Ballade en \kA \Flat majeur, \Opus{47}~;
 Mazurka~; Polonaise.
 \item[\DateWithWeekDay{1937-02-24}]
 Leningrad.
 Concert pour la radio avec des œuvres de \Scriabine{}.
 \item[B1937-02 (fin) -- 1937-03 (début)]
 \VSofronitsky{} à Moskva.
 Selon le fils du pianiste, \ASofronitsky{}, il a été malade au début du
 mois de mars (lettre datée du~15~mars).
 \item[\DateWithWeekDay{1937-04-03}]
 Leningrad.
 Concert et conférence.

 \textsc{\Scriabine{}}~: Six Préludes~; Sonate en \kF \Sharp mineur,
 \Opus{23}.
 \item[B1937-05 (début, Pâques)]
 \VSofronitsky{} à Moskva.
 \item[\DateWithWeekDay{1937-05-20}]
 Leningrad~: société philharmonique.
 Concert.
 \item[\DateWithWeekDay{1937-05-24}]
 Leningrad~: grande salle de la société philharmonique.
 Concert de \VSofronitsky{} et \VSlivinsky{}.
 \item[\DateWithWeekDay{1937-05-28}]
 Moskva~: grande salle du conservatoire.
 Recensions de ce concert par G.~\Neuhaus{}%
 \footnote{\foreignlanguage{russian}{\emph{Советское искусство}},
 \Number{29} (23~juin~1937).}
 et par Ju.~Motylev%
 \footnote{\foreignlanguage{russian}{\emph{Музыка}}, \Number{12} (1937).},
 dont des extraits sont retranscrits par \citet[p.~428-429]{Milshteyn82a}
 et, pour Ju.~Motylev, par \citet[p.~387]{Nikonovich08}.

 \textsc{\Chopin{}}~: Polonaise en \kC \Sharp mineur, \Opus{26} \Number{1}~;
 Nocturne en \kC \Sharp mineur, \Opus{27} \Number{1}~; Nocturne en \kF
 \Sharp majeur, \Opus{15} \Number{2}~; Sonate en \kB \Flat mineur,
 \Opus{35}~; Fantaisie en \kF mineur, \Opus{49}~; Nocturne en \kC mineur,
 \Opus{48} \Number{1}~; Nocturne en \kF majeur, \Opus{15} \Number{1}~;
 Barcarolle en \kF \Sharp majeur, \Opus{60}~; Scherzo en \kB \Flat mineur,
 \Opus{31}~; Scherzo en \kB mineur, \Opus{20}.
 \emph{Bis} -- \textsc{\Chopin{}}~: Mazurka en \kE mineur, \Opus{41}
 \Number{2}~; Prélude en \kG \Sharp mineur, \Opus{28} \Number{12}~; Étude en
 \kC \Sharp mineur, \Opus{25} \Number{7}~; Mazurka en \kC \Sharp mineur,
 \Opus{30} \Number{4}~; Prélude en \kB \Flat mineur, \Opus{28} \Number{16}~;
 Prélude en \kD mineur, \Opus{28} \Number{24}~; Mazurka en \kD \Flat majeur,
 \Opus{30} \Number{3}.
 \item[B\DateWithWeekDay{1937-06-04}]
 Naissance, à Moskva, de la fille aînée de \VSofronitsky{},
 \RKoganSofronitskaya{}.
 \item[\DateWithWeekDay{1937-06-16}]
 Leningrad.
 Première session d'enregistrements en studio.

 \textsc{\Chopin{}}~: Étude \Number{4} en \kC \Sharp mineur, \Opus{10}
 \Number{4}~; Mazurka \Number{27} en \kE mineur, \Opus{41} \Number{2}.
 \item[\DateWithWeekDay{1937-07-06}]
 Leningrad.
 Concert.
 \item[\DateWithWeekDay{1937-07-09}]
 Moskva~: grande salle du conservatoire.
 Concert symphonique.
 Soliste~: \VSofronitsky{}.

 \textsc{\ARubinstein{}}~: Concerto pour piano et orchestre.
 \item[\DateWithWeekDay{1937-08-22}]
 Leningrad (lieu incertain).
 Concert pour la radio.
 \item[B1937-08 (fin)]
 \VSofronitsky{} dans une maison de convalescence à Luga.
 \item[B\DateWithWeekDay{1937-09-05}]
 Maladie du père de \VSofronitsky{}~; celui-ci lui rend visite avec son
 fils, Aleksandr.
 \item[\DateWithWeekDay{1937-09-07}]
 Leningrad.
 Concert pour la radio.

 \textsc{\Chopin{}}~: Polonaise en \kA \Flat majeur, \Opus{53}~; Deux
 Mazurkas.
 \item[\DateWithWeekDay{1937-10-11}]
 Leningrad.
 Concert privé.
 \item[B\DateWithWeekDay{1937-10-21} -- \DateWithWeekDay{1937-10-23}]
 Maladie grave du père de \VSofronitsky{}.
 \item[\DateWithWeekDay{1937-10-28}]
 Leningrad.
 Concert.
 \item[\DateWithWeekDay{1937-11-07}]
 Leningrad.
 Concert pour la radio.
 \item[\DateWithWeekDay{1937-11-26}]
 Leningrad~: grande salle de la société philharmonique.
 Dernier concert de la décennie de musique soviétique.
 Concert avec des œuvres de \Prokofiev{}.
 \item[B1937-1938 (saison)]
 Durant la saison~1937-1938, au conservatoire de Leningrad, \VSofronitsky{}
 annonce un cycle de douze concerts avec des programmes différents.
 Ce cycle de récitals, à l'\Quote{apogée de la terreur stalinienne}
 \citep[p.~540]{Voskobojnikov09b}, a eu un grand retentissement, comparable
 à celui des sept \Quote{concerts historiques} donnés par \ARubinstein{} en
 Rossija, en Europe et aux États-Unis d'Amérique, au
 \textsc{xix}\ieme{}~siècle.
 Voir, en particulier, un article de L.~Poljuta et L.~Potexin%
 \footnote{\foreignlanguage{russian}{\emph{Смена}}, \Number{40} (1938).}
 retranscrit par \citet[p.~430]{Milshteyn82a} puis par
 \citet[p.~387-388]{Nikonovich08} et un article de V.~\BogdanovBerezovsky{}%
 \footnote{\foreignlanguage{russian}{\emph{Советское искусство}},
 \Number{98} (26~juin~1938).}
 retranscrit par \citet[p.~430-431]{Milshteyn82a}.
 Voir aussi \citet[p.~38-41]{Sofronitsky82a}.
 \item[\DateWithWeekDay{1937-12-11}]
 Leningrad~: petite salle du conservatoire.
 Premier concert du cycle de douze \Quote{concerts historiques}.
 \citet[p.~158]{Nekrasova08} et \citet[p.~541]{Voskobojnikov09b} donnent la
 date du~11 décembre~1937, tandis que \citet[p.~48]{White},
 \citet[p.~16]{Artese} et \citet[p.~410-411]{Scriabine} indiquent celle
 du~1\ier{} décembre~1937, de même que \Sofronitsky{} dans une lettre du~20
 octobre~1937 à sa famille \citep[p.~20]{Kogan08}.

 \textsc{\Schumann{}}~: Études symphoniques, \Opus{13}.
 \textsc{\Chopin{}}~: Études en \kC majeur, en \kE majeur, en \kC \Sharp
 mineur, en \kE \Flat mineur et en \kA \Flat majeur~; Fantaisie en \kF
 mineur, \Opus{49}~; Vingt-quatre Préludes, \Opus{28}.
 \item[\DateWithWeekDay{1937-12-14}]
 Leningrad~: petite salle du conservatoire.
 Deuxième concert du cycle.
 Concert évoqué, de même que le suivant, dans une lettre du~17 décembre~1937
 de \VSofronitsky{} à sa famille \citep[p.~21]{Kogan08}.

 \textsc{\Schumann{}}~: Arabesque, \Opus{18}~; Fantaisie en \kC majeur,
 \Opus{17}.
 \textsc{\Chopin{}}~: Nocturnes en \kC mineur, \Opus{48} \Number{1}, et en
 \kF majeur, \Opus{15} \Number{1}~; Scherzo en \kB mineur, \Opus{20}.
 \textsc{\Liszt{}}~: \emph{Sposalizio}, S~161 \Number{1}~; \emph{Sonetto~104
 del Petrarca}, S~161 \Number{5}~; Valse-caprice en \kA \Flat majeur
 (d'après \Schubert{})~; Valse oubliée \Number{1}, S~215 \Number{1}~; Feux
 follets, S~139 \Number{5}~; \emph{Gnomenreigen}, S~145 \Number{2}~;
 Méphisto-valse.
 \item[\DateWithWeekDay{1937-12-23}]
 Leningrad~: petite salle du conservatoire.
 Troisième concert du cycle.
 Le cycle de \VSofronitsky{} se poursuit jusqu'au mois de juin~1938.

 \textsc{\Beethoven{}}~: \emph{Andante favori} en \kF majeur, WoO~57.
 \textsc{\Schumann{}}~: \emph{Kreisleriana}, \Opus{16}.
 \textsc{\Chopin{}}~: Ballades en \kG mineur, \Opus{23}, et en \kF mineur,
 \Opus{52}~; Mazurka~; Barcarolle en \kF \Sharp majeur, \Opus{60}.
 \textsc{\Debussy{}}~: \emph{Serenade of the Doll}, L~113 \Number{III}~;
 \emph{General Lavine -- eccentric}, L~123 \Number{VI}~; Canope, L~123
 \Number{X}~; Feux d'artifice, L~123 \Number{XII}.
\end{description}

\section{Année~1938}

\begin{description}
 \item[\DateWithWeekDay{1938-01-06}]
 Leningrad~: petite salle du conservatoire.
 Quatrième concert du cycle.
 Ce concert n'a pas eu lieu~: concert reporté au~20 janvier.

 \textsc{\Chopin{}}~: Polonaise en \kC \Sharp mineur, \Opus{26} \Number{1}~;
 Nocturnes en \kC \Sharp mineur, \Opus{27} \Number{1}, et en \kF \Sharp
 majeur, \Opus{15} \Number{2}~; Sonate en \kB \Flat mineur, \Opus{35}~; Cinq
 Mazurkas~; Ballade en \kA \Flat majeur, \Opus{47}~; Scherzo en \kB \Flat
 mineur, \Opus{31}~; Polonaise en \kA \Flat majeur, \Opus{53}.
 \item[\DateWithWeekDay{1938-01-09}]
 Moskva~: grande salle du conservatoire.

 \textsc{\Schumann{}}~: Études symphoniques, \Opus{13}.
 \textsc{\Chopin{}}~: Scherzo en \kB \Flat mineur, \Opus{31}~; Deux
 Mazurkas~; Ballade en \kA \Flat majeur, \Opus{47}~; Vingt-quatre Préludes,
 \Opus{28}.
 \emph{Bis} -- \textsc{\Chopin{}}~: Mazurka en \kE mineur, \Opus{41}
 \Number{2}~; Mazurka en \kF mineur, \Opus{7} \Number{3}~; Étude en \kC
 \Sharp mineur, \Opus{10} \Number{4}.
 \textsc{\Liszt{}}~: Feux follets, S~139 \Number{5}~; \emph{Gnomenreigen},
 S~145 \Number{2}.
 \textsc{\Debussy{}}~: Feux d'artifice, L~123 \Number{12}.
 \textsc{\Scriabine{}}~: Étude en \kD \Flat majeur, \Opus{8} \Number{10}.
 \item[B\DateWithWeekDay{1938-01-17} ou~18]
 \VSofronitsky{} revient à Leningrad de Moskva (lettre datée du~19
 janvier~1938).
 \item[\DateWithWeekDay{1938-01-19}]
 Leningrad.
 Pour le jubilé du conservatoire de Leningrad, \VSofronitsky{} joue le
 Concerto en \kD mineur d'\ARubinstein{}, \Opus{70}.
 L'orchestre symphonique est placé sous la direction d'\hbox{Isaj} Moiseevič
 Al'terman,\index[ndxnames]{Al'terman, Isaj Moiseevič} disciple d'\AGauk{}.
 \item[\DateWithWeekDay{1938-01-20}]
 Leningrad~: petite salle du conservatoire.
 Quatrième concert du cycle.
 Concert reporté du~6 janvier~; programme indiqué \emph{supra}.
 \item[\DateWithWeekDay{1938-01-28}]
 Leningrad~: petite salle du conservatoire.
 Cinquième concert du cycle.
 \citet[p.~411]{Scriabine} indiquent la date du~28 janvier pour ce concert
 prévu, au départ, le~24 janvier \citep[voir][p.~158]{Nekrasova08}.
 Selon \citeauthor{Nekrasova08}, la première œuvre jouée de \Liszt{}, en \kE
 mineur, n'est pas spécifiée.

 \textsc{\Haendel{}}~: Variations en \kE majeur sur le finale (Air) de la
 Suite \Number{5} en \kE majeur.
 \textsc{\Mozart{}}~: Fantaisie en \kC mineur.
 \textsc{\Beethoven{}}~: Sonate en \kC mineur, \Opus{111}.
 \textsc{\Liszt{}}~: Funérailles, S~173 \Number{7}~; Deux lieder (d'après
 \Schubert{})~; \emph{Sonetto~123 del Petrarca}, S~161 \Number{6}~;
 \emph{Tarantella} de \emph{Venezia e Napoli}, S~162 \Number{3}.
 \item[\DateWithWeekDay{1938-02-01}]
 Leningrad.
 Concert pour la radio avec une Sonate de \Haydn{}.
 \item[\DateWithWeekDay{1938-02-08}]
 Leningrad~: petite salle du conservatoire.
 Sixième concert du cycle.
 \citet[p.~158]{Nekrasova08} et \citet[p.~411]{Scriabine} donnent la date
 du~8 février~1938, tandis que \citet[p.~49]{White} indique celle du~2
 février~1938.
 La référence la plus récente mentionne que le concert a été reporté du~2
 février.

 \textsc{\Buxtehude{}/\Nikolaiev{}}~: Prélude et fugue pour orgue en \kF
 \Sharp mineur.
 \textsc{\Schumann{}}~: Sonate en \kF \Sharp mineur, \Opus{11}.
 \textsc{\Scriabine{}}~: Six Préludes~; Sonate en \kF \Sharp mineur,
 \Opus{23}~; Sonate en \kF \Sharp majeur, \Opus{30}.
 \item[1938-02 (mi)]
 Concert symphonique mentionné dans une lettre datée du~8~mars.
 \item[\DateWithWeekDay{1938-02-24}]
 Leningrad~: petite salle du conservatoire.
 Septième concert du cycle.

 \textsc{\JBach{}/\Busoni{}}~: Préludes de chorals en \kG mineur et en \kG
 majeur.
 \textsc{\Scarlatti{}}~: Quatre Sonates.
 \textsc{\Beethoven{}}~: Sonate en \kF mineur, \Opus{57}.
 \textsc{\Liszt{}}~: Sonate en \kB mineur, S~178.
 \item[B1938-03 (début)]
 \VSofronitsky{} malade.
 \item[\DateWithWeekDay{1938-03-18}]
 Leningrad~: petite salle du conservatoire.
 Huitième concert du cycle.
 \citet[p.~158]{Nekrasova08} et \citet[p.~411]{Scriabine} donnent la date
 du~18 mars~1938, tandis que \citet[p.~49]{White} indique celle du~12
 mars~1938.
 La référence la plus récente mentionne que le concert a été reporté du~12
 mars.

 \textsc{\Mendelssohn{}}~: Variations sérieuses en \kD mineur, \Opus{54}.
 \textsc{\Chopin{}}~: Sonate en \kB mineur, \Opus{58}.
 \textsc{\Scriabine{}}~: Dix Études~; Deux Morceaux~; Deux Poèmes~; Poème
 satanique, \Opus{36}.
 \item[\DateWithWeekDay{1938-03-24}]
 Leningrad~: grande salle de la société philharmonique.
 Concert symphonique auquel participe \VSofronitsky{}, dans des œuvres en
 solo~; le chef d'orchestre est Boris Xajkin (œuvres de \Borodine{},
 \Liadov{} et \Tchaikovski{}).

 \textsc{\Scriabine{}}~: Six Préludes~; Sonate en \kF \Sharp mineur,
 \Opus{23}.
 \item[\DateWithWeekDay{1938-04-01}]
 Moskva~: grande salle du conservatoire.

 \textsc{\Mendelssohn{}}~: Variations sérieuses en \kD mineur, \Opus{54}.
 \textsc{\Schumann{}}~: Fantaisie en \kC majeur, \Opus{17}.
 \textsc{\Scriabine{}}~: Cinq Préludes~; Sonate en \kF \Sharp mineur,
 \Opus{23}~; Deux Poèmes, \Opus{32}~; Poème en \kC majeur, \Opus{52}
 \Number{1}~; Énigme, \Opus{52} \Number{2}~; Poème, \Opus{59} \Number{1}~;
 Un Poème extrait de l'\Opus{69}~; Poème satanique, \Opus{36}.
 \emph{Bis} -- \textsc{\Scriabine{}}~: Prélude en \kG mineur, \Opus{27}
 \Number{1}.
 \textsc{\Rachmaninov{}}~: Prélude en \kG majeur, \Opus{32} \Number{5}~;
 Prélude en \kG \Sharp mineur, \Opus{32} \Number{12}.
 \textsc{\Scriabine{}}~: Étude en \kD \Flat majeur, \Opus{8} \Number{10}~;
 Mazurka extraite de l'\Opus{25}~; Étude en \kD \Sharp mineur, \Opus{8}
 \Number{12}.
 \item[\DateWithWeekDay{1938-04-03}]
 Moskva~: grande salle du conservatoire.

 \textsc{\CBach{}}~: \emph{Rondo espressivo}.
 \textsc{\JBach{}/\Ziloti{}}~: Deux Préludes en \kE mineur et en \kB mineur.
 \textsc{\Haydn{}}~: Sonate en \kD majeur, Hob.XVI:37.
 \textsc{\Schumann{}}~: Carnaval, \Opus{9}.
 \textsc{\Scriabine{}}.
 \item[\DateWithWeekDay{1938-04-12}]
 Leningrad~: petite salle du conservatoire.
 Neuvième concert du cycle.
 \citet[p.~158]{Nekrasova08} et \citet[p.~412]{Scriabine} donnent la date
 du~12 avril~1938, tandis que \citet[p.~49]{White} indique celle du~29
 mars~1938.
 La référence la plus récente mentionne que le concert a été reporté du~29
 mars.
 Concert évoqué, de même que le suivant avec \DChostakovitch{}, dans une
 lettre du~16 avril~1938 de \VSofronitsky{} à sa famille
 \citep[p.~23]{Kogan08}.

 \textsc{\CBach{}}~: \emph{Rondo espressivo}.
 \textsc{\JBach{}/\Ziloti{}}~: Deux Préludes en \kE mineur et en \kB mineur.
 \textsc{\Haydn{}}~: Sonate en \kD majeur, Hob.XVI:37.
 \textsc{\Schumann{}}~: \emph{Humoreske} en \kB \Flat majeur, \Opus{20}.
 \textsc{\Chopin{}}~: Dix Mazurkas.
 \textsc{\Schumann{}}~: Carnaval, \Opus{9}.
 \item[\DateWithWeekDay{1938-04-15}]
 Leningrad~: grande salle de la société philharmonique.
 Interprétation pour deux pianos avec \DChostakovitch{} lors d'un concert de
 pianistes issus de la classe de \LNikolaiev{}.

 \textsc{\Nikolaiev{}}~: Variations sur un thème de quatre notes.
 \item[\DateWithWeekDay{1938-04-23}]
 Leningrad~: petite salle du conservatoire.
 Dixième concert du cycle.

 \textsc{\Schumann{}}~: \emph{Kinderszenen}, \Opus{15}~; Deux
 \emph{Novelettes} en \kE majeur, \Opus{21} \Number{7}, et en \kF \Sharp
 mineur, \Opus{21} \Number{8}.
 \textsc{\Chopin{}}~: Nocturne en \kF \Sharp mineur, \Opus{48} \Number{2}~;
 Deux Études extraites de l'\Opus{25}~; Scherzo.
 \textsc{\Liszt{}}~: Au Lac de Wallenstadt, S~160 \Number{2}~;
 \emph{Waldesrauschen}, S~145 \Number{1}~; Deux Valses-caprices (d'après
 \Schubert{})~; Deux Études.
 \item[\DateWithWeekDay{1938-05-15}]
 Leningrad~: grande salle de la société philharmonique.
 Concert avec des œuvres de compositeurs russes.

 \textsc{\Bortnianski{}}~: Sonate.
 \textsc{\Borodine{}}~: Petite Suite.
 \textsc{\Glazounov{}}~: Sonate en \kB \Flat mineur, \Opus{74}.
 \textsc{\Scriabine{}}~: Sonate-fantaisie en \kG \Sharp mineur, \Opus{19}~;
 Sonate, \Opus{70}.
 \textsc{\Medtner{}}~: \emph{Skazka}.
 \textsc{\Rachmaninov{}}~: Deux Préludes.
 \textsc{\Balakirev{}}~: \emph{Islamey}.
 \item[\DateWithWeekDay{1938-05-18}]
 Leningrad~: petite salle du conservatoire.
 Onzième concert du cycle.

 \textsc{\Glazounov{}}~: Prélude et fugue en \kC majeur, \Opus{101}.
 \textsc{\Medtner{}}~: Marche funèbre, \Opus{31} \Number{2}~; Idylle,
 \Opus{7} \Number{1}~; \emph{Novelette} en \kC mineur, \Opus{17}
 \Number{2}~; \emph{Skazka} en \kB mineur~; Dithyrambe, \Opus{10}.
 \textsc{\Rachmaninov{}}~: Deux Préludes.
 \textsc{\Prokofiev{}}~: Deux Études, \Opus{2}~; Deux Contes de la vieille
 grand-mère extraits de l'\Opus{31}~; Sarcasmes, \Opus{17} \Number{4} et
 \Number{5}.
 \textsc{\Chostakovitch{}}~: Quatre Préludes, \Opus{34}.
 \textsc{\Scriabine{}}~: Sonate, \Opus{53}.
 \item[\DateWithWeekDay{1938-05-24}]
 Leningrad~: grande salle de la société philharmonique.
 Concert de \VSlivinsky{} (première partie) et \VSofronitsky{} (seconde
 partie).

 \textsc{\Chopin{}}~: Quatre Mazurkas.
 \textsc{\Schumann{}}~: Carnaval, \Opus{9}.
 \textsc{\Rachmaninov{}}~: Moments musicaux~; Étude-tableau~; Deux Préludes.
 \textsc{\Scriabine{}}~: Deux Études.
 \textsc{\Debussy{}}~: Feux d'artifice, L~123 \Number{XII}.
 \item[B\DateWithWeekDay{1938-06-07}]
 \VSofronitsky{} reçoit le grade académique de docteur en musique.
 Évoqué dans une lettre de \VSofronitsky{} à sa famille le~8 juin~1938
 \citep[p.~24]{Kogan08}.
 \item[\DateWithWeekDay{1938-06-08}]
 Leningrad.
 Concert pour la radio avec le Carnaval de \Schumann{}, \Opus{9}.
 \item[\DateWithWeekDay{1938-06-21}]
 Leningrad~: petite salle du conservatoire.
 Douzième et dernier concert du cycle.
 \citet[p.~159]{Nekrasova08} et \citet[p.~412]{Scriabine} donnent la date
 du~21 juin~1938, tandis que \citet[p.~50]{White} indique celle du~8
 juin~1938.
 La référence la plus récente mentionne que le concert a été reporté du~8
 juin.
 Concert et cycle évoqués par \VSofronitsky{} dans une lettre à sa famille
 le~22 juin~1938 \citep[p.~24]{Kogan08}.
 À propos des œuvres jouées de \Miaskovski{}, \citet[p.~159]{Nekrasova08}
 indique plutôt Deux \foreignlanguage{russian}{Причуды} [Pričudy], \cad{}
 Deux Excentricités, extraites de l'\Opus{25}.
 Toujours selon \citeauthor{Nekrasova08}, les œuvres jouées de \Prokofiev{}
 étaient plutôt~: Légende, \Opus{12} \Number{6}~; Contes de la vieille
 grand-mère, \Opus{31}~; Deux Marches~; Gavotte~; Sonate \Number{3} en \kA
 mineur, \Opus{28}.
 Voir \citet{Bogdanov67a} pour les œuvres jouées de ce compositeur.

 \textsc{\Prokofiev{}}~: Deux Marches, \Opus{3} \Number{3} et \Opus{12}
 \Number{1}~; Visions fugitives, \Opus{22}~; Pensées, \Opus{62}~; Sarcasmes,
 \Opus{17}~; Sonate \Number{3} en \kA mineur, \Opus{28}.
 \textsc{\Kabalevski{}}~: Sonatine en \kC majeur, \Opus{13} \Number{1}.
 \textsc{\Miaskovski{}/\Alyavdina{}}~: Scherzo de la Symphonie \Number{5} en
 \kD majeur, \Opus{18}.
 \textsc{\BogdanovBerezovsky{}}~: Deux Études, \Opus{16}~; Petite Suite
 (\Number{1}~: Danse en \kA majeur~; \Number{2}~: Marche en \kD majeur).
 \textsc{\Goltz{}}~: Six Préludes~; Scherzo en \kE mineur.
 \item[\DateWithWeekDay{1938-06-28}]
 Leningrad.
 \ASofronitsky{} a indiqué seulement la date et la ville~: le lieu précis du
 concert et son programme ne sont pas désignés.
 \item[B1938-07]
 \VSofronitsky{} et sa famille en séjour à Malojaroslavec, selon
 \citet[p.~159]{Nekrasova08}.
 \item[B1938-08]
 \VSofronitsky{} en séjour dans la famille \Vizel{}, au village de Rapti,
 près de Luga, selon \citet[p.~159]{Nekrasova08}.
 \item[B1938-09]
 \ESofronitskaya{} et \RKoganSofronitskaya{} vivent à Leningrad.
 \item[\DateWithWeekDay{1938-10-12}]
 Leningrad~: grande salle de la société philharmonique.

 \textsc{\Beethoven{}}~: Sonate en \kD majeur, \Opus{28}.
 \textsc{\Schumann{}}~: Études symphoniques, \Opus{13}.
 \textsc{\Chopin{}}~: Fantaisie en \kF mineur, \Opus{49}~; Deux Mazurkas~;
 Ballade~; Barcarolle en \kF \Sharp majeur, \Opus{60}~; Sonate en \kB \Flat
 mineur, \Opus{35}.
 \item[\DateWithWeekDay{1938-10-16}]
 Leningrad~: maison des arts.
 Concert.
 \item[\DateWithWeekDay{1938-10-20}]
 Leningrad.
 Concert pour la radio avec des œuvres de \Scriabine{}~: Préludes,
 \Opus{22}~; Sonate \Number{3} en \kF \Sharp mineur, \Opus{23}.
 \item[B1938-1939 (saison)]
 Pour la saison~1938-1939, annonce du projet d'un cycle de concerts consacré
 à la musique russe et soviétique, et donné à la grande salle de la société
 philharmonique de Leningrad.
 Le premier concert du cycle a lieu le~28 novembre~1938.
 Dates des concerts suivants~: le~20 décembre~1938, le~31 janvier~1939,
 le~21 avril~1939 et le~7 mai~1939.
 Compositeurs joués à l'occasion de ce cycle~: \DBortnianski{}, \MGlinka{},
 \MMoussorgski{}, \ABorodine{}, \ARubinstein{}, \PTchaikovski{},
 \MBalakirev{}, \SLiapounov{}, \ALiadov{}, \AGlazounov{}, \AAleksandrov{},
 \SFeinberg{}, \NMiaskovski{}, \VChtcherbatchiov{}, \DKabalevski{} et
 d'autres auteurs.
 Certains concerts du cycle ont été repris à Moskva.
 Voir \citet[p.~160]{Nekrasova08} et \citet[p.~413]{Scriabine}.
 \item[\DateWithWeekDay{1938-11-28}]
 Leningrad~: grande salle de la société philharmonique.
 Concert avec des œuvres de compositeurs soviétiques~; premier concert du
 cycle annoncé ci-dessus.

 \textsc{\Miaskovski{}}~: Sonate en \kF \Sharp mineur, \Opus{13}.
 \textsc{\Prokofiev{}}~: Quinze des Visions fugitives, \Opus{22}~; Sonate en
 \kA mineur, \Opus{28}~; Marche, Gavotte, Rigaudon et Allemande des Dix
 Pièces, \Opus{12}~; Contes de la vieille grand-mère, \Opus{31}~; Deux des
 Sarcasmes, \Opus{17}.
 \textsc{\Kabalevski{}}~: Sonatine en \kC majeur, \Opus{13} \Number{1}.
 \textsc{\BogdanovBerezovsky{}}~: Deux Études, \Opus{16}.
 \textsc{\Goltz{}}~: Quatre Préludes~; Scherzo en \kE mineur.
 \item[\DateWithWeekDay{1938-12-05}]
 Leningrad.
 Concert pour la radio.
 Évoqué en termes très peu élogieux par \VSofronitsky{} dans une lettre à sa
 famille le~6 décembre~1938 \citep[p.~25]{Kogan08}~; le musicien a joué une
 Ballade.
 \item[\DateWithWeekDay{1938-12-20}]
 Leningrad~: grande salle de la société philharmonique.
 Deuxième concert du cycle de musique russe et soviétique.

 \textsc{\Scriabine{}}~: \emph{Allegro} de concert en \kB \Flat mineur,
 \Opus{18}~; Deux Impromptus~: \Opus{12} \Number{2} et \Opus{14}
 \Number{2}~; Sonate en \kF \Sharp mineur, \Opus{23}~; Dix Études extraites
 de l'\Opus{8}~; Deux Poèmes, \Opus{32}~; Deux Préludes, \Opus{27}~; Sonate
 en \kF \Sharp majeur, \Opus{30}~; Deux Poèmes~: \Opus{59} \Number{1} et
 \Opus{69} \Number{1}~; Énigme, \Opus{52} \Number{2}~; Préludes, \Opus{74}~;
 Vers la flamme, \Opus{72}~; Sonate, \Opus{53}.
 \item[\DateWithWeekDay{1938-12-29}]
 Moskva~: grande salle du conservatoire.
 Concert qui reprend le programme de celui du~20 décembre.
 Concert reporté pour des raisons de santé évoquées par \VSofronitsky{} dans
 une lettre du~30 décembre~1938 à sa famille \citep[p.~25]{Kogan08}.
\end{description}

\section{Année~1939}

\begin{description}
 \item[B1939-01]
 \VSofronitsky{} malade.
 \item[\DateWithWeekDay{1939-01-31}]
 Leningrad~: grande salle de la société philharmonique.
 Troisième concert du cycle de musique russe et soviétique.
 Concert mentionné par \citet[p.~160]{Nekrasova08}, mais absent de la
 chronologie établie par \citet[p.~413]{Scriabine}.
 \item[\DateWithWeekDay{1939-03-02}]
 Concert pour la radio avec des œuvres de \Scriabine{}~: Deux Impromptus.
 \item[B1939-03 (début, avant le~20)]
 Appendicectomie évoquée par \VSofronitsky{} dans ses lettres du~27 février
 et du~20 mars~1939 à sa famille \citep[p.~26]{Kogan08}.
 \item[\DateWithWeekDay{1939-04-02}]
 Moskva~: conservatoire.

 \textsc{\JBach{}/\Busoni{}}~: Trois Préludes.
 \textsc{\Mozart{}}~: Fantaisie.
 \textsc{\Chopin{}}~: Sonate en \kB \Flat mineur, \Opus{35}.
 \item[\DateWithWeekDay{1939-04-15}]
 Leningrad~: petite salle de la société philharmonique.
 Interprétation pour deux pianos avec \DChostakovitch{} lors d'un concert de
 pianistes issus de la classe de \LNikolaiev{}.

 \textsc{\Nikolaiev{}}~: Variations pour deux pianos sur un thème de quatre
 notes.
 \item[\DateWithWeekDay{1939-04-21}]
 Leningrad~: grande salle de la société philharmonique.
 Quatrième concert du cycle de musique russe et soviétique.
 Concert mentionné par \citet[p.~160]{Nekrasova08}, mais absent de la
 chronologie établie par \citet[p.~413]{Scriabine}.
 \item[\DateWithWeekDay{1939-04-23}]
 Concert.
 Peut-être est-ce celui évoqué par \VSofronitsky{} dans une lettre à sa
 famille le~21 février~1939 \citep[p.~26]{Kogan08}.

 \textsc{\Bortnianski{}}~: Sonate (1784).
 \textsc{\Glinka{}}~: Variations sur un thème écossais (1847).
 \textsc{\Borodine{}}~: Petite Suite (sept pièces).
 \textsc{\Glazounov{}}~: Sonate en \kB \Flat mineur, \Opus{74}.
 \textsc{\Scriabine{}}~: Sonate-fantaisie en \kG \Sharp mineur, \Opus{19}~;
 Deux Études extraites de l'\Opus{42}~; Sonate, \Opus{70}.
 \textsc{\Medtner{}}~: Deux \emph{Skazki} en \kB mineur et en \kE mineur.
 \textsc{\Rachmaninov{}}~: Deux Préludes.
 \textsc{\Balakirev{}}~: \emph{Islamey}, \Opus{18}.
 \item[1939-05 et 1939-06]
 Bakou et Kiïv (plusieurs concerts).
 Selon \citet[p.~414]{Scriabine}, le ou les concerts du mois de juin~1939
 ont peut-être eu lieu à Tbilissi.
 \item[\DateWithWeekDay{1939-05-06}]
 Kiïv~: société philharmonique.

 \textsc{\Schumann{}}~: Études symphoniques, \Opus{13}.
 \textsc{\Chopin{}}~: Six Études~; Fantaisie en \kF mineur, \Opus{49}~;
 Vingt-quatre Préludes, \Opus{28}.
 \item[\DateWithWeekDay{1939-05-07}]
 Leningrad~: grande salle de la société philharmonique.
 Cinquième et dernier concert du cycle de musique russe et soviétique.
 Concert mentionné par \citet[p.~160]{Nekrasova08}, mais absent de la
 chronologie établie par \citet[p.~414]{Scriabine}.
 Ce concert à Leningrad est très surprenant, entre deux dates contiguës à
 Kiïv.
 \item[\DateWithWeekDay{1939-05-08}]
 Kiïv~: société philharmonique.

 \textsc{\Scriabine{}}~: Deux Préludes~; Deux Impromptus~; Sonate en \kF
 \Sharp mineur, \Opus{23}~; Huit Études extraites de l'\Opus{8}~;
 Sonate-fantaisie en \kG \Sharp mineur, \Opus{19}~; Deux Préludes,
 \Opus{27}~; Deux Poèmes, \Opus{32}~; Sonate en \kF \Sharp majeur,
 \Opus{30}~; Poème, \Opus{59} \Number{1}~; Énigme, \Opus{52} \Number{2}~;
 Nuances, \Opus{56} \Number{3}~; Danse languide, \Opus{51} \Number{4}~;
 Poème satanique, \Opus{36}.
 \item[\DateWithWeekDay{1939-05-18}]
 Moskva~: grande salle du conservatoire.

 \textsc{\Scriabine{}}~: Deux Impromptus~; Sonate en \kF \Sharp mineur,
 \Opus{23}~; Huit Études extraites de l'\Opus{8}~; Deux Préludes,
 \Opus{27}~; Deux Poèmes, \Opus{32}~; Sonate en \kF \Sharp majeur,
 \Opus{30}~; Poème, \Opus{59} \Number{1}~; Énigme, \Opus{52} \Number{2}~;
 Nuances, \Opus{56} \Number{3}~; Danse languide, \Opus{51} \Number{4}~;
 Sonate, \Opus{70}~; Poème satanique, \Opus{36}.
 \item[\DateWithWeekDay{1939-05-24}]
 Bakou~: société philharmonique.
 Reprise du concert du~6 mai~1939 à Kiïv.

 \textsc{\Schumann{}}~: Études symphoniques, \Opus{13}.
 \textsc{\Chopin{}}~: Six Études~; Fantaisie en \kF mineur, \Opus{49}~;
 Vingt-quatre Préludes, \Opus{28}.
 \item[\DateWithWeekDay{1939-05-26}]
 Bakou~: société philharmonique.
 Reprise du concert du~8 mai~1939 à Kiïv.

 \textsc{\Scriabine{}}~: Deux Préludes~; Deux Impromptus~; Sonate en \kF
 \Sharp mineur, \Opus{23}~; Huit Études extraites de l'\Opus{8}~;
 Sonate-fantaisie en \kG \Sharp mineur, \Opus{19}~; Deux Préludes,
 \Opus{27}~; Deux Poèmes, \Opus{32}~; Sonate en \kF \Sharp majeur,
 \Opus{30}~; Poème, \Opus{59} \Number{1}~; Énigme, \Opus{52} \Number{2}~;
 Nuances, \Opus{56} \Number{3}~; Danse languide, \Opus{51} \Number{4}~;
 Poème satanique, \Opus{36}.
 \item[1939-06]
 Tbilissi.
 Concerts incertains.
 \item[\DateWithWeekDay{1939-09-15}]
 Moskva~: grande salle du conservatoire.
 Premier concert d'un cycle \Chopin{} comportant neuf concerts.
 \item[\DateWithWeekDay{1939-10-09}]
 Moskva~: grande salle du conservatoire.
 Concert pour le~130\ieme{} anniversaire de la naissance de \Schumann{}.

 \textsc{\Schumann{}}~: Sonate en \kF mineur, \Opus{14}~;
 \emph{Kreisleriana}, \Opus{16}~; Études symphoniques, \Opus{13}~; Carnaval,
 \Opus{9}.
 \emph{Bis} -- \textsc{\Schumann{}}~: Romance en \kF \Sharp majeur,
 \Opus{28} \Number{2}~; \emph{Träumerei} extraite des \emph{Kinderszenen},
 \Opus{15}~; \emph{Traumes Wirren} extraits des \emph{Phantasiestücke},
 \Opus{12} \Number{7}.
 \item[\DateWithWeekDay{1939-10-15}]
 Moskva~: grande salle du conservatoire.
 Deuxième des neuf concerts du cycle \Chopin{}.

 \textsc{\Chopin{}}~: Polonaise en \kC mineur, \Opus{40} \Number{2}~; Deux
 Nocturnes en \kC mineur et en \kF majeur, \Opus{48} \Number{1} et \Opus{15}
 \Number{1}~; Vingt-quatre Préludes, \Opus{28}~; Fantaisie en \kF mineur,
 \Opus{49}~; Deux Mazurkas en \kC majeur et en \kF mineur~; Scherzo en \kB
 mineur, \Opus{20}~; Sonate en \kB \Flat mineur, \Opus{35}.
 \emph{Bis} -- \textsc{\Chopin{}}~: Mazurka~; Trois Études.
 \item[\DateWithWeekDay{1939-10-31}]
 Leningrad~: grande salle de la société philharmonique.
 Concert pour le~130\ieme{} anniversaire de la naissance de \Schumann{}.

 \textsc{\Schumann{}}~: Sonate en \kF mineur, \Opus{14}~;
 \emph{Kreisleriana}, \Opus{16}~; Études symphoniques, \Opus{13}~; Carnaval,
 \Opus{9}.
 \item[\DateWithWeekDay{1939-11-15}]
 Moskva~: grande salle du conservatoire.

 \textsc{\Chopin{}}~: Polonaise en \kC mineur, \Opus{40} \Number{2}~;
 Nocturne en \kC mineur, \Opus{48} \Number{1}~; Nocturne en \kF majeur,
 \Opus{15} \Number{1}~; Vingt-quatre Préludes, \Opus{28}~; Fantaisie en \kF
 mineur, \Opus{49}~; Deux Mazurkas en \kC majeur et en \kF mineur~; Scherzo
 en \kB \Flat mineur, \Opus{31}~; Sonate en \kB \Flat mineur, \Opus{35}.
 \item[\DateWithWeekDay{1939-12-02}]
 Leningrad~: grande salle du conservatoire.
 Reprise du programme du concert du~15 novembre à Moskva.

 \textsc{\Chopin{}}~: Polonaise en \kC mineur, \Opus{40} \Number{2}~;
 Nocturne en \kC mineur, \Opus{48} \Number{1}~; Nocturne en \kF majeur,
 \Opus{15} \Number{1}~; Vingt-quatre Préludes, \Opus{28}~; Fantaisie en \kF
 mineur, \Opus{49}~; Deux Mazurkas en \kC majeur et en \kF mineur~; Sonate
 en \kB \Flat mineur, \Opus{35}.
 \item[\DateWithWeekDay{1939-12-07}]
 Leningrad~: petite salle de la société philharmonique.
 Concert académique donné par les professeurs, professeurs associés et
 assistants du département dirigé par \LNikolaiev{}.
 Concert ouvrant la décennie de musique soviétique.
 \VSofronitsky{} joue la dernière partie de ce concert.
 Voir \citet[p.~161]{Nekrasova08}.

 \textsc{\Prokofiev{}}~: Deux Marches~; Cinq Sarcasmes, \Opus{17}.
 \item[B\DateWithWeekDay{1939-12-17}]
 Approbation de \VSofronitsky{} au poste de professeur au conservatoire de
 Leningrad.
 \item[\DateWithWeekDay{1939-12-25}]
 Moskva~: grande salle du conservatoire.
 Deuxième concert d'un cycle de six concerts (concert philharmonique).

 \textsc{\Schumann{}}~: Fantaisie en \kC majeur, \Opus{17}.
 \textsc{\Chopin{}}~: Douze Études extraites des \Opus{10 et~25}.
 \textsc{\Liszt{}}~: Sonate en \kB mineur, S~178.
 \emph{Bis} -- \textsc{\Chopin{}}~: Nocturne en \kF \Sharp majeur, \Opus{15}
 \Number{2}.
 \textsc{\JBach{}}~: Prélude.
 \textsc{\Liszt{}}~: Valse oubliée.
 \textsc{\Debussy{}}~: Feux d'artifice, L~123 \Number{XII}.
 \textsc{\Scriabine{}}~: Étude.
 \textsc{\Chopin{}}~: Étude.
 \item[\DateWithWeekDay{1939-12-29}]
 Leningrad~: grande salle de la société philharmonique.
 Reprise du programme du concert du~25 décembre à Moskva.

 \textsc{\Schumann{}}~: Fantaisie en \kC majeur, \Opus{17} (dédiée à
 \FLiszt{}).
 \textsc{\Chopin{}}~: Douze Études extraites des \Opus{10 et~25} (dédiées à
 \FLiszt{} et à Marie d'Agoult).
 \textsc{\Liszt{}}~: Sonate en \kB mineur, S~178 (dédiée à \RSchumann{}).
\end{description}

\section{Année~1940}

\begin{description}
 \item[\DateWithWeekDay{1940-01-10}]
 Moskva~: salle \Tchaikovski{}.
 Concert symphonique placé sous la direction d'\IMiklashevsky{}.

 \textsc{\Scriabine{}}~: Concerto en \kF \Sharp mineur, \Opus{20}~; œuvres
 de forme brève jouées après le concert.

 \Comment{Lors d'une communication privée avec \citet{TADGO19400110}, au
 mois de janvier~\citeyear{TADGO19400110}, ce dernier m'a indiqué que cette
 entrée, le~10 janvier~1940, est douteuse au plus haut degré.
 L'argumentation d'\citeauthor{TADGO19400110} est très convaincante et, avec
 son autorisation, en voici un résumé.
 Cette entrée avance tout d'abord deux informations qui sont impossibles.
 D'une part, il est impossible que \IMiklashevsky{} ait été chef d'orchestre
 de ce concert à Moskva le~10 janvier~1940~: \Miklashevsky{} était interdit
 de direction d'orchestre jusqu'à la fin de l'année~1940, et de toute façon
 la date de sa première apparition en tant que chef d'orchestre à Moskva est
 le~10 janvier~1941 -- voir l'entrée correspondante, \hbox{1941-01-10}, plus
 loin dans ce document.
 D'autre part, il est impossible qu'un tel concert ait eu lieu à la salle
 \Tchaikovski{} à Moskva~: en effet, après l'assassinat de \VMeyerhold{} par
 la police politique de \Staline{} en février~1940, le bâtiment du théâtre
 qu'il occupait à Moskva a été réaffecté en salle de concert, mais celle-ci
 n'a ouvert ses portes, après les transformations, que le~12 octobre~1940,
 sous le nom \emph{salle \PTchaikovski{}}.
 Pour le reste de l'entrée, il n'est bien entendu pas impossible que
 \VSofronitsky{} ait joué le Concerto de \Scriabine{} à cette date et dans
 cette ville -- mais alors dans une autre salle et sous la direction d'un
 autre chef d'orchestre.
 Le plus probable est que cette entrée, citée par \citet[p.~415]{Scriabine},
 soit un double erroné de l'entrée~\hbox{1941-01-10} ci-dessous, laquelle
 est référencée de longue date \citep[voir par exemple][p.~52]{White}.}
 \item[\DateWithWeekDay{1940-01-21}]
 Leningrad~: conservatoire.
 Concert.
 \item[B\DateWithWeekDay{1940-02-02}]
 \VMeyerhold{} fusillé par la police politique de \Staline{}.
 \item[\DateWithWeekDay{1940-02-07}]
 Leningrad~: grande salle de la société philharmonique.

 \textsc{\Chopin{}}~: Quatre Impromptus en \kF \Sharp majeur, \Opus{36}, en
 \kA \Flat majeur, \Opus{29}, en \kG \Flat majeur, \Opus{51}, et en \kC
 \Sharp mineur, \Opus{66} (Fantaisie-impromptu)~; Sonate en \kB mineur,
 \Opus{58}~; Ballade en \kG mineur, \Opus{23}~; Ballade en \kA \Flat majeur,
 \Opus{47}~; Une Mazurka extraite de l'\Opus{50}~; Scherzo en \kB mineur,
 \Opus{20}~; Scherzo en \kC \Sharp mineur, \Opus{39}~; Polonaise en \kA
 \Flat majeur, \Opus{53}.
 \item[\DateWithWeekDay{1940-02-11}]
 Moskva~: grande salle du conservatoire.
 Troisième concert d'un cycle de six concerts consacré à \Chopin{} selon
 \citet[p.~162]{Nekrasova08}~; reprise du programme du concert du~7 février.

 \textsc{\Chopin{}}~: Quatre Impromptus en \kF \Sharp majeur, \Opus{36}, en
 \kA \Flat majeur, \Opus{29}, en \kG \Flat majeur, \Opus{51}, et en \kC
 \Sharp mineur, \Opus{66} (Fantaisie-impromptu)~; Sonate en \kB mineur,
 \Opus{58}~; Ballade en \kG mineur, \Opus{23}~; Ballade en \kA \Flat majeur,
 \Opus{47}~; Une Mazurka extraite de l'\Opus{50}~; Scherzo en \kB mineur,
 \Opus{20}~; Scherzo en \kC \Sharp mineur, \Opus{39}~; Polonaise en \kA
 \Flat majeur, \Opus{53}.
 \item[\DateWithWeekDay{1940-02-20}]
 Moskva~: grande salle du conservatoire.
 Troisième concert d'abonnement \Number{10}, \Quote{Musique pour piano},
 consacré à \Schumann{} pour commémorer le~130\ieme{} anniversaire de sa
 naissance.
 Concert et programme mentionnés par \citet[p.~162]{Nekrasova08}.

 \textsc{\Schumann{}}~: Sonate en \kF mineur, \Opus{14}~; Sonate en \kG
 mineur, \Opus{22}~; \emph{Kreisleriana}, \Opus{16}~; Carnaval, \Opus{9}.
 \item[\DateWithWeekDay{1940-02-24}]
 Leningrad~: petite salle de la société philharmonique.
 Concert évoqué par \VSofronitsky{} dans une lettre du~1\ier{} mars~1940 à
 sa famille \citep[p.~28]{Kogan08}.

 \textsc{\Schumann{}}~: Fantaisie en \kC majeur, \Opus{17}.
 \textsc{\Chopin{}}~: Dix Études.
 \textsc{\Liszt{}}~: Sonate en \kB mineur, S~178.
 \item[\DateWithWeekDay{1940-03-22}]
 Leningrad~: grande salle de la société philharmonique.
 Concert pour commémorer le~25\ieme{} anniversaire de la mort de
 \Scriabine{}.

 \textsc{\Scriabine{}}~: Huit Préludes extraits des \Opus{22, 27, 35
 et~37}~; Sonate en \kF \Sharp mineur, \Opus{23}~; Dix Études extraites de
 l'\Opus{8}~; Deux Poèmes, \Opus{32}~; Un Prélude extrait de l'\Opus{48}~;
 Rêverie en \kC majeur, \Opus{49} \Number{3}~; Ironies en \kC majeur,
 \Opus{56} \Number{2}~; Nuances, \Opus{56} \Number{3}~; Sonate en \kF \Sharp
 majeur, \Opus{30}~; Sonate, \Opus{70}~; Poème satanique, \Opus{36}.
 \item[\DateWithWeekDay{1940-03-25}]
 Moskva~: grande salle du conservatoire.
 Concert philharmonique.
 Quatrième concert d'un cycle de six concerts de \VSofronitsky{}.
 Voir \citet{Lazarev20}.

 \textsc{\Scriabine{}}~: Huit Préludes extraits des \Opus{22, 27, 35
 et~37}~; Sonate en \kF \Sharp mineur, \Opus{23}~; Dix Études extraites de
 l'\Opus{8}~; Deux Poèmes, \Opus{32}~; Un Prélude extrait de l'\Opus{48}~;
 Rêverie en \kC majeur, \Opus{49} \Number{3}~; Prélude~; Ironies en \kC
 majeur, \Opus{56} \Number{2}~; Nuances, \Opus{56} \Number{3}~; Danse
 languide, \Opus{51} \Number{4}~; Sonate en \kF \Sharp majeur, \Opus{30}~;
 Sonate, \Opus{70}~; Poème satanique, \Opus{36}.
 \item[\DateWithWeekDay{1940-03-27}]
 Moskva~: grande salle du conservatoire.
 Quatrième concert d'un cycle de cinq concerts pour le~25\ieme{}
 anniversaire de la mort de \Scriabine{}.
 Reprise du programme des concerts du~22 et du~25~mars.
 Voir \citet{Lazarev20}.
 \item[\DateWithWeekDay{1940-04-06}]
 Leningrad~: petite salle du conservatoire.
 Concert.
 Reprise du programme des concerts du~22 et du~25~mars.
 \item[\DateWithWeekDay{1940-04-22}]
 Leningrad~: grande salle de la société philharmonique.

 \textsc{\Schubert{}}~: Quatre Impromptus.
 \textsc{\Schumann{}}~: \emph{Humoreske} en \kB \Flat majeur, \Opus{20}.
 \textsc{\Chopin{}}~: Un Nocturne extrait de l'\Opus{15}~; Impromptu en \kG
 \Flat majeur, \Opus{51}~; Une Mazurka extraite de l'\Opus{30}~; Barcarolle
 en \kF \Sharp majeur, \Opus{60}~; Un Scherzo \citep[selon][]{Nekrasova08}.
 \textsc{\Rachmaninov{}}~: Deux Préludes.
 \textsc{\Liszt{}}~: \emph{Sonetto~104 del Petrarca}, S~161 \Number{5}~;
 \emph{Sonetto~123 del Petrarca}, S~161 \Number{6}~; Méphisto-valse.
 \item[\DateWithWeekDay{1940-04-27}]
 Moskva~: musée \Scriabine{}.
 Concert à la mémoire de \Scriabine{}.
 Premier concert mentionné de \VSofronitsky{} au musée \Scriabine{}, après
 un concert collectif le~8 mai~1926.
 \item[\DateWithWeekDay{1940-04-29}]
 Leningrad.
 Concert pour la radio.
 Concert diffusé en direct~; début à~20\up{h}~30\up{m}.

 \textsc{\Scriabine{}}~: Préludes~; Deux Études~; Sonate en \kF \Sharp
 majeur, \Opus{30}.
 \item[\DateWithWeekDay{1940-05-03}]
 Moskva~: grande salle du conservatoire.
 Concert philharmonique.
 Cinquième concert d'un cycle de six concerts de \VSofronitsky{}.

 \textsc{\Schubert{}}~: Quatre Impromptus.
 \textsc{\Schumann{}}~: \emph{Humoreske} en \kB \Flat majeur, \Opus{20}.
 \textsc{\Chopin{}}~: Nocturne en \kF \Sharp majeur, \Opus{15} \Number{2}~;
 Barcarolle en \kF \Sharp majeur, \Opus{60}~; Une Mazurka extraite de
 l'\Opus{30}~; Scherzo.
 \textsc{\Rachmaninov{}}~: Deux Préludes.
 \textsc{\Liszt{}}~: \emph{Sonetto~104 del Petrarca}, S~161 \Number{5}~;
 Méphisto-valse.
 \emph{Bis} -- \textsc{\Liszt{}}~: Étude~; \emph{Tarantella} extraite de
 \emph{Venezia e Napoli}, S~162 \Number{3}.
 \item[\DateWithWeekDay{1940-05-09}]
 Moskva~: musée polytechnique.
 Concert.
 Reprise du programme du~3~mai.
 \item[\DateWithWeekDay{1940-05-19}]
 Moskva~: grande salle du conservatoire.
 Concert pour le festival d'art de Leningrad.

 \textsc{\Schumann{}}~: Études symphoniques, \Opus{13}.
 \textsc{\Chopin{}}~: Ballade en \kG mineur, \Opus{23}~; Deux Mazurkas~;
 Scherzo en \kB mineur, \Opus{20}.
 \textsc{\Prokofiev{}}~: Sonate en \kA mineur, \Opus{28}.
 \textsc{\Goltz{}}~: Deux Préludes~; Scherzo en \kE mineur.
 \textsc{\Liszt{}}~: \emph{Sonetto~104 del Petrarca}, S~161 \Number{5}~;
 Feux follets, S~139 \Number{5}.
 \textsc{\Scriabine{}}~: Deux Études~; Sonate, \Opus{53}.
 \item[\DateWithWeekDay{1940-05-22}]
 Moskva~: grande salle du conservatoire.
 Concert philharmonique, abonnement \Number{13}, quatrième et dernier
 concert.

 \textsc{\JBach{}/\Busoni{}}~: Deux Préludes de chorals en \kG mineur et en
 \kG majeur.
 \textsc{\Beethoven{}}~: Sonate en \kF mineur, \Opus{57}~; Sonate en \kC
 mineur, \Opus{111}.
 \textsc{\Chopin{}}~: Vingt-quatre Préludes, \Opus{28}.
 \item[\DateWithWeekDay{1940-05-23}]
 Moskva~: maison des architectes.
 Concert avec des œuvres de \Schubert{}, \Mendelssohn{} et \Schumann{}.
 Selon \citet[p.~162]{Nekrasova08}, l'œuvre de \Mendelssohn{} était les
 Variations sérieuses en \kD mineur, \Opus{54}.
 \item[\DateWithWeekDay{1940-05-26}]
 Moskva~: théâtre Bol'šoj.
 Concert final du festival d'art de Leningrad.

 \textsc{\Scriabine{}}~: Étude en \kD \Sharp mineur, \Opus{8} \Number{12}.
 \item[\DateWithWeekDay{1940-05-29}]
 Moskva~: Club des écrivains.
 Concert (début à~20\up{h}~30\up{m}).
 \item[\DateWithWeekDay{1940-05-30}]
 Moskva.
 Reprise du concert final du festival d'art de Leningrad.
 \item[\DateWithWeekDay{1940-06-05}]
 Leningrad.
 Concert pour la radio.

 \textsc{\Schumann{}}~: Études symphoniques, \Opus{13}.
 \item[\DateWithWeekDay{1940-06-07}]
 Leningrad.
 Concert pour la radio.

 \textsc{\Schumann{}}~: Fantaisie en \kC majeur, \Opus{17}.
 \item[\DateWithWeekDay{1940-06-17}]
 Leningrad.
 Concert pour la radio.

 \textsc{\Schumann{}}~: Carnaval, \Opus{9}.
 \item[\DateWithWeekDay{1940-06-20}]
 Leningrad.
 Concert pour la radio.

 \textsc{\Schumann{}}~: \emph{Kreisleriana}, \Opus{16}.
 \item[B1940-06]
 Dans la revue \foreignlanguage{russian}{\emph{Искусство и жизнь}}
 [\emph{Iskusstvo i žizn'}], publication de la recension intitulée
 \Quote{\Sofronitsky{} joue \Scriabine{}}.
 \item[\DateWithWeekDay{1940-07-31}]
 Leningrad.
 Concert pour la radio.

 \textsc{\Scriabine{}}~: Poème~; Prélude~; Mazurka~; Étude.
 \textsc{\Chopin{}}~: Nocturne.
 \textsc{\Schumann{}}~: \emph{Traumes Wirren}, \Opus{12} \Number{7}.
 \textsc{\Prokofiev{}}~: Sarcasmes, \Opus{17}.
 \textsc{\Liszt{}}~: Feux follets, S~139 \Number{5}.
 \textsc{\Goltz{}}~: Scherzo en \kE mineur.
 \item[\DateWithWeekDay{1940-09-18}]
 Leningrad.
 Concert pour la radio à partir de~18\up{h}~00\up{m}.

 \textsc{\Schubert{}}~: Quatre Impromptus.
 \item[\DateWithWeekDay{1940-09-24}]
 Leningrad.
 Concert pour la radio à partir de~13\up{h}~20\up{m}.

 \textsc{\Scriabine{}}~: Mazurka~; Étude.
 \item[\DateWithWeekDay{1940-10-05}]
 Leningrad.
 Concert pour la radio.

 \textsc{\Chopin{}}~: Huit Mazurkas.
 \item[\DateWithWeekDay{1940-10-12}]
 Leningrad.
 Concert pour la radio durant l'après-midi.

 \textsc{\Scriabine{}}~: Préludes~; Mazurkas~; Études.
 \item[\DateWithWeekDay{1940-10-12}]
 Leningrad~: grande salle de la société philharmonique.
 Concert%
 \footnote{Pour l'affiche et le programme de ce concert, voir
 \href{https://100philharmonia.spb.ru/historical-poster/15169/}%
 {https://100philharmonia.spb.ru/historical-poster/15169/}.}
 de pianistes issus de la classe de \LNikolaiev{}.
 Ce concert a eu lieu le même jour que le concert précédent.

 \textsc{\Chopin{}}~: Scherzo~; Deux Mazurkas~; Polonaise.
 \item[\DateWithWeekDay{1940-10-30}]
 Leningrad~: grande salle de la société philharmonique.
 Concert évoqué, de même que le suivant à Moskva, dans deux lettres de
 \VSofronitsky{} à sa famille, le~16 octobre puis le~25 novembre~1940
 \citep[p.~29 et~30]{Kogan08}.

 \textsc{\Mendelssohn{}}~: Variations sérieuses en \kD mineur, \Opus{54}.
 \textsc{\Schumann{}}~: Sonate en \kF mineur, \Opus{14}~; Toccata,
 \Opus{7}~; Arabesque en \kC majeur, \Opus{18}.
 \textsc{\Ravel{}}~: Sonatine.
 \textsc{\Scriabine{}}~: Poème, \Opus{59} \Number{1}~; Masque, \Opus{63}
 \Number{1}~; Prélude en \kD \Flat majeur, \Opus{17} \Number{3}~;
 Guirlandes, \Opus{73} \Number{1}~; Flammes sombres, \Opus{73} \Number{2}~;
 Sonate, \Opus{70}.
 \textsc{\Rachmaninov{}}~: Prélude en \kC \Sharp mineur, \Opus{3}
 \Number{2}.
 \textsc{\Prokofiev{}}~: Quatre pièces extraites de la Suite \emph{Roméo et
 Juliette}, \Opus{75}.
 \textsc{\Liszt{}}~: Valse oubliée.
 \textsc{\Chopin{}}~: Prélude en \kC \Sharp mineur, \Opus{28} \Number{10}~;
 Étude \Number{4}.
 \textsc{\Debussy{}}~: Canope, L~123 \Number{X}~; Feux d'artifice, L~123
 \Number{XII}~; \emph{General Lavine -- eccentric}, L~123 \Number{VI}.
 \item[\DateWithWeekDay{1940-11-02}]
 Moskva~: grande salle du conservatoire.
 Concert philharmonique, abonnement \Number{13}, premier concert.

 \textsc{\Mendelssohn{}}~: Variations sérieuses en \kD mineur, \Opus{54}.
 \textsc{\Schumann{}}~: Arabesque en \kC majeur, \Opus{18}~; Sonate en \kF
 mineur, \Opus{14}.
 \textsc{\Ravel{}}~: Sonatine.
 \textsc{\Scriabine{}}~: Poème, \Opus{59} \Number{1}~; Guirlandes, \Opus{73}
 \Number{1}~; Sonate, \Opus{70}.
 \textsc{\Prokofiev{}}~: Quatre pièces extraites de la Suite \emph{Roméo et
 Juliette}, \Opus{75}.
 \textsc{\Debussy{}}~: Canope, L~123 \Number{X}~; \emph{General Lavine --
 eccentric}, L~123 \Number{VI}~; Feux d'artifice, L~123 \Number{XII}.
 \emph{Bis} -- \textsc{\Liszt{}}~: Valse oubliée \Number{1}, S~215
 \Number{1}.
 \textsc{\Rachmaninov{}}~: Prélude en \kC \Sharp mineur, \Opus{3}
 \Number{2}.
 \textsc{\Chopin{}}~: Étude.
 \textsc{\Scriabine{}}~: Prélude en \kC \Sharp mineur, \Opus{11}
 \Number{10}~; Étude en \kD \Flat majeur, \Opus{8} \Number{10}.
 \item[\DateWithWeekDay{1940-11-24}]
 Leningrad~: grande salle de la société philharmonique.
 Concert et programme consignés dans les archives d'\AVizel{} et mentionnés
 par \citet[p.~162]{Nekrasova08}.

 \textsc{\Schubert{}}~: Impromptu.
 \textsc{\Schumann{}}~: Arabesque en \kC majeur, \Opus{18}~;
 \emph{Kreisleriana}, \Opus{16}.
 \textsc{\Medtner{}}~: \emph{Skazka} \Quote{fantastique}~(?) \Number{3}.
 \textsc{\Scriabine{}}~: Sonate \Number{5}, \Opus{53}.
 \item[\DateWithWeekDay{1940-11-25}]
 Leningrad~: petite salle du conservatoire.
 Concert.
 \item[\DateWithWeekDay{1940-11-29}]
 Leningrad~: conservatoire.
 Concert reporté le~13 décembre.
 Concert privé avec des œuvres de \Prokofiev{}.
 \item[\DateWithWeekDay{1940-12-02}]
 Moskva~: grande salle du conservatoire.
 Concert philharmonique, abonnement \Number{3}, troisième
 \citep[p.~417]{Scriabine} ou premier \citep[p.~162]{Nekrasova08} concert.
 Concert consacré à des œuvres de \Liszt{}~; \citeauthor{Nekrasova08}
 indique en outre la \emph{Canzonetta del Salvator Rosa}, S~161 \Number{3}.

 \textsc{\Liszt{}}~: Funérailles, S~173 \Number{7}~; Chapelle de Guillaume
 Tell, S~160 \Number{1}~; \emph{Sposalizio}, S~161 \Number{1}~; \emph{Il
 penseroso}, S~161 \Number{2}~; Après une lecture de Dante, S~161
 \Number{7}~; Au Lac de Wallenstadt, S~160 \Number{2}~; \emph{Sonetto~104
 del Petrarca}, S~161 \Number{5}~; \emph{Sonetto~123 del Petrarca}, S~161
 \Number{6}~; Sonate en \kB mineur, S~178.
 \emph{Bis} -- \textsc{\Liszt{}}~: Valse oubliée \Number{1}, S~215
 \Number{1}~; Feux follets, S~139 \Number{5}.
 \textsc{\Liszt{}}~: \emph{Waldesrauschen}, S~145 \Number{1}.
 \item[\DateWithWeekDay{1940-12-07}]
 Leningrad~: grande salle de la société philharmonique.
 Apparition lors d'un concert symphonique dirigé par \IMiklashevsky{}.
 Voir \citet{Lazarev20}.
 Lors de ce même concert, Aleksandr Danilovič \Kamensky{} a joué, toujours
 sous la direction d'\IMiklashevsky{}, la partie de piano de
 \emph{Prométhée}, \Opus{60} \citep[voir aussi][p.~163]{Nekrasova08}.
 Concert évoqué par \VSofronitsky{} dans une lettre du~16 octobre~1940 à sa
 famille \citep[p.~29]{Kogan08}, mais il y parle d'un concert à Moskva.

 \textsc{\Scriabine{}}~: Concerto en \kF \Sharp mineur, \Opus{20}.
 \item[\DateWithWeekDay{1940-12-13}]
 Première moitié d'un concert à l'\hbox{Union} des compositeurs, pour les
 étudiants du conservatoire.
 Ensuite~: Quintette de \DChostakovitch{}, \Opus{57}, en présence du
 compositeur.
 Concert évoqué par \VSofronitsky{} dans une lettre du~16 décembre~1940 à sa
 famille \citep[p.~30]{Kogan08}.
 \item[\DateWithWeekDay{1940-12-21}]
 Leningrad~: grande salle de la société philharmonique.

 \textsc{\Schubert{}}~: Quatre Impromptus.
 \textsc{\Beethoven{}}~: Sonate en \kC \Sharp mineur, \Opus{27} \Number{2}.
 \textsc{\Liszt{}}~: Au Lac de Wallenstadt, S~160 \Number{2}~;
 \emph{Waldesrauschen}, S~145 \Number{1}~; Feux follets, S~139 \Number{5}~;
 Sonate en \kB mineur, S~178.
 \textsc{\Chopin{}}~: Nocturne~; Mazurka.
 \textsc{\Prokofiev{}}~: Prélude~; Suite \emph{Roméo et Juliette},
 \Opus{75}.
 \item[\DateWithWeekDay{1940-12-26}]
 Leningrad.
 Concert pour la radio.

 \textsc{\Beethoven{}}~: Sonate en \kC \Sharp mineur, \Opus{27} \Number{2}.
 \textsc{\Liszt{}}~: \emph{Sposalizio}, S~161 \Number{1}~; \emph{Il
 penseroso}, S~161 \Number{2}~; \emph{Canzonetta del Salvator Rosa}, S~161
 \Number{3}.
\end{description}

\section{Année~1941}

\begin{description}
 \item[\DateWithWeekDay{1941-01-06}]
 Moskva~: grande salle du conservatoire.
 Abonnement \Number{13} (troisième concert).

 \textsc{\Beethoven{}}~: Sonate en \kC \Sharp mineur, \Opus{27} \Number{2}.
 \textsc{\Chopin{}}~: Sonate en \kB \Flat mineur, \Opus{35}.
 \textsc{\Medtner{}}~: Deux \emph{Skazki} en \kF mineur, \Opus{26}
 \Number{3}, et en \kE \Flat majeur extrait de l'\Opus{26}~;
 \emph{Novelette} en \kC mineur, \Opus{17} \Number{2}~; \emph{Skazka} en \kE
 mineur, \Opus{14} \Number{2}~; \emph{Skazka} en \kB mineur, \Opus{20}
 \Number{2}.
 \textsc{\Prokofiev{}}~: Contes de la vieille grand-mère, \Opus{31}~; Deux
 Pièces extraites de la Suite \emph{Roméo et Juliette}, \Opus{75}.
 \textsc{\Scriabine{}}~: Sonate, \Opus{53}.
 \emph{Bis} -- \textsc{\Scriabine{}}~: Prélude en \kG mineur, \Opus{27}
 \Number{1}.
 \textsc{\Rachmaninov{}}~: Préludes en \kG majeur, \Opus{32} \Number{5}~;
 Prélude en \kG \Sharp mineur, \Opus{32} \Number{12}.
 \textsc{\Prokofiev{}}~: Sarcasme, \Opus{17} \Number{5}.
 \item[\DateWithWeekDay{1941-01-10}]
 Moskva~: salle \Tchaikovski{}.
 Concert symphonique placé sous la direction de \IMiklashevsky{}.
 Même programme que le~7 décembre~1940.

 \textsc{\Scriabine{}}~: Concerto en \kF \Sharp mineur, \Opus{20}.
 \emph{Bis} -- \textsc{\Scriabine{}}~: Deux Études.
 \item[\DateWithWeekDay{1941-01-11}]
 Moskva~: Commissariat du peuple aux Affaires étrangères.

 \textsc{\Schumann{}}~: Études symphoniques, \Opus{13}~; Carnaval, \Opus{9}.
 \textsc{\Chopin{}}~: Mazurka.
 \textsc{\Scriabine{}}~: Deux Études.
 \item[\DateWithWeekDay{1941-01-12}]
 Moskva~: grand auditoire de l'institut polytechnique.
 Le programme ci-dessous est mentionné par \citet[p.~417]{Scriabine}, mais
 \citet[p.~163]{Nekrasova08} indique plutôt une reprise du programme du
 concert du~10 janvier~1941 à la salle \Tchaikovski{}.

 \textsc{\Schubert{}}~: Deux Impromptus.
 \textsc{\Schumann{}}~: Études symphoniques, \Opus{13}.
 \textsc{\Beethoven{}}~: Sonate en \kC \Sharp mineur, \Opus{27} \Number{2}.
 \textsc{\Chopin{}}~: Sonate en \kB \Flat mineur, \Opus{35}.
 \emph{Bis} -- \textsc{\Chopin{}}~: Deux Mazurkas.
 \textsc{\Medtner{}}~: \emph{Skazka} en \kF mineur, \Opus{26} \Number{3}.
 \textsc{\Scriabine{}}.
 \textsc{\Schumann{}}.
 \item[\DateWithWeekDay{1941-01-14}]
 Moskva~: maison des écrivains.
 Concert à partir de~20\up{h}~30\up{m}.
 \item[\DateWithWeekDay{1941-02-04}]
 Leningrad~: grande salle de la société philharmonique.
 Concert symphonique placé sous la direction de \IMiklashevsky{}.

 \textsc{\Scriabine{}}~: Concerto en \kF \Sharp mineur, \Opus{20}.
 \item[\DateWithWeekDay{1941-02-20}]
 Moskva~: grande salle du conservatoire.

 \textsc{\Schumann{}}~: Sonate en \kF mineur, \Opus{14}~; Sonate en \kG
 mineur, \Opus{22}~; \emph{Kreisleriana}, \Opus{16}~; Carnaval, \Opus{9}~;
 \emph{Novelette}~; \emph{Bunte Blätter} (œuvre incertaine).
 \item[\DateWithWeekDay{1941-03-01}]
 Leningrad~: grande salle de la société philharmonique.
 Même programme \Schumann{} que le~20 février~1941.
 \item[\DateWithWeekDay{1941-03-03}]
 Leningrad.
 Concert avec des œuvres de \Schumann{}.
 Selon \ASofronitsky{}, il pourrait s'agir de la reprise du programme du
 concert du~1\ier{}~mars.
 \item[\DateWithWeekDay{1941-03-24}]
 Moskva~: grande salle du conservatoire.
 \citet[p.~163]{Nekrasova08} souligne une construction de programme très
 caractéristique de \Sofronitsky{}, avec la symétrie et le contraste des
 formes classiques de la variation et de la fantaisie, et de leurs formes
 romantiques.

 \textsc{\Haendel{}}~: Variations en \kB \Flat majeur.
 \textsc{\Mozart{}}~: Fantaisie en \kC mineur.
 \textsc{\Schumann{}}~: Variations en \kF mineur~; Fantaisie en \kC majeur,
 \Opus{17}.
 \textsc{\Chopin{}}~: Ballade en \kF mineur, \Opus{52}~; Ballade en \kA
 \Flat majeur, \Opus{47}~; Barcarolle en \kF \Sharp majeur, \Opus{60}~; Deux
 Valses en \kG \Flat majeur, \Opus{70} \Number{1}, et en \kD \Flat majeur~;
 Polonaise en \kA \Flat majeur, \Opus{53}.
 \emph{Bis} -- \textsc{\Chopin{}}~: Valse en \kA \Flat majeur, \Opus{69}
 \Number{1}~; Tarentelle en \kA \Flat majeur, \Opus{43}~; Deux Études en \kF
 majeur et en \kC \Sharp mineur.
 \item[\DateWithWeekDay{1941-04-03}]
 Leningrad~: grande salle de la société philharmonique.

 \textsc{\Haendel{}}~: Variations.
 \textsc{\Mozart{}}~: Fantaisie en \kC mineur.
 \textsc{\Schumann{}}~: Fantaisie en \kC majeur, \Opus{17}.
 \textsc{\Chopin{}}~: Ballade en \kF mineur, \Opus{52}~; Ballade en \kA
 \Flat majeur, \Opus{47}~; Barcarolle en \kF \Sharp majeur, \Opus{60}~;
 Tarentelle en \kA \Flat majeur, \Opus{43}~; Deux Valses~; Polonaise en \kA
 \Flat majeur, \Opus{53}.
 \item[\DateWithWeekDay{1941-04-11}]
 Moskva~: petite salle du conservatoire.
 Concert avec des œuvres de \Schumann{}.
 Concert mentionné par \citet[p.~163]{Nekrasova08}.
 \item[\DateWithWeekDay{1941-04-12}]
 Leningrad~: conservatoire.
 Concert privé.

 \textsc{\Schumann{}}~: Fantaisie en \kC majeur, \Opus{17}~; Variations.
 \item[\DateWithWeekDay{1941-04-25}]
 Moskva~: grande salle de la maison des acteurs.
 \citet{Lazarev20} présente un billet d'invitation à la session scientifique
 et au concert consacrés aux œuvres de \Scriabine{}.
 \item[B1941-04 (fin) -- 1941-05 (début)]
 Selon \ASofronitsky{}, \VSofronitsky{} malade.
 \item[\DateWithWeekDay{1941-05-22}]
 Moskva~: grande salle du conservatoire.
 Concert philharmonique.

 \textsc{\JBach{}/\Busoni{}}~: Deux Préludes de chorals en \kG mineur et en
 \kG majeur.
 \textsc{\Beethoven{}}~: Sonate en \kF mineur, \Opus{57}~; Sonate en \kC
 mineur, \Opus{111}.
 \textsc{\Chopin{}}~: Vingt-quatre Préludes, \Opus{28}.
 \emph{Bis} -- \textsc{\JBach{}/\Busoni{}}~: Prélude de choral en \kG
 majeur.
 \item[\DateWithWeekDay{1941-05-25}]
 Moskva~: grande salle du conservatoire.
 Dernier concert à Moskva avant la Deuxième Guerre mondiale.

 \textsc{\JBach{}}~: Préludes et fugues en \kG mineur et en \kG majeur.
 \textsc{\Beethoven{}}~: Sonate en \kF mineur, \Opus{57}~; Sonate en \kC
 mineur, \Opus{111}.
 \textsc{\Chopin{}}~: Vingt-quatre Préludes, \Opus{28}.
 \item[\DateWithWeekDay{1941-05-26}]
 Moskva~: théâtre Bol'šoj.
 Concert final de la Décennie des arts de Leningrad, mentionné par
 \citet[p.~163]{Nekrasova08}.

 \textsc{\Scriabine{}}~: Étude en \kD \Sharp mineur, \Opus{8} \Number{12}.
 \item[\DateWithWeekDay{1941-05-28}]
 Leningrad~: grande salle de la société philharmonique.
 Dernier concert avant la guerre.
 Reprise du programme du concert du~25~mai à Moskva.
 \item[B1941-06 (début)]
 Selon \ASofronitsky{}, \VSofronitsky{} se trouve à Kiïv.
 Aucune information à propos de concerts.
 \item[B\DateWithWeekDay{1941-06-22}]
 Début de la Grande Guerre patriotique (ou front de l'\hbox{Est} de la
 Deuxième Guerre mondiale), suite à l'opération d'invasion Barbarossa du
 \textsc{iii}\ieme{}~Reich en Union soviétique.
 \VSofronitsky{} a donné plus de quarante concerts en solo durant les années
 de guerre, de l'automne~1941 au printemps~1945 \citep[p.~167]{Nekrasova08}.
 \item[B1941-07 (fin)]
 \VSofronitsky{} réside au conservatoire de Leningrad.
 Le~29 juillet~1941, \DChostakovitch{} et \VSofronitsky{} sont photographiés
 sur les toits du conservatoire, en tenue de pompiers, pour les journaux.
 Lettres du pianiste à sa famille les~27 et~29 juillet~1941
 \citep[p.~31]{Kogan08}.
 \item[B1941 (automne)]
 Début du siège de Leningrad.
 \item[B\DateWithWeekDay{1941-09-01}]
 Soirée au domicile de \VBogdanovBerezovsky{}.
 \IMiklashevsky{} et d'autres musiciens sont présents, dont le compositeur
 Vladimir Mixajlovič Deševov \citep[voir][]{Bogdanov67a}.
 Dans une lettre à son épouse, \VSofronitsky{} dit~: \Quote{Je joue beaucoup
 et bien...}
 Leonid Abramovič Portov, cité par \citet[p.~202-203]{Sofronitsky13b},
 évoque le programme suivant, joué par \Sofronitsky{} (\BogdanovBerezovsky{}
 mentionne plutôt des pièces de \Prokofiev{}, \Scriabine{}, \Schumann{} et
 \Debussy{}).

 \textsc{\Schumann{}}~: Arabesque en \kC majeur, \Opus{18}~;
 \emph{Kreisleriana}, \Opus{16}.
 \textsc{\Scriabine{}}~: Un Prélude extrait de l'\Opus{33}~; Prélude en \kG
 mineur, \Opus{27} \Number{1} (\emph{Patetico}).
 \item[B1941-09]
 Leningrad.
 \VSofronitsky{} s'installe à l'étage inférieur du théâtre \Pouchkine{}, où
 vivent de nombreux artistes qui n'étaient pas encore partis, en particulier
 \DShafran{} et \DChostakovitch{}.
 Voir \citet[p.~58]{Juban}.
 \citet[p.~165]{Nekrasova08} indique le mois d'octobre, plutôt que celui de
 septembre.
 \item[\DateWithWeekDay{1941-11-07}]
 Leningrad~: grande salle de la société philharmonique.
 Concert interrompu en raison de bombardements.
 Descente du public et du musicien à l'abri.
 Ensuite, couvre-feu prolongé puis reprise du concert
 \citep[voir][p.~164]{Nekrasova08}.
 \citet[p.~342]{Savkevich}, citée par \citet[p.~203]{Sofronitsky13b},
 précise que le concert a été interrompu au milieu d'\hbox{Études} de
 \Chopin{}.
 \item[\DateWithWeekDay{1941-12-12}]
 Leningrad~: salle du théâtre \Pouchkine{}.
 Concert pour les défenseurs de Leningrad assiégée.
 Il s'agit du concert évoqué par \citet{Sofronitsky42, Sofronitsky61}
 lui-même.
\end{description}

\section{Année~1942}

\begin{description}
 \item[B1942 (hiver)]
 Siège de Leningrad.
 \VSofronitsky{} admis à l'hôpital, situé dans les murs de l'hôtel Astoria,
 où l'on traitait les victimes de la famine et du froid.
 Sans donner de date précise, Jurij Lazarevič Aljanskij, cité par
 \citet[p.~204]{Sofronitsky13b}, indique que \VSofronitsky{} jouait parfois
 du piano à l'hôtel Astoria, pendant une dizaine ou une quinzaine de minutes
 tout au plus -- il jouait \Scriabine{} ou \Rachmaninov{}, \Liszt{} ou
 \Chopin{}, dans la pénombre~; les patients alités l'y écoutaient.
 Témoignage confirmé par celui d'\hbox{Andrej} Nikolaevič Krjukov
 \citep[voir][p.~204]{Sofronitsky13b}.
 \item[B\DateWithWeekDay{1942-03-02}]
 Décès du père de \VSofronitsky{} à Leningrad.
 \item[B\DateWithWeekDay{1942-03-31}]
 \VSofronitsky{} reçoit de Boris Ivanovič Zagurskij (chef du département
 artistique du comité exécutif de la ville de Leningrad) l'annonce de son
 évacuation imminente, \emph{a~priori} par route et chemin de fer, prévue
 dans les tout premiers jours d'avril \citep[p.~165]{Nekrasova08}.
 Adieu de \VSofronitsky{} à sa grande amie \EDaugovet{}
 (\Dates{1882}{1942}), pianiste et professeur au conservatoire de Leningrad,
 trop affaiblie pour pouvoir être sauvée \citep[p.~165-166]{Nekrasova08}.
 \item[B\DateWithWeekDay{1942-04-08}]
 \VSofronitsky{}, évacué par pont aérien de Leningrad assiégée, arrive à
 Moskva.
 \item[\DateWithWeekDay{1942-04-26}]
 Moskva~: salle \Tchaikovski{}.
 Premier concert de \VSofronitsky{} à Moskva après son évacuation de
 Leningrad.
 Voir \citet[p.~301]{Zhukova82} et \citet[p.~216]{Zhukova08}.

 \textsc{\JBach{}/\Busoni{}}~: Deux Préludes de chorals en \kG mineur et en
 \kG majeur.
 \textsc{\Beethoven{}}~: Sonate en \kF mineur, \Opus{57}.
 \textsc{\Chopin{}}~: Fantaisie en \kF mineur, \Opus{49}~; Ballade en \kG
 mineur, \Opus{23}~; Quatre Mazurkas~; Deux Valses en \kD \Flat majeur et en
 \kG \Flat majeur~; Scherzo en \kB mineur, \Opus{20}~; Polonaise en \kA
 \Flat majeur, \Opus{53}.
 \emph{Bis} -- \textsc{\Chopin{}}~: Deux Mazurkas en \kC \Sharp mineur et en
 \kD \Flat majeur~; Étude en \kC \Sharp mineur.
 \textsc{\Scriabine{}}~: Étude en \kD \Flat majeur, \Opus{8} \Number{10}.
 \item[\DateWithWeekDay{1942-05-03}]
 Moskva~: maison des architectes.
 Deuxième concert de \VSofronitsky{} à Moskva après son évacuation de
 Leningrad~; deuxième programme.

 \textsc{\Scriabine{}}~: Six Préludes extraits des \Opus{22, 33, 35 et~37}~;
 Sonate en \kF \Sharp mineur, \Opus{23}~; Six Études extraites de
 l'\Opus{8}~: en \kC \Sharp majeur (\Number{1}), en \kF \Sharp mineur
 (\Number{2}), en \kE majeur (\Number{5}), en \kA \Flat majeur (\Number{8}),
 en \kG \Sharp mineur (\Number{9}) et en \kB majeur (\Number{4})~; Deux
 Poèmes, \Opus{32}~; Deux Préludes, \Opus{27}~; Poème, \Opus{59}
 \Number{1}~; Prélude en \kE majeur et Nuances, \Opus{56} \Number{1} et
 \Number{3}~; Désir, \Opus{57} \Number{1}~; Poème satanique, \Opus{36}.
 \emph{Bis} -- \textsc{\Scriabine{}}~: Deux Études extraites de l'\Opus{8}~:
 en \kB \Flat mineur et en \kD \Flat majeur (\Number{10})~; Prélude en \kE
 \Flat mineur, \Opus{16} \Number{4}~; Étude en \kD \Sharp mineur, \Opus{8}
 \Number{12}.
 \item[\DateWithWeekDay{1942-05-10}]
 Moskva~: salle \Tchaikovski{}.
 Troisième concert de \VSofronitsky{} à Moskva après son évacuation de
 Leningrad~; troisième programme différent.

 \textsc{\Schumann{}}~: Études symphoniques, \Opus{13}.
 \textsc{\Chopin{}}~: Polonaise en \kC \Sharp mineur, \Opus{26} \Number{1}~;
 Deux Mazurkas~; Ballade en \kA \Flat majeur, \Opus{47}.
 \textsc{\Scriabine{}}~: Cinq Études extraites de l'\Opus{8}~: en \kC \Sharp
 majeur (\Number{1}), en \kF \Sharp mineur (\Number{2}), en \kB majeur
 (\Number{4}), en \kG \Sharp mineur (\Number{9}) et en \kB \Flat mineur~;
 Sonate en \kF \Sharp majeur, \Opus{30}.
 \textsc{\Rachmaninov{}}~: Deux Préludes en \kG majeur, \Opus{32}
 \Number{5}, et en \kG \Sharp mineur, \Opus{32} \Number{12}.
 \textsc{\Debussy{}}~: \emph{Serenade of the Doll}, L~113 \Number{III}~;
 \emph{General Lavine -- eccentric}, L~123 \Number{VI}~; Canope, L~123
 \Number{X}~; Feux d'artifice, L~123 \Number{XII}.
 \emph{Bis} -- \textsc{\Chopin{}}~: Deux Mazurkas en \kC \Sharp mineur et en
 \kD \Flat majeur~; Deux Valses en \kG \Flat majeur et en \kD \Flat majeur.
 \textsc{\Scriabine{}}~: Étude en \kD \Flat majeur, \Opus{8} \Number{10}.
 \textsc{\Prokofiev{}}~: Un Sarcasme extrait de l'\Opus{17}.
 \textsc{\Scriabine{}}~: Poème en \kD majeur, \Opus{32} \Number{2}.
 \item[\DateWithWeekDay{1942-05-23}]
 Moskva~: grande salle du conservatoire.
 Concert philharmonique.

 \textsc{\Mendelssohn{}}~: Variations sérieuses en \kD mineur, \Opus{54}.
 \textsc{\Liszt{}}~: Sonate en \kB mineur, S~178.
 \textsc{\Schumann{}}~: Arabesque en \kC majeur, \Opus{18}.
 \textsc{\Chopin{}}~: Ballade en \kF mineur, \Opus{52}~; Deux Mazurkas en
 \kB mineur et en \kC majeur~; Scherzo en \kB mineur, \Opus{20}.
 \textsc{\Chostakovitch{}}~: Deux Préludes, \Opus{34} \Number{13} et
 \Number{10}, en \kF \Sharp majeur et en \kC \Sharp mineur.
 \textsc{\Prokofiev{}}~: Deux Sarcasmes extraits de l'\Opus{17}, \Number{4}
 et \Number{5}.
 \textsc{\Scriabine{}}~: Sonate, \Opus{53}.
 \emph{Bis} -- \textsc{\Rachmaninov{}}~: Préludes en \kG majeur, \Opus{32}
 \Number{5}, et en \kG \Sharp mineur, \Opus{32} \Number{12}.
 \textsc{\Liszt{}}~: Feux follets, S~139 \Number{5}.
 \textsc{\Chopin{}}~: Mazurka en \kF mineur.
 \item[\DateWithWeekDay{1942-05-31}]
 Moskva~: grande salle du conservatoire.
 Concert philharmonique pour le~27\ieme{} anniversaire de la mort de
 \Scriabine{}.

 \textsc{\Scriabine{}}~: Prélude en \kG \Sharp mineur, \Opus{22}
 \Number{1}~; Deux Préludes extraits de l'\Opus{35}~: \Number{2} en \kB
 \Flat majeur et \Number{1} en \kD \Flat majeur~; Deux Préludes extraits de
 l'\Opus{37}~: \Number{1} en \kB \Flat mineur et \Number{2} en \kF \Sharp
 majeur~; Prélude en \kB mineur, \Opus{22} \Number{4}~; Sonate en \kF \Sharp
 mineur, \Opus{23}~; Six Études extraites de l'\Opus{8}~: en \kC \Sharp
 majeur (\Number{1}), en \kF \Sharp mineur (\Number{2}), en \kB majeur
 (\Number{4}), en \kE majeur (\Number{5}), en \kA \Flat majeur (\Number{8})
 et en \kG \Sharp mineur (\Number{9})~; Deux Poèmes, \Opus{32}~; Poème,
 \Opus{59} \Number{1}~; Poème, \Opus{63} \Number{1}~; Sonate en \kF \Sharp
 majeur, \Opus{30}~; Sonate, \Opus{70}~; Poème satanique, \Opus{36}.
 \emph{Bis} -- \textsc{\Scriabine{}}~: Valse en \kA \Flat majeur,
 \Opus{38}~; Prélude en \kC \Sharp mineur, \Opus{11} \Number{10}~; Deux
 Mazurkas, \Opus{40}.
 \textsc{\Rachmaninov{}}~: Prélude en \kG majeur, \Opus{32} \Number{5}.
 \textsc{\Liszt{}}~: Feux follets, S~139 \Number{5}.
 \item[\DateWithWeekDay{1942-06-05}]
 Moskva.
 Concert.
 Programme inconnu.
 \item[\DateWithWeekDay{1942-06-12}]
 Moskva~: grande salle du conservatoire.
 Concert philharmonique.
 Voir en particulier \citet[p.~442]{Milshteyn82a} et
 \citet[p.~393]{Nikonovich08}.

 \textsc{\Beethoven{}}~: Andante favori en \kF majeur, WoO~57.
 \textsc{\Schumann{}}~: \emph{Kreisleriana}, \Opus{16}.
 \textsc{\Chopin{}}~: Ballade en \kA \Flat majeur, \Opus{47}~; Vingt-quatre
 Préludes, \Opus{28}.
 \textsc{\Liszt{}}~: \emph{Sonetto~123 del Petrarca}, S~161 \Number{6}~;
 Méphisto-valse.
 \emph{Bis} -- \textsc{\Chopin{}}~: Étude en \kE majeur, \Opus{10}
 \Number{3}~; Étude en \kC \Sharp mineur, \Opus{10} \Number{4}.
 \textsc{\Liszt{}}~: Valse oubliée \Number{1}, S~215 \Number{1}.
 \textsc{\Chopin{}}~: Valse en \kG \Flat majeur, \Opus{70} \Number{1}.
 \textsc{\Scriabine{}}~: Poème en \kF \Sharp majeur, \Opus{32} \Number{1}.
 \textsc{\Chopin{}}~: Prélude.
 \item[B\DateWithWeekDay{1942-07-10}]
 La revue \emph{Sovetskaja muzyka}%
 \footnote{\foreignlanguage{russian}{\emph{Советская музыка}}, vol.~499,
 \Number{6} (1980), p.~97.}
 reproduit une lettre qui aurait été écrite et envoyée par \VSofronitsky{}
 au rédacteur du journal \emph{Daily Express}, à Londres~: elle a pour sujet
 le devoir de l'artiste en temps de guerre et de siège de Leningrad.
 Cette lettre et plusieurs autres, écrites par d'autres musiciens russes,
 sont introduites et présentées par \citet{Krasilchtchik80}, rédacteur en
 chef de l'agence de presse \emph{Novosti}.
 \item[\DateWithWeekDay{1942-08-09}]
 Moskva~: salle \Tchaikovski{}.
 Concert philharmonique%
 \footnote{\foreignlanguage{russian}{\emph{Известия}}, 6~août~1942.}.
 Voir en particulier \citet[p.~442]{Milshteyn82a} et
 \citet[p.~393]{Nikonovich08}.

 \textsc{\Schubert{}}~: Quatre Impromptus, D~935 et D~899.
 \textsc{\Schumann{}}~: \emph{Humoreske} en \kB \Flat majeur, \Opus{20}.
 \textsc{\Prokofiev{}}~: Sonate en \kA mineur, \Opus{28}~; Contes de la
 vieille grand-mère, \Opus{31}~; Quatre Pièces de la Suite \emph{Roméo et
 Juliette}, \Opus{75}~: \emph{Montaigu et Capulet}, \emph{Menuet},
 \emph{Frère Laurent} et \emph{Mercutio}.
 \textsc{\Scriabine{}}~: Sonate, \Opus{53}.
 \emph{Bis} -- \textsc{\Scriabine{}}~: Étude en \kE majeur (œuvre
 incertaine)~; Étude en \kB \Flat mineur.
 \textsc{\Rachmaninov{}}~: Prélude en \kC \Sharp mineur, \Opus{3}
 \Number{2}.
 \textsc{\Scriabine{}}~: Étude en \kD \Sharp mineur, \Opus{8} \Number{12}.
 \textsc{\Prokofiev{}}~: \emph{Montaigu et Capulet}.
 \item[\DateWithWeekDay{1942-09-27}]
 Moskva~: grande salle du conservatoire%
 \footnote{\foreignlanguage{russian}{\emph{Известия}}, 27~septembre~1942.}.
 \citet{Lazarev20} présente une recension de ce concert par S.~Semenov%
 \footnote{\foreignlanguage{russian}{\emph{Комсомольская правда}},
 29~septembre~1942.}.

 \textsc{\Glazounov{}}~: Sonate en \kB \Flat mineur, \Opus{74}.
 \textsc{\Rachmaninov{}}~: Deux Moments musicaux~; Deux Préludes~; Deux
 Études-tableaux.
 \textsc{\Prokofiev{}}~: Dix Pièces extraites des Visions fugitives,
 \Opus{22}~; Deux Pièces extraites de la Suite \emph{Roméo et Juliette},
 \Opus{75}~; Un Sarcasme extrait de l'\Opus{17}.
 \textsc{\Scriabine{}}~: Deux Poèmes, \Opus{32}~; Poème, \Opus{59}
 \Number{1}~; Énigme, \Opus{52} \Number{2}~; Flammes sombres, \Opus{73}
 \Number{2}.
 \emph{Bis} -- \textsc{\Rachmaninov{}}~: Prélude en \kC \Sharp mineur,
 \Opus{3} \Number{2}~; Prélude en \kG \Sharp mineur, \Opus{32} \Number{12}~;
 Moment musical~; Étude-tableau.
 \item[\DateWithWeekDay{1942-09-30}]
 Moskva~: maison des écrivains.

 \textsc{\Scriabine{}}~: Sonate en \kF \Sharp mineur, \Opus{23}.
 \textsc{\Rachmaninov{}}~: Deux Moments musicaux en \kB mineur, \Opus{16}
 \Number{3}, et en \kE \Flat mineur, \Opus{16} \Number{2}~; Deux Préludes en
 \kG majeur et en \kG \Sharp mineur, \Opus{32} \Number{5} et \Opus{32}
 \Number{12}~; Étude-tableau.
 \textsc{\Scriabine{}}~: Six Études extraites de l'\Opus{8}~; Deux Poèmes,
 \Opus{32}~; Poème satanique, \Opus{36}.
 \emph{Bis} -- \textsc{\Prokofiev{}}~: \emph{Montaigu et Capulet} extrait de
 l'\Opus{75}~; Cinquième Sarcasme, \Opus{17} \Number{5}.
 \textsc{\Scriabine{}}~: Étude en \kD \Flat majeur, \Opus{8} \Number{10}~;
 Prélude en \kE mineur, \Opus{11} \Number{4}~; Étude en \kD \Sharp mineur,
 \Opus{8} \Number{12}.
 \textsc{\Rachmaninov{}}~: Moment musical.
 \item[\DateWithWeekDay{1942-10-15}]
 Moskva~: maison des architectes.

 \textsc{\Glazounov{}}~: Sonate.
 \textsc{\Rachmaninov{}}~: Deux Moments musicaux en \kB mineur, \Opus{16}
 \Number{3}, et en \kE \Flat mineur, \Opus{16} \Number{2}~; Deux Préludes en
 \kG majeur, \Opus{32} \Number{5}, et en \kG \Sharp mineur, \Opus{32}
 \Number{12}~; Étude-tableau.
 \textsc{\Prokofiev{}}~: Six Pièces extraites des Visions fugitives,
 \Opus{22}~; \emph{Montaigu et Capulet} extrait de l'\Opus{75}~; Cinquième
 Sarcasme, \Opus{17} \Number{5}.
 \textsc{\Scriabine{}}~: Deux Poèmes, \Opus{32}~; Deux Études extraites de
 l'\Opus{8}~: en \kB \Flat mineur et en \kD \Sharp mineur (\Number{12}).
 \emph{Bis} -- \textsc{\Rachmaninov{}}~: Prélude en \kC \Sharp mineur,
 \Opus{3} \Number{2}.
 \textsc{\Scriabine{}}~: Étude en \kD \Flat majeur, \Opus{8} \Number{10}.
 \textsc{\Rachmaninov{}}~: Moment musical en \kE \Flat mineur, \Opus{16}
 \Number{2}.
 \item[B\DateWithWeekDay{1942-10-27}]
 Attribution à \VSofronitsky{} du titre d'artiste émérite de la~RSFSR
 (République socialiste fédérative soviétique de Rossija).
 \item[\DateWithWeekDay{1942-10-31}]
 Moskva~: grande salle du conservatoire.
 Concert philharmonique%
 \footnote{\foreignlanguage{russian}{\emph{Известия}}, 27~octobre~1942.}.
 Voir en particulier \citet[p.~442]{Milshteyn82a} et
 \citet[p.~394]{Nikonovich08}.

 \textsc{\Borodine{}}~: Six Pièces extraites de la Petite Suite~: \emph{Au
 couvent}, \emph{Intermezzo}, \emph{Mazurka}, \emph{Rêverie},
 \emph{Sérénade} et \emph{Nocturne}.
 \textsc{\Prokofiev{}}~: Marche et Légende extraites des Dix Pièces,
 \Opus{12} \Number{1} et \Number{6}~; Contes de la vieille grand-mère,
 \Opus{31}~; Sarcasmes, \Opus{17} \Number{2} et \Number{5}.
 \textsc{\Scriabine{}}~: Études en \kA \Flat majeur, \Opus{8} \Number{8}, et
 en \kG \Sharp mineur, \Opus{8} \Number{9}~; Sonate, \Opus{53}.
 \textsc{\Glazounov{}}~: Sonate en \kB \Flat mineur, \Opus{74}.
 \textsc{\Balakirev{}}~: \emph{Islamey}.
 \emph{Bis} -- \textsc{\Rachmaninov{}}~: Prélude en \kC \Sharp mineur,
 \Opus{3} \Number{2}~; Moment musical~; Étude-tableau.
 \textsc{\Scriabine{}}~: Prélude en \kC \Sharp mineur, \Opus{11}
 \Number{10}~; Deux Études en \kD \Flat majeur et en \kD \Sharp mineur,
 \Opus{8} \Number{10} et \Number{12}.
 \item[B1942-11]
 Début du travail pédagogique de \VSofronitsky{} au conservatoire de Moskva
 (professeur de la classe de piano).
 \citet[p.~214]{Zhukova08} indique plutôt le mois de décembre~1942.
 \item[B\DateWithWeekDay{1942-11-07}]
 Dans le journal \foreignlanguage{russian}{\emph{Литература и искусство}}
 [\emph{Literatura i iskusstvo}], publication d'un article rédigé par
 \VSofronitsky{} et intitulé~: \Quote{Le devoir de l'artiste}.
 Voir aussi \citet{Sofronitsky61}.
 Il s'agit d'un discours prononcé par le pianiste devant le public du
 théâtre \Pouchkine{} à Leningrad en décembre~1941, pendant le siège de la
 ville.
 \citet[p.~420]{Scriabine} indiquent le~4 novembre plutôt que le~7.
 \item[\DateWithWeekDay{1942-11-15}]
 Moskva~: salle \Tchaikovski{}.
 Concert conjoint avec \DZhuravlev{} (récitant).
 Concert avec des œuvres de \Prokofiev{}, \Chostakovitch{} et \Scriabine{}.
 \citet{Lazarev20} présente une note de \VSofronitsky{} et de \DZhuravlev{}
 évoquant leur concert pour les défenseurs de Leningrad%
 \footnote{\foreignlanguage{russian}{\emph{Вечерняя Москва}},
 14~novembre~1942.}.
 \item[\DateWithWeekDay{1942-12-27}]
 Moskva~: salle \Tchaikovski{}.
 Voir en particulier \citet[p.~442]{Milshteyn82a} et
 \citet[p.~394]{Nikonovich08}.

 \textsc{\Beethoven{}}~: Sonate en \kF mineur, \Opus{57}.
 \textsc{\Chopin{}}~: Sonate en \kB \Flat mineur, \Opus{35}.
 \textsc{\Schumann{}}~: Six \emph{Albumblätter}, \Opus{124}~; Carnaval,
 \Opus{9}.
 \textsc{\Liadov{}}~: Deux Préludes.
 \textsc{\Balakirev{}}~: \emph{Islamey}.
 \emph{Bis} -- \textsc{\Glinka{}}~: Nocturne.
 \textsc{\Prokofiev{}}~: Marche.
 \textsc{\Schubert{}}~: Moment musical en \kA \Flat majeur.
 \textsc{\Chostakovitch{}}~: Prélude.
 \textsc{\Rachmaninov{}}~: Moment musical en \kE \Flat mineur, \Opus{16}
 \Number{2}~; Étude-tableau en \kE \Flat majeur, \Opus{33} \Number{6}.
 \textsc{\Scriabine{}}~: Prélude en \kE \Flat mineur, \Opus{16} \Number{4}~;
 Étude en \kD \Sharp mineur, \Opus{8} \Number{12}.
\end{description}

\section{Année~1943}

\begin{description}
 \item[\DateWithWeekDay{1943-01-10}]
 Moskva~: maison centrale des artistes.

 \textsc{\JBach{}/\Busoni{}}~: Deux Préludes de chorals en \kG mineur et en
 \kG majeur.
 \textsc{\Beethoven{}}~: Sonate en \kF mineur, \Opus{57}.
 \textsc{\Chopin{}}~: Sonate en \kB \Flat mineur, \Opus{35}.
 \textsc{\Rachmaninov{}}~: Préludes en \kC \Sharp mineur, \Opus{3}
 \Number{2}, et en \kG \Sharp mineur, \Opus{32} \Number{12}.
 \textsc{\Balakirev{}}~: \emph{Islamey}.
 \emph{Bis} -- \textsc{\Glinka{}}~: Nocturne.
 \textsc{\Prokofiev{}}~: Marche.
 \textsc{\Schubert{}}~: Moment musical en \kA \Flat majeur.
 \textsc{\Chopin{}}~: Mazurka.
 \textsc{\Prokofiev{}}~: Cinquième Sarcasme, \Opus{17} \Number{5}.
 \textsc{\Rachmaninov{}}~: Étude-tableau en \kE \Flat majeur, \Opus{33}
 \Number{6}.
 \item[\DateWithWeekDay{1943-01-15}]
 Moskva~: maison centrale des artistes.
 Reprise du programme du concert du~10 janvier.
 \item[\DateWithWeekDay{1943-02-07}]
 Moskva~: salle \Tchaikovski{}.

 \textsc{\Beethoven{}}~: Sonate en \kC \Sharp mineur, \Opus{27} \Number{2}.
 \textsc{\Liszt{}}~: Sonate en \kB mineur, S~178~; Funérailles, S~173
 \Number{7}.
 \textsc{\Schubert{}/\Liszt{}}~: \emph{Der Müller und der Bach}~;
 \emph{Erlkönig}.
 \textsc{\Liszt{}}~: Feux follets, S~139 \Number{5}~; \emph{Gnomenreigen},
 S~145 \Number{2}~; \emph{Sonetto~123 del Petrarca}, S~161 \Number{6}.
 \item[B\DateWithWeekDay{1943-03-19}]
 Attribution à \VSofronitsky{} du Prix \Staline{} de premier grade, pour
 des réalisations exceptionnelles en concert et dans les arts de la scène.
 \citet{Lazarev20} présente un article de I.~\Martinov{}%
 \footnote{\foreignlanguage{russian}{\emph{Вечерняя Москва}}, 9~avril~1943.}
 qui évoque cette attribution~; voir aussi \citet{Zagursky43}.
 \item[\DateWithWeekDay{1943-03-30}]
 Moskva~: grande salle du conservatoire.
 Participation à un concert symphonique placé sous la direction de
 \NRakhline{}.

 \textsc{\Scriabine{}}~: Concerto en \kF \Sharp mineur, \Opus{20}.
 \item[\DateWithWeekDay{1943-05-03}]
 Moskva~: salle \Tchaikovski{}.

 \textsc{\Glinka{}}~: Variations sur un thème écossais.
 \textsc{\Glazounov{}}~: Sonate.
 \textsc{\Rachmaninov{}}~: Prélude en \kC \Sharp mineur, \Opus{3}
 \Number{2}.
 \textsc{\Balakirev{}}~: \emph{Islamey}.
 \textsc{\Scriabine{}}~: Sonate en \kF \Sharp mineur, \Opus{23}~; Sonate en
 \kF \Sharp majeur, \Opus{30}~; Deux Études extraites de l'\Opus{8}~: en \kB
 \Flat mineur et en \kD \Sharp mineur (\Number{12})~; Deux Poèmes,
 \Opus{32}.
 \emph{Bis} -- \textsc{\Rachmaninov{}}~: Deux Préludes en \kG majeur,
 \Opus{32} \Number{5}, et en \kG \Sharp mineur, \Opus{32} \Number{12}~;
 Moment musical en \kE \Flat mineur, \Opus{16} \Number{2}~; Étude-tableau
 en \kE \Flat majeur, \Opus{33} \Number{6}.
 \item[\DateWithWeekDay{1943-05-09}]
 Moskva (lieu exact non précisé).
 Participation au concert de la Troisième Réunion pan\-slave
 (III~\foreignlanguage{russian}{Всеславянский Митинг}), aux côtés de
 \DZhuravlev{} (récitant), \GBarinova{} (violon), Vera Dulova (harpe), Ivan
 Kozlovskij (chant) et d'autres musiciens.

 \textsc{\Chopin{}}~: Polonaise en \kA \Flat majeur.
 \item[\DateWithWeekDay{1943-05-19}]
 Moskva~: maison des architectes.
 Concert conjoint avec \DZhuravlev{} (récitant).

 \textsc{\Beethoven{}}~: Sonate en \kF mineur, \Opus{57}.
 \textsc{\Scriabine{}}~: Sonate, \Opus{53}.
 \item[\DateWithWeekDay{1943-05-20}]
 Moskva~: société théâtrale de l'\hbox{Union}.
 Récital.
 Programme inconnu.
 \item[\DateWithWeekDay{1943-05-31}]
 Moskva~: société théâtrale de l'\hbox{Union}.
 Participation à un concert dédié à la ville de Leningrad.
 Programme inconnu.
 \item[\DateWithWeekDay{1943-10-13}]
 Moskva~: maison du cinématographe.
 Concert conjoint avec \MZochtchenko{}.
 Concert avec des œuvres de \Chopin{}, \Rachmaninov{}, \Scriabine{} et
 \Chostakovitch{}%
 \footnote{\foreignlanguage{russian}{\emph{Литератур и искусство}},
 16~octobre~1943.}.
 \item[\DateWithWeekDay{1943-10-18}]
 Moskva~: grande salle du conservatoire.
 Concert symphonique placé sous la direction de \NRakhline{}.
 Peut-être la dernière interprétation de \VSofronitsky{} avec un orchestre
 \citep[voir][p.~116]{White}.

 \textsc{\Scriabine{}}~: Concerto en \kF \Sharp mineur, \Opus{20}.
 \item[\DateWithWeekDay{1943-10-21}]
 Moskva~: grande salle du conservatoire (lieu incertain).
 Concert avec des œuvres de \Schumann{}.
 \item[B1943]
 \VSofronitsky{} et \EScriabina{} se séparent.
 \VSofronitsky{} vit chez des amis et des connaissances.
\end{description}

\section{Année~1944}

\begin{description}
 \item[\DateWithWeekDay{1944-01-07}]
 Moskva~: musée \Scriabine{}.
 Concert avec des œuvres de \Beethoven{}, \Schumann{}, \Liszt{},
 \Rachmaninov{} et \Scriabine{}, selon \ASofronitsky{}.
 Selon un ouvrage de \citeauthor{Berkovskaya13}, un programme bref --
 \Scriabine{} et \Chopin{} --, avec de nombreux \emph{bis}.
 \citet{Lazarev20} présente un bref article%
 \footnote{\foreignlanguage{russian}{\emph{Вечерняя Москва}},
 15~janvier~1944.}
 qui indique un programme avec des œuvres de \Beethoven{}, \Schumann{},
 \Liszt{} et \Rachmaninov{}.
 \item[\DateWithWeekDay{1944-01-28}]
 Moskva~: grande salle du conservatoire.

 \textsc{\Schubert{}}~: Impromptus.
 \textsc{\Schumann{}}~: \emph{Humoreske} en \kB \Flat majeur, \Opus{20}.
 \textsc{\Chopin{}}~: Mazurkas~; Nocturne en \kE \Flat majeur, \Opus{9}
 \Number{2}~; Ballade en \kA \Flat majeur, \Opus{47}.
 \emph{Bis} -- \textsc{\Liszt{}}~: Feux follets, S~139 \Number{5}.
 \item[\DateWithWeekDay{1944-06-01}]
 Moskva~: grande salle du conservatoire.

 \textsc{\Schumann{}}~: Fantaisie en \kC majeur, \Opus{17}.
 \textsc{\Schubert{}}~: Fantaisie \Quote{Wanderer} en \kC majeur, D~760.
 \textsc{\Liszt{}}~: Sonate en \kB mineur, S~178.
 \textsc{\Schubert{}/\Liszt{}}~: \emph{Aufenthalt}~; \emph{Der Müller und
 der Bach}~; \emph{Die Post}~; \emph{Gretchen am Spinnrade}~;
 \emph{Erlkönig}.
 \item[\DateWithWeekDay{1944-06-07}]
 Moskva~: maison centrale des artistes.
 Récital avec des œuvres de \Schumann{}, \Liszt{} et \Scriabine{}.
 \item[\DateWithWeekDay{1944-06-29}]
 Moskva~: grande salle du conservatoire.

 \textsc{\Chopin{}}~: Vingt-quatre Préludes, \Opus{28}~; Fantaisie en \kF
 mineur, \Opus{49}.
 \textsc{\Liszt{}}~: Après une lecture de Dante, S~161 \Number{7}~;
 \emph{Sonetto~104 del Petrarca}, S~161 \Number{5}~; \emph{Sonetto~123 del
 Petrarca}, S~161 \Number{6}~; \emph{Tarantella} extraite de \emph{Venezia
 e Napoli}, S~162 \Number{3}.
 \textsc{\Schubert{}/\Liszt{}}~: \emph{Frühlingsglaube}~; \emph{Auf dem
 Wasser zu singen}.
 \emph{Bis} -- \textsc{\Beethoven{}}~: Sonate en \kC mineur, \Opus{111}.
 \textsc{\Rachmaninov{}}~: Étude (œuvre incertaine).
 \item[1944-07 (date incertaine)]
 Moskva~: musée de la culture musicale (portant aujourd'hui le nom de
 \MGlinka{}).
 Concert dédié à la mémoire de \Scriabine{}.
 Programme inconnu.
 \item[\DateWithWeekDay{1944-07-14}]
 Moskva~: grande salle du conservatoire.

 \textsc{\Chopin{}}~: Polonaise en \kC \Sharp mineur, \Opus{26} \Number{1}~;
 Fantaisie en \kF mineur, \Opus{49}~; Sonate en \kB \Flat mineur,
 \Opus{35}~; Nocturne en \kC mineur, \Opus{48} \Number{1}~; Ballade en \kG
 mineur, \Opus{23}~; Six Mazurkas.
 \emph{Bis} -- \textsc{\Chopin{}}~: Prélude en \kB \Flat mineur, \Opus{28}
 \Number{16}~; Prélude en \kE mineur, \Opus{28} \Number{4}.
 \item[\DateWithWeekDay{1944-07-20}]
 Moskva~: grande salle du conservatoire.

 \textsc{\Chopin{}}~: Polonaise en \kC \Sharp mineur, \Opus{26} \Number{1}~;
 Deux Nocturnes~; Sonate en \kB \Flat mineur, \Opus{35}~; Vingt-quatre
 Préludes, \Opus{28}~; Scherzo~; Trois Mazurkas~; Nocturne en \kE \Flat
 majeur, \Opus{9} \Number{2}~; Valse \Number{8} en \kA \Flat majeur,
 \Opus{64} \Number{3}.
 \item[B1944 (été, automne)]
 Selon \citeauthor{Berkovskaya13}, \VSofronitsky{} épouse son élève,
 \VDushinova{}.
 \item[\DateWithWeekDay{1944-10-26}]
 Moskva~: grande salle du conservatoire.

 \textsc{\Schumann{}}~: \emph{Allegro} en \kB mineur, \Opus{8}~; Sonate en
 \kF \Sharp mineur, \Opus{11}.
 \textsc{\Chopin{}}~: Nocturne~; Valse \Number{6} en \kD \Flat majeur,
 \Opus{64} \Number{1}~; Valse \Number{11} en \kG \Flat majeur, \Opus{70}
 \Number{1}~; Mazurka en \kB mineur~; Scherzo en \kB mineur, \Opus{20}.
 \textsc{\Rachmaninov{}}~: Deux Études-tableaux.
 \textsc{\Debussy{}}~: \emph{Serenade of the Doll}, L~113 \Number{III}~;
 \emph{General Lavine -- eccentric}, L~123 \Number{VI}~; Canope, L~123
 \Number{X}~; Feux d'artifice, L~123 \Number{XII}.
 \emph{Bis} -- \textsc{\Chopin{}}~: Mazurka~; Deux Valses.
 \item[\DateWithWeekDay{1944-11-21}]
 Moskva~: grande salle du conservatoire.

 \textsc{\Schubert{}}~: Fantaisie \Quote{Wanderer} en \kC majeur, D~760.
 \textsc{\Chopin{}}~: Six Études~; Ballade en \kA \Flat majeur, \Opus{47}.
 \textsc{\Liszt{}}~: Funérailles, S~173 \Number{7}~; Feux follets, S~139
 \Number{5}~; \emph{Tarantella} extraite de \emph{Venezia e Napoli}, S~162
 \Number{3}~; \emph{Sonetto del Petrarca}.
 \textsc{\Schubert{}/\Liszt{}}~: \emph{Der Müller und der Bach}~; \emph{Auf
 dem Wasser zu singen}.
 \item[\DateWithWeekDay{1944-12-12}]
 Moskva~: grande salle du conservatoire.

 \textsc{\Schumann{}}~: Études symphoniques, \Opus{13}.
 \textsc{\Chopin{}}~: Sonate en \kB \Flat mineur, \Opus{35}.
 \textsc{\Beethoven{}}~: Sonate en \kC \Sharp mineur, \Opus{27} \Number{2}.
 \textsc{\Liszt{}}~: Après une lecture de Dante, S~161 \Number{7}~;
 Rhapsodie hongroise, S~244 \Number{12}.
 \item[\DateWithWeekDay{1944-12-18}]
 Moskva~: musée \Scriabine{}.
 Concert avec des œuvres de \Scriabine{}.
 Concert en l'honneur de l'ouverture du musée après une interruption de
 trois ans.
 \citet{Lazarev20} présente deux articles successifs%
 \footnote{\foreignlanguage{russian}{\emph{Вечерняя Москва}}, 29~avril~1944
 et~19 juillet~1944.}
 qui évoquaient les découvertes réalisées au musée \Scriabine{} pendant les
 années de guerre et annonçaient sa réouverture imminente et un futur
 concert de \VSofronitsky{}.
\end{description}

\section{Année~1945}

\begin{description}
 \item[B1945]
 \VSofronitsky{} vit avec son épouse \VDushinova{} au boulevard Tverskoj.
 \item[\DateWithWeekDay{1945-01-09}]
 Moskva~: grande salle du conservatoire.

 \textsc{\Glazounov{}}~: Sonate en \kB \Flat mineur, \Opus{74}.
 \textsc{\Rachmaninov{}}~: Quatre Préludes~: en \kC majeur, \Opus{32}
 \Number{1}, en \kA mineur, \Opus{32} \Number{8}, en \kG majeur, \Opus{32}
 \Number{5}, et en \kB \Flat mineur, \Opus{32} \Number{2}.
 \textsc{\Scriabine{}}~: Sonate en \kF \Sharp mineur, \Opus{23}~; Trois
 Préludes (œuvres incertaines)~; Quatre Préludes (œuvres incertaines)~;
 Sonate, \Opus{53}.
 \emph{Bis} -- \textsc{\Scriabine{}}~: Valse en \kA \Flat majeur,
 \Opus{38}~; Mazurka~; Étude en \kD \Sharp mineur, \Opus{8} \Number{12}.
 \item[\DateWithWeekDay{1945-01-12}]
 Moskva~: grande salle du conservatoire.
 Reprise du programme du concert du~9 janvier, mais une annonce dans la
 presse%
 \footnote{\foreignlanguage{russian}{\emph{Вечерняя Москва}},
 10~janvier~1945.}
 évoque aussi des Poèmes de \Scriabine{} et six Préludes de \Rachmaninov{}
 au lieu de quatre.
 \item[\DateWithWeekDay{1945-02-09}]
 Leningrad~: grande salle de la société philharmonique.
 Concert non mentionné par \citet[p.~423]{Scriabine}.
 Premier concert d'un cycle de \foreignlanguage{german}{\emph{Klavierabend}}
 annoncé dans la presse%
 \footnote{\foreignlanguage{russian}{\emph{Ленинградская Правда}},
 3~février~1945.}
 puis critiqué par le compositeur Jurij Vladimirovič Kočurov.

 \textsc{\Schumann{}}~: Études symphoniques, \Opus{13}.
 \textsc{\Chopin{}}~: Sonate en \kB \Flat mineur, \Opus{35}.
 \textsc{\Liszt{}}~: Après une lecture de Dante, S~161 \Number{7}.
 \textsc{\Prokofiev{}}~: Six Visions fugitives extraites de l'\Opus{22}.
 \textsc{\Scriabine{}}~: Sonate \Number{4} en \kF \Sharp majeur, \Opus{30}~;
 Sonate \Number{5}, \Opus{53}.
 \item[\DateWithWeekDay{1945-02-11}]
 Leningrad~: grande salle de la société philharmonique.
 Concert non mentionné par \citet[p.~423]{Scriabine}.
 Reprise du programme du concert du~9 février.
 \item[\DateWithWeekDay{1945-02-15}]
 Leningrad~: grande salle de la société philharmonique.
 Concert non mentionné par \citet[p.~423]{Scriabine}.
 Deuxième concert d'un cycle de
 \foreignlanguage{german}{\emph{Klavierabend}}.

 \textsc{\Schumann{}}~: \emph{Allegro} en \kB mineur, \Opus{8}~; Sonate en
 \kF \Sharp mineur, \Opus{11}.
 \textsc{\Chopin{}}~: Mazurka en \kB mineur~; Deux Valses~; Scherzo en \kB
 mineur, \Opus{20}.
 \textsc{\Liszt{}}~: Sonate en \kB mineur, S~178~; \emph{Tarantella}
 extraite de \emph{Venezia e Napoli}, S~162 \Number{3}.
 \item[\DateWithWeekDay{1945-02-18}]
 Leningrad~: grande salle de la société philharmonique.
 Concert non mentionné par \citet[p.~423]{Scriabine}.
 Reprise du programme du concert du~15 février.
 \item[\DateWithWeekDay{1945-02-25}]
 Leningrad~: grande salle de la société philharmonique.
 Concert non mentionné par \citet[p.~423]{Scriabine}.
 Troisième et dernier concert d'un cycle de
 \foreignlanguage{german}{\emph{Klavierabend}}.

 \textsc{\Glazounov{}}~: Sonate \Number{1} en \kB \Flat mineur, \Opus{74}.
 \textsc{\Rachmaninov{}}~: Six Préludes extraits des \Opus{23 et~32} (en \kE
 \Flat majeur, \kC majeur, \kF majeur, \kA mineur, \kG \Sharp mineur et \kB
 \Flat majeur).
 \textsc{\Scriabine{}}~: Dix Études.
 \textsc{\Chostakovitch{}}~: Préludes.
 \textsc{\Balakirev{}}~: \emph{Islamey}, \Opus{18}.
 \item[\DateWithWeekDay{1945-02-27}]
 Leningrad~: grande salle de la société philharmonique.
 Concert non mentionné par \citet[p.~423]{Scriabine}.
 Reprise du programme du concert du~25 février.
 \item[\DateWithWeekDay{1945-03-04}]
 Leningrad~: grande salle de la société philharmonique.
 Concert non mentionné par \citet[p.~423]{Scriabine}.

 \textsc{\Chopin{}}~: Fantaisie en \kF mineur, \Opus{49}~; Deux Nocturnes
 extraits des \Opus{48 et~9}~; Ballade \Number{3} en \kA \Flat majeur,
 \Opus{47}~; Quatre Mazurkas~; Polonaise en \kA \Flat majeur, \Opus{53}.
 \textsc{\Liszt{}}~: Funérailles, S~173 \Number{7}~; Deux Lieder d'après
 \Schubert{} (\emph{Der Müller und der Bach}, S~565 \Number{2}~; \emph{Auf
 dem Wasser zu singen}, S~558 \Number{2})~; Deux \emph{Sonetto del
 Petrarca}~; Valse oubliée~; Méphisto-valse.
 \item[\DateWithWeekDay{1945-03-09}]
 Leningrad~: grande salle de la société philharmonique.
 Concert non mentionné par \citet[p.~423]{Scriabine}.
 Reprise du programme du concert du~4 mars.
 \item[\DateWithWeekDay{1945-03-23}]
 Moskva~: grande salle du conservatoire.

 \textsc{\Chopin{}}~: Fantaisie en \kF mineur, \Opus{49}~; Deux Nocturnes~;
 Deux Mazurkas~; Ballade en \kA \Flat majeur, \Opus{47}.
 \textsc{\Liszt{}}~: Funérailles, S~173 \Number{7}~; Deux \emph{Sonetto del
 Petrarca}~; \emph{Tarantella} extraite de \emph{Venezia e Napoli}, S~162
 \Number{3}~; Feux follets, S~139 \Number{5}~; Méphisto-valse~; Valse
 oubliée.
 \textsc{\Schubert{}/\Liszt{}}~: \emph{Der Müller und der Bach}~; \emph{Auf
 dem Wasser zu singen}.
 \item[\DateWithWeekDay{1945-03-28}]
 Date d'un concert mentionné par \citet[p.~167]{Nekrasova08}.
 \item[\DateWithWeekDay{1945-04-03}]
 Moskva~: grande salle du conservatoire.
 Reprise du programme du concert du~23~mars.
 \item[\DateWithWeekDay{1945-04-11}]
 Moskva~: grande salle du conservatoire.
 \citet{Lazarev20} présente deux affiches qui annoncent, pour le~11 et le~19
 avril~1945, deux concerts aux programmes quasi identiques.

 \emph{Affiche du concert du~11 avril}.--
 Dix Préludes extraits des \Opus{22, 27, 31, 33, 35, 39 et~48}~; Sonates,
 \Opus{23 et~30}~; Poèmes, \Opus{34 et~36}~; Deux Poèmes, \Opus{69}~;
 Sonate, \Opus{70}~; Préludes, \Opus{74}~; Sonate, \Opus{53}.
 \item[\DateWithWeekDay{1945-04-19}]
 Moskva~: grande salle du conservatoire.
 Voir \citet{Lazarev20}.

 \textsc{\Scriabine{}}~: Dix Préludes extraits des \Opus{22, 27, 31, 33, 35
 et~39}~; Sonate en \kF \Sharp mineur, \Opus{23}~; Deux Poèmes extraits des
 \Opus{59} (\Number{1}) et~69~; Sonate en \kF \Sharp majeur, \Opus{30}~;
 Étude en \kB majeur, \Opus{8} \Number{4}~; Étude en \kA \Flat majeur,
 \Opus{8} \Number{8}~; Étude en \kG \Sharp mineur, \Opus{8} \Number{9}~;
 Sonate, \Opus{70}.
 \emph{Bis} -- \textsc{\Scriabine{}}~: Poème satanique, \Opus{36}~;
 Mazurka~; Prélude.
 \item[\DateWithWeekDay{1945-05-17}]
 Moskva~: grande salle du conservatoire.
 Reprise du programme du concert du~19~avril.
 \citet{Lazarev20} présente une affiche du concert du~17 mai, donné à
 l'occasion du~30\ieme{} anniversaire de la mort de \Scriabine{}.
 Huit jours après la victoire à l'issue de la Grande Guerre patriotique.
 \item[\DateWithWeekDay{1945-06-05}]
 Moskva~: grande salle du conservatoire.

 \textsc{\Haendel{}}~: Variations en \kE majeur sur le finale (Air) de la
 Suite \Number{5} en \kE majeur.
 \textsc{\Mozart{}}~: Fantaisie en \kC mineur.
 \textsc{\Chopin{}}~: Vingt-quatre Préludes, \Opus{28}~; Ballade en \kF
 mineur, \Opus{52}.
 \textsc{\Scriabine{}}~: Sonate, \Opus{68}.
 \textsc{\Debussy{}}~: Danseuses de Delphes, L~117 \Number{I}~; Le Vent dans
 la plaine, L~117 \Number{III}~; Feuilles mortes, L~123 \Number{II}~;
 \emph{General Lavine -- eccentric}, L~123 \Number{VI}~; Canope, L~123
 \Number{X}~; Feux d'artifice, L~123 \Number{XII}.
 \emph{Bis} -- \textsc{\Debussy{}}~: \emph{Serenade of the Doll}, L~113
 \Number{III}.
 \textsc{\Chopin{}}~: Valse~; Mazurka~; Étude.
 \textsc{\Scriabine{}}~: Prélude.
 \item[\DateWithWeekDay{1945-06-12}]
 Moskva~: grande salle du conservatoire.
 Reprise du programme du concert du~5~juin, hormis un Scherzo de \Chopin{}.
 Voir en particulier \citet[p.~442]{Milshteyn82a},
 \citet[p.~394]{Nikonovich08} et \citet{Lazarev20}.

 \textsc{\Haendel{}}~: Variations en \kE majeur sur le finale (Air) de la
 Suite \Number{5} en \kE majeur.
 \textsc{\Mozart{}}~: Fantaisie en \kC mineur.
 \textsc{\Chopin{}}~: Ballade en \kF mineur, \Opus{52}.
 \textsc{\Scriabine{}}~: Sonate, \Opus{68}.
 \textsc{\Chopin{}}~: Vingt-quatre Préludes, \Opus{28}~; Scherzo en \kB
 \Flat mineur, \Opus{31}.
 \textsc{\Debussy{}}~: Six Préludes.
 \emph{Bis} -- \textsc{\Debussy{}}~: \emph{Serenade of the Doll}, L~113
 \Number{III}.
 \textsc{\Chopin{}}~; Valse~; Mazurka~; Étude.
 \textsc{\Scriabine{}}~: Prélude.
 \item[\DateWithWeekDay{1945-06-22}]
 Leningrad~: grande salle de la société philharmonique.
 Concert non mentionné par \citet[p.~423]{Scriabine}%
 \footnote{Pour la date et le programme de ce récital, voir
 \href{https://100philharmonia.spb.ru/historical-poster/19025/}%
 {https://100philharmonia.spb.ru/historical-poster/19025/}.}.

 \textsc{\Haendel{}}~: Variations.
 \textsc{\Mozart{}}~: Fantaisie.
 \textsc{\Chopin{}}~: Vingt-quatre Préludes, \Opus{28}~; Ballade \Number{4}
 en \kF mineur, \Opus{52}~; Scherzo \Number{2} en \kB \Flat mineur,
 \Opus{31}.
 \textsc{\Scriabine{}}~: Sonate \Number{3} en \kF \Sharp mineur, \Opus{23}.
 \textsc{\Debussy{}}~: Six Préludes.
 \item[B\DateWithWeekDay{1945-07-17} -- \DateWithWeekDay{1945-08-02}]
 Conférence de Potsdam.
 \item[\DateWithWeekDay{1945-07-21}]
 Potsdam~: Résidence officielle de \Staline{} durant la conférence.
 Apparition au banquet d'\hbox{État} pour une interprétation sur ordre.
 Selon \citet[p.~423]{Scriabine}, le séjour de \VSofronitsky{} à Berlin a
 duré du~20 juillet~1945 au~28 juillet~1945 et il y a eu trois concerts,
 avec la participation de \VSofronitsky{}, \EGuilels{}, \GBarinova{} et
 \ADedioukhine{}, devant les dirigeants de la coalition antinazie.
 Selon \RKoganSofronitskaya{}, en revanche, le séjour à Potsdam n'a duré que
 trois jours \citep[p.~95]{Artese}.
 \citet[p.~166 et~167]{Nekrasova08} confirme les trois concerts de
 \Sofronitsky{} à Berlin et Potsdam, de même que le séjour du~20 au~28
 \Quote{août} [sic] (lettre à \AVizel{} datée du~2 septembre~1945).

 \textsc{\Scriabine{}}~: Étude en \kD \Sharp mineur, \Opus{8} \Number{12}.
 D'autres œuvres ont été jouées, en particulier peut-être la Polonaise en
 \kA \Flat majeur de \Chopin{}, \Opus{53}, selon \INikonovich{}
 \citep[voir][p.~54, note~7]{White} et Varvara \citet[p.~166]{Nekrasova08}.
 \item[\DateWithWeekDay{1945-09-20}]
 Moskva~: grande salle du conservatoire.
 Concert avec des œuvres de \Beethoven{}, \Schumann{}, \Ravel{} et
 \Debussy{}.
 \item[\DateWithWeekDay{1945-09-26}]
 Moskva~: grande salle du conservatoire.

 \textsc{\Beethoven{}}~: Sonate en \kE \Flat majeur, \Opus{81a}.
 \textsc{\Schumann{}}~: \emph{Kreisleriana}, \Opus{16}.
 \textsc{\Ravel{}}~: Sonatine.
 \textsc{\Debussy{}}~: Clair de lune, extrait de la Suite bergamasque, L~75
 \Number{III}~; Reflets dans l'eau, extraits des Images, L~110 \Number{I}.
 \textsc{\Rachmaninov{}}~: Deux Préludes extraits de l'\Opus{23}~; Deux
 Études-tableaux extraites de l'\Opus{39}~: en \kA mineur et en \kB mineur
 (\Number{4}).
 \textsc{\Balakirev{}}~: \emph{Islamey}.
 \item[\DateWithWeekDay{1945-10-04}]
 Moskva~: grande salle du conservatoire.
 Sans doute un concert évoqué par \VSofronitsky{} dans son entretien avec
 \citet{Vitsinsky}.

 \textsc{\Beethoven{}}~: Sonate en \kE \Flat majeur, \Opus{81a}.
 \textsc{\Schumann{}}~: \emph{Kreisleriana}, \Opus{16}.
 \textsc{\Ravel{}}~: Sonatine.
 \textsc{\Debussy{}}~: Reflets dans l'eau, extraits des Images, L~110
 \Number{I}.
 \textsc{\Medtner{}}~: Deux \emph{Skazki}.
 \textsc{\Rachmaninov{}}~: Deux Préludes extraits de l'\Opus{32}~; Deux
 Études-tableaux extraites de l'\Opus{39}.
 \textsc{\Balakirev{}}~: \emph{Islamey}.
 \item[\DateWithWeekDay{1945-10-22}]
 Moskva~: grande salle du conservatoire.
 Reprise du programme du concert du~4 octobre.
 Sans doute un concert évoqué par \VSofronitsky{} dans son entretien avec
 \citet{Vitsinsky}.
 \item[\DateWithWeekDay{1945-11-12}]
 Moskva~: grande salle du conservatoire.
 Voir en particulier \citet[p.~442]{Milshteyn82a} et
 \citet[p.~394]{Nikonovich08}.

 \textsc{\Scarlatti{}}~: Cinq Sonates.
 \textsc{\Schumann{}}~: \emph{Humoreske} en \kB \Flat majeur, \Opus{20}.
 \textsc{\Liszt{}}~: \emph{Sposalizio}, S~161 \Number{1}~; Deux
 Valses-caprices (d'après \Schubert{})~; \emph{Sonetto~123 del Petrarca},
 S~161 \Number{6}~; Sonate en \kB mineur, S~178.
 \emph{Bis} -- \textsc{\Liszt{}}~: Valse oubliée.
 \textsc{\Chopin{}}~: Étude en \kG \Flat majeur, \Opus{10} \Number{5}~;
 Valse \Number{8} en \kA \Flat majeur, \Opus{64} \Number{3}.
 \textsc{\Rachmaninov{}}~: Étude-tableau en \kA mineur, \Opus{39}
 \Number{6}.
 \item[\DateWithWeekDay{1945-12-12}]
 Moskva~: grande salle du conservatoire.

 \textsc{\Chopin{}}~: Polonaise en \kC \Sharp mineur, \Opus{26} \Number{1}~;
 Ballade en \kG mineur, \Opus{23}~; Vingt-quatre Préludes, \Opus{28}~;
 Barcarolle en \kF \Sharp majeur, \Opus{60}~; Tarentelle en \kA \Flat
 majeur, \Opus{43}~; Valse \Number{8} en \kA \Flat majeur, \Opus{64}
 \Number{3}~; Scherzo en \kB \Flat mineur, \Opus{31}~; Scherzo en \kB
 mineur, \Opus{20}.
 \emph{Bis} -- \textsc{\Chopin{}}~: Valse \Number{11} en \kG \Flat majeur,
 \Opus{70} \Number{1}~; Valse \Number{8} in \kA \Flat majeur, \Opus{64}
 \Number{3}~; Étude en \kC \Sharp mineur, \Opus{10} \Number{4}~; Étude en
 \kF majeur, \Opus{10} \Number{8}.
\end{description}

\section{Année~1946}

\begin{description}
 \item[\DateWithWeekDay{1946-01-07}]
 Moskva~: grande salle du conservatoire.
 Voir en particulier \citet[p.~442]{Milshteyn82a} et
 \citet[p.~394]{Nikonovich08}.

 \textsc{\Chopin{}}~: Polonaise en \kC \Sharp mineur, \Opus{26} \Number{1}~;
 Ballade en \kG mineur, \Opus{23}~; Vingt-quatre Préludes, \Opus{28}~;
 Barcarolle en \kF \Sharp majeur, \Opus{60}~; Tarentelle en \kA \Flat
 majeur, \Opus{43}~; Valse \Number{8} en \kA \Flat majeur, \Opus{64}
 \Number{3}~; Deux Mazurkas~; Scherzos en \kB mineur, \Opus{20}, et en \kB
 \Flat mineur, \Opus{31}.
 \emph{Bis} -- \textsc{\Chopin{}}~: Études en \kG \Flat majeur, \Opus{10}
 \Number{5}, et en \kC \Sharp mineur, \Opus{10} \Number{4}~; Étude en \kF
 majeur, \Opus{25} \Number{3}~; Valse \Number{8} en \kA \Flat majeur,
 \Opus{64} \Number{3}~; Étude en \kF majeur, \Opus{10} \Number{8}.
 \item[\DateWithWeekDay{1946-01-14}]
 Moskva~: grande salle du conservatoire.

 \textsc{\Chopin{}}~: Polonaise en \kC \Sharp mineur, \Opus{26} \Number{1}~;
 Deux Nocturnes extraits de l'\Opus{15}~: en \kF \Sharp majeur (\Number{2})
 et en \kF majeur (\Number{1})~; Sonate en \kB \Flat mineur, \Opus{35}~;
 Fantaisie en \kF mineur, \Opus{49}~; Ballade en \kA \Flat majeur,
 \Opus{47}~; Mazurka en \kC \Sharp mineur, \Opus{30} \Number{4}~; Polonaise
 en \kA \Flat majeur, \Opus{53}.
 \emph{Bis} -- \textsc{\Chopin{}}~: Une Mazurka extraite de l'\Opus{30}~;
 Valse \Number{11} en \kG \Flat majeur, \Opus{70} \Number{1}~; Valse
 \Number{6} en \kD \Flat majeur, \Opus{64} \Number{1}~; Prélude en \kD
 mineur, \Opus{28} \Number{24}.
 \item[\DateWithWeekDay{1946-01-19}]
 Moskva~: musée \Scriabine{}.
 Concert avec des œuvres de \Chopin{} et \Scriabine{}.
 \item[\DateWithWeekDay{1946-02-07}]
 Moskva~: grande salle du conservatoire.
 Voir en particulier \citet[p.~442]{Milshteyn82a} et
 \citet[p.~394]{Nikonovich08}.

 \textsc{\Chopin{}}~: Polonaise en \kC \Sharp mineur, \Opus{26} \Number{1}~;
 Deux Nocturnes extraits de l'\Opus{15}~; Sonate en \kB \Flat mineur,
 \Opus{35}~; Fantaisie en \kF mineur, \Opus{49}~; Une Mazurka extraite de
 l'\Opus{50}~; Ballade en \kA \Flat majeur, \Opus{47}~; Polonaise en \kA
 \Flat majeur, \Opus{53}.
 \item[1946-02 (date incertaine)]
 Moskva~: maison des syndicats.
 Concert avec des œuvres de \Scriabine{}.
 \item[\DateWithWeekDay{1946-03-18}]
 Moskva~: grande salle du conservatoire.
 Voir en particulier \citet[p.~442]{Milshteyn82a} et
 \citet[p.~394]{Nikonovich08}.

 \textsc{\Schumann{}}~: \emph{Allegro} en \kB mineur, \Opus{8}~; Sonate en
 \kF \Sharp mineur, \Opus{11}~; \emph{Kreisleriana}, \Opus{16}~; Carnaval,
 \Opus{9}.
 \emph{Bis} -- \textsc{\Chopin{}}~: Deux Mazurkas~; Valse.
 \item[\DateWithWeekDay{1946-04-04}]
 Concert en un lieu inconnu.
 Selon un article paru dans \emph{Sovetskaja muzyka}%
 \footnote{\foreignlanguage{russian}{\emph{Советская музыка}}, vol.~101,
 \Number{7} (1946), p.~108.},
 le concert a eu lieu à la grande salle du conservatoire de Moskva, le
 programme incluait aussi l'\emph{Humoreske} en \kB \Flat majeur, \Opus{20},
 et les deux \emph{Novelettes} en question étaient la septième et la
 huitième~; en outre, seules six \emph{Bunte Blätter} auraient été jouées.

 \textsc{\Schumann{}}~: \emph{Allegro} en \kB mineur, \Opus{8}~; Carnaval,
 \Opus{9}~; Deux \emph{Novelettes} extraites de l'\Opus{21}~; \emph{Bunte
 Blätter}, \Opus{99}.
 \item[\DateWithWeekDay{1946-04-17}]
 Moskva~: grande salle du conservatoire.
 Voir en particulier \citet[p.~442]{Milshteyn82a} et
 \citet[p.~394]{Nikonovich08}.
 Récital évoqué et critiqué par \citet[p.~91]{Zitomirsky46}.

 \textsc{\Beethoven{}}~: Sonate en \kE \Flat majeur, \Opus{81a}~; Sonate en
 \kF mineur, \Opus{57}~; Sonate en \kC \Sharp mineur, \Opus{27} \Number{2}~;
 Sonate en \kC mineur, \Opus{111}.
 \emph{Bis} -- \textsc{\Schubert{}}~: Moment musical.
 \textsc{\Schumann{}}~: \emph{Novelette} en \kE majeur, \Opus{21}
 \Number{7}.
 \textsc{\Chopin{}}~: Deux Préludes extraits de l'\Opus{28}~: \Number{7} en
 \kA majeur et \Number{8} en \kF \Sharp mineur.
 \textsc{\Scriabine{}}~: Étude en \kD \Flat majeur, \Opus{8} \Number{10}.
 \item[\DateWithWeekDay{1946-04-28}]
 Moskva~: musée \Scriabine{}.

 \textsc{\Scriabine{}}~: Six Préludes extraits des \Opus{11, 13 et~17}~;
 Sonate en \kF \Sharp mineur, \Opus{23}~; Huit Études extraites de
 l'\Opus{8}~; Poème tragique, \Opus{34}~; Poème satanique, \Opus{36}~; Deux
 Poèmes, \Opus{32}~; Poème, \Opus{59} \Number{1}~; Sonate, \Opus{53}.
 \emph{Bis} -- \textsc{\Scriabine{}}~: Trois Études extraites de l'\Opus{8}.
 \item[\DateWithWeekDay{1946-05-15}]
 Moskva~: grande salle du conservatoire.

 \textsc{\Scriabine{}}~: Préludes extraits de l'\Opus{13}, de l'\Opus{16}
 (en \kF \Sharp majeur, \Number{5}) et de l'\Opus{17} (en \kF mineur,
 \Number{5}, en \kB \Flat mineur, \Number{4}, et en \kD \Flat majeur,
 \Number{3})~; Sonate en \kF \Sharp mineur, \Opus{23}~; Trois Études
 extraites de l'\Opus{8}~: \Number{4} en \kB majeur, \Number{7} en \kB \Flat
 mineur et \Number{9} en \kG \Sharp mineur~; Sonate en \kF \Sharp majeur,
 \Opus{30}~; Poème tragique, \Opus{34}~; Poème satanique, \Opus{36}~;
 Masque, \Opus{63} \Number{1}~; Énigme, \Opus{52} \Number{2}~; Sonate,
 \Opus{70}~; Vers la flamme, \Opus{72} (à la place de l'œuvre annoncée~:
 l'\hbox{Étude} \Quote{en quintes}, \Opus{65} \Number{3}).
 \emph{Bis} -- \textsc{\Scriabine{}}~: Prélude en \kG mineur, \Opus{27}
 \Number{1}~; Prélude en \kE \Flat majeur, \Opus{11} \Number{19}~; Prélude
 en \kE \Flat mineur, \Opus{11} \Number{14}~; Étude en \kD \Flat majeur,
 \Opus{8} \Number{10}~; Poème en \kF \Sharp majeur, \Opus{32} \Number{1}~;
 Étude en \kD \Sharp mineur, \Opus{8} \Number{12}.
 \item[\DateWithWeekDay{1946-05-17}]
 Moskva~: maison centrale des artistes.
 Concert avec des œuvres de \Scriabine{}%
 \footnote{\foreignlanguage{russian}{\emph{Вечерняя Москва}}, 15~mai~1946.}.
 \item[\DateWithWeekDay{1946-05-19}]
 Moskva~: musée \Scriabine{}.

 \textsc{\Scriabine{}}~: Sonate en \kF \Sharp mineur, \Opus{23}~; Études
 extraites de l'\Opus{8}~; Poème tragique, \Opus{34}~; Deux Poèmes,
 \Opus{32}~; Sonate en \kF \Sharp majeur, \Opus{30}.
 \emph{Bis} -- \textsc{\Scriabine{}}~: Préludes~; Étude en \kD \Flat majeur,
 \Opus{8} \Number{10}.
 \item[\DateWithWeekDay{1946-05-23}]
 Moskva~: grande salle du conservatoire.
 Voir en particulier \citet[p.~442]{Milshteyn82a} et
 \citet[p.~394]{Nikonovich08}.

 \textsc{\Schubert{}}~: Deux Impromptus en \kC mineur et en \kA \Flat
 majeur, D~899 \Number{1} et \Number{4}~; Moments musicaux, D~780~;
 Fantaisie \Quote{Wanderer} en \kC majeur, D~760.
 \textsc{\Schubert{}/\Liszt{}}~: \emph{Du bist die Ruh}~; \emph{Der Müller
 und der Bach}~; \emph{Aufenthalt}~; \emph{Erstarrung}~; \emph{Der
 Doppelgänger}~; \emph{Gretchen am Spinnrade}~; \emph{Auf dem Wasser zu
 singen}~; \emph{Die Post}~; \emph{Frühlingsglaube}~; \emph{Erlkönig}.
 \emph{Bis} -- \textsc{\Schubert{}}~: Impromptu en \kE \Flat majeur, D~899
 \Number{2}~; Moment musical.
 \textsc{\Schubert{}/\Liszt{}}~: Valse-caprice~; \emph{Erlkönig}.
 \item[\DateWithWeekDay{1946-06-21}]
 Participation à une émission de radio enregistrée et intitulée~:
 \foreignlanguage{russian}{\emph{Мое любимое произведение}} \Quote{Mon œuvre
 préférée}.
 \VSofronitsky{} évoque et joue \Schubert{}/\Liszt{} (\emph{Erlkönig}, S~558
 \Number{4}) et \Scriabine{}.
 Voir \citet[p.~167]{Nekrasova08}, lettre du~22 juin.
 \item[\DateWithWeekDay{1946-06-22}]
 Moskva~: grande salle du conservatoire.
 Voir en particulier \citet[p.~442]{Milshteyn82a} et
 \citet[p.~394-395]{Nikonovich08}.

 \textsc{\Mendelssohn{}}~: Variations sérieuses en \kD mineur, \Opus{54}.
 \textsc{\Brahms{}}~: Ballade en \kG mineur, \Opus{118} \Number{3}.
 \textsc{\Schumann{}}~: Fantaisie en \kC majeur, \Opus{17}.
 \textsc{\Liszt{}}~: Funérailles, S~173 \Number{7}~; \emph{Du bist die
 Ruh}~; \emph{Auf dem Wasser zu singen}~; \emph{Frühlingsglaube}~;
 \emph{Erlkönig} (d'après \Schubert{})~; Feux follets, S~139 \Number{5}~;
 \emph{Gnomenreigen}, S~145 \Number{2}~; Valse oubliée \Number{1}, S~215
 \Number{1}~; \emph{Tarantella} extraite de \emph{Venezia e Napoli}, S~162
 \Number{3}.
 \item[B\DateWithWeekDay{1946-08-15}]
 Lettre à \AVizel{} \citep[voir][p.~167-168]{Nekrasova08} où \VSofronitsky{}
 évoque l'enregistrement en studio de la Troisième Sonate de \Scriabine{},
 en \kF \Sharp mineur, \Opus{23}, très réussi selon lui.
 \item[\DateWithWeekDay{1946-09-01}]
 Moskva~: musée \Scriabine{}.
 \citet{Lazarev20} présente un programme de ce concert.

 \textsc{\Scriabine{}}~: Huit Préludes extraits des \Opus{31, 33, 35, 37,
 39 et~48}~; Deux Poèmes, \Opus{44}~; Deux Morceaux, \Opus{57} (Désir et
 Caresse dansée)~; Quasi-valse en \kF majeur, \Opus{47}~; Ironies, \Opus{56}
 \Number{2}~; Rêverie en \kC majeur, \Opus{49} \Number{3}~; Sonate,
 \Opus{53}~; Sonate en \kF \Sharp majeur, \Opus{30}~; Poème, \Opus{59}
 \Number{1}~; Poème en \kF \Sharp majeur, \Opus{32} \Number{1}~; Énigme,
 \Opus{52} \Number{2}~; Sonate, \Opus{70}~; Poème satanique, \Opus{36}.
 \emph{Bis} -- \textsc{\Scriabine{}}~: Prélude en \kE mineur, \Opus{11}
 \Number{4}~; Prélude en \kD majeur, \Opus{11} \Number{5}~; Un Prélude
 extrait de l'\Opus{13}~; Mazurka en \kE mineur, \Opus{25} \Number{3}~;
 Valse en \kA \Flat majeur, \Opus{38}.
 \item[\DateWithWeekDay{1946-10-25}]
 Moskva~: grande salle du conservatoire.
 Voir en particulier \citet[p.~442-443]{Milshteyn82a} et
 \citet[p.~395]{Nikonovich08}.

 \textsc{\Glazounov{}}~: Sonate en \kB \Flat mineur, \Opus{74}.
 \textsc{\Rachmaninov{}}~: Préludes en \kD majeur et en \kC mineur,
 \Opus{23} \Number{4} et \Number{7}~; Préludes en \kE majeur et en \kG
 majeur, \Opus{32} \Number{3} et \Number{5}~; Deux Études-tableaux en \kA
 mineur et en \kE \Flat majeur, \Opus{39}.
 \textsc{\Prokofiev{}}~: Sonate en \kA mineur, \Opus{28}~; Sarcasmes,
 \Opus{17}.
 \textsc{\Scriabine{}}~: Sonate, \Opus{53}.
 \emph{Bis} -- \textsc{\Rachmaninov{}}~: Moment musical.
 \textsc{\Scriabine{}}~: Ironies, \Opus{56} \Number{2}~; Prélude en \kC
 mineur, \Opus{17}.
 \textsc{\Rachmaninov{}}~: Étude-tableau en \kA mineur.
 \item[\DateWithWeekDay{1946-11-12}]
 Moskva~: grande salle du conservatoire.

 \textsc{\Scarlatti{}}~: Cinq Sonates.
 \textsc{\Schumann{}}~: \emph{Humoreske} en \kB \Flat majeur, \Opus{20}.
 \textsc{\Liszt{}}~: \emph{Sposalizio}, S~161 \Number{1}.
 \textsc{\Schubert{}/\Liszt{}}~: Deux Valses-caprices.
 \textsc{\Liszt{}}~: \emph{Sonetto~123 del Petrarca}, S~161 \Number{6}~;
 Sonate en \kB mineur, S~178.
 \item[\DateWithWeekDay{1946-11-18}]
 Moskva~: grande salle du conservatoire.
 Voir en particulier \citet[p.~443]{Milshteyn82a} et
 \citet[p.~395]{Nikonovich08}.

 \textsc{\Rachmaninov{}}~: Deux Moments musicaux en \kB mineur et en \kD
 \Flat majeur, \Opus{16} \Number{3} et \Number{5}~; Préludes en \kC \Sharp
 mineur, \Opus{3} \Number{2}, en \kF \Sharp mineur, \Opus{23} \Number{1}, en
 \kD majeur, \Opus{23} \Number{4}, en \kC mineur, \Opus{23} \Number{7}, en
 \kA \Flat majeur, \Opus{23} \Number{8}, en \kE \Flat majeur, \Opus{23}
 \Number{6}, en \kE majeur, \Opus{32} \Number{3}, en \kG majeur, \Opus{32}
 \Number{5}, en \kA mineur, \Opus{32} \Number{8}, en \kG \Sharp mineur,
 \Opus{32} \Number{12}, en \kE \Flat mineur, \Opus{23} \Number{9} et en \kB
 \Flat majeur, \Opus{23} \Number{2}~; Trois Études-tableaux en \kG mineur,
 en \kB mineur et en \kE \Flat majeur, \Opus{33} \Number{7}, \Opus{39}
 \Number{4} et \Opus{33} \Number{6}.
 \textsc{\Prokofiev{}}~: Douze Pièces extraites des Visions fugitives,
 \Opus{22}.
 \textsc{\Scriabine{}}~: Sonate, \Opus{66}.
 \textsc{\Balakirev{}}~: \emph{Islamey}.
 \emph{Bis} -- \textsc{\Rachmaninov{}}~: Étude-tableau en \kB mineur,
 \Opus{39} \Number{4}.
 \textsc{\Scriabine{}}~: Prélude~; Mazurka~; Ironies, \Opus{56} \Number{2}~;
 Étude en \kD \Sharp mineur, \Opus{8} \Number{12}.
 \item[\DateWithWeekDay{1946-11-24}]
 Moskva~: musée \Scriabine{}.

 \textsc{\Rachmaninov{}}~: Deux Moments musicaux en \kD \Flat majeur et en
 \kB mineur, \Opus{16} \Number{5} et \Number{3}~; Préludes~;
 Études-tableaux.
 \textsc{\Scriabine{}}~: Sonate en \kF \Sharp mineur, \Opus{23}~; Sonate,
 \Opus{70}~; œuvres de forme brève.
 \item[B1946-10 -- 1946-11]
 \VSofronitsky{} séjourne à l'hôtel Jakor' (actuellement rue Tverskoj), où
 se trouvait alors la maison des scientifiques.
 \item[\DateWithWeekDay{1946-12-16}]
 Leningrad~: grande salle de la société philharmonique.

 \textsc{\Rachmaninov{}}~: Deux Moments musicaux extraits de l'\Opus{16}~;
 Douze Préludes.
 \textsc{\Scriabine{}}~: Sonate en \kF \Sharp mineur, \Opus{23}~; Six
 Poèmes~; Sonate, \Opus{70}.
 \item[\DateWithWeekDay{1946-12-17}]
 Leningrad~: grande salle de la société philharmonique.

 \textsc{\Glazounov{}}~: Sonate en \kB \Flat mineur, \Opus{74}.
 \textsc{\Rachmaninov{}}~: Quatre Préludes extraits de l'\Opus{23}, en
 particulier ceux en \kD majeur (\Number{4}) et en \kC mineur (\Number{7})~;
 Deux Études-tableaux extraites de l'\Opus{39}.
 \textsc{\Prokofiev{}}~: Sonate en \kA mineur, \Opus{28}~; Sarcasme,
 \Opus{17} \Number{5}.
 \textsc{\Scriabine{}}~: Sonate, \Opus{53}.
 \item[B\DateWithWeekDay{1946-12-28}]
 Attribution de l'\hbox{Ordre} de \Lenin{} à \VSofronitsky{}.
 \item[\DateWithWeekDay{1946-12-29}]
 Leningrad~: salle \Glazounov{} (petite salle du conservatoire).
 Concert pour le~75\ieme{} anniversaire de la naissance de \Scriabine{}%
 \footnote{\foreignlanguage{russian}{\emph{Смена (Ленинград)}},
 29~décembre~1946.}.

 \textsc{\Scriabine{}}~: Six Préludes extraits des \Opus{11, 13, 15 et~17}~;
 Huit Études extraites de l'\Opus{8}~; Sonate en \kF \Sharp mineur,
 \Opus{23}~; Poème tragique, \Opus{34}~; Poème satanique, \Opus{36}~; Sonate
 en \kF \Sharp majeur, \Opus{30}~; Sonate, \Opus{53}~; Sonate, \Opus{70}~;
 Deux Morceaux extraits de l'\Opus{56}.
\end{description}

\section{Année~1947}

\begin{description}
 \item[\DateWithWeekDay{1947-01-07}]
 Moskva~: musée \Scriabine{}.
 Concert pour le~75\ieme{} anniversaire de la naissance de \Scriabine{}.

 \textsc{\Scriabine{}}~: Quatre Préludes, \Opus{22}~; Douze Études extraites
 des \Opus{42 et~8}~; Deux Poèmes, \Opus{32}~; Poème, \Opus{59} \Number{1}~;
 Énigme, \Opus{52} \Number{2}~; Sonate, \Opus{70}.
 \emph{Bis} -- \textsc{\Scriabine{}}~: Valse~; Feuillet d'album, \Opus{45}
 \Number{1}~; Deux Préludes extraits de l'\Opus{11}~: en \kE \Flat mineur
 (\Number{14}) et en \kC \Sharp mineur (\Number{10})~; Étude en \kD \Flat
 majeur, \Opus{8} \Number{10}.
 \item[\DateWithWeekDay{1947-01-09}]
 Moskva~: grande salle du conservatoire.

 \textsc{\Scriabine{}}~: Prélude en \kC majeur, \Opus{13} \Number{1}~;
 Prélude en \kB mineur, \Opus{22} \Number{4}~; Prélude en \kD \Flat majeur,
 \Opus{17} \Number{3}~; Prélude en \kB \Flat mineur, \Opus{17} \Number{4}~;
 Prélude en \kG mineur, \Opus{17} \Number{7}~; Prélude en \kB mineur,
 \Opus{13} \Number{3}~; Sonate en \kF \Sharp mineur, \Opus{23}~; Neuf Études
 extraites de l'\Opus{8}~: en \kC \Sharp majeur (\Number{1}), en \kF \Sharp
 mineur (\Number{2}), en \kB majeur (\Number{4}), en \kE majeur
 (\Number{5}), en \kB \Flat mineur (\Number{7}), en \kA \Flat majeur
 (\Number{8}), en \kG \Sharp mineur (\Number{9}), en \kB \Flat mineur
 (\Number{11}) et en \kD \Sharp mineur (\Number{12})~; Sonate en \kF \Sharp
 majeur, \Opus{30}~; Deux Poèmes, \Opus{32}~; Poème, \Opus{59} \Number{1}~;
 Énigme, \Opus{52} \Number{2}~; Poème satanique, \Opus{36}~; Sonate,
 \Opus{70}~; Vers la flamme, \Opus{72}.
 \emph{Bis} -- \textsc{\Scriabine{}}~: Étude en \kD \Flat majeur, \Opus{8}
 \Number{10}~; Valse (œuvre incertaine).
 \item[B\DateWithWeekDay{1947-01-11}]
 Moskva.
 \VSofronitsky{} au Moskovskij Kreml' (en français, Kremlin de Moscou) afin
 d'y recevoir l'\hbox{Ordre} de \Lenin{}.
 Au soir~: concert de gala au conservatoire.
 Voir \citet[p.~167 et~168]{Nekrasova08}, lettres de janvier~1947.
 \item[\DateWithWeekDay{1947-02-06}]
 Moskva~: grande salle du conservatoire.

 \textsc{\Schubert{}}~: Deux Impromptus en \kC mineur, D~899 \Number{1}, et
 en \kA \Flat majeur, D~899 \Number{4}.
 \textsc{\Beethoven{}}~: Sonate en \kC \Sharp mineur, \Opus{27} \Number{2}.
 \textsc{\Chopin{}}~: Fantaisie en \kF mineur, \Opus{49}~; Polonaise en \kC
 \Sharp mineur, \Opus{26} \Number{1}~; Polonaise en \kA majeur, \Opus{40}
 \Number{1}~; Quatre Mazurkas extraites des \Opus{30 et~50}.
 \textsc{\Schumann{}}~: Carnaval, \Opus{9}.
 \emph{Bis} -- \textsc{\Schumann{}}~: \emph{Aufschwung}, \Opus{12}
 \Number{2}~; \emph{Träumerei}, \Opus{15} \Number{7}.
 \textsc{\Chopin{}}~: Mazurka ou Étude extraite de l'\Opus{10} (œuvre
 incertaine).
 \textsc{\Rachmaninov{}}~: Moment musical en \kE \Flat mineur, \Opus{16}
 \Number{2}~; Prélude en \kG majeur, \Opus{32} \Number{5}.
 \item[\DateWithWeekDay{1947-02-16}]
 Moskva~: maison des scientifiques.

 \textsc{\Chopin{}}~: Une Polonaise extraite de l'\Opus{40}~; Deux Nocturnes
 extraits des \Opus{9 et~15}~; Sonate en \kB \Flat mineur, \Opus{35}~;
 Scherzo en \kB mineur, \Opus{20}~; Scherzo en \kB \Flat mineur, \Opus{31}~;
 Ballade en \kA \Flat majeur, \Opus{47}~; Quatre Mazurkas extraites des
 \Opus{30 et~50}~; Valse \Number{6} en \kD \Flat majeur, \Opus{64}
 \Number{1}~; Valse \Number{8} en \kA \Flat majeur, \Opus{64} \Number{3}~;
 Polonaise en \kA \Flat majeur, \Opus{53}.
 \item[\DateWithWeekDay{1947-02-19}]
 Leningrad~: grande salle de la société philharmonique.
 Selon \ASofronitsky{}, la première partie du concert était soit la reprise
 de la première partie du programme du~16 février (Polonaise, Nocturnes et
 Sonate de \Chopin{}), soit la liste suivante.

 \textsc{\Schubert{}}~: Impromptu en \kC mineur, D~899 \Number{1}~;
 Impromptu en \kA \Flat majeur, D~899 \Number{4}.
 \textsc{\Beethoven{}}~: Sonate en \kC \Sharp mineur, \Opus{27} \Number{2}.
 \textsc{\Chopin{}}~: Fantaisie en \kF mineur, \Opus{49}.
 \emph{Deuxième partie} -- \textsc{\Chopin{}}~: Polonaise en \kC \Sharp
 mineur, \Opus{26} \Number{1}~; Polonaise en \kA majeur, \Opus{40}
 \Number{1}~; Quatre Mazurkas.
 \textsc{\Schumann{}}~: Carnaval, \Opus{9}.
 \item[\DateWithWeekDay{1947-02-23}]
 Leningrad~: grande salle de la société philharmonique.

 \textsc{\Chopin{}}~: Polonaise en \kC mineur, \Opus{40} \Number{2}~; Deux
 Nocturnes extraits des \Opus{9 et~15}~; Sonate en \kB mineur, \Opus{58}~;
 Scherzo en \kB mineur, \Opus{20}~; Scherzo en \kB \Flat mineur, \Opus{31}~;
 Ballade en \kA \Flat majeur, \Opus{47}~; Mazurka en \kC \Sharp mineur~;
 Valse \Number{8} en \kA \Flat majeur, \Opus{64} \Number{3}~; Polonaise en
 \kA \Flat majeur, \Opus{53}.
 \item[\DateWithWeekDay{1947-03-01}]
 Moskva~: petite salle du conservatoire.
 Participation de \VSofronitsky{} à un concert pour le~120\ieme{}
 anniversaire de la mort de \Beethoven{}.

 \textsc{\Beethoven{}}~: Sonate en \kC \Sharp mineur, \Opus{27} \Number{2}.
 \item[\DateWithWeekDay{1947-03-04}]
 Moskva~: grande salle du conservatoire.
 Concert enregistré en partie (\Schumann{}~: Carnaval, \Opus{9}~;
 \Scriabine{}~: Prélude en \kB mineur, \Opus{13} \Number{6}).
 Parmi les concerts de \VSofronitsky{}, il s'agit du premier qui ait été
 enregistré~: il a été radiodiffusé depuis la salle de concert.
 Les enregistrements antérieurs ont été réalisés en studio, et non pas
 pendant une représentation en public.

 \textsc{\Schubert{}}~: Deux Impromptus en \kC mineur et en \kA \Flat
 majeur, D~899 \Number{1} et \Number{4}.
 \textsc{\Beethoven{}}~: Sonate en \kC \Sharp mineur, \Opus{27} \Number{2}.
 \textsc{\Chopin{}}~: Fantaisie en \kF mineur, \Opus{49}~; Polonaise en \kC
 \Sharp mineur, \Opus{26} \Number{1}~; Polonaise en \kA majeur, \Opus{40}
 \Number{1}~; Cinq Mazurkas extraites des \Opus{30 et~50}.
 \textsc{\Schumann{}}~: Carnaval, \Opus{9}.
 \emph{Bis} -- \textsc{\Rachmaninov{}}~: Préludes en \kE \Flat majeur et en
 \kE majeur, \Opus{23} \Number{6} et \Opus{32} \Number{3}~; Étude-tableau en
 \kB mineur, \Opus{39} \Number{4}~; Prélude en \kG majeur, \Opus{32}
 \Number{5}.
 \textsc{\Scriabine{}}~: Prélude en \kB mineur, \Opus{13} \Number{6}.
 \item[\DateWithWeekDay{1947-03-15}]
 Moskva~: maison des syndicats.

 \textsc{\Scriabine{}}~: Sonate-fantaisie \Number{2} en \kG \Sharp mineur,
 \Opus{19}~; Sonate \Number{3} en \kF \Sharp mineur, \Opus{23}~; Sonate
 \Number{4} en \kF \Sharp majeur, \Opus{30}~; Sonate \Number{9}, \Opus{68}~;
 Sonate \Number{10}, \Opus{70}~; Sonate \Number{5}, \Opus{53}.
 \item[\DateWithWeekDay{1947-03-18}]
 Moskva~: musée \Scriabine{}.

 \textsc{\Scriabine{}}~: Douze Préludes extraits des \Opus{31, 33, 35, 37,
 39 et~48}~; Six Poèmes extraits des \Opus{44, 51, 52, 69 et~59}~; Sonate,
 \Opus{68}~; Sonate, \Opus{70}~; Deux Études extraites de l'\Opus{8}~: en
 \kB \Flat mineur (\Number{7}) et en \kB \Flat mineur (\Number{11})~; Valse
 en \kA \Flat majeur, \Opus{38}~; Prélude, \Opus{74} \Number{3}~; Prélude,
 \Opus{74} \Number{2}~; Vers la flamme, \Opus{72}.
 \emph{Bis} -- \textsc{\Scriabine{}}~: Deux Morceaux, \Opus{57}~; Énigme,
 \Opus{52} \Number{2}~; Ironies, \Opus{56} \Number{2}~; Feuillet d'album,
 \Opus{58}~; Prélude en \kF mineur, \Opus{17} \Number{5}.
 \item[\DateWithWeekDay{1947-03-26}]
 Moskva~: grande salle du conservatoire.

 \textsc{\Mozart{}}~: Sonate en \kE \Flat majeur, K~282.
 \textsc{\Schumann{}}~: \emph{Humoreske} en \kB \Flat majeur, \Opus{20}.
 \textsc{\Scriabine{}}~: Sonate, \Opus{68}.
 \textsc{\Debussy{}}~: Reflets dans l'eau, L~110 \Number{I}~; \emph{General
 Lavine -- eccentric}, L~123 \Number{VI}~; \emph{Minstrels}, L~117
 \Number{XII}~; La Danse de Puck, L~117 \Number{XI}.
 \textsc{\Rachmaninov{}}~: Deux Études-tableaux.
 \textsc{\Prokofiev{}}~: Danse.
 \textsc{\Liszt{}}~: Méphisto-valse.
 \item[\DateWithWeekDay{1947-04-13}]
 Moskva~: grande salle du conservatoire.
 Ce concert pourrait avoir été reporté au~18 avril.

 \textsc{\Mozart{}}~: Sonate en \kB \Flat majeur.
 \textsc{\Schumann{}}~: \emph{Humoreske} en \kB \Flat majeur, \Opus{20}.
 \textsc{\Scriabine{}}~: Sonate, \Opus{68}.
 \textsc{\Debussy{}}~: Reflets dans l'eau, L~110 \Number{I}~;
 \emph{Minstrels}, L~117 \Number{XII}~; \emph{General Lavine -- eccentric},
 L~123 \Number{VI}.
 \textsc{\Rachmaninov{}}~: Deux Études-tableaux.
 \textsc{\Prokofiev{}}~: Danse.
 \textsc{\Liszt{}}~: Méphisto-valse.
 \item[\DateWithWeekDay{1947-04-18}]
 Moskva~: grande salle du conservatoire.
 Reprise ou report du concert du~13 avril.
 Programme identique.
 \item[\DateWithWeekDay{1947-04-20}]
 Moskva~: musée \Scriabine{}.

 \textsc{\Scriabine{}}~: Préludes extraits des \Opus{11, 13, 17, 22 et~39}~;
 Étude en \kB majeur, \Opus{8} \Number{4}~; Étude en \kE majeur, \Opus{8}
 \Number{5}~; Étude en \kA \Flat majeur, \Opus{8} \Number{8}~; Étude en \kG
 \Sharp mineur, \Opus{8} \Number{9}~; Valse~; Sonate, \Opus{68}~; Sonate,
 \Opus{53}.
 \item[\DateWithWeekDay{1947-04-24}]
 Leningrad~: grande salle de la société philharmonique.

 \textsc{\Mozart{}}~: Sonate en \kB \Flat majeur.
 \textsc{\Schumann{}}~: \emph{Humoreske} en \kB \Flat majeur, \Opus{20}.
 \textsc{\Scriabine{}}~: Sonate, \Opus{68}.
 \textsc{\Debussy{}}~: Reflets dans l'eau, L~110 \Number{I}~;
 \emph{Minstrels}, L~117 \Number{XII}~; \emph{General Lavine -- eccentric},
 L~123 \Number{VI}.
 \textsc{\Rachmaninov{}}~: Deux Études-tableaux.
 \textsc{\Prokofiev{}}~: Danse.
 \textsc{\Liszt{}}~: Méphisto-valse.
 \item[B\DateWithWeekDay{1947-04-26}]
 À partir de cette date, et jusqu'à la fin de l'année~1947, la chronologie
 de \citet[p.~168]{Nekrasova08} diverge totalement de celles de
 \citet[p.~460]{Milshteyn82a} et de \citet[p.~426-428]{Scriabine}, adoptées
 ci-dessous.
 \item[\DateWithWeekDay{1947-04-28}]
 Leningrad~: grande salle de la société philharmonique.

 \textsc{\Chopin{}}~: Polonaise en \kC \Sharp mineur, \Opus{26} \Number{1}~;
 Nocturne en \kF \Sharp majeur, \Opus{15} \Number{2}~; Nocturne en \kF
 majeur, \Opus{15} \Number{1}~; Sonate en \kB \Flat mineur, \Opus{35}~;
 Fantaisie en \kF mineur, \Opus{49}~; Deux Études~; Ballade en \kA \Flat
 majeur, \Opus{47}~; Deux Mazurkas~; Polonaise en \kA \Flat majeur,
 \Opus{53}.
 \item[\DateWithWeekDay{1947-05-15}]
 Moskva~: grande salle du conservatoire.
 Concert philharmonique.

 \textsc{\Scriabine{}}~: Quatre Préludes extraits de l'\Opus{11}~: en \kB
 majeur (\Number{11}), en \kG \Flat majeur (\Number{13}), en \kD \Flat
 majeur (\Number{15}) et en \kA \Flat majeur (\Number{17})~; Deux autres
 Préludes extraits aussi de l'\Opus{11}~; Sonate en \kF \Sharp mineur,
 \Opus{23}~; Trois Études extraites de l'\Opus{8}~: en \kB majeur
 (\Number{4}), en \kB \Flat mineur (\Number{7}) et en \kG \Sharp mineur
 (\Number{9})~; Peut-être, selon un manuscrit, les Études en \kA \Flat
 majeur, \Opus{8} \Number{8}, et en \kD \Sharp mineur, \Opus{8}
 \Number{12}~; Sonate en \kF \Sharp majeur, \Opus{30}~; Poème tragique,
 \Opus{34}~; Poème satanique, \Opus{36}~; Masque, \Opus{63} \Number{1}~;
 Énigme, \Opus{52} \Number{2}~; Sonate, \Opus{70}~; Vers la flamme,
 \Opus{72}~; Flammes sombres, \Opus{73} \Number{2}~; Poème en \kF \Sharp
 majeur, \Opus{32} \Number{1}.
 \item[\DateWithWeekDay{1947-06-18}]
 Moskva~: musée \Scriabine{}.

 \textsc{\Schumann{}}~: \emph{Bunte Blätter}, \Opus{99}~; Études
 symphoniques, \Opus{13}.
 \textsc{\Rachmaninov{}}~: Prélude en \kD majeur, \Opus{23} \Number{4}~;
 Prélude en \kC mineur, \Opus{23} \Number{7}.
 \textsc{\Medtner{}}~: \emph{Novelette} en \kC mineur, \Opus{17}
 \Number{2}~; Deux \emph{Skazki} extraits des \Opus{26 et~20}.
 \textsc{\Prokofiev{}}~: Trois Pièces extraites des Contes de la vieille
 grand-mère, \Opus{31}.
 \textsc{\Scriabine{}}~: Prélude en \kF \Sharp majeur, \Opus{39}
 \Number{1}~; Prélude en \kF \Sharp majeur, \Opus{48} \Number{1}~; Prélude,
 \Opus{74} \Number{1} (ou \Number{2})~; Vers la flamme, \Opus{72}.
 \item[\DateWithWeekDay{1947-07-10}]
 Moskva~: musée \Scriabine{}.

 \textsc{\Schumann{}}~: \emph{Bunte Blätter}, \Opus{99} \Number{10}
 (\emph{Präludium})~; Une Romance extraite de l'\Opus{28}~;
 \emph{Intermezzo} en \kE \Flat mineur extrait du \emph{Faschingsschwank aus
 Wien}, \Opus{26} \Number{4}~; \emph{Kreisleriana}, \Opus{16}.
 \textsc{\Rachmaninov{}}~: Moment musical en \kE \Flat mineur, \Opus{16}
 \Number{2}~; Moment musical en \kB mineur, \Opus{16} \Number{3}~; Quatre
 Préludes extraits des \Opus{23 et~32}.
 \textsc{\Scriabine{}}~: Valse en \kA \Flat majeur, \Opus{38}~; Deux
 Préludes, \Opus{67}~; Vers la flamme, \Opus{72}.
 \item[B1947 (été)]
 \VSofronitsky{} et son épouse \VDushinova{} en vacances à Plës, sur les
 rives de la Volga.
 \item[\DateWithWeekDay{1947-10-19}]
 Moskva~: musée \Scriabine{}.

 \textsc{\BogdanovBerezovsky{}}~: \emph{Grave} extrait de la Sonate pour
 piano.
 \textsc{\Chostakovitch{}}~: Douze Préludes.
 \textsc{\Kabalevski{}}~: Douze Préludes.
 \textsc{\Prokofiev{}}~: Dix Pièces extraites des Visions fugitives,
 \Opus{22}~; Cinq Sarcasmes, \Opus{17}.
 \textsc{\Scriabine{}}~: Sonate, \Opus{53}~; Danse languide, \Opus{51}
 \Number{4}~; Deux Préludes~; Mazurka en \kE mineur, \Opus{25} \Number{3}~;
 Un Prélude extrait de l'\Opus{17}.
 \item[\DateWithWeekDay{1947-11-21}]
 Moskva~: musée \Scriabine{}.

 \textsc{\Beethoven{}}~: Sonate en \kF mineur, \Opus{57}.
 \textsc{\Rachmaninov{}}~: Deux Moments musicaux extraits de l'\Opus{16}~;
 Prélude en \kC mineur, \Opus{23} \Number{7}~; Deux Études-tableaux
 extraites de l'\Opus{33}.
 \textsc{\Scriabine{}}~: Sonate, \Opus{68}.
 \textsc{\Prokofiev{}}~: Six Pièces extraites de la Suite \emph{Roméo et
 Juliette}, \Opus{75}~: \emph{Montaigu et Capulet}, \emph{Scène},
 \emph{Menuet}, \emph{Mercutio}, \emph{Frère Laurent} et \emph{Juliette et
 sa nourrice}.
 \item[\DateWithWeekDay{1947-11-26}]
 Moskva~: grande salle du conservatoire.
 Concert faisant partie du cycle~: \Quote{Œuvres pour le~30\ieme{}
 anniversaire de la grande révolution socialiste d'\hbox{Octobre}}.

 \textsc{\Beethoven{}}~: Sonate en \kF mineur, \Opus{57}.
 \textsc{\Rachmaninov{}}~: Deux Moments musicaux extraits de l'\Opus{16}~;
 Prélude en \kC mineur, \Opus{23} \Number{7}~; Prélude en \kG majeur,
 \Opus{32} \Number{5}.
 \textsc{\Scriabine{}}~: Sonate, \Opus{68}.
 \textsc{\Prokofiev{}}~: Dix Pièces extraites des Visions fugitives,
 \Opus{22}~; Six Pièces extraites de la Suite \emph{Roméo et Juliette},
 \Opus{75}~; Toccata en \kD mineur, \Opus{11}.
 \item[\DateWithWeekDay{1947-12-17}]
 Moskva~: grande salle du conservatoire.

 \textsc{\Schubert{}}~: Impromptu en \kG \Flat majeur, D~899 \Number{3}~;
 Impromptu en \kA \Flat majeur, D~899 \Number{4}.
 \textsc{\Schumann{}}~: Sonate en \kF \Sharp mineur, \Opus{11} (jouée sans
 l'\hbox{Introduction}).
 \textsc{\Chopin{}}~: Polonaise en \kF \Sharp mineur, \Opus{44}~; Scherzo en
 \kB mineur, \Opus{20}.
 \textsc{\Liszt{}}~: Sonate en \kB mineur, S~178.
 \textsc{\Schubert{}/\Liszt{}}~: \emph{Der Müller und der Bach}, \emph{Der
 Doppelgänger}, \emph{Auf dem Wasser zu singen} et \emph{Erlkönig}.
 \item[\DateWithWeekDay{1947-12-28}]
 Moskva~: maison des scientifiques.
 Concert faisant partie du cycle~: \Quote{Musique occidentale pour piano}.

 \textsc{\Schumann{}}~: Études symphoniques, \Opus{13}~;
 \emph{Kreisleriana}, \Opus{16}~; Carnaval, \Opus{9}.
 \emph{Bis} -- \textsc{\Schumann{}}~: Romance en \kF \Sharp majeur,
 \Opus{28} \Number{2}.
 \textsc{\Chopin{}}~: Polonaise en \kF \Sharp mineur, \Opus{44}~; Mazurka en
 \kD \Flat majeur, \Opus{30} \Number{3}.
\end{description}

\section{Année~1948}

\begin{description}
 \item[1948 (début de l'année)]
 Moskva~: musée \Scriabine{}.
 \citet{Bogdanov67a} évoque un concert donné par \VSofronitsky{} au musée
 \Scriabine{}, lors duquel le pianiste a joué la Sonate pour piano à deux
 voix de \VBogdanovBerezovsky{}~; \Sofronitsky{} l'a ensuite enregistrée sur
 bande.
 \item[\DateWithWeekDay{1948-01-05}]
 Moskva~: grande salle du conservatoire.
 Concert enregistré en totalité, sauf le dernier \emph{bis}.

 \textsc{\Schubert{}}~: Deux Impromptus en \kG \Flat majeur et en \kA \Flat
 majeur, D~899 \Number{3} et \Number{4}.
 \textsc{\Schumann{}}~: Sonate en \kF \Sharp mineur, \Opus{11}.
 \textsc{\Chopin{}}~: Polonaise en \kF \Sharp mineur, \Opus{44}.
 \textsc{\Liszt{}}~: Sonate en \kB mineur, S~178.
 \textsc{\Schubert{}/\Liszt{}}~: \emph{Der Doppelgänger}~; \emph{Auf dem
 Wasser zu singen}~; \emph{Erlkönig}.
 \emph{Bis} -- \textsc{\Schubert{}/\Liszt{}}~: \emph{Der Müller und der
 Bach}.
 \textsc{\Schumann{}}~: \emph{Kreisleriana}, \Opus{16}, dernière pièce.
 \item[\DateWithWeekDay{1948-01-07}]
 Moskva~: musée \Scriabine{}.

 \textsc{\Scriabine{}}~: Huit Préludes extraits des \Opus{22, 31, 33
 et~39}~; Sonate en \kF \Sharp mineur, \Opus{23}~; Sonate, \Opus{68}~;
 Sonate, \Opus{53}~; Poème en \kF \Sharp majeur, \Opus{32} \Number{1}~;
 Poème, \Opus{59} \Number{1}~; Deux Préludes~; Valse en \kA \Flat majeur,
 \Opus{38}.
 \item[\DateWithWeekDay{1948-01-15}]
 Moskva~: Rédaction du journal \emph{Komsomol'skaja pravda}.

 \textsc{\Beethoven{}}~: Sonate en \kF mineur, \Opus{57}.
 \textsc{\Chopin{}}~: Fantaisie en \kF mineur, \Opus{49}.
 \textsc{\Schumann{}}~: Carnaval, \Opus{9}.
 \textsc{\Scriabine{}}~: Étude en \kD \Sharp mineur, \Opus{8} \Number{12}.
 \item[B1948-02 (à partir de)]
 \VSofronitsky{} de plus en plus préoccupé par l'état de santé de sa sœur
 jumelle, Vera Vladimirovna \citep[p.~168]{Nekrasova08}.
 \item[\DateWithWeekDay{1948-02-01}]
 Moskva~: musée \Scriabine{}.

 \textsc{\Chopin{}}~: Polonaise en \kC \Sharp mineur, \Opus{26} \Number{1}~;
 Six Préludes~; Sonate en \kB \Flat mineur, \Opus{35}~; Six Mazurkas.
 \textsc{\Schumann{}}~: Carnaval, \Opus{9}.
 \item[\DateWithWeekDay{1948-03-05}]
 Moskva~: grande salle du conservatoire.
 Concert radiodiffusé.
 Dans une lettre du~11 mars à \AVizel{} \citep[voir][p.~168]{Nekrasova08},
 \VSofronitsky{} mentionne que le concert a dû être interrompu en raison de
 problèmes cardiaques.

 \textsc{\Chopin{}}~: Fantaisie en \kF mineur, \Opus{49}~; Huit Préludes
 extraits de l'\Opus{28}~: en \kD \Flat majeur (\Number{15}), en \kB \Flat
 mineur (\Number{16}), en \kA \Flat majeur (\Number{17}), en \kF mineur
 (\Number{18}), en \kE \Flat majeur (\Number{19}), en \kC mineur
 (\Number{20}), en \kB \Flat majeur (\Number{21}) et en \kG mineur
 (\Number{22})~; Sonate en \kB \Flat mineur, \Opus{35}~; Ballade en \kF
 mineur, \Opus{52}~; Ballade en \kA \Flat majeur, \Opus{47}~; Une Mazurka
 extraite de l'\Opus{50}~; Valse en \kC \Sharp mineur, \Opus{64}
 \Number{2}~; Valse en \kD \Flat majeur, \Opus{64} \Number{1}.
 Le programme annonçait aussi, en fin de récital, la Polonaise en \kA \Flat
 majeur, \Opus{53}.
 \item[\DateWithWeekDay{1948-03-26}]
 Moskva~: grande salle du conservatoire.
 Reprise du programme du concert \Chopin{} du~5~mars -- dans la deuxième
 partie, à la place de la Mazurka, \VSofronitsky{} a joué la Polonaise en
 \kA \Flat majeur, \Opus{53}.
 \item[\DateWithWeekDay{1948-04-11}]
 Moskva~: musée \Scriabine{}.

 \textsc{\Glazounov{}}~: Sonate en \kB \Flat mineur, \Opus{74}.
 \textsc{\Liadov{}}~: Prélude en \kD \Flat majeur, \Opus{57} \Number{1}~;
 Prélude en \kF \Sharp majeur, \Opus{36} \Number{1}~; Barcarolle en \kF
 \Sharp majeur, \Opus{44}.
 \textsc{\Rachmaninov{}}~: Deux Moments musicaux extraits de l'\Opus{16}~:
 en \kE mineur (\Number{4}) et en \kE \Flat mineur (\Number{2}).
 \textsc{\Scriabine{}}~: Douze Études, \Opus{8}.
 \item[B1948-04 (mi) -- 1948-05 (mi)]
 \VSofronitsky{} à nouveau malade~; deux concerts ont dû être reportés.
 Voir lettre du~27 avril \citep[p.~168]{Nekrasova08}.
 \item[\DateWithWeekDay{1948-05-19}]
 Moskva~: grande salle du conservatoire.

 \textsc{\Glazounov{}}~: Sonate en \kB \Flat mineur, \Opus{74}.
 \textsc{\Liadov{}}~: Prélude en \kD \Flat majeur, \Opus{57} \Number{1}~;
 Prélude~; Barcarolle en \kF \Sharp majeur, \Opus{44}.
 \textsc{\Rachmaninov{}}~: Sérénade, \Opus{3} \Number{5}~; Polichinelle,
 \Opus{3} \Number{4}.
 \textsc{\Scriabine{}}~: Quatre Préludes extraits des \Opus{11, 16 et~22}~;
 Neuf Études extraites de l'\Opus{8} (tout le recueil sauf le \Number{3}, le
 \Number{6} et le \Number{10}).
 \emph{Bis} -- \textsc{\Rachmaninov{}}~: Moment musical en \kE \Flat mineur,
 \Opus{16} \Number{2}.
 \textsc{\Scriabine{}}~: Étude en \kD \Flat majeur, \Opus{8} \Number{10}.
 \textsc{\Liadov{}}~: Grimace, \Opus{64} \Number{1}.
 \textsc{\Scriabine{}}~: Ironies en \kC majeur, \Opus{56} \Number{2}~;
 Nuances, \Opus{56} \Number{3}~; Mazurka en \kE mineur, \Opus{25}
 \Number{3}~; Un Prélude extrait de l'\Opus{17}.
 \item[\DateWithWeekDay{1948-05-23}]
 Moskva~: maison des scientifiques.
 Concert reporté du~18 avril.

 \textsc{\Beethoven{}}~: Sonate en \kC \Sharp mineur, \Opus{27} \Number{2}~;
 Sonate en \kF mineur, \Opus{57}.
 \textsc{\Liszt{}}~: Sonate en \kB mineur, S~178.
 \emph{Bis} -- \textsc{\Liszt{}}~: Valse oubliée~; \emph{Gnomenreigen},
 S~145 \Number{2}.
 \textsc{\Rachmaninov{}}~: Sérénade, \Opus{3} \Number{5}~; Polichinelle,
 \Opus{3} \Number{4}.
 \textsc{\Scriabine{}}~: Prélude pour la main gauche en \kC \Sharp mineur,
 \Opus{9} \Number{1}~; Une Étude extraite de l'\Opus{8}.
 \item[\DateWithWeekDay{1948-05-28}]
 Moskva~: musée \Scriabine{}.

 \textsc{\Scriabine{}}~: Douze Préludes extraits des \Opus{9, 11, 13, 15, 17
 et~22}~; Huit Préludes extraits des \Opus{31, 33, 35, 37 et~48}~; Un Poème
 extrait de l'\Opus{52}~; Un Poème extrait de l'\Opus{69}~; Un Poème extrait
 de l'\Opus{71}~; Énigme, \Opus{52} \Number{2}~; Masque, \Opus{63}
 \Number{1}~; Vers la flamme, \Opus{72}~; Poème satanique, \Opus{36}.
 \emph{Bis} -- \textsc{\Scriabine{}}~: Poème en \kF \Sharp majeur, \Opus{32}
 \Number{1}~; Prélude en \kC \Sharp mineur, \Opus{11} \Number{10}~; Mazurka
 en \kF \Sharp majeur, \Opus{40} \Number{2}~; Mazurka en \kE mineur,
 \Opus{25} \Number{3}~; Deux Préludes extraits de l'\Opus{33}~; Deux
 Préludes extraits de l'\Opus{17}~; Deux Préludes extraits de l'\Opus{11}.
 \item[\DateWithWeekDay{1948-06-15}]
 Moskva~: musée \Scriabine{}.

 \textsc{\Scriabine{}}~: Huit Préludes extraits des \Opus{16, 11 et~17}~;
 Étude en \kB majeur, \Opus{8} \Number{4}~; Trois Études extraites de
 l'\Opus{42}~; Valse en \kA \Flat majeur, \Opus{38}~; Poème tragique,
 \Opus{34}~; Deux Préludes extraits de l'\Opus{35}~; Un Prélude extrait de
 l'\Opus{37}~; Ironies en \kC majeur, \Opus{56} \Number{2}~; Danse languide,
 \Opus{51} \Number{4}~; Mazurka en \kE mineur, \Opus{25} \Number{3}~; Poème
 ailé, \Opus{51} \Number{3}~; Poème, \Opus{59} \Number{1}~; Deux Danses,
 \Opus{73}~; Énigme, \Opus{52} \Number{2}~; Une Mazurka extraite de
 l'\Opus{40}~; Nuances, \Opus{56} \Number{3}~; Rêverie en \kC majeur,
 \Opus{49} \Number{3}~; Étude en \kB \Flat mineur extraite de l'\Opus{8}~;
 Un Prélude extrait de l'\Opus{17}.
 \item[\DateWithWeekDay{1948-10-03}]
 Moskva~: musée \Scriabine{}.

 \textsc{\Beethoven{}}~: Sonate en \kC mineur, \Opus{111}.
 \textsc{\Schumann{}}~: Arabesque en \kC majeur, \Opus{18}.
 \textsc{\Chopin{}}~: Dix Mazurkas.
 \textsc{\Scriabine{}}~: Valse en \kA \Flat majeur, \Opus{38}~; Vingt-quatre
 Préludes (peut-être l'\Opus{11}).
 \item[\DateWithWeekDay{1948-10-12}]
 Moskva~: grande salle du conservatoire.

 \textsc{\Beethoven{}}~: Sonate en \kC mineur, \Opus{111}.
 \textsc{\Chopin{}}~: Fantaisie en \kF mineur, \Opus{49}~; Six Études~: en
 \kC majeur, en \kE mineur (\Opus{25} \Number{5}), en \kA mineur, en \kF
 majeur, en \kF mineur et en \kA \Flat majeur~; Ballade en \kA \Flat majeur,
 \Opus{47}~; Deux Mazurkas~; Polonaise en \kF \Sharp mineur, \Opus{44}.
 \emph{Bis} -- \textsc{\Chopin{}}~: Valses \Number{7} en \kC \Sharp mineur,
 \Opus{64} \Number{2}, \Number{8} en \kA \Flat majeur, \Opus{64} \Number{3},
 et peut-être aussi \Number{6} en \kD \Flat majeur, \Opus{64} \Number{1}.
 \textsc{\Rachmaninov{}}~: Sérénade, \Opus{3} \Number{5}~; Polichinelle,
 \Opus{3} \Number{4}~; Polka de~W.R. (1911).
 \item[\DateWithWeekDay{1948-10-17}]
 Moskva~: maison des scientifiques.

 \textsc{\Beethoven{}}~: Sonate en \kC mineur, \Opus{13}~; Sonate en \kC
 mineur, \Opus{111}.
 \textsc{\Chopin{}}~: Six Études~: en \kC majeur (\Opus{10}), en \kA mineur
 (\Opus{10} \Number{2}), en \kE majeur (\Opus{10} \Number{3}), en \kE mineur
 (\Opus{25} \Number{5}), en \kC \Sharp mineur (\Opus{25} \Number{7}) et en
 \kA \Flat majeur (\Opus{25} \Number{1})~; Ballade en \kA \Flat majeur,
 \Opus{47}~; Deux Mazurkas~; Polonaise en \kF \Sharp mineur, \Opus{44}.
 \item[\DateWithWeekDay{1948-10-25}]
 Leningrad~: grande salle de la société philharmonique.

 \textsc{\Beethoven{}}~: Sonate en \kF mineur, \Opus{57}~; Sonate en \kC
 mineur, \Opus{111}.
 \textsc{\Chopin{}}~: Fantaisie en \kF mineur, \Opus{49}~; Quatre Études~;
 Ballade en \kA \Flat majeur, \Opus{47}~; Trois Mazurkas~; Deux Valses~;
 Polonaise.
 \textsc{\Rachmaninov{}}~: Sérénade, \Opus{3} \Number{5}.
 \item[\DateWithWeekDay{1948-10-29}]
 Leningrad.
 Enregistrement pour la radio du premier mouvement de la Sonate en \kF
 mineur de \Beethoven{}, \Opus{57}, dite \emph{Appassionata}.
 \item[\DateWithWeekDay{1948-11-03}]
 Leningrad~: grande salle de la société philharmonique.

 \textsc{\Schubert{}}~: Deux Impromptus extraits du recueil D~899.
 \textsc{\Beethoven{}}~: Sonate en \kC \Sharp mineur, \Opus{27} \Number{2}.
 \textsc{\Chopin{}}~: Fantaisie en \kF mineur, \Opus{49}~; Dix Mazurkas~;
 Nocturne~; Deux Valses.
 \textsc{\Schumann{}}~: Carnaval, \Opus{9}.
 \item[\DateWithWeekDay{1948-11-17}]
 Moskva~: petite salle du conservatoire.

 \textsc{\Chopin{}}~: Polonaise en \kC \Sharp mineur, \Opus{26} \Number{1}~;
 Nocturne en \kF majeur, \Opus{15} \Number{1}~; Nocturne en \kF \Sharp
 mineur, \Opus{48} \Number{2}~; Sonate en \kB \Flat mineur, \Opus{35}~;
 Trois Mazurkas~; Ballade en \kG mineur, \Opus{23}~; Scherzo en \kB mineur,
 \Opus{20}.
 \emph{Bis} -- \textsc{\Scriabine{}}~: Valse en \kA \Flat majeur,
 \Opus{38}~; Étude en \kB \Flat mineur~; Poème en \kF \Sharp majeur,
 \Opus{32} \Number{1}.
 \textsc{\Rachmaninov{}}~: Prélude en \kG mineur, \Opus{23} \Number{5}.
 \textsc{\Chopin{}}~: Mazurka en \kA mineur.
 \item[B\DateWithWeekDay{1948-11-27}]
 Décès de la sœur jumelle de \VSofronitsky{}, Vera Vladimirovna, à
 Leningrad, de la même maladie -- un cancer -- que celle qui allait emporter
 le musicien en~1961 et leur sœur aînée, Natal'ja Vladimirovna, en~1979.
 Voir \citet[p.~118-119, note en bas de page]{Nikonovich08a}.
 \item[B\DateWithWeekDay{1948-11-29}]
 Moskva~: grande salle du conservatoire.
 Concert annulé.
 La liste ci-dessous indique le programme prévu de ce concert annulé.

 \textsc{\Schubert{}}~: Impromptu en \kA \Flat majeur, D~935 \Number{2}~;
 Impromptu en \kA \Flat majeur, D~899 \Number{4}.
 \textsc{\Beethoven{}}~: Sonate \Number{14} en \kC \Sharp mineur, \Opus{27}
 \Number{2}.
 \textsc{\Chopin{}}~: Sonate \Number{2} en \kB \Flat mineur, \Opus{35}~;
 Huit Mazurkas.
 \textsc{\Schumann{}}~: Carnaval, \Opus{9}.
 \item[\DateWithWeekDay{1948-12-02}]
 Concert dont la date n'apparaît que dans les archives d'\AVizel{}
 \citep[voir][p.~169]{Nekrasova08}.
 \item[\DateWithWeekDay{1948-12-21}]
 Concert à la mémoire de Vera Vladimirovna Sofronickaja.
 \item[\DateWithWeekDay{1948-12-23}]
 Moskva~: grande salle du conservatoire.
 Participation de \VSofronitsky{} à un concert et conférence.
 Conférencier~: \IMartinov{}.
 Concert à la mémoire de Vera Vladimirovna Sofronickaja.

 \textsc{\Chopin{}}~: Polonaise en \kC \Sharp mineur, \Opus{26} \Number{1}~;
 Deux Nocturnes~; Deux Mazurkas~; Six Préludes~; Sonate en \kB \Flat mineur,
 \Opus{35}.
\end{description}

\section{Année~1949}

\begin{description}
 \item[\DateWithWeekDay{1949-01-06}]
 Moskva~: grande salle du conservatoire.

 \textsc{\Schumann{}}~: \emph{Bunte Blätter}, \Opus{99} (six pièces)~;
 Variations sur un thème de \CWieck{}, \Opus{14 (III)}~;
 \emph{Kreisleriana}, \Opus{16}.
 \textsc{\Scriabine{}}~: Neuf Préludes extraits des \Opus{13, 11, 17, 16,
 17 et~22}~; Quatre Études extraites des \Opus{8 et~42}.
 \textsc{\Rachmaninov{}}~: Trois Moments musicaux extraits de l'\Opus{16}~:
 en \kE \Flat mineur (\Number{2}), en \kE mineur (\Number{4}) et en \kC
 majeur (\Number{6}).
 \emph{Bis} -- \textsc{\Rachmaninov{}}~: Valse.
 \textsc{\Scriabine{}}~: Impromptu~; Étude en \kD \Sharp mineur, \Opus{8}
 \Number{12}.
 \item[\DateWithWeekDay{1949-01-16}]
 Moskva~: maison des scientifiques.

 \textsc{\Liszt{}}~: Funérailles, S~173 \Number{7}~; Sonate en \kB mineur,
 S~178~; \emph{Sposalizio}, S~161 \Number{1}~; \emph{Il penseroso}, S~161
 \Number{2}~; \emph{Sonetto~104 del Petrarca}, S~161 \Number{5}~; Valse
 oubliée~; Méphisto-valse.
 \emph{Bis} -- \textsc{\Liszt{}}~: \emph{Gnomenreigen}, S~145 \Number{2}~;
 \emph{Tarantella} extraite de \emph{Venezia e Napoli}, S~162 \Number{3}.
 \item[\DateWithWeekDay{1949-01-21}]
 Moskva~: grande salle du conservatoire.

 \textsc{\Beethoven{}}~: Sonate en \kF mineur, \Opus{57}~; Sonate en \kC
 \Sharp mineur, \Opus{27} \Number{2}.
 \textsc{\Chopin{}}~: Nocturne en \kC mineur, \Opus{48} \Number{1}~;
 Nocturne en \kG mineur, \Opus{37} \Number{1}~; Ballade en \kG mineur,
 \Opus{23}~; Scherzo en \kB mineur, \Opus{20}.
 \textsc{\Rachmaninov{}}~: Cinq Moments musicaux extraits de l'\Opus{16}.
 \item[B\DateWithWeekDay{1949-03-24}]
 Lettre de \VSofronitsky{} à \AVizel{} \citep[p.~169]{Nekrasova08}~: trois
 grippes et autres problèmes de santé~; tous les concerts annulés.
 Évocation de l'enregistrement en studio de trois œuvres de \Liszt{} le~9
 mars~: \emph{Sposalizio}, \emph{Il penseroso} et \emph{Canzonetta del
 Salvator Rosa}.
 Dans une lettre suivante, sans doute fin avril~1949, \Sofronitsky{} indique
 qu'il n'a pas eu de concerts depuis trois mois, mais qu'il a accepté de
 jouer le~31 mai à la grande salle du conservatoire de Moskva
 \citep[p.~169-170]{Nekrasova08} -- ce qui jette un doute sur le concert
 du~31 mars, ci-dessous, avec le même programme que le~31 mai.
 \item[\DateWithWeekDay{1949-03-31}]
 Moskva~: grande salle du conservatoire.

 \textsc{\Borodine{}}~: Petite Suite (sept pièces).
 \textsc{\Liadov{}}~: Six Préludes.
 \textsc{\Glazounov{}}~: Étude en \kE majeur, \Opus{31} \Number{3}
 (\Quote{La Nuit}).
 \textsc{\Rachmaninov{}}~: Deux Études.
 \textsc{\Scriabine{}}~: Douze Préludes~; Sonate, \Opus{53}.
 \item[1949-05 (jour inconnu)]
 Moskva~: musée \Scriabine{}.
 Concert mentionné dans les archives d'\AVizel{}
 \citep[voir][p.~169]{Nekrasova08}.
 \item[\DateWithWeekDay{1949-05-31}]
 Moskva~: grande salle du conservatoire.
 Concert radiodiffusé.
 Selon \citet[p.~170]{Nekrasova08}, il était difficile d'imaginer qu'un
 homme tourmenté par la maladie jouait.

 \textsc{\Borodine{}}~: Petite Suite.
 \textsc{\Liadov{}}~: Six Préludes extraits des \Opus{31, 36, 39 et~46}~;
 Barcarolle en \kF \Sharp majeur, \Opus{44}~; \emph{A Musical Snuffbox},
 \Opus{32}.
 \textsc{\Glazounov{}}~: Étude en \kE majeur, \Opus{31} \Number{3}
 (\Quote{La Nuit}).
 \textsc{\Rachmaninov{}}~: Deux Études-tableaux extraites de l'\Opus{33}.
 \textsc{\Scriabine{}}~: Douze Préludes extraits des \Opus{16, 37, 39, 45
 (\Number{3}) et~56 (\Number{1})}~; Sonate, \Opus{53}.
 \item[B1949 (été)]
 \VSofronitsky{} au sanatorium de Bolševo.
 Dans une lettre, annonce des concerts d'un cycle \Chopin{} pour le centième
 anniversaire de la mort du compositeur \citep[voir][p.~170]{Nekrasova08}.
 \item[\DateWithWeekDay{1949-09-09}]
 Moskva~: musée \Scriabine{}.

 \textsc{\Schubert{}}~: Impromptu~; Moment musical.
 \textsc{\Liadov{}}~: Deux Préludes~; Trois Mazurkas~; \emph{Novelette} en
 \kC majeur, \Opus{20}.
 \textsc{\Borodine{}}~: Petite Suite.
 \textsc{\Scriabine{}}~: Six Préludes~; Trois Morceaux, \Opus{45}~; Deux
 Morceaux, \Opus{57}~; Poème satanique, \Opus{36}~; Mazurka en \kE mineur,
 \Opus{25} \Number{3}.
 \textsc{\Liadov{}}~: Valse~; Prélude.
 \textsc{\Scriabine{}}~: Étude en \kD \Flat majeur, \Opus{8} \Number{10}~;
 Un Prélude extrait de l'\Opus{17}.
 \item[\DateWithWeekDay{1949-10-09}]
 Moskva~: musée \Scriabine{}.

 \textsc{\Schubert{}}~: Impromptu en \kA \Flat majeur, D~935 \Number{2}~;
 Moment musical en \kA \Flat majeur, D~780 \Number{6}.
 \textsc{\Liadov{}}~: Deux Préludes~; Trois Mazurkas extraites des \Opus{38
 et~57 (\Number{3})}.
 \textsc{\Borodine{}}~: Petite Suite.
 \textsc{\Scriabine{}}~: Six Préludes extraits des \Opus{11, 13 et~17}~;
 Trois Morceaux, \Opus{45}~; Deux Morceaux, \Opus{57}~; Poème satanique,
 \Opus{36}.
 \item[1949-10 -- 1949-12]
 En dépit de ses problèmes de santé, \VSofronitsky{} présente un cycle de
 cinq concerts pour le centième anniversaire de la mort de \Chopin{}.
 Dates des concerts~: le~20 et le~29 octobre~; le~21 novembre~; le~19 et
 le~24 décembre.
 Les programmes comportaient les Deux Sonates (\Opus{35 et~58}), les Quatre
 Ballades, les Quatre Scherzos, les Vingt-quatre Préludes (\Opus{28}), douze
 Études, trente Mazurkas, six Valses, six Nocturnes, des Impromptus, la
 Fantaisie en \kF mineur, \Opus{49}, la Barcarolle en \kF \Sharp majeur,
 \Opus{60}, la Tarentelle en \kA \Flat majeur, \Opus{43}, l'\emph{Allegro}
 de concert en \kA majeur, \Opus{46}, six Polonaises et d'autres œuvres%
 \footnote{Lettre du~11 août~1949 à \AVizel{}~:
 \foreignlanguage{russian}{\emph{Шопена люблю сейчас ужасно и по-новому...}}
 \Quote{J'aime \Chopin{} terriblement et d'une nouvelle façon à présent...}
 \citep[p.~171.]{Nekrasova08}}
 \citep[voir][p.~170]{Nekrasova08}.
 \item[\DateWithWeekDay{1949-10-20}]
 Moskva~: grande salle du conservatoire.
 Premier concert du cycle \Chopin{} pour commémorer le centième anniversaire
 de la mort du compositeur.
 Concert enregistré en partie.
 Pendant le concert, un fragment du récital a été enregistré à l'occasion de
 la radiodiffusion~: la Fantaisie, le Nocturne, la Barcarolle, les Mazurkas
 et quelques autres pièces non mentionnées ci-dessous.

 \textsc{\Chopin{}}~: Fantaisie en \kF mineur, \Opus{49}~; Nocturne en \kD
 \Flat majeur, \Opus{27} \Number{2}~; Sonate en \kB \Flat mineur,
 \Opus{35}~; Barcarolle en \kF \Sharp majeur, \Opus{60}~; Deux Mazurkas~;
 Ballade en \kA \Flat majeur, \Opus{47}~; Impromptu en \kG \Flat majeur,
 \Opus{51}~; Scherzo en \kB \Flat mineur, \Opus{31}~; Scherzo en \kB mineur,
 \Opus{20}.
 \item[\DateWithWeekDay{1949-10-29}]
 Moskva~: maison des scientifiques.
 Deuxième concert du cycle \Chopin{} pour commémorer le centième
 anniversaire de la mort du compositeur.
 Reprise du programme du~20 octobre~1949 à la grande salle du conservatoire
 de Moskva.
 \item[\DateWithWeekDay{1949-11-01}]
 Leningrad~: grande salle de la société philharmonique.
 Enregistrement de l'\hbox{Impromptu} en \kG \Flat majeur, D~899 \Number{3},
 de \Schubert{}.
 \item[\DateWithWeekDay{1949-11-21}]
 Moskva~: grande salle du conservatoire.
 Concert philharmonique.
 Troisième concert du cycle \Chopin{} pour commémorer le centième
 anniversaire de la mort du compositeur.
 Concert enregistré pour la radiodiffusion.

 \textsc{\Chopin{}}~: Nocturne en \kC mineur, \Opus{48} \Number{1}~;
 Nocturne en \kG majeur, \Opus{37} \Number{2}~; Vingt-quatre Préludes,
 \Opus{28}~; Dix Mazurkas extraites des \Opus{7, 17, 24, 30, 33, 41, 67
 et~68}~; Valse en \kA \Flat majeur, \Opus{42}~; Valse en \kD \Flat majeur,
 \Opus{64} \Number{1}~; Polonaise en \kA \Flat majeur, \Opus{53}.
 \emph{Bis} -- \textsc{\Chopin{}}~: Nocturne en \kE \Flat majeur, \Opus{9}
 \Number{2}~; Étude en \kF majeur, \Opus{25} \Number{3}~; Mazurka en \kA
 mineur, \Opus{68} \Number{2}~; Valse \Number{8} en \kA \Flat majeur,
 \Opus{64} \Number{3}~; Mazurka en \kD majeur, \Opus{33} \Number{3}.
 \item[\DateWithWeekDay{1949-11-27}]
 Moskva~: musée \Scriabine{}.
 Concert avec des œuvres de \Chopin{}.
 \item[\DateWithWeekDay{1949-12-12}]
 Moskva~: grande salle du conservatoire.
 Il est possible qu'il s'agisse du troisième concert du cycle \Chopin{} pour
 commémorer le centième anniversaire de la mort du compositeur, celui du~29
 octobre~1949 étant une reprise du premier (le~20 octobre~1949) en une autre
 salle.

 \textsc{\Chopin{}}~: Polonaise en \kC \Sharp mineur, \Opus{26} \Number{1}~;
 Nocturne en \kC \Sharp mineur, \Opus{27} \Number{1}~; Nocturne~; Sonate en
 \kB mineur, \Opus{58}~; Dix Mazurkas~; Deux Valses~; Polonaise en \kF
 \Sharp mineur, \Opus{44}.
 \emph{Bis} -- \textsc{\Chopin{}}~: Deux Nocturnes~; Étude en \kC \Sharp
 mineur.
 \item[\DateWithWeekDay{1949-12-19}]
 Moskva~: grande salle du conservatoire.
 Quatrième concert du cycle \Chopin{} pour commémorer le centième
 anniversaire de la mort du compositeur.
 \item[\DateWithWeekDay{1949-12-21}]
 Moskva~: grande salle du conservatoire.
 \VSofronitsky{} participe à un concert collectif.
 Programme inconnu.
 \item[\DateWithWeekDay{1949-12-24}]
 Moskva~: grande salle du conservatoire.
 Cinquième et dernier concert du cycle \Chopin{} pour commémorer le centième
 anniversaire de la mort du compositeur.
\end{description}

\section{Année~1950}

\begin{description}
 \item[B1950]
 Au début des années~1950, \VSofronitsky{} est membre de la commission
 d'\hbox{État} pour l'attribution du prix \Staline{}.
 \item[\DateWithWeekDay{1950-01-07}]
 Moskva~: musée \Scriabine{}.

 \textsc{\Scriabine{}}~: Cinq Préludes extraits des \Opus{11, 16 et~17}~;
 Douze Études, \Opus{8}~; Six Poèmes extraits des \Opus{51, 52 (\Number{1}
 en \kC majeur), 59 (\Number{1}), 69 et~71}~; Vers la flamme, \Opus{72}.
 \item[\DateWithWeekDay{1950-01-12}]
 Moskva ou Leningrad.
 Concert mentionné par \citet[p.~171]{Nekrasova08}.
 \item[\DateWithWeekDay{1950-01-21}]
 Moskva~: grande salle du conservatoire.

 \textsc{\JBach{}/\Busoni{}}~: Deux Préludes de chorals.
 \textsc{\JBach{}/\Ziloti{}}~: Prélude de choral en \kB mineur.
 \textsc{\Beethoven{}}~: Sonate en \kF mineur, \Opus{57}.
 \textsc{\Rachmaninov{}}~: Deux Moments musicaux extraits de l'\Opus{16}~;
 Deux Études-tableaux extraites de l'\Opus{33}.
 \textsc{\Scriabine{}}~: Six Poèmes extraits des \Opus{32, 51, 52 et~59
 (\Number{1})}~; Sonate, \Opus{53}.
 \item[\DateWithWeekDay{1950-02-04}]
 Moskva~: musée \Scriabine{}.

 \textsc{\Scriabine{}}~: Douze Préludes extraits des \Opus{13 et~11}~;
 Sonate en \kF \Sharp mineur, \Opus{23}~; Deux Morceaux extraits de
 l'\Opus{56}, dont le Prélude en \kE majeur (\Number{1})~; Poème fantastique
 en \kC majeur, \Opus{45}~; Poème, \Opus{51} \Number{4} (Danse languide)~;
 Poème, \Opus{63} \Number{1} (Masque)~; Danse, \Opus{73} \Number{1}
 (Guirlandes)~; Sonate, \Opus{70}.
 \item[\DateWithWeekDay{1950-02-15}]
 Moskva~: grande salle du conservatoire.

 \textsc{\Chopin{}}~: Prélude en \kC \Sharp mineur, \Opus{45}~; Huit
 Préludes extraits de l'\Opus{28}~: en \kD \Flat majeur (\Number{15}), en
 \kB \Flat mineur (\Number{16}), en \kA \Flat majeur (\Number{17}), en \kF
 mineur (\Number{18}), en \kE \Flat majeur (\Number{19}), en \kC mineur
 (\Number{20}), en \kB \Flat majeur (\Number{21}) et en \kG mineur
 (\Number{22})~; Nocturne en \kE \Flat majeur, \Opus{9} \Number{2}~; Ballade
 en \kF majeur, \Opus{38}~; Ballade en \kF mineur, \Opus{52}~; Quatre
 Scherzos~: \Number{1} en \kB mineur, \Opus{20}, \Number{2} en \kB \Flat
 mineur, \Opus{31}, \Number{3} en \kC \Sharp mineur, \Opus{39}, et
 \Number{4} en \kE majeur, \Opus{54}.
 \item[\DateWithWeekDay{1950-04-29}]
 Moskva ou Leningrad.
 Concert mentionné par \citet[p.~171]{Nekrasova08}.
 \item[\DateWithWeekDay{1950-05-07}]
 Moskva~: musée \Scriabine{}.

 \textsc{\Scriabine{}}~: Cinq Préludes extraits des \Opus{31, 44 (Poèmes),
 48 et~39}~; Cinq Études extraites de l'\Opus{42}~: \Number{2} en \kF \Sharp
 mineur, \Number{4} en \kF \Sharp majeur, \Number{5} en \kC \Sharp mineur,
 \Number{6} en \kD \Flat majeur et \Number{8} en \kE \Flat majeur~; Sonate
 en \kF \Sharp majeur, \Opus{30}~; Un Poème extrait de l'\Opus{52}~; Deux
 Morceaux, \Opus{57} (Désir et Caresse dansée)~; Sonate, \Opus{70}~; Vers la
 flamme, \Opus{72}.
 \item[\DateWithWeekDay{1950-05-19}]
 Moskva~: grande salle du conservatoire.
 Concert pour le~35\ieme{} anniversaire de la mort de \Scriabine{}.
 Concert enregistré en grande partie.

 \textsc{\Scriabine{}}~: Huit Préludes extraits de l'\Opus{11}~; Douze
 Études, \Opus{8}~; Cinq Poèmes extraits des \Opus{32, 51, 52 et~59
 (\Number{1})}~; Sonate en \kF \Sharp majeur, \Opus{30}~; Sonate, \Opus{53}.

 Le programme ci-dessous est extrait de la discographie établie par
 \citet[p.~29]{Nikonovich11} \citep[voir aussi][p.~392]{Scriabine}.

 \textsc{\Scriabine{}}~: Huit Préludes (en \kC majeur, \Opus{13}
 \Number{1}~; en \kA mineur, \Opus{11} \Number{2}~; en \kE mineur, \Opus{11}
 \Number{4}~; en \kD majeur, \Opus{11} \Number{5}~; en \kC \Sharp mineur,
 \Opus{11} \Number{10}~; en \kA \Flat majeur, \Opus{11} \Number{17}~; en
 \kE \Flat majeur, \Opus{11} \Number{19}~; en \kC mineur, \Opus{11}
 \Number{20})~; Neuf Études extraites de l'\Opus{8} (en \kC \Sharp majeur,
 \Number{1}~; en \kF \Sharp mineur, \Number{2}~; en \kB majeur, \Number{4}~;
 en \kE majeur, \Number{5}~; en \kA \Flat majeur, \Number{8}~; en \kB \Flat
 mineur, \Number{7}~; en \kG \Sharp mineur, \Number{9}~; en \kD \Flat
 majeur, \Number{10}~; en \kD \Sharp mineur, \Number{12})~; Poème en \kD
 majeur, \Opus{32} \Number{2}~; Poème en \kF \Sharp majeur, \Opus{32}
 \Number{1}~; Poème en \kC majeur, \Opus{52} \Number{1}~; Poème, \Opus{59}
 \Number{1}~; Sonate \Number{4} en \kF \Sharp majeur, \Opus{30}~; Sonate
 \Number{5}, \Opus{53}~; Valse en \kA \Flat majeur, \Opus{38}~; Mazurka en
 \kF \Sharp majeur, \Opus{40} \Number{2}~; Étude en \kC \Sharp mineur,
 \Opus{42} \Number{5}~; Étude en \kD \Sharp mineur, \Opus{8} \Number{12}.
 \item[\DateWithWeekDay{1950-06-02}]
 Leningrad~: grande salle de la société philharmonique.

 \textsc{\Scriabine{}}~: Huit Préludes extraits de l'\Opus{11}~; Douze
 Études, \Opus{8}~; Deux Poèmes~; Valse (œuvre incertaine)~; Sonate en \kF
 \Sharp majeur, \Opus{30}~; Sonate, \Opus{53}.
 \emph{Bis} -- \textsc{\Scriabine{}}~: Une Mazurka extraite de l'\Opus{40}~;
 Une Étude extraite de l'\Opus{42}~; Prélude~; Mazurka~; Énigme, \Opus{52}
 \Number{2}~; Trois Préludes.
 \item[\DateWithWeekDay{1950-06-12}]
 Moskva~: musée \Scriabine{}.
 Seul le programme de la première partie du concert est indiqué par
 \citet[p.~432]{Scriabine}.

 \textsc{\Scriabine{}}~: Sept Préludes~; Sonate en \kF \Sharp mineur,
 \Opus{23}.
 \item[B1950 (été)]
 \VSofronitsky{} séjourne à Plës, sur les rives de la Volga.
 \item[\DateWithWeekDay{1950-09-28}]
 Moskva~: musée \Scriabine{}.

 \textsc{\Scriabine{}}~: Polonaise en \kB \Flat mineur, \Opus{21}~;
 Impromptu en \kB \Flat mineur, \Opus{12} \Number{2}~; Quatre Préludes
 extraits des \Opus{35 et~48}~; Deux Études extraites de l'\Opus{42}~; Une
 Mazurka extraite de l'\Opus{40}~; Poème tragique, \Opus{34}~; Sonate,
 \Opus{68}~; Flammes sombres, \Opus{73} \Number{2}~; Vers la flamme,
 \Opus{72}.
 \item[\DateWithWeekDay{1950-10-05}]
 Moskva ou Leningrad.
 Concert mentionné par \citet[p.~171]{Nekrasova08}.
 \item[\DateWithWeekDay{1950-10-19}]
 Moskva~: petite salle du conservatoire.

 \textsc{\Scriabine{}}~: Huit Préludes~; Dix Études~; Sonate en \kF \Sharp
 mineur, \Opus{23}~; Poèmes~; Sonate, \Opus{53}~; Valse en \kA \Flat majeur,
 \Opus{38}.
 \item[\DateWithWeekDay{1950-10-29}]
 Moskva~: maison des scientifiques.

 \textsc{\Schubert{}}~: Deux Impromptus, dont un en \kA \Flat majeur.
 \textsc{\Liszt{}}~: Sonate en \kB mineur, S~178.
 \textsc{\Chopin{}}~: Impromptu~; Nocturne en \kF \Sharp majeur, \Opus{15}
 \Number{2}~; Mazurka.
 \textsc{\Schumann{}}~: Carnaval, \Opus{9}.
 \emph{Bis} -- \textsc{\Liszt{}}~: \emph{Gnomenreigen}, S~145 \Number{2}~;
 Valse oubliée \Number{1}, S~215 \Number{1}.
 \textsc{\Rachmaninov{}}~: Polka de~W.R. (1911).
 \item[\DateWithWeekDay{1950-11-26}]
 Moskva~: grande salle du conservatoire.
 La date et le lieu de ce concert sont indiqués par \ASofronitsky{}, mais
 pas le programme.
 \item[\DateWithWeekDay{1950-12-05}]
 Moskva~: musée \Scriabine{}.
 Dans la première partie du concert, les œuvres de \Scriabine{} présentées
 étaient des enregistrements réalisés par le compositeur lui-même sur des
 rouleaux pour le dispositif Welte-Mignon~: Prélude en \kG \Sharp mineur,
 \Opus{22} \Number{1}~; Préludes extraits des \Opus{11 et~13}~; Impromptu
 extrait de l'\Opus{14}~; Mazurka en \kF \Sharp majeur, \Opus{40}
 \Number{2}~; Poème en \kF \Sharp majeur, \Opus{32} \Number{1}~; Un Morceau
 extrait de l'\Opus{57}~; Étude en \kD \Sharp mineur, \Opus{8} \Number{12}.

 Dans la seconde partie du concert, \VSofronitsky{} a joué des œuvres du
 compositeur.
 \textsc{\Scriabine{}}~: Sonate, \Opus{68}.
 Pour la suite du programme, les sources disponibles divergent.
 Selon \NKalinenko{} -- \textsc{\Scriabine{}}~: Poèmes extraits des \Opus{71
 et~59 (\Number{1})}~; Vers la flamme, \Opus{72}~; Masque, \Opus{63}
 \Number{1}~; Énigme, \Opus{52} \Number{2}~; Poème satanique, \Opus{36}.
 Selon \ASofronitsky{} -- \textsc{\Scriabine{}}~: Un Prélude extrait de
 l'\Opus{22}~; Prélude en \kG \Flat majeur, \Opus{11} \Number{13}~; Prélude
 en \kE \Flat mineur, \Opus{11} \Number{14}~; Poème en \kF \Sharp majeur,
 \Opus{32} \Number{1}~; Mazurka en \kF \Sharp majeur, \Opus{40} \Number{2}~;
 Deux Morceaux, \Opus{57} (Désir et Caresse dansée)~; Étude en \kD \Sharp
 mineur, \Opus{8} \Number{12}~; Deux Poèmes extraits des \Opus{71 et~59
 (\Number{1})}~; Poème satanique, \Opus{36}.
 \item[\DateWithWeekDay{1950-12-28}]
 Moskva~: musée \Scriabine{}.

 \textsc{\Scriabine{}}~: Vingt-quatre Préludes, \Opus{11}.
 Selon \ASofronitsky{} -- \textsc{\Scriabine{}}~: Poèmes~; Préludes~;
 Sonate, \Opus{70}.
\end{description}

\section{Année~1951}

\begin{description}
 \item[1951-01 (jour inconnu)]
 Moskva~: union des artistes (lieu incertain).
 Soirée en hommage à \KLipskerov{} (\Dates{1889}{1954}), poète, dramaturge,
 traducteur et artiste russe de l'Âge d'argent
 (\foreignlanguage{russian}{Серебряный век}).
 Voir \citet[p.~171]{Nekrasova08}.
 \item[\DateWithWeekDay{1951-01-07}]
 Moskva~: musée \Scriabine{}.

 \textsc{\Scriabine{}}~: Sept Préludes~: en \kD \Flat majeur, \Opus{35}
 \Number{1}, en \kB \Flat mineur, \Opus{37} \Number{1}, en \kF \Sharp
 majeur, \Opus{37} \Number{2}, en \kB majeur, \Opus{37} \Number{3}, en \kD
 majeur, \Opus{39} \Number{2}, en \kG majeur, \Opus{39} \Number{3}, et en
 \kF \Sharp mineur, \Opus{31} \Number{2}~; Quatre Études extraites de
 l'\Opus{42}~; Poème ailé, \Opus{51} \Number{3}~; Poème languide en \kB
 majeur, \Opus{52} \Number{3}~; Sonate en \kF \Sharp majeur, \Opus{30}~;
 Deux Préludes extraits de l'\Opus{48}~; Cinq Poèmes~; Sonate, \Opus{70}.
 \item[\DateWithWeekDay{1951-01-22}]
 Moskva~: maison des scientifiques.

 \textsc{\Mozart{}}~: Fantaisie en \kC mineur.
 \textsc{\Beethoven{}}~: Sonate en \kF mineur, \Opus{57}.
 \textsc{\Rachmaninov{}}~: Moment musical en \kE \Flat mineur, \Opus{16}
 \Number{2}~; Étude-tableau en \kC mineur~; Étude-tableau en \kE \Flat
 majeur, \Opus{33} \Number{6}.
 \textsc{\Chopin{}}~: Nocturne en \kC mineur, \Opus{48} \Number{1}~;
 Nocturne en \kE \Flat majeur, \Opus{9} \Number{2}~; Ballade en \kA \Flat
 majeur, \Opus{47}.
 \textsc{\Scriabine{}}~: Deux Préludes.
 \item[\DateWithWeekDay{1951-01-28}]
 Selon \ASofronitsky{}, reprise du programme du concert du~22 janvier.
 Le lieu du concert est inconnu.
 \item[\DateWithWeekDay{1951-03-26}]
 Leningrad~: grande salle de la société philharmonique.

 \textsc{\Mozart{}}~: Fantaisie \Number{1}.
 \textsc{\Beethoven{}}~: Sonate en \kC \Sharp mineur, \Opus{27} \Number{2}~;
 Sonate en \kC mineur, \Opus{111}.
 \textsc{\Schubert{}}~: Fantaisie \Quote{Wanderer} en \kC majeur, D~760.
 \textsc{\Chopin{}}~: Fantaisie en \kF mineur, \Opus{49} (œuvre
 incertaine)~; Sonate en \kB \Flat mineur, \Opus{35}.
 \item[\DateWithWeekDay{1951-03-28}]
 Leningrad~: grande salle de la société philharmonique.
 Reprise du programme du concert du~26~mars.
 \item[\DateWithWeekDay{1951-03-31}]
 Moskva~: musée \Scriabine{}.

 \textsc{\Chopin{}}~: Nocturne en \kD \Flat majeur, \Opus{27} \Number{2}~;
 Sonate en \kB mineur, \Opus{58}.
 \textsc{\Rachmaninov{}}~: Moment musical en \kD \Flat majeur, \Opus{16}
 \Number{5}~; Moment musical en \kE \Flat mineur, \Opus{16} \Number{2}~;
 Deux Études-tableaux~; Deux Préludes.
 \textsc{\Scriabine{}}~: Sonate en \kF \Sharp majeur, \Opus{30}.
 \item[\DateWithWeekDay{1951-04-27}]
 Moskva~: musée \Scriabine{}.
 Concert donné le jour anniversaire de la mort de \Scriabine{}.

 \textsc{\Scriabine{}}~: Deux Préludes extraits de l'\Opus{35}~; Quatre
 Préludes, \Opus{37}~; Quatre Préludes, \Opus{39}~; Fragilité, \Opus{51}
 \Number{1}~; Trois Morceaux extraits de l'\Opus{56}~; Étude en \kE \Flat
 majeur, \Opus{49} \Number{1}~; Deux Morceaux, \Opus{57} (Désir et Caresse
 dansée)~; Poème satanique, \Opus{36}~; Poème en \kC majeur, \Opus{52}
 \Number{1}~; Deux Poèmes, \Opus{71}~; Guirlandes, \Opus{73} \Number{1}~;
 Sonate, \Opus{70}~; Prélude, \Opus{74} \Number{2} (Très lent,
 contemplatif)~; Vers la flamme, \Opus{72}.
 \item[B\DateWithWeekDay{1951-07-18} à \DateWithWeekDay{1951-08-14}]
 \VSofronitsky{} en vacances à Schodnja, près de Moskva.
 \item[\DateWithWeekDay{1951-08-28}]
 Moskva.
 Enregistrement pour la radio.
 La date ne me semble pas certaine~; il s'agit peut-être plutôt du~28
 juillet \citep[voir][p.~171]{Nekrasova08}, parce que les vacances ont été
 interrompues par les pluies et \Sofronitsky{} est parti pour Moskva le~24
 juillet, où il a effectué deux enregistrements avec ce même programme.

 \textsc{\Beethoven{}}~: Sonate en \kC mineur, \Opus{111}.
 \textsc{\Chopin{}}~: Ballade en \kA \Flat majeur, \Opus{47}.
 \item[B1951]
 \VSofronitsky{} réfléchit à de nouveaux programmes pour les prochains
 concerts à Leningrad.
 La lettre qu'il a envoyée à \AVizel{} le~9 août \citep[p.~171]{Nekrasova08}
 en donne un exemple, avec le programme suivant.

 \textsc{\Beethoven{}}~: Sonate \emph{quasi una fantasia} \Number{1}
 (\Opus{27} \Number{1})~; Sonate \emph{quasi una fantasia} \Number{2}
 (\Opus{27} \Number{2}).
 \textsc{\Schumann{}}~: Fantaisie en \kC majeur, \Opus{17}.
 \textsc{\Chopin{}}~: Ballade \Number{4} en \kF mineur, \Opus{52}~; Sept
 Études extraites de l'\Opus{25}.
 \textsc{\Liszt{}}~: \emph{Sonetto~104 del Petrarca}, S~161 \Number{5}~;
 Étude \emph{Waldesrauschen}, S~145 \Number{1}~; Étude \Number{10}.
 \item[\DateWithWeekDay{1951-09-22}]
 Moskva~: musée \Scriabine{}.
 Concert mentionné dans les archives d'\AVizel{}
 \citep[voir][p.~171]{Nekrasova08}.
 \item[\DateWithWeekDay{1951-10-20}]
 Moskva~: musée \Scriabine{}.

 \textsc{\Glazounov{}}~: Sonate en \kE mineur, \Opus{75}.
 \textsc{\Blumenfeld{}}~: Deux Fragments lyriques, \Opus{47}.
 \textsc{\Liadov{}}~: Deux Préludes~; \emph{Novelette} en \kC majeur,
 \Opus{20}~; Barcarolle en \kF \Sharp majeur, \Opus{44}~; \emph{A Musical
 Snuffbox}, \Opus{32}.
 \textsc{\Scriabine{}}~: Deux Mazurkas, \Opus{40}~; Valse en \kA \Flat
 majeur, \Opus{38}~; Polonaise en \kB \Flat mineur, \Opus{21}.
 \textsc{\Rachmaninov{}}~: Moment musical en \kE \Flat mineur, \Opus{16}
 \Number{2}~; Quatre Études-tableaux extraites des \Opus{33 et~39}.
 \item[\DateWithWeekDay{1951-10-26}]
 Moskva~: petite salle du conservatoire.
 Concert enregistré pour la radiodiffusion, enregistré en grande partie~;
 date du concert confirmée par \citet[p.~80]{Nikonovich08a}.
 Reprise du programme du concert du~20 octobre au musée \Scriabine{}.
 Le programme ci-dessous est extrait de la discographie établie par
 \citet[p.~29]{Nikonovich11} \citep[voir aussi][p.~392]{Scriabine}.

 \textsc{\Glazounov{}}~: Sonate en \kE mineur, \Opus{75}.
 \textsc{\Liadov{}}~: Prélude en \kD \Flat majeur, \Opus{57} \Number{1}~;
 Barcarolle en \kF \Sharp majeur, \Opus{44}~; \emph{A Musical Snuffbox},
 \Opus{32}~; \emph{Novelette} en \kC majeur, \Opus{20}.
 \textsc{\Scriabine{}}~: Polonaise en \kB \Flat mineur, \Opus{21}~; Valse en
 \kA \Flat majeur, \Opus{38}.
 \textsc{\Rachmaninov{}}~: Moment musical en \kE \Flat mineur, \Opus{16}
 \Number{2}~; Étude-tableau en \kE \Flat mineur, \Opus{39} \Number{5}~;
 Étude-tableau en \kC majeur, \Opus{33} \Number{2}~; Étude-tableau en \kG
 mineur, \Opus{33} \Number{7}.
 \textsc{\Blumenfeld{}}~: Deux Fragments lyriques, \Opus{47}.
 \textsc{\Scriabine{}}~: Mazurka en \kF \Sharp majeur, \Opus{40}
 \Number{2}~; Étude en \kD \Flat majeur, \Opus{8} \Number{10}.
 \textsc{\Rachmaninov{}}~: Prélude en \kG \Sharp mineur, \Opus{32}
 \Number{12}.
 \textsc{\Scriabine{}}~: Prélude en \kF mineur, \Opus{17} \Number{5}.
 \item[\DateWithWeekDay{1951-11-22}]
 Moskva~: musée \Scriabine{}.

 \textsc{\Mozart{}}~: Fantaisie en \kC mineur \Number{2}.
 \textsc{\Schumann{}}~: Fantaisie en \kC majeur, \Opus{17}.
 \textsc{\Liszt{}}~: Après une lecture de Dante, S~161 \Number{7}.
 \textsc{\Chopin{}}~: Deux Nocturnes, \Opus{27}~: en \kC \Sharp mineur
 (\Number{1}) et en \kD \Flat majeur (\Number{2})~; Impromptu en \kG \Flat
 majeur, \Opus{51}~; Ballade en \kA \Flat majeur, \Opus{47}.
 \textsc{\Scriabine{}}~: Sonate en \kF \Sharp majeur, \Opus{30}.
 \emph{Bis} -- \textsc{\Scriabine{}}~: Poème, \Opus{59} \Number{1}~;
 Ironies en \kC majeur, \Opus{56} \Number{2}.
 \textsc{\Prokofiev{}}~: Sarcasme, \Opus{17} \Number{5}.
 \item[\DateWithWeekDay{1951-11-26}]
 Moskva~: grande salle du conservatoire.
 Concert enregistré en grande partie.
 Reprise du programme du concert du~22 novembre au musée \Scriabine{}.

 Selon \ASofronitsky{} -- \textsc{\Debussy{}}~: Reflets dans l'eau, L~110
 \Number{I}.
 Selon \NKalinenko{}, cette œuvre n'a pas été jouée au musée \Scriabine{}
 ni à la grande salle du conservatoire de Moskva~; le disque compact Moscow
 State Conservatoire SMC CD~0019 semble lui donner raison.

 \emph{Bis} -- \textsc{\Chopin{}}~: Mazurka en \kE mineur, \Opus{41}
 \Number{1}.
 \textsc{\Debussy{}}~: \emph{Serenade of the Doll}, L~113 \Number{III}.
 \textsc{\Liadov{}}~: \emph{A Musical Snuffbox}, \Opus{32}.
 \textsc{\Prokofiev{}}~: Sarcasme, \Opus{17} \Number{5}.
 \item[\DateWithWeekDay{1951-12-16}]
 Moskva~: grande salle du conservatoire.
 Concert mentionné par \VSofronitsky{} dans une lettre du~3 décembre~1951
 \citep[voir][p.~172]{Nekrasova08}, avec une partie de son programme.

 \textsc{\Mozart{}}~: Fantaisie (une autre que celle jouée précédemment).
 \textsc{\Beethoven{}}~: Sonate \Number{14} en \kC \Sharp mineur, \Opus{27}
 \Number{2}.
 \textsc{\Liszt{}}~: Après une lecture de Dante, S~161 \Number{7}.
 Autres œuvres.
\end{description}

\section{Année~1952}

\begin{description}
 \item[B1952]
 \VSofronitsky{} habite rue Novopesčanaja.
 Année oblitérée par la maladie, durant laquelle le musicien donne pourtant
 dix-huit concerts dans les grandes salles de Moskva, Leningrad et Kiïv,
 sans compter sept récitals au musée \Scriabine{}.
 \item[\DateWithWeekDay{1952-01-07}]
 Moskva~: grande salle du conservatoire.
 Concert avec des œuvres de \Scriabine{} pour commémorer le quatre-vingtième
 anniversaire de la naissance du compositeur.

 \textsc{\Scriabine{}}~: \emph{Allegro} de concert en \kB \Flat mineur,
 \Opus{18}~; Sonate en \kF \Sharp mineur, \Opus{23}~; Douze Études,
 \Opus{8}~; Deux Poèmes, \Opus{32}~: en \kD majeur (\Number{2}) et en \kF
 \Sharp majeur (\Number{1})~; Poème, \Opus{59} \Number{1}~; Poème extrait de
 l'\Opus{52}~; Sonate en \kF \Sharp majeur, \Opus{30}~; Sonate, \Opus{53}.
 \item[\DateWithWeekDay{1952-01-09}]
 Moskva~: musée \Scriabine{}.
 Concert avec des œuvres de \Scriabine{} pour commémorer le quatre-vingtième
 anniversaire de la naissance du compositeur.
 Programme inconnu.
 \item[\DateWithWeekDay{1952-01-27}]
 Moskva~: musée \Scriabine{}.
 Concert à la mémoire de Vladimir Il'ič Ul'janov (\Lenin{}), à partir de ses
 œuvres préférées \citep[voir][p.~172]{Nekrasova08}.
 \item[\DateWithWeekDay{1952-01-28}]
 Leningrad~: grande salle de la société philharmonique.
 Concert reporté du~12 janvier et enregistré en grande partie.
 En plus des œuvres enregistrées, il semble que la Valse de \Chopin{} en \kD
 \Flat majeur, \Opus{70} \Number{3}, ait aussi été jouée lors de ce concert
 \citep[p.~57]{White}.
 Voir en particulier \citet[p.~444]{Milshteyn82a}.

 \textsc{\Mozart{}}~: Fantaisie en \kC mineur, K~396.
 \textsc{\Schumann{}}~: Fantaisie en \kC majeur, \Opus{17}.
 \textsc{\Liszt{}}~: Après une lecture de Dante, S~161 \Number{7}.
 \textsc{\Chopin{}}~: Deux Nocturnes, \Opus{27}~; Impromptu en \kG \Flat
 majeur, \Opus{51}~; Deux Mazurkas extraites de l'\Opus{41}~; Valse
 \Number{13} en \kD \Flat majeur, \Opus{70} \Number{3}~; Ballade en \kA
 \Flat majeur, \Opus{47}.
 \item[\DateWithWeekDay{1952-01-30}]
 Leningrad~: grande salle de la société philharmonique.
 Concert reporté du~15 janvier et enregistré en grande partie.
 En plus des œuvres enregistrées, il semble que l'\emph{Allegro} de concert
 de \Scriabine{}, \Opus{18}, et quatre Poèmes de \Scriabine{}, \Opus{52},
 \Opus{59} et \Opus{69}, aient aussi été joués lors de ce concert
 \citep[p.~57-58]{White}.
 Voir en particulier \citet[p.~444]{Milshteyn82a}.

 \textsc{\Scriabine{}}~: \emph{Allegro} de concert en \kB \Flat mineur,
 \Opus{18}~; Sonate en \kF \Sharp mineur, \Opus{23}~; Douze Études,
 \Opus{8}~; Six Poèmes extraits des \Opus{32, 52, 59 (\Number{1}) et~69}~;
 Sonate en \kF \Sharp majeur, \Opus{30}.
 \item[\DateWithWeekDay{1952-02-03}]
 Leningrad~: grande salle de la société philharmonique.
 Concert enregistré au moins en partie, mais non mentionné par
 \citet[p.~434]{Scriabine}.

 \textsc{\Mozart{}}~: Fantaisie en \kC mineur, K~475.
 \textsc{\Beethoven{}}~: Sonate en \kC \Sharp mineur, \Opus{27} \Number{2}~;
 Sonate en \kC mineur, \Opus{111}.
 \textsc{\Liszt{}}~: Funérailles, S~173 \Number{7}~; Valse oubliée
 \Number{1}, S~215 \Number{1}~; \emph{Gnomenreigen}, S~145 \Number{2}.
 \textsc{\Scriabine{}}~: Vers la flamme, \Opus{72}.
 \item[\DateWithWeekDay{1952-02-24}]
 Moskva~: musée \Scriabine{}.

 \textsc{\Scriabine{}}~: Deux Poèmes, \Opus{32}~; Dix Préludes extraits des
 \Opus{27, 37, 11 et~16}~; Trois Études extraites de l'\Opus{8}~; Sonate en
 \kF \Sharp majeur, \Opus{30}~; Fragilité, \Opus{51} \Number{1}~; Trois
 Morceaux, \Opus{45}~; Deux Morceaux, \Opus{57}~; Une Mazurka extraite de
 l'\Opus{25}~; Un Prélude extrait de l'\Opus{39}~; Deux Mazurkas,
 \Opus{40}~; Étude en \kC \Sharp mineur, \Opus{42} \Number{5}~; Sonate,
 \Opus{70}~; Vers la flamme, \Opus{72}~; Un Prélude extrait de l'\Opus{33}~;
 Une Étude en \kF \Sharp majeur extraite de l'\Opus{42}~; Prélude en \kC
 \Sharp mineur et Nocturne en \kD \Flat majeur, \Opus{9} \Number{1} et
 \Number{2}~; Prélude en \kE \Flat mineur, \Opus{16} \Number{4}~; Étude en
 \kD \Sharp mineur, \Opus{8} \Number{12}.
 \item[\DateWithWeekDay{1952-03-02}]
 Moskva~: grande salle du conservatoire.

 \textsc{\Mozart{}}~: Fantaisie en \kC mineur.
 \textsc{\Beethoven{}}~: Sonate en \kC \Sharp mineur, \Opus{27} \Number{2}.
 \textsc{\Liszt{}}~: Funérailles, S~173 \Number{7}~; \emph{Sonetto~123 del
 Petrarca}, S~161 \Number{6}~; Valse oubliée \Number{1}, S~215 \Number{1}~;
 \emph{Gnomenreigen}, S~145 \Number{2}.
 \textsc{\Debussy{}}~: \emph{Serenade of the Doll}, L~113 \Number{III}
 (concert du~14 mars~: voir ci-dessous)~; Prélude en \kA mineur, L~95
 \Number{I}~; \emph{General Lavine -- eccentric}, L~123 \Number{VI}~;
 Canope, L~123 \Number{X}~; Feux d'artifice, L~123 \Number{XII}.
 \emph{Bis} (concert du~14 mars~: voir ci-dessous) -- \textsc{\Chopin{}}~:
 Mazurka en \kF mineur.
 \textsc{\Scriabine{}}~: Étude en \kB \Flat mineur, \Opus{8}.
 \item[\DateWithWeekDay{1952-03-07}]
 Moskva~: grande salle du conservatoire.
 Concert et programme mentionnés par \citet[p.~173]{Nekrasova08}.
 Étant donné la similitude des programmes et des dates, il s'agit peut-être
 du concert du~7 mai (et non mars), ici daté de manière incorrecte.

 \textsc{\Schumann{}}~: Papillons, \Opus{2}~; Fantaisie en \kC majeur,
 \Opus{17}~; Carnaval, \Opus{9}~; \emph{Kreisleriana}, \Opus{16}.
 \item[\DateWithWeekDay{1952-03-14}]
 Moskva~: grande salle du conservatoire.
 Programme du concert~: voir concert du~2 mars plus haut.
 Récital évoqué et critiqué par \citet{Aleksandrova52}.
 \item[\DateWithWeekDay{1952-03-20}]
 Kiïv~: salle philharmonique (piano Bösendorfer).
 Concert mentionné par \VSofronitsky{} dans ses lettres du~9 et du~31 mars à
 \AVizel{} \citep[p.~173]{Nekrasova08}.
 \item[\DateWithWeekDay{1952-03-22}]
 Kiïv~: salle philharmonique (piano Bösendorfer).
 Concert mentionné par \VSofronitsky{} dans ses lettres du~9 et du~31 mars à
 \AVizel{} \citep[p.~173]{Nekrasova08}.
 \item[1952-03 (jour inconnu)]
 Kiïv~: salle philharmonique (piano Bösendorfer).
 Concert mentionné par \VSofronitsky{} dans sa lettre du~31 mars à \AVizel{}
 \citep[p.~173]{Nekrasova08}.
 \item[\DateWithWeekDay{1952-04-04}]
 Moskva~: musée \Scriabine{}.
 Concert figurant dans les archives d'\AVizel{} \citep[p.~173]{Nekrasova08}.
 \item[\DateWithWeekDay{1952-04-07}]
 Moskva~: grande salle du conservatoire.
 Concert mentionné par \ASofronitsky{}, mais sans indication de programme.
 Concert évoqué par \VSofronitsky{} dans une lettre du~31 mars à \AVizel{}
 \citep[p.~173]{Nekrasova08}, avec le programme suivant.

 \textsc{\Schumann{}}~: Papillons, \Opus{2}~; Fantaisie en \kC majeur,
 \Opus{17}~; \emph{Kreisleriana}, \Opus{16}~; Carnaval, \Opus{9}.
 \item[\DateWithWeekDay{1952-04-16}]
 Moskva~: salle \Tchaikovski{}.
 Concert non répertorié mais mentionné dans un article paru dans
 \emph{Sovetskaja muzyka}%
 \footnote{\foreignlanguage{russian}{\emph{Советская музыка}}, vol.~163,
 \Number{6} (1952), p.~96.}.
 Le programme aurait comporté des œuvres de \Schumann{}, \Chopin{}, \Liszt{}
 et \Debussy{}.
 \item[\DateWithWeekDay{1952-05-07}]
 Moskva~: grande salle du conservatoire.

 \textsc{\Schumann{}}~: \emph{Bunte Blätter}, \Opus{99}~; Fantaisie en \kC
 majeur, \Opus{17}~; \emph{Kreisleriana}, \Opus{16}~; Carnaval, \Opus{9}.
 \item[B\DateWithWeekDay{1952-05-10}]
 Rencontre de \VSofronitsky{} avec les artistes du théâtre Bol'šoj
 \citep[voir][p.~173]{Nekrasova08}.
 \item[\DateWithWeekDay{1952-05-14}]
 Leningrad~: grande salle de la société philharmonique.
 Même programme que le~7 mai~1952, avec des œuvres de \Schumann{}.
 Voir en particulier \citet[p.~444]{Milshteyn82a}.
 \item[\DateWithWeekDay{1952-05-16}]
 Leningrad~: grande salle de la société philharmonique.
 Concert avec des œuvres de \Liszt{} \citep[voir][p.~223]{Zhukova08}.

 \textsc{\Liszt{}}~: Funérailles, S~173 \Number{7}~; Au Lac de Wallenstadt,
 S~160 \Number{2}~; Églogue, S~160 \Number{7}~; Sonate en \kB mineur,
 S~178~; \emph{Sposalizio}, S~161 \Number{1}~; Deux \emph{Sonetto del
 Petrarca}~; \emph{Gnomenreigen}, S~145 \Number{2}~; Feux follets, S~139
 \Number{5}~; Méphisto-valse \Number{1}, S~514.
 \item[\DateWithWeekDay{1952-05-18}]
 Leningrad~: grande salle de la société philharmonique.
 Concert avec des œuvres de \Liszt{}.
 Le programme ci-dessous est indiqué par \citet[p.~66]{Gakkel86}.

 \textsc{\Liszt{}}~: Chapelle de Guillaume Tell, S~160 \Number{1}~; Au Lac
 de Wallenstadt, S~160 \Number{2}~; Églogue, S~160 \Number{7}~; Sonate en
 \kB mineur, S~178~; \emph{Sposalizio}, S~161 \Number{1}~; \emph{Sonetto~123
 del Petrarca}, S~161 \Number{6}~; \emph{Sonetto~104 del Petrarca}, S~161
 \Number{5}~; \emph{Gnomenreigen}, S~145 \Number{2}~; Feux follets, S~139
 \Number{5}~; Méphisto-valse.
 \emph{Bis} -- \textsc{\Liszt{}}~: Valse oubliée.
 \textsc{\Chopin{}}~: Étude \Number{15} en \kF majeur, \Opus{25} \Number{3}.
 \textsc{\Debussy{}}~: Prélude en \kA mineur, L~95 \Number{I}.
 \item[\DateWithWeekDay{1952-06-15}]
 Moskva~: musée \Scriabine{}.
 Concert mentionné dans l'archive d'\AVizel{} \citep[p.~174]{Nekrasova08},
 mais sans indication de programme.
 \item[\DateWithWeekDay{1952-06-18}]
 Moskva~: musée \Scriabine{}.
 Concert mentionné dans l'archive d'\AVizel{} \citep[p.~174]{Nekrasova08},
 mais sans indication de programme.
 \item[\DateWithWeekDay{1952-06-29}]
 Moskva~: musée \Scriabine{}.
 Concert mentionné dans l'archive d'\AVizel{} \citep[p.~174]{Nekrasova08},
 avec le programme suivant.

 \textsc{\Scriabine{}}~: \emph{Allegro} de concert en \kB \Flat mineur,
 \Opus{18}~; Polonaise en \kB \Flat mineur, \Opus{21}~; Quatre Préludes
 (\Opus{16} \Number{1} en \kB majeur~; \Opus{17} \Number{4} en \kB \Flat
 mineur~; \Opus{11} \Number{13} en \kG \Flat majeur~; \Opus{11} \Number{6}
 en \kB mineur)~; Mazurka en \kG \Sharp mineur, \Opus{3} \Number{9}~; Trois
 Mazurkas extraites de l'\Opus{25} (\Number{3} en \kE mineur~; \Number{8} en
 \kB majeur~; \Number{7} en \kF \Sharp mineur)~; Valse en \kA \Flat majeur,
 \Opus{38}~; Un Prélude extrait de l'\Opus{39}~; Prélude en \kF \Sharp
 majeur, \Opus{33} \Number{2}~; Prélude en \kE \Flat mineur, \Opus{31}
 \Number{3}~; Poème ailé, \Opus{51} \Number{3}~; Poème languide en \kB
 majeur, \Opus{52} \Number{3}~; Sonate \Number{4} en \kF \Sharp majeur,
 \Opus{30}~; Sonate \Number{10}, \Opus{70}.
 \item[B1952 (été)]
 \VSofronitsky{} en vacances à Marfino, district municipal de Moskva.
 \item[B1952 (automne~?)]
 \DSerov{} (\Dates{1924}{1991}) est assistant de \VSofronitsky{} au
 conservatoire de Moskva.
 Voir lettre de \Sofronitsky{} à \AVizel{} le~20 novembre~1952, citée par
 \citet[p.~174]{Nekrasova08}.
 \item[\DateWithWeekDay{1952-10-10}]
 Moskva~: grande salle du conservatoire.
 Concert enregistré en partie.

 \textsc{\Beethoven{}}~: Sonate en \kD majeur, \Opus{28}~; Sonate en \kF
 mineur, \Opus{57}.
 \textsc{\Schumann{}}~: Carnaval, \Opus{9}.
 \textsc{\Rachmaninov{}}~: \emph{Oriental Sketch} en \kB \Flat majeur
 (1917)~; Deux Études-tableaux, \Opus{39}~: en \kB mineur (\Number{4}) et en
 \kA mineur (\Number{6}).
 \textsc{\Prokofiev{}}~: Toccata en \kD mineur, \Opus{11}.
 \item[\DateWithWeekDay{1952-11-09}]
 Moskva~: grande salle du conservatoire.

 \textsc{\Schubert{}}~: Fantaisie \Quote{Wanderer} en \kC majeur, D~760.
 \textsc{\Chopin{}}~: Ballade en \kG mineur, \Opus{23}~; Nocturne en \kG
 majeur, \Opus{37} \Number{2}~; Scherzo en \kB mineur, \Opus{20}.
 \textsc{\Kabalevski{}}~: Sonatine, \Opus{13}.
 \textsc{\Rachmaninov{}}~: Étude-tableau en \kC majeur, \Opus{33}
 \Number{2}~; Étude-tableau en \kE \Flat mineur, \Opus{39} \Number{5}~;
 Étude-tableau en \kA mineur, \Opus{39} \Number{6}.
 \textsc{\Scriabine{}}~: Six Études extraites des \Opus{8, 42 et~65}.
\end{description}

\section{Année~1953}

\begin{description}
 \item[B1953-01]
 Les concerts de janvier~1953 ne sont répertoriés ni dans la chronologie, ni
 dans les archives d'\AVizel{} \citep[p.~174]{Nekrasova08}.
 \item[\DateWithWeekDay{1953-01-07}]
 Moskva~: musée \Scriabine{}.
 La seconde partie du programme ci-dessous ne coïncide pas avec celle donnée
 par \citet[p.~174]{Nekrasova08}, qui indique, sans plus de détails, des
 œuvres de \Kabalevski{}, \Rachmaninov{} et \Scriabine{}~; la seconde partie
 commence avec la Sonate \Number{10}, \Opus{70}.

 \textsc{\Scriabine{}}~: \emph{Allegro} de concert en \kB \Flat mineur,
 \Opus{18}~; Quatre Préludes~: en \kB \Flat mineur, \Opus{11} \Number{16},
 en \kD mineur, \Opus{17} \Number{1}, en \kB \Flat mineur, \Opus{17}
 \Number{4}, et en \kB mineur, \Opus{13} \Number{6}~; Sept Études~: en \kB
 majeur, \Opus{8} \Number{4}, en \kB \Flat mineur, \Opus{8} \Number{7}, en
 \kF \Sharp mineur, \Opus{42} \Number{2}, en \kF \Sharp majeur, \Opus{42}
 \Number{4}, en \kC \Sharp mineur, \Opus{42} \Number{5}, et Deux Études
 extraites de l'\Opus{65}~; Poème en \kC majeur, \Opus{52} \Number{1}~;
 Un Poème (peut-être l'\Opus{59} \Number{1})~; Poème ailé, \Opus{51}
 \Number{3}~; Énigme, \Opus{52} \Number{2}~; Sonate, \Opus{70}~; Vers la
 flamme, \Opus{72}.
 \emph{Bis} -- \textsc{\Scriabine{}}~: Feuillet d'album en \kE \Flat majeur,
 \Opus{45} \Number{1}~; Prélude en \kA \Flat majeur, \Opus{31} \Number{4}~;
 Prélude en \kA \Flat majeur, \Opus{39} \Number{4}~; Prélude en \kF mineur,
 \Opus{17} \Number{5}~; Étude en \kD \Flat majeur, \Opus{8} \Number{10}.
 \item[\DateWithWeekDay{1953-01-15}]
 Moskva~: petite salle du conservatoire.
 Concert évoqué par \VSofronitsky{} dans une lettre à \AVizel{} le~2
 février~1953 \citep[voir][p.~174]{Nekrasova08}~; il mentionne une Sonate de
 \Glazounov{}.
 \item[\DateWithWeekDay{1953-01-30}]
 Moskva~: petite salle du conservatoire.
 Programme mentionné par \VSofronitsky{} dans une lettre à \AVizel{} le~2
 février~1953 \citep[voir][p.~174]{Nekrasova08}.

 \textsc{\Mendelssohn{}}~: Variations sérieuses en \kD mineur, \Opus{54}.
 \textsc{\Schumann{}}~: Trois \emph{Novelettes} extraites de l'\Opus{21}
 (\Number{1}, \Number{7} et \Number{8}).
 \textsc{\Chopin{}}~: Ballade en \kF mineur, \Opus{52}.
 \textsc{\Kabalevski{}}~: Sonatine.
 \textsc{\Rachmaninov{}}~: Trois Études-tableaux.
 \textsc{\Scriabine{}}~: Trois Morceaux, \Opus{45}~; Deux Préludes extraits
 de l'\Opus{33}~; Deux Préludes extraits de l'\Opus{17}~; Étude en \kB \Flat
 mineur extraite de l'\Opus{8}~; Énigme, \Opus{52} \Number{2}~; Étude en \kD
 \Flat majeur, \Opus{8} \Number{10}.
 \item[\DateWithWeekDay{1953-02-24}]
 Moskva~: musée \Scriabine{}.
 Concert.
 Programme inconnu.
 \item[\DateWithWeekDay{1953-02-25}]
 Moskva~: grande salle du conservatoire.
 Dans la lettre ci-dessus du~2 février~1953 à \AVizel{}, \VSofronitsky{}
 mentionne son \Quote{prochain concert prévu le~25 février dans la grande
 salle du conservatoire} \citep[voir][p.~174]{Nekrasova08} -- un programme
 qu'il souhaite répéter à Leningrad, en y apportant peut-être un troisième
 concert avec la Sonate Pastorale de \Beethoven{}, \Opus{28}.

 \textsc{\Schubert{}}~: Fantaisie en \kC majeur, D~760.
 \textsc{\Chopin{}}~: Nocturne en \kG majeur, \Opus{37} \Number{2}~;
 Barcarolle en \kF \Sharp majeur, \Opus{60}~; Scherzo \Number{2} en \kB
 \Flat mineur, \Opus{31}.
 \textsc{\Liszt{}}~: Chapelle de Guillaume Tell, S~160 \Number{1}~;
 \emph{Sonetto~104 del Petrarca}, S~161 \Number{5}~; \emph{Venezia e
 Napoli}, S~162 (\Number{1}, \Number{2} et \Number{3}).
 \item[\DateWithWeekDay{1953-03-01}]
 Moskva~: grande salle du conservatoire.
 Concert.
 Programme non mentionné par \citet[p.~435]{Scriabine}.

 \textsc{\Schubert{}}~: Deux Impromptus extraits du recueil D~899~;
 Fantaisie en \kC majeur, D~760.
 \textsc{\Chopin{}}~: Ballade \Number{1} en \kG mineur, \Opus{23}.
 \textsc{\Liszt{}}~: Chapelle de Guillaume Tell, S~160 \Number{1}~; Au Lac
 de Wallenstadt, S~160 \Number{2}~; Deux Études de concert~; \emph{Venezia e
 Napoli} (\emph{Gondoliera}, S~162 \Number{1}~; \emph{Canzone}, S~162
 \Number{2}~; \emph{Tarantella}, S~162 \Number{3}).
 \emph{Bis} -- \textsc{\Schubert{}/\Liszt{}}~: \emph{Der Müller und der
 Bach}, S~565 \Number{2}~; \emph{Auf dem Wasser zu singen}, S~558
 \Number{2}.
 \textsc{\Borodine{}}~: Étude~(?).
 \textsc{\Rachmaninov{}}.
 \item[\DateWithWeekDay{1953-03-19}]
 Moskva~: grande salle du conservatoire.

 \textsc{\JBach{}/\Busoni{}}~: Deux Préludes de chorals.
 \textsc{\Chopin{}}~: Fantaisie en \kF mineur, \Opus{49}~; Six Préludes
 extraits de l'\Opus{28}~: en \kF \Sharp majeur (\Number{13}), en \kE \Flat
 mineur (\Number{14}), en \kD \Flat majeur (\Number{15}), en \kB \Flat
 mineur (\Number{16}), en \kA \Flat majeur (\Number{17}), en \kF mineur
 (\Number{18}), Ballade en \kF mineur, \Opus{52}~; Un Nocturne extrait de
 l'\Opus{48}~; Un Nocturne extrait de l'\Opus{37}.
 \textsc{\Rachmaninov{}}~: Prélude en \kD majeur, \Opus{23} \Number{4}~;
 Prélude en \kG majeur, \Opus{32} \Number{5}~; Étude-tableau en \kE \Flat
 mineur, \Opus{39} \Number{5}~; Étude-tableau en \kG mineur, \Opus{33}
 \Number{7}~; Étude-tableau en \kC \Sharp mineur, \Opus{33} \Number{8}.
 \item[\DateWithWeekDay{1953-03-28}]
 Leningrad~: grande salle de la société philharmonique.
 Voir en particulier \citet[p.~444]{Milshteyn82a}.

 \textsc{\JBach{}/\Busoni{}}~: Deux Préludes de chorals.
 \textsc{\Chopin{}}~: Fantaisie en \kF mineur, \Opus{49}~; Six Préludes~;
 Ballade en \kF mineur, \Opus{52}~; Deux Nocturnes extraits des \Opus{48} et
 \Opus{37}.
 \textsc{\Rachmaninov{}}~: Moment musical en \kB mineur, \Opus{16}
 \Number{3}~; Six Études-tableaux.
 \item[\DateWithWeekDay{1953-03-30}]
 Leningrad~: grande salle de la société philharmonique.
 Voir en particulier \citet[p.~444]{Milshteyn82a}.

 \textsc{\Mendelssohn{}}~: Variations sérieuses en \kD mineur, \Opus{54}.
 \textsc{\Schumann{}}~: Trois \emph{Novelettes} extraites de l'\Opus{21}.
 \textsc{\Chopin{}}~: Ballade en \kG mineur, \Opus{23}~; Scherzo en \kB
 \Flat mineur, \Opus{31}~; Impromptu en \kG \Flat majeur, \Opus{51}.
 \textsc{\Liszt{}}~: \emph{La leggierezza}, S~144 \Number{2}.
 \textsc{\Rachmaninov{}}~: Moments musicaux en \kD \Flat majeur, \Opus{16}
 \Number{5}, et en \kE \Flat mineur, \Opus{16} \Number{2}.
 \textsc{\Scriabine{}}~: Six Études extraites de l'\Opus{8}, de l'\Opus{42}
 et de l'\Opus{65}.
 \item[\DateWithWeekDay{1953-04-25}]
 Moskva~: salle \Tchaikovski{}.

 \textsc{\Haendel{}}~: Variations.
 \textsc{\Haydn{}}~: Sonate.
 \textsc{\Schumann{}}~: \emph{Kreisleriana}, \Opus{16}~; \emph{Bunte
 Blätter}, \Opus{99}~; Carnaval, \Opus{9}.
 \item[\DateWithWeekDay{1953-04-27}]
 Moskva~: musée \Scriabine{}.

 \textsc{\Scriabine{}}~: Huit Préludes~; Sonate en \kF \Sharp mineur,
 \Opus{23}~; \emph{Allegro} de concert en \kB \Flat mineur, \Opus{18}~; Six
 Poèmes~; Sonate, \Opus{70}.
 \emph{Bis} -- \textsc{\Scriabine{}}~: Prélude, \Opus{74} \Number{2}~;
 Énigme, \Opus{52} \Number{2}~; Feuillet d'album en \kE \Flat majeur,
 \Opus{45} \Number{1}~; Désir, \Opus{57} \Number{1}~; Prélude en \kA \Flat
 majeur, \Opus{39} \Number{4}~; Mazurka en \kF \Sharp majeur, \Opus{40}
 \Number{2}~; Un Prélude extrait de l'\Opus{17}.
 \item[\DateWithWeekDay{1953-05-30}]\phantomsection\label{rec:1953-05-30}
 Moskva~: grande salle du conservatoire.
 Selon \VSofronitsky{}, il s'agissait de l'un des meilleurs concerts de sa
 vie \citep[voir][p.~175]{Nekrasova08}, mais il n'a pas été enregistré.

 \textsc{\Liszt{}}~: Rhapsodie hongroise \Number{3}, S~244 \Number{3}~;
 Sonate en \kB mineur, S~178 (centenaire de l'œuvre)~; \emph{Sonetto~123 del
 Petrarca}, S~161 \Number{6}~; \emph{Sonetto~104 del Petrarca}, S~161
 \Number{5}~; Étude de concert en \kF mineur~; Feux follets, S~139
 \Number{5}~; Méphisto-valse~; Rhapsodie hongroise \Number{2}, S~244
 \Number{2}.
 \item[\DateWithWeekDay{1953-07-08}]
 Moskva~: musée \Scriabine{}.

 \textsc{\Scriabine{}}~: Six Préludes~: en \kF \Sharp majeur, \Opus{39}
 \Number{1}, en \kD \Flat majeur/\kC majeur, \Opus{31} \Number{1}, en \kD
 \Flat majeur, \Opus{35} \Number{1}, en \kB \Flat mineur, \Opus{37}
 \Number{1}, en \kF \Sharp majeur, \Opus{48} \Number{1}, et en \kC majeur,
 \Opus{48} \Number{2}~; Deux Morceaux extraits de l'\Opus{56}~; Caresse
 dansée, \Opus{57} \Number{2}~; Poème satanique, \Opus{36}~; Deux Préludes,
 \Opus{27}~; Prélude en \kG majeur, \Opus{39} \Number{3}~; Prélude en \kE
 \Flat majeur, \Opus{31} \Number{3}~; Quatre Poèmes, \Opus{69} et
 \Opus{71}~; Prélude, \Opus{74} \Number{2}~; Vers la flamme, \Opus{72}.
 \item[B1953 (été)]
 \VSofronitsky{} en vacances à Melluži (Jūrmala), non loin de Riga, dans la
 maison de l'académie des sciences agricoles.
 \item[\DateWithWeekDay{1953-10-15}]
 Moskva~: grande salle du conservatoire.

 \textsc{\Beethoven{}}~: \emph{Andante} favori en \kF majeur, WoO~57.
 \textsc{\Schumann{}}~: \emph{Faschingsschwank aus Wien}, \Opus{26}.
 \textsc{\Liszt{}}~: Rhapsodie hongroise \Number{9} en \kE \Flat majeur,
 S~244 \Number{9} (Carnaval de Pest).
 \textsc{\Schubert{}/\Liszt{}}~: Soirées de Vienne~: Valses-caprices
 \Number{4}, \Number{5}, \Number{6} et \Number{7}.
 \textsc{\Liszt{}}~: Méphisto-valse~; Rhapsodie hongroise \Number{2} en \kC
 \Sharp mineur/\kF \Sharp majeur, S~244 \Number{2}~; \emph{La Campanella},
 S~141 \Number{3}.
 \item[B\DateWithWeekDay{1953-11-06}]
 Lettre d'\hbox{Aleksandr} Aleksandrovič Fadeev à \VSofronitsky{}, exprimant
 ses impressions après un concert du pianiste%
 \footnote{\foreignlanguage{russian}{\emph{Советская музыка}}, vol.~277,
 \Number{12} (1961), p.~85.}.
 Voir \citet[p.~432]{Milshteyn82a} pour la retranscription et
 \citet[p.~12-13]{White} pour une traduction en anglais.
 Selon \citet[p.~175]{Nekrasova08}, il s'agit du concert du~30 mai~1953,
 évoqué en page~\pageref{rec:1953-05-30}.
 \item[B\DateWithWeekDay{1953-11-10} et~12]
 Dans une lettre du~31 octobre~1953, \VSofronitsky{} invite à l'écouter
 le~10 novembre, et évoque même un départ en Allemagne le~12 novembre, pour
 un séjour de trois semaines au minimum, avec au moins dix concerts dans
 plusieurs villes allemandes \citep[voir][p.~175]{Nekrasova08}.
 Le voyage en Allemagne a été annulé, sur ordre des médecins, en raison de
 l'état de santé cardiaque du musicien.
 Le programme ci-dessous est indiqué par \Sofronitsky{} pour le~10 novembre,
 dans la lettre du~31 octobre.

 \textsc{\Schubert{}}~: Trois Impromptus~; Fantaisie en \kC majeur, D~760.
 \textsc{\Schumann{}}~: Études symphoniques, \Opus{13}.
 \textsc{\Schubert{}/\Liszt{}}~: Deux Lieder~; Valse-caprice \Number{8}.
 \item[\DateWithWeekDay{1953-12-25}]
 Moskva~: salle \Tchaikovski{}.
 Concert pour commémorer le~125\ieme{} anniversaire de la mort de
 \Schubert{}, enregistré en totalité, sauf le \emph{bis}.

 \textsc{\Schubert{}}~: Sonate en \kA mineur, D~784~; Fantaisie en \kC
 majeur, D~760 (\Quote{Wanderer}).
 \textsc{\Schubert{}/\Liszt{}}~: \emph{Die Stadt}~; \emph{Die junge Nonne}~;
 \emph{Am Meer}~; \emph{Frühlingsglaube}~; \emph{Auf dem Wasser zu singen}~;
 \emph{Der Doppelgänger}~; \emph{Erlkönig}.
 \emph{Bis} -- \textsc{\Schubert{}/\Liszt{}}~: Valse-caprice en \kD majeur.
\end{description}

\section{Année~1954}

\begin{description}
 \item[\DateWithWeekDay{1954-01-07}]
 Moskva~: musée \Scriabine{}.

 \textsc{\Scriabine{}}~: Cinq Préludes, \Opus{74}~; Deux Préludes,
 \Opus{67}~; Deux Poèmes, \Opus{71}~; Guirlandes, \Opus{73} \Number{1}~;
 Flammes sombres, \Opus{73} \Number{2}~; Vers la flamme, \Opus{72}~; Trois
 Poèmes~: Poème en \kC majeur, \Opus{52} \Number{1}, Poème, \Opus{59}
 \Number{1}, et Poème, \Opus{69} \Number{1}~; Guirlandes, \Opus{73}
 \Number{1}~; Sonate en \kF \Sharp majeur, \Opus{30}.
 \item[\DateWithWeekDay{1954-01-19}]
 Leningrad~: grande salle de la société philharmonique.
 Voir en particulier \citet[p.~444]{Milshteyn82a}.

 \textsc{\Beethoven{}}~: \emph{Andante} favori en \kF majeur, WoO~57.
 \textsc{\Schumann{}}~: \emph{Faschingsschwank aus Wien}, \Opus{26}.
 \textsc{\Chopin{}}~: Ballade en \kG mineur, \Opus{23}.
 \textsc{\Liszt{}}~: Rhapsodie hongroise \Number{9} en \kE \Flat majeur,
 S~244 \Number{9} (Carnaval de Pest)~; Quatre Valses-caprices (d'après
 \Schubert{})~: \Number{4}, \Number{5}, \Number{6} et \Number{7}~; Rhapsodie
 hongroise \Number{2} en \kC \Sharp mineur/\kF \Sharp majeur, S~244
 \Number{2}.
 \item[\DateWithWeekDay{1954-01-22}]
 Leningrad~: grande salle de la société philharmonique.
 Même programme que le~25 décembre~1953.
 Voir en particulier \citet[p.~444]{Milshteyn82a}.
 \item[\DateWithWeekDay{1954-01-24}]
 Leningrad~: petite salle de la société philharmonique.
 Voir en particulier \citet[p.~444]{Milshteyn82a}.

 \textsc{\Schubert{}}~: Impromptu en \kA \Flat majeur, D~935 \Number{2}~;
 Impromptu en \kG \Flat majeur, D~899 \Number{3}.
 \textsc{\Schumann{}}~: \emph{Bunte Blätter}, \Opus{99}~;
 \emph{Kreisleriana}, \Opus{16}.
 \textsc{\Chopin{}}~: Scherzo.
 \textsc{\Rachmaninov{}}~: Deux Préludes.
 \textsc{\Medtner{}}~: Marche funèbre en \kB mineur, \Opus{31} \Number{2}~;
 Deux \emph{Skazki} extraits de l'\Opus{20}.
 \textsc{\Scriabine{}}~: Sonate en \kF \Sharp majeur, \Opus{30}.
 \item[\DateWithWeekDay{1954-02-21}]
 Moskva~: musée \Scriabine{}.

 \textsc{\Scriabine{}}~: Préludes extraits des \Opus{39, 48, 49 (\Number{2})
 et~33}~; Quatre Études extraites de l'\Opus{42}~; Sonate, \Opus{68}~; Deux
 Préludes, \Opus{27}~; Deux Poèmes, \Opus{32}~; Quatre Préludes, \Opus{31}.
 \item[\DateWithWeekDay{1954-02-27}]
 Moskva~: musée \Scriabine{}.
 Concert mentionné par \citet[p.~175]{Nekrasova08}.
 \item[\DateWithWeekDay{1954-03-01}]
 Moskva~: grande salle du conservatoire.
 Concert radiodiffusé \citep[voir][p.~175-176]{Nekrasova08}.

 \textsc{\Beethoven{}}~: \emph{Rondo} en \kG majeur, \Opus{51} \Number{2}.
 \textsc{\Schumann{}}~: Arabesque en \kC majeur, \Opus{18}~; Études
 symphoniques, \Opus{13}.
 \textsc{\Ravel{}}~: Sonatine.
 \textsc{\Medtner{}}~: Marche funèbre en \kB mineur, \Opus{31} \Number{2}~;
 \emph{Novelette} en \kC mineur, \Opus{17} \Number{2}~; \emph{Skazka} en \kF
 mineur, \Opus{26} \Number{3}~; \emph{Skazka} en \kB mineur, \Opus{20}
 \Number{2}.
 \textsc{\Balakirev{}}~: \emph{Islamey}, \Opus{18}.
 \emph{Bis} -- \textsc{\Mendelssohn{}}~: Étude en \kA mineur, \Opus{104}
 \Number{3}.
 \textsc{\Grieg{}}~: Nocturne en \kC majeur, extrait du livre~V des Pièces
 lyriques, \Opus{54} \Number{4}.
 \textsc{\Beethoven{}}~: Marche extraite des Ruines d'\hbox{Athènes}.
 \item[\DateWithWeekDay{1954-03-25}]
 Moskva~: grande salle du conservatoire.

 \textsc{\Mozart{}}~: Sonate.
 \textsc{\Schumann{}}~: \emph{Humoreske} en \kB \Flat majeur, \Opus{20}.
 \textsc{\Scriabine{}}~: Sonate, \Opus{68}~; Deux Danses, \Opus{73}~:
 Flammes sombres (\Number{2}) et Guirlandes (\Number{1}).
 \textsc{\Prokofiev{}}~: Sarcasmes, \Opus{17} \Number{2}, \Number{3} et
 \Number{5}.
 \textsc{\Debussy{}}~: \emph{Doctor Gradus ad Parnassum}, L~113 \Number{I}~;
 Prélude en \kA mineur, L~95 \Number{I}~; Deux Préludes.
 \item[\DateWithWeekDay{1954-03-28}]
 Moskva~: musée \Scriabine{}.
 Concert.
 Programme inconnu.
 \item[\DateWithWeekDay{1954-03-30}]
 Leningrad~: société philharmonique.
 Concert mentionné par \citet[p.~177]{Nekrasova08}, à partir des archives
 d'\AVizel{}.

 \textsc{\Mendelssohn{}}~: Variations sérieuses en \kD mineur, \Opus{54}.
 \textsc{\Schumann{}}~: Trois \emph{Novelettes} extraites de l'\Opus{21}.
 \textsc{\Chopin{}}~: Ballade \Number{1} en \kG mineur, \Opus{23}~; Scherzo
 \Number{2} en \kB \Flat mineur, \Opus{31}~; Impromptu \Number{3} en \kG
 \Flat majeur, \Opus{51}.
 \textsc{\Liszt{}}~: Étude de concert en \kF mineur.
 \textsc{\Rachmaninov{}}~: Deux Moments musicaux.
 \textsc{\Scriabine{}}~: Six Études extraites des \Opus{8, 42 et~65}.
 \item[\DateWithWeekDay{1954-04-04}]
 Moskva~: grande salle du conservatoire.
 Concert.
 Programme inconnu.
 \item[\DateWithWeekDay{1954-04-27}]
 Moskva~: musée \Scriabine{}.

 \textsc{\Scriabine{}}~: Étude en \kC \Sharp mineur, \Opus{2} \Number{1}~;
 Prélude pour la main gauche en \kC \Sharp mineur, \Opus{9} \Number{1}~;
 Huit Études extraites de l'\Opus{8}~; Prélude en \kE majeur, \Opus{56}
 \Number{1}~; Trois Morceaux, \Opus{45}~; Trois Morceaux, \Opus{49}~; Deux
 Morceaux extraits de l'\Opus{56}~; Poème satanique, \Opus{36}.
 \emph{Bis} -- \textsc{\Scriabine{}}~: Un Prélude extrait de l'\Opus{48}
 (œuvre incertaine)~; Un Poème extrait de l'\Opus{32} (œuvre incertaine)~;
 Prélude en \kE \Flat majeur, \Opus{31} \Number{3}~; Étude en \kD \Sharp
 mineur, \Opus{8} \Number{12}~; Poème en \kF \Sharp majeur, \Opus{32}
 \Number{1}~; Prélude en \kB majeur, \Opus{27} \Number{2}~; Poème en \kD
 majeur, \Opus{32} \Number{2}~; Prélude en \kF mineur, \Opus{17} \Number{5}.
 \item[\DateWithWeekDay{1954-05-10}]
 Moskva~: maison des scientifiques.

 \textsc{\Schumann{}}~: \emph{Novelette} en \kF majeur, \Opus{21}
 \Number{1}~; \emph{Novelette} en \kE majeur, \Opus{21} \Number{7}~;
 \emph{Bunte Blätter}, \Opus{99} \Number{1, 3, 4, 6 et~8}~; Carnaval,
 \Opus{9}.
 \textsc{\Chopin{}}~: Nocturne en \kF \Sharp majeur, \Opus{15} \Number{2}~;
 Étude en \kG \Flat majeur, \Opus{10} \Number{5}~; Étude en \kE \Flat
 mineur, \Opus{10} \Number{6}~; Étude en \kF mineur, \Opus{10} \Number{9}~;
 Étude en \kF majeur, \Opus{25} \Number{3}~; Mazurka en \kF mineur,
 \Opus{63} \Number{2}~; Mazurka en \kC majeur, \Opus{33} \Number{2}.
 \textsc{\Debussy{}}~: \emph{Minstrels}, L~117 \Number{XII}~; \emph{General
 Lavine -- eccentric}, L~123 \Number{VI}~; Canope, L~123 \Number{X}~;
 Feux d'artifice, L~123 \Number{XII}.
 \emph{Bis} -- \textsc{\Goltz{}}~: Scherzo en \kE mineur.
 \textsc{\Tchaikovski{}}~: \emph{Ruines d'un château} extraites du
 \emph{Souvenir de Hapsal}, \Opus{2} \Number{1}~; \emph{Nuits de mai}
 extraites des \emph{Saisons}, \Opus{37a} \Number{5}.
 \textsc{\Moussorgski{}/\Rachmaninov{}}~: \emph{Gopak} extrait de \emph{La
 Foire de Soročinskij}.
 \item[\DateWithWeekDay{1954-05-25}]
 Kiïv.
 Concert.
 Le lieu exact et le programme sont inconnus.
 Selon \ASofronitsky{}, ce concert comportait les œuvres suivantes de
 \Chopin{}~: Nocturne~; Six Études~; Scherzo.
 \item[\DateWithWeekDay{1954-05-27}]
 Kiïv.
 Concert.
 Lieu exact et programme inconnus.
 \item[\DateWithWeekDay{1954-06-11}]
 Moskva~: petite salle du conservatoire.
 Concert enregistré en grande partie.
 \citet[p.~437]{Scriabine} indiquent la date du~12~juin (le lendemain), de
 même qu'un article paru dans \emph{Sovetskaja muzyka}%
 \footnote{\foreignlanguage{russian}{\emph{Советская музыка}}, vol.~189,
 \Number{8} (1954), p.~93.}.
 Toutes les autres sources consultées indiquent cependant la date
 du~11~juin, en particulier \citet[p.~177]{Nekrasova08}.

 \textsc{\Borodine{}}~: Petite Suite.
 \textsc{\Liadov{}}~: Quatre Préludes, dont les \Opus{46} \Number{1} en \kB
 \Flat majeur, \Opus{39} \Number{2} et \Opus{36} \Number{3} en \kG majeur.
 \textsc{\Rachmaninov{}}~: Prélude en \kD majeur, \Opus{23} \Number{4}~;
 Prélude en \kG majeur, \Opus{32} \Number{5}~; Deux Études-tableaux, dont
 l'\Opus{39} \Number{4} en \kB mineur.
 \textsc{\Scriabine{}}~: Deux Études extraites de l'\Opus{42}~: en \kF
 \Sharp mineur (\Number{2}) et en \kF \Sharp majeur (\Number{3})~; Sonate en
 \kF \Sharp majeur, \Opus{30}.
 \textsc{\Prokofiev{}}~: Cinq Visions fugitives, \Opus{22} \Number{1, 2, 3,
 7 et~11}.
 \textsc{\Kabalevski{}}~: Sonatine en \kC majeur, \Opus{13} \Number{1}.
 \textsc{\Krioukov{}}~: Rhapsodie russe (l'œuvre est jouée pour la première
 fois).
 \textsc{\Goltz{}}~: Scherzo en \kE mineur.
 \textsc{\Balakirev{}}~: \emph{Islamey}, \Opus{18}.
 \emph{Bis} -- \textsc{\Mendelssohn{}}~: Étude en \kA mineur, \Opus{104}
 \Number{3}.
 \textsc{\Debussy{}}~: \emph{General Lavine -- eccentric}, L~123
 \Number{VI}.
 \textsc{\Chopin{}}~: Étude en \kG \Flat majeur, \Opus{10} \Number{5}.
 \textsc{\Scriabine{}}~: Poème en \kF \Sharp majeur, \Opus{32} \Number{1}~;
 Prélude en \kG \Flat majeur, \Opus{11} \Number{13}~; Prélude en \kE \Flat
 mineur, \Opus{11} \Number{14}~; Prélude en \kC majeur extrait de
 l'\Opus{48}~; Prélude en \kA mineur~; Étude en \kD \Flat majeur, \Opus{8}
 \Number{10}.
 \textsc{\Rachmaninov{}}~: Prélude en \kG \Sharp mineur, \Opus{32}
 \Number{12}.
 \item[\DateWithWeekDay{1954-06-19}]
 Leningrad~: petite salle du conservatoire.
 Concert.
 Reprise du programme du concert du~11~juin.
 \item[\DateWithWeekDay{1954-06-21}]
 Leningrad~: grande salle de la société philharmonique.
 Voir en particulier \citet[p.~444]{Milshteyn82a}.

 \textsc{\Mozart{}}~: Sonate en \kE \Flat majeur, K~282.
 \textsc{\Schumann{}}~: \emph{Humoreske} en \kB \Flat majeur, \Opus{20}.
 \textsc{\Chopin{}}~: Nocturne, \Opus{15}~; Mazurka, \Opus{50}~; Quatre
 Études~; Scherzo en \kB mineur, \Opus{20}.
 \textsc{\Debussy{}}~: \emph{Minstrels}, L~117 \Number{XII}~; \emph{General
 Lavine -- eccentric}, L~123 \Number{VI}~; Prélude en \kA mineur, L~95
 \Number{I}~; Feux d'artifice, L~123 \Number{XII}.
 \textsc{\Tchaikovski{}}~: Les Saisons, \Opus{37a} (\Number{5}, Nuits de
 mai).
 \textsc{\Moussorgski{}/\Rachmaninov{}}~: \emph{Gopak}, extrait de \emph{La
 Foire de Soročinskij}.
 \item[B1954-08]
 Bref séjour à Dubulti (Jūrmala), non loin de Riga.
 Travail sur les programmes de concerts de la saison~1954-1955.
 \item[B\DateWithWeekDay{1954-10-02}]
 Moskva~: rue Novopesčanaja.
 Enregistrement d'une leçon de \VSofronitsky{} à deux de ses anciens élèves,
 \PLobanov{} et \NFeigina{} \citep[voir][p.~350]{Lobanov08b}.
 Publication sur disque compact~: Prometheus Editions EDITION003.
 \item[\DateWithWeekDay{1954-10-10}]
 Moskva~: grande salle du conservatoire.

 \textsc{\Grieg{}}~: Ballade en \kG mineur, \Opus{24}.
 \textsc{\Chopin{}}~: Sonate en \kB \Flat mineur, \Opus{35}.
 \textsc{\Scriabine{}}~: Sonate, \Opus{68}.
 \textsc{\Debussy{}}~: Reflets dans l'eau, L~110 \Number{I}.
 \textsc{\Liszt{}}~: Étude en \kF mineur~; \emph{Gnomenreigen}, S~145
 \Number{2}~; Méphisto-valse \Number{1}, S~514.
 \emph{Bis} -- \textsc{\Chopin{}}~: Nocturne en \kF \Sharp majeur, \Opus{15}
 \Number{2}~; Scherzo en \kB mineur, \Opus{20}.
 \textsc{\Scriabine{}}~: Mazurka en \kE mineur, \Opus{25} \Number{3}.
 \item[\DateWithWeekDay{1954-10-16}]
 Moskva~: musée \Scriabine{}.
 Voir en particulier \citet[p.~448]{Milshteyn82a}.

 \textsc{\Borodine{}}~: Petite Suite.
 \textsc{\Liadov{}}~: Cinq Préludes.
 \textsc{\Rachmaninov{}}~: Deux Préludes en \kD majeur et en \kG majeur,
 \Opus{23} \Number{4} et \Opus{32} \Number{5}.
 \textsc{\Chopin{}}~: Scherzo en \kB mineur, \Opus{20}.
 \textsc{\Scriabine{}}~: Sonate, \Opus{68}~; Deux Poèmes, \Opus{71}~; Deux
 Préludes, \Opus{74} \Number{4} et \Opus{74} \Number{2}~; Vers la flamme,
 \Opus{72}.
 \citet[p.~178]{Nekrasova08} y ajoute des Études et une Mazurka.
 \item[\DateWithWeekDay{1954-11-12}]
 Leningrad~: grande salle de la société philharmonique.
 Dernier concert de \VSofronitsky{} à Leningrad.
 Voir en particulier \citet[p.~444-445]{Milshteyn82a}.

 \textsc{\Grieg{}}~: Ballade en \kG mineur, \Opus{24}.
 \textsc{\Chopin{}}~: Sonate en \kB \Flat mineur, \Opus{35}.
 \textsc{\Scriabine{}}~: Sonate, \Opus{68}.
 \textsc{\Debussy{}}~: Reflets dans l'eau, L~110 \Number{I}~; La Soirée dans
 Grenade, L~100 \Number{II}.
 \textsc{\Liszt{}}~: Méphisto-valse.
 \item[B\DateWithWeekDay{1954-11-14}]
 Un concert annoncé le~14 novembre à la grande salle de la société
 philharmonique de Leningrad, avec des œuvres de \Haendel{}, \Schumann{} et
 \Chopin{}, n'a pas eu lieu.
 Sans mentionner cette annulation du concert, \citet[p.~178]{Nekrasova08} en
 fournit le programme suivant.

 \textsc{\Haendel{}}~: \emph{Aria} et variations en \kE majeur.
 \textsc{\Schumann{}}~: \emph{Bunte Blätter}, \Opus{99} (œuvre entière).
 \textsc{\Chopin{}}~: Nocturne~; Deux Mazurkas~; Barcarolle en \kF \Sharp
 majeur, \Opus{60}~; Tarentelle en \kA \Flat majeur, \Opus{43}~; Scherzo
 \Number{2} en \kB \Flat mineur, \Opus{31}~; Scherzo \Number{3} en \kC
 \Sharp mineur, \Opus{39}.
 \item[\DateWithWeekDay{1954-12-29}]
 Moskva~: musée \Scriabine{}.
 Concert enregistré en partie.

 \textsc{\Debussy{}}~: \emph{Doctor Gradus ad Parnassum}, L~113 \Number{I}~;
 Reflets dans l'eau, L~110 \Number{I}.
\end{description}

\section{Année~1955}

\begin{description}
 \item[B1955]\phantomsection\label{bio:1955}
 Année charnière qui voit \Sofronitsky{} s'éloigner des \Quote{grandes
 salles} de concert, pour la première fois de sa vie artistique et six ans
 avant son décès.
 \item[B1955 (hiver)]
 Début du travail de \VSofronitsky{} sur la Septième Sonate en \kB \Flat
 majeur de \SProkofiev{}, \Opus{83} \citep[voir][p.~388-394]{Shiryaeva}.
 \item[B1955-1958]
 La période~1955-1958 comporte tout le travail éditorial de \VSofronitsky{},
 qui y consacre une partie de son temps, tandis qu'il ne joue qu'au musée
 \Scriabine{} et dans d'autres petites salles de concert.
 Il travaille ainsi à l'édition des œuvres pour piano des compositeurs
 suivants~: \AScriabine{} (Huit Études, \Opus{42}, en~1955~; Troisième
 Sonate en \kF \Sharp mineur, \Opus{23}, en~1956~; Quatrième Sonate en \kF
 \Sharp majeur, \Opus{30}, en~1957~; Valse en \kA \Flat majeur, \Opus{38},
 en~1957~; Trois Études, \Opus{65}, en~1957), \NMedtner{} (Sonate en \kG
 mineur, \Opus{22}, en~1957~; Triade de Sonates, \Opus{11}, en~1958) et
 \MRavel{} (Œuvres pour piano, en deux volumes, en~1958-1962).
 Toutes ces éditions ont été publiées par le département moscovite de
 \foreignlanguage{russian}{Музгиз} [Muzgiz], la maison d'édition musicale
 d'\hbox{État}.
 Voir \citet[p.~126 et note en bas de page]{Nikonovich08a}.
 \item[\DateWithWeekDay{1955-01-07}]
 Moskva~: musée \Scriabine{}.
 Voir en particulier \citet[p.~448]{Milshteyn82a}.

 \textsc{\Scriabine{}}~: Un Poème extrait de l'\Opus{52}~; Deux Poèmes,
 \Opus{69}~; Deux Poèmes, \Opus{71}~; Deux Danses, \Opus{73}~; Sonate,
 \Opus{70}~; Prélude en \kE majeur, \Opus{56} \Number{1}~; Deux Préludes,
 \Opus{67}~; Deux Morceaux extraits de l'\Opus{51}~; Mazurka en \kE mineur,
 \Opus{25} \Number{3}~; Valse en \kA \Flat majeur, \Opus{38}~; Deux
 Préludes, \Opus{74} \Number{4} et \Opus{74} \Number{2}~; Vers la flamme,
 \Opus{72}.
 \emph{Bis} -- \textsc{\Scriabine{}}~: Cinq Préludes extraits des \Opus{11,
 22, 33 et~39}~; Deux Morceaux, \Opus{57}.
 \item[\DateWithWeekDay{1955-01-14}]
 Moskva~: grande salle du conservatoire.
 Concert enregistré en partie (œuvres de \Scriabine{}).
 Il s'agit peut-être du dernier concert de \VSofronitsky{} à la grande salle
 du conservatoire de Moskva \citep[voir][p.~14]{White}, mais voir cependant
 page~\pageref{bio:LateMCGH}.

 \textsc{\Schumann{}}~: Arabesque en \kC majeur, \Opus{18}~; Sonate en \kF
 mineur, \Opus{14}.
 \textsc{\Chopin{}}~: Fantaisie en \kF mineur, \Opus{49}.
 \textsc{\Chostakovitch{}}~: Dix Préludes extraits de l'\Opus{34}.
 \textsc{\Scriabine{}}~: Deux Préludes, \Opus{67}~; Deux Danses, \Opus{73}~:
 Flammes sombres (\Number{2}) et Guirlandes (\Number{1})~; Sonate,
 \Opus{53}.
 \item[\DateWithWeekDay{1955-01-26}]
 Moskva~: salle \Tchaikovski{}.

 \textsc{\Schubert{}}~: Impromptu en \kA \Flat majeur, D~935 \Number{2}~;
 Impromptu en \kG (\Flat) majeur, D~899 \Number{3}~; Impromptu en \kA \Flat
 majeur, D~899 \Number{4}.
 \textsc{\Liszt{}}~: Sonate en \kB mineur, S~178.
 \textsc{\Scriabine{}}~: Deux Poèmes, \Opus{32}~; Sonate, \Opus{53}~;
 Sonate, \Opus{70}~; Vers la flamme, \Opus{72}.
 \item[\DateWithWeekDay{1955-04-10}]
 Moskva~: musée \Scriabine{}.
 Concert enregistré en partie.
 Voir en particulier \citet[p.~448]{Milshteyn82a}.

 \textsc{\Prokofiev{}}~: Douze Visions fugitives extraites de l'\Opus{22}~;
 Cinq Sarcasmes, \Opus{17}~; Pièce, \Opus{3} \Number{1} (Conte).
 \textsc{\Scriabine{}}~: Prélude en \kF majeur, \Opus{49} \Number{2}~;
 Prélude en \kC majeur, \Opus{33} \Number{3}~; Masque, \Opus{63}
 \Number{1}~; Vers la flamme, \Opus{72}~; Douze Études extraites des
 \Opus{2 (\Number{1} en \kC \Sharp mineur), 8 et~42}~; Valse en \kA \Flat
 majeur, \Opus{38}~; Deux Morceaux, \Opus{57}~: Désir (\Number{1}) et
 Caresse dansée (\Number{2})~; Un Prélude extrait de l'\Opus{17}.
 \emph{Bis} -- \textsc{\Scriabine{}}~: Mazurka en \kE mineur, \Opus{25}
 \Number{3}~; Prélude en \kC majeur, \Opus{31} \Number{4}~; Énigme,
 \Opus{52} \Number{2}.
 \item[\DateWithWeekDay{1955-04-29}]
 Moskva~: musée \Scriabine{}.
 Voir en particulier \citet[p.~449]{Milshteyn82a}.

 \emph{Première partie} -- \textsc{\Scriabine{}}~: œuvres de la première
 période de la vie du compositeur.
 \emph{Deuxième partie} -- \textsc{\Scriabine{}}~: œuvres de la période
 médiane et de la fin de la vie du compositeur.
 \item[\DateWithWeekDay{1955-05-14}]
 Moskva~: musée \Scriabine{}.
 Voir en particulier \citet[p.~449]{Milshteyn82a}.
 Selon \citet[p.~391]{Shiryaeva}, première interprétation en public de la
 Septième Sonate de \Prokofiev{} par \Sofronitsky{}.

 \textsc{\Chostakovitch{}}~: Deux Préludes et fugues extraits de
 l'\Opus{87}~; Huit Préludes.
 \textsc{\Prokofiev{}}~: Sonate en \kB \Flat majeur, \Opus{83}.
 \textsc{\Scriabine{}}~: Sonate, \Opus{68}~; Deux Poèmes, \Opus{69}~;
 Ironies en \kC majeur, \Opus{56} \Number{2}~; Poème satanique, \Opus{36}.
 \item[\DateWithWeekDay{1955-05-22}]
 Moskva~: musée \Scriabine{}.
 Date mentionnée par \citet[p.~178]{Nekrasova08}, qui cite les archives de
 \citeauthor{Shiryaeva}, pour un concert de \Sofronitsky{} au musée~; cette
 date n'apparaît pourtant pas dans la liste de \citet[p.~449]{Milshteyn82a},
 en note de l'article de \citet{Shiryaeva}.
 Un peu plus loin, \citeauthor{Nekrasova08} évoque même onze concerts au
 musée durant cette \Quote{année fatidique}~1955, en plus des autres
 concerts.
 \item[\DateWithWeekDay{1955-06-11}]
 Moskva~: musée \Scriabine{}.
 Concert enregistré en partie.
 Voir en particulier \citet[p.~449]{Milshteyn82a}.
 Selon \citet[p.~391]{Shiryaeva}, deuxième interprétation en public de la
 Septième Sonate de \Prokofiev{} par \Sofronitsky{}~; cet auteur évoque, en
 outre, des Visions fugitives de \Prokofiev{}, \Opus{22}.

 \textsc{\Glazounov{}}~: Prélude et fugue extrait de l'\Opus{101}.
 \textsc{\Chostakovitch{}}~: Deux Préludes et fugues, \Opus{87} \Number{3}
 (en \kG majeur) et \Opus{87} \Number{9} (en \kE majeur).
 \textsc{\Prokofiev{}}~: Sonate en \kB \Flat majeur, \Opus{83}~; Sonate en
 \kA mineur, \Opus{28}~; Contes de la vieille grand-mère, \Opus{31}.
 \textsc{\Scriabine{}}~: Mazurka en \kE mineur, \Opus{25} \Number{3}~;
 Mazurka en \kB majeur, \Opus{25} \Number{8}~; Étude en \kF \Sharp majeur,
 \Opus{42} \Number{4}~; Étude en \kC \Sharp mineur, \Opus{42} \Number{5}~;
 Six Poèmes, \Opus{51} \Number{3} (Poème ailé), \Opus{52}, \Opus{63}
 \Number{1} (Masque), \Opus{69}, \Opus{71} \Number{2} et \Opus{72} (Vers la
 flamme).
 \emph{Bis} -- \textsc{\Scriabine{}}~: Étude en \kC \Sharp majeur, \Opus{8}
 \Number{1}~; Prélude en \kE \Flat mineur, \Opus{16} \Number{4}~; Prélude en
 \kE \Flat mineur, \Opus{11} \Number{14}~; Prélude en \kF \Sharp majeur,
 \Opus{33} \Number{2}~; Prélude en \kE \Flat mineur, \Opus{31} \Number{3}~;
 Prélude en \kC majeur, \Opus{31} \Number{4}~; Prélude en \kF mineur,
 \Opus{17} \Number{5}.
 \item[\DateWithWeekDay{1955-06-25}]
 Moskva~: musée \Scriabine{}.
 Concert enregistré en partie.
 Voir en particulier \citet[p.~449]{Milshteyn82a}.

 \textsc{\Scarlatti{}}~: Deux Sonates.
 \textsc{\Mendelssohn{}}~: Variations sérieuses en \kD mineur, \Opus{54}.
 \textsc{\Schumann{}}~: Carnaval, \Opus{9}.
 \textsc{\Chopin{}}~: Ballade en \kG mineur, \Opus{23}.
 \textsc{\Schubert{}/\Liszt{}}~: \emph{Der Müller und der Bach}~; \emph{Auf
 dem Wasser zu singen}~; \emph{Der Doppelgänger} (en \emph{bis}).
 \textsc{\Liszt{}}~: Feux follets, S~139 \Number{5}~; \emph{Gnomenreigen},
 S~145 \Number{2}~; Valse oubliée \Number{1}, S~215 \Number{1} (en
 \emph{bis}).
 \textsc{\Scriabine{}}~: Six Études extraites des \Opus{42 et~65}.
 \item[B1955 (été)]
 \VSofronitsky{} séjourne à Nikolo-Prozorovo, au nord de Moskva.
 Une carte postale, envoyée par \Sofronitsky{} à \citet[p.~398]{Shiryaeva},
 depuis sa villégiature de Marfino, évoque en termes peu amènes son travail
 pédagogique avec les élèves et étudiants.
 \item[B\DateWithWeekDay{1955-08-31}]
 \VSofronitsky{} écrit une lettre au bureau du procureur militaire en charge
 de la réhabilitation de \VMeyerhold{}.
 Voir \citet[p.~367]{NikonovichScriabine08}.
 \item[\DateWithWeekDay{1955-10-02}]
 Moskva~: musée \Scriabine{}.
 Voir en particulier \citet[p.~449]{Milshteyn82a}.

 \textsc{\Chopin{}}~: Polonaise en \kC \Sharp mineur, \Opus{26} \Number{1}~;
 Vingt-cinq Préludes~; Scherzo en \kB mineur, \Opus{20}.
 \textsc{\Scriabine{}}~: Étude en \kB majeur, \Opus{8} \Number{4}~; Étude en
 \kB \Flat mineur, \Opus{8} \Number{7}~; Étude en \kB \Flat mineur, \Opus{8}
 \Number{11}~; Deux Danses, \Opus{73}~: Flammes sombres (\Number{2}) et
 Guirlandes (\Number{1}).
 \textsc{\Debussy{}}~: L'\hbox{Isle} joyeuse, L~106.
\end{description}

\section{Année~1956}

\begin{description}
 \item[1956]
 Concert en un lieu inconnu et à une date imprécise.

 \textsc{\Chopin{}}~: Polonaise en \kC \Sharp mineur, \Opus{26} \Number{1}~;
 Vingt-quatre Préludes~; Scherzo en \kB mineur, \Opus{20}.
 \textsc{\Scriabine{}}~: Étude en \kB majeur, \Opus{8} \Number{4}~; Étude en
 \kB \Flat mineur, \Opus{8} \Number{7}~; Étude en \kB \Flat mineur, \Opus{8}
 \Number{11}~; Deux Danses, \Opus{73}~: Flammes sombres (\Number{2}) et
 Guirlandes (\Number{1}).
 \textsc{\Debussy{}}~: L'\hbox{Isle} joyeuse, L~106.
 \item[\DateWithWeekDay{1956-02-12}]
 Moskva~: musée \Scriabine{}.
 \citet[p.~179]{Nekrasova08} mentionne cette date pour un concert de
 \VSofronitsky{} et indique le programme suivant.

 \textsc{\Schubert{}}~: Sonate en \kB \Flat majeur, D~960.
 \textsc{\Mendelssohn{}}~: Variations sérieuses en \kD mineur, \Opus{54}.
 \textsc{\Schumann{}}~: Carnaval, \Opus{9}.
 \item[\DateWithWeekDay{1956-02-29}]
 Moskva~: musée \Scriabine{}.
 Concert enregistré en partie.

 \textsc{\Schumann{}}~: Arabesque en \kC majeur, \Opus{18}~; Sonate en \kF
 mineur, \Opus{14} (troisième mouvement).
 \item[\DateWithWeekDay{1956-03-07}]
 Moskva~: musée \Scriabine{}.
 Concert enregistré en grande partie.
 Voir en particulier \citet[p.~449]{Milshteyn82a}.

 \textsc{\Schubert{}}~: Sonate en \kB \Flat majeur, D~960.
 \textsc{\Chopin{}}~: Huit Mazurkas~: en \kD \Flat majeur, \Opus{30}
 \Number{3}, en \kC \Sharp mineur, \Opus{30} \Number{4}, en \kC majeur,
 \Opus{33} \Number{3}, en \kB mineur, \Opus{33} \Number{4}, en \kC \Sharp
 mineur, \Opus{41} \Number{1}, en \kE mineur, \Opus{41} \Number{2}, en \kC
 \Sharp mineur, \Opus{50} \Number{3}, et en \kF mineur, \Opus{63}
 \Number{2}~; Cinq Valses~: en \kB mineur, \Opus{69} \Number{2}, en \kG \Flat
 majeur, \Opus{70} \Number{1}, en \kF mineur, \Opus{70} \Number{2}, en \kD
 \Flat majeur, \Opus{70} \Number{3}, et en \kD \Flat majeur, \Opus{64}
 \Number{1}~; Deux Préludes~: en \kB \Flat mineur, \Opus{28} \Number{16}, et
 en \kF mineur, \Opus{28} \Number{18}.
 \textsc{\Scriabine{}}~: Mazurka en \kF \Sharp majeur, \Opus{40}
 \Number{2}~; Mazurka en \kE mineur, \Opus{25} \Number{3}~; Valse en \kA
 \Flat majeur, \Opus{38}.
 \item[\DateWithWeekDay{1956-04-27}]
 Moskva~: musée \Scriabine{}.
 Concert enregistré en partie.
 Voir en particulier \citet[p.~449]{Milshteyn82a}.

 \textsc{\Scriabine{}}~: \emph{Allegro} de concert en \kB \Flat mineur,
 \Opus{18}~; Prélude en \kB \Flat majeur, \Opus{35} \Number{2}~; Prélude en
 \kD \Flat majeur, \Opus{35} \Number{1}~; Prélude en \kD \Flat majeur,
 \Opus{11} \Number{15}~; Prélude en \kF \Sharp majeur, \Opus{16}
 \Number{5}~; Sonate en \kF \Sharp mineur, \Opus{23}~; Sonate en \kF \Sharp
 majeur, \Opus{30}~; Six Poèmes extraits des \Opus{32, 51, 71 et~63}~;
 Sonate, \Opus{53}.
 \item[\DateWithWeekDay{1956-05-12}]
 Moskva~: musée \Scriabine{}.
 Voir en particulier \citet[p.~449]{Milshteyn82a}.

 \textsc{\Liszt{}}~: \emph{Sposalizio}, S~161 \Number{1}~; \emph{Il
 penseroso}, S~161 \Number{2}~; \emph{Canzonetta del Salvator Rosa}, S~161
 \Number{3}~; Sonate en \kB mineur, S~178.
 \textsc{\Rachmaninov{}}~: Moment musical en \kE \Flat mineur, \Opus{16}
 \Number{2}~; Moment musical en \kD \Flat majeur, \Opus{16} \Number{5}.
 \textsc{\Scriabine{}}~: Étude en \kB \Flat mineur, \Opus{8} \Number{7}~;
 Étude en \kB \Flat mineur, \Opus{8} \Number{11}~; Deux Danses, \Opus{73}.
 \textsc{\Debussy{}}~: L'\hbox{Isle} joyeuse, L~106.
 \emph{Bis} -- \textsc{\Debussy{}}~: Feuilles mortes, L~123 \Number{II}.
 \textsc{\Medtner{}}~: Marche funèbre en \kB mineur, \Opus{31} \Number{2}.
 \textsc{\Scriabine{}}~: Poèmes~; Préludes.
 \item[\DateWithWeekDay{1956-05-31}]
 Moskva~: musée \Scriabine{}.
 Concert enregistré en partie.
 Voir en particulier \citet[p.~449]{Milshteyn82a}.

 \textsc{\Schubert{}}~: Sonate en \kA mineur, D~784.
 \textsc{\Schumann{}}~: \emph{Kreisleriana}, \Opus{16}.
 \textsc{\Scriabine{}}~: Sonate, \Opus{68}.
 \textsc{\Debussy{}}~: Reflets dans l'eau, L~110 \Number{I}~; Prélude en \kA
 mineur, L~95 \Number{I}~; \emph{Serenade of the Doll}, L~113 \Number{III}~;
 \emph{Minstrels}, L~117 \Number{XII}~; \emph{General Lavine -- eccentric},
 L~123 \Number{VI}~; Canope, L~123 \Number{X}~; Feux d'artifice, L~123
 \Number{XII}.
 \emph{Bis} -- \textsc{\Chopin{}}~: Nocturne~; Étude~; Prélude.
 \textsc{\Scriabine{}}~: Prélude en \kE \Flat mineur, \Opus{11} \Number{14}.
 \item[\DateWithWeekDay{1956-06-14}]
 Moskva~: musée \Scriabine{}.
 Concert enregistré en partie.
 Voir en particulier \citet[p.~449]{Milshteyn82a}.

 \textsc{\Beethoven{}}~: \emph{Andante} favori en \kF majeur, WoO~57~;
 Sonate en \kC mineur, \Opus{111}.
 \textsc{\Medtner{}}~: Sonate en \kC majeur, \Opus{11} \Number{3}~; Deux
 \emph{Skazki}, \Opus{26}~: \Number{1} en \kE \Flat majeur et \Number{3} en
 \kF mineur~; \emph{Skazka} en \kE mineur, \Opus{14} \Number{2}~;
 \emph{Skazka} en \kB mineur, \Opus{20} \Number{2}.
 \textsc{\Rachmaninov{}}~: Moment musical en \kE \Flat mineur, \Opus{16}
 \Number{2}~; Moment musical en \kB mineur, \Opus{16} \Number{3}~; Prélude
 en \kG majeur, \Opus{32} \Number{5}~; Prélude en \kG \Sharp mineur,
 \Opus{32} \Number{12}~; Étude-tableau en \kB mineur, \Opus{39} \Number{4}~;
 Étude-tableau en \kA mineur, \Opus{39} \Number{6}.
 \textsc{\Liszt{}}~: Méphisto-valse.
 \emph{Bis} -- \textsc{\Scriabine{}}.
 \item[\DateWithWeekDay{1956-06-25}]
 Moskva~: musée \Scriabine{}.
 Concert enregistré en partie.

 \textsc{\Schumann{}}~: \emph{Kreisleriana}, \Opus{16}.
 \textsc{\Medtner{}}~: Sonate Élégie en \kD mineur, \Opus{11} \Number{2}~;
 Idylle en \kB mineur, \Opus{7} \Number{1}~; \emph{Skazka} en \kE mineur,
 \Opus{14} \Number{2}~; \emph{Novelette} en \kC mineur, \Opus{17}
 \Number{2}~; Deux \emph{Skazki}, \Opus{20}~: \Number{1} en \kB \Flat
 mineur et \Number{2} en \kB mineur.
 \textsc{\Scriabine{}}~: Sonate en \kF \Sharp majeur, \Opus{30}.
 \item[\DateWithWeekDay{1956-06-29}]
 Moskva~: musée \Scriabine{}.
 Voir en particulier \citet[p.~449]{Milshteyn82a}.
 La date du~29~juin est indiquée par \citeauthor{Shiryaeva}, citée par
 \citet[p.~439]{Scriabine}.
 Il s'agit sans doute du même concert que le précédent, le~25~juin, mais
 avec une date erronée.
 Certains enregistrements spécifient néanmoins la date du~25 ou du~29~juin.

 \textsc{\Schumann{}}~: \emph{Humoreske} en \kB \Flat majeur, \Opus{20}.
 \textsc{\Medtner{}}~: Sonate Élégie en \kD mineur, \Opus{11} \Number{2}~;
 Idylle en \kB mineur, \Opus{7} \Number{1}~; \emph{Skazka} en \kE mineur,
 \Opus{14} \Number{2}~; \emph{Novelette} en \kC mineur, \Opus{17}
 \Number{2}~; Deux \emph{Skazki}, \Opus{20}~: \Number{1} en \kB \Flat
 mineur et \Number{2} en \kB mineur.
 \textsc{\Scriabine{}}~: Sonate en \kF \Sharp majeur, \Opus{30}.
 \item[B1956 (été)]
 \VSofronitsky{} séjourne non loin de Moskva.
 \item[\DateWithWeekDay{1956-09-25}]
 Moskva~: musée \Scriabine{}.
 Voir en particulier \citet[p.~449]{Milshteyn82a}.

 \textsc{\Schubert{}}~: Un Impromptu extrait du recueil D~935~; Deux
 Impromptus extraits du recueil D~899.
 \textsc{\Chopin{}}~: Cinq Préludes~; Barcarolle en \kF \Sharp majeur,
 \Opus{60}~; Tarentelle en \kA \Flat majeur, \Opus{43}~; Ballade en \kF
 mineur, \Opus{52}.
 \textsc{\Scriabine{}}~: Prélude en \kD \Flat majeur, \Opus{17} \Number{3}~;
 Prélude en \kE \Flat mineur, \Opus{11} \Number{14}~; Prélude en \kE \Flat
 majeur, \Opus{11} \Number{19}~; Trois Études extraites de l'\Opus{42}~;
 Sonate en \kF \Sharp majeur, \Opus{30}.
 \item[\DateWithWeekDay{1956-10-14}]
 Moskva~: musée \Scriabine{}.
 Voir en particulier \citet[p.~449]{Milshteyn82a} et
 \citet[p.~395]{Shiryaeva}.

 \textsc{\Haendel{}}~: Air et variations en \kE majeur.
 \textsc{\Beethoven{}}~: Rondo en \kC majeur, \Opus{51} \Number{1}.
 \textsc{\Chopin{}}~: Vingt-quatre Préludes.
 \textsc{\Scriabine{}}~: Vingt-quatre Préludes.
 \emph{Bis} -- \textsc{\Scriabine{}}~: Valse~; Une Mazurka extraite de
 l'\Opus{3}.
 \textsc{\Prokofiev{}}~: Sonate en \kB \Flat majeur, \Opus{83} (troisième
 mouvement).
 \item[\DateWithWeekDay{1956-12-02}]
 Moskva~: musée \Scriabine{}.
 Concert enregistré en partie.
 Voir en particulier \citet[p.~449]{Milshteyn82a} et
 \citet[p.~395]{Shiryaeva}~: \Quote{Désormais, je commencerai chaque concert
 par [des œuvres de] \Bach{}.}

 \textsc{\JBach{}}~: Suite française \Number{5} en \kG majeur.
 \textsc{\Chopin{}}~: Sonate en \kB mineur, \Opus{58}~; Nocturne en \kG
 majeur, \Opus{37} \Number{2}~; Nouvelle Étude \Number{2} en \kA \Flat
 majeur~; Étude en \kE \Flat mineur, \Opus{10} \Number{6}~; Étude en \kF
 mineur, \Opus{10} \Number{9}~; Étude en \kF majeur, \Opus{25} \Number{3}~;
 Étude en \kA mineur, \Opus{25} \Number{4}~; Deux Études~; Mazurka en \kF
 mineur, \Opus{7} \Number{3}~; Mazurka en \kC \Sharp mineur, \Opus{30}
 \Number{4}~; Mazurka en \kG \Sharp mineur, \Opus{33} \Number{1}~; Mazurka
 en \kC \Sharp mineur, \Opus{41} \Number{1}~; Mazurka en \kA \Flat majeur,
 \Opus{41} \Number{4}~; Mazurka en \kB majeur, \Opus{63} \Number{1}.
 \textsc{\Scriabine{}}~: Mazurka en \kD \Flat majeur, \Opus{40} \Number{1}~;
 Mazurka en \kF \Sharp majeur, \Opus{40} \Number{2}~; Polonaise en \kB \Flat
 mineur, \Opus{21}.
 \item[\DateWithWeekDay{1956-12-09}]
 Moskva~: musée \Scriabine{}.
 Concert enregistré en partie.
 Voir en particulier \citet[p.~450]{Milshteyn82a}.

 \textsc{\JBach{}}~: Quatre Préludes de chorals en \kG mineur, en \kE \Flat
 mineur, en majeur et en \kF mineur.
 \textsc{\Beethoven{}}~: Sonate en \kC \Sharp mineur, \Opus{27} \Number{2}~;
 Sonate en \kF mineur, \Opus{57}.
 \textsc{\Chopin{}}~: Fantaisie en \kF mineur, \Opus{49}.
 \textsc{\Scriabine{}}~: Deux Préludes extraits de l'\Opus{35}~; Quatre
 Préludes, \Opus{37}~; Huit Poèmes.
 \item[\DateWithWeekDay{1956-12-25}]
 Moskva~: musée \Scriabine{}.
 Concert enregistré en partie.
 Voir en particulier \citet[p.~450]{Milshteyn82a} et
 \citet[p.~395]{Shiryaeva}.

 \textsc{\Schubert{}}~: Trois Moments musicaux, D~780~: \Number{2} en \kA
 \Flat majeur, \Number{5} en \kF mineur et \Number{6} en \kA \Flat majeur~;
 Deux Impromptus, D~899~: \Number{2} en \kE \Flat majeur et \Number{4} en
 \kA \Flat majeur.
 \textsc{\Schumann{}}~: Fantaisie en \kC majeur, \Opus{17}.
 \textsc{\Chopin{}}~: Nocturne en \kC mineur, \Opus{48} \Number{1}~;
 Nocturne en \kF majeur, \Opus{15} \Number{1}.
 \textsc{\Scriabine{}}~: Sonate, \Opus{68}.
 \textsc{\Liszt{}}~: \emph{Klavierstück} en \kF \Sharp majeur~; \emph{In
 Festo Transfigurationis Domini Nostri Jesu Christi}, S~188~; Nuages gris,
 S~199.
 \emph{Bis} -- \textsc{\Debussy{}}~: \emph{Doctor Gradus ad Parnassum},
 L~113 \Number{I}~; \emph{General Lavine -- eccentric}, L~123 \Number{VI}.
 \textsc{\Schubert{}}~: Deux Moments musicaux, D~780~: \Number{3} en \kF
 mineur et \Number{5} en \kF mineur.
 \item[\DateWithWeekDay{1956-12-30}]
 Moskva~: musée \Scriabine{}.
 \citet[p.~179]{Nekrasova08} mentionne cette date pour un concert de
 \VSofronitsky{} et indique le programme suivant.

 \textsc{\Beethoven{}}~: Sonate en \kC \Sharp mineur, \Opus{27} \Number{2}~;
 Sonate en \kC mineur, \Opus{111}.
 \textsc{\Liszt{}}.
\end{description}

\section{Année~1957}

\begin{description}
 \item[\DateWithWeekDay{1957-01-07}]
 Moskva~: musée \Scriabine{}.
 Voir en particulier \citet[p.~450]{Milshteyn82a}.

 \textsc{\Scriabine{}}~: Vingt Préludes extraits des \Opus{13, 16, 17, 22,
 27, 31, 33, 37, 39, 45 et~48}~; Vingt-quatre Préludes, \Opus{11}.
 \emph{Bis} -- \textsc{\Scriabine{}}~: Préludes.
 \item[\DateWithWeekDay{1957-01-12}]
 Moskva~: musée \Scriabine{}.
 Voir en particulier \citet[p.~450]{Milshteyn82a}.

 \textsc{\Beethoven{}}~: Sonate en \kC \Sharp mineur, \Opus{27} \Number{2}~;
 Sonate en \kC mineur, \Opus{111}.
 \textsc{\Liszt{}}~: Au Lac de Wallenstadt, S~160 \Number{2}~;
 \emph{Sonetto~123 del Petrarca}, S~161 \Number{6}~; \emph{Sonetto~104 del
 Petrarca}, S~161 \Number{5}~; Méphisto-valse.
 \item[\DateWithWeekDay{1957-01-19}]
 Moskva~: musée \Scriabine{}.
 Concert enregistré en partie.
 Voir en particulier \citet[p.~450]{Milshteyn82a}.

 \textsc{\Mozart{}}~: Fantaisie en \kD mineur, K~397.
 \textsc{\Haydn{}}~: Sonate en \kD majeur, Hob.XVI:37.
 \textsc{\Chopin{}}~: Sonate en \kB \Flat mineur, \Opus{35}~; Ballade en \kG
 mineur, \Opus{23}~; Ballade en \kA \Flat majeur, \Opus{47}.
 \textsc{\Liszt{}}~: \emph{Sonetto~123 del Petrarca}, S~161 \Number{6}~;
 \emph{Sonetto~104 del Petrarca}, S~161 \Number{5}~; Méphisto-valse.
 \emph{Bis} -- \textsc{\Scriabine{}}~: Prélude en \kB mineur, \Opus{11}
 \Number{6}~; Prélude en \kD majeur, \Opus{11} \Number{5}~; Prélude en \kB
 majeur, \Opus{2} \Number{2}~; Prélude en \kE mineur, \Opus{11} \Number{4}.
 \textsc{\Schubert{}/\Liszt{}}~: Valse-caprice (Soirée de Vienne \Number{7},
 S~427 \Number{7}).
 \textsc{\Schubert{}}~: Moment musical en \kF mineur, D~780 \Number{3}.
 \textsc{\Chopin{}}~: Prélude en \kE \Flat mineur, \Opus{28} \Number{14}.
 \textsc{\Scriabine{}}~: Étude en \kB \Flat mineur, \Opus{8} \Number{11}~;
 Prélude en \kF \Sharp majeur, \Opus{33} \Number{2}~; Énigme, \Opus{52}
 \Number{2}~; Masque, \Opus{63} \Number{1}~; Prélude en \kF mineur,
 \Opus{17} \Number{5}.
 \item[\DateWithWeekDay{1957-01-31}]
 Moskva~: musée \Scriabine{}.
 Concert enregistré en partie.
 Voir en particulier \citet[p.~450]{Milshteyn82a}.

 \textsc{\Scriabine{}}~: Étude en \kC \Sharp mineur, \Opus{2} \Number{1}~;
 Douze Études, \Opus{8}~; Six Études extraites de l'\Opus{42}~; Étude en \kE
 \Flat majeur, \Opus{49} \Number{1}~; Étude, \Opus{56} \Number{4}~; Trois
 Études, \Opus{65}.
 \emph{Bis} -- \textsc{\Scriabine{}}~: Deux Morceaux, \Opus{57}~: \Number{1}
 Désir et \Number{2} Caresse dansée~; Feuillet d'album en \kE \Flat majeur,
 \Opus{45} \Number{1}.
 \textsc{\Chopin{}}~: Mazurka en \kF mineur, \Opus{63} \Number{2}.
 \textsc{\Scriabine{}}~: Valse en \kA \Flat majeur, \Opus{38}.
 \item[\DateWithWeekDay{1957-02-09}]
 Moskva~: musée \Scriabine{}.
 Concert enregistré en partie.
 Voir en particulier \citet[p.~450]{Milshteyn82a}.

 \textsc{\Scriabine{}}~: Vingt-quatre Études (même programme que le~31
 janvier~1957).
 \emph{Bis} -- \textsc{\Scriabine{}}~: Ironies en \kC majeur, \Opus{56}
 \Number{2}.
 \textsc{\Liadov{}}~: Grimaces, \Opus{64} \Number{1}.
 \textsc{\Prokofiev{}}~: Sarcasme, \Opus{17} \Number{5}~; Légende, \Opus{12}
 \Number{6}.
 \textsc{\Liszt{}}~: Nuages gris, S~199.
 \item[\DateWithWeekDay{1957-03-30}]
 Moskva~: musée \Scriabine{}.
 Voir en particulier \citet[p.~450]{Milshteyn82a}.

 \textsc{\Mozart{}}~: Sonate \Number{4} en \kE \Flat majeur, K~282.
 \textsc{\Schumann{}}~: \emph{Humoreske} en \kB \Flat majeur, \Opus{20}.
 \textsc{\Debussy{}}~: Arabesque \Number{1} en \kE majeur, L~66 \Number{I}.
 \textsc{\Liszt{}}~: Méphisto-valse.
 \textsc{\Scriabine{}}~: Sonate en \kF \Sharp majeur, \Opus{30}.
 \item[\DateWithWeekDay{1957-05-06}]
 Moskva~: musée \Scriabine{}.
 Concert avec des œuvres de \Scriabine{}.
 Programme exact inconnu.
 Voir en particulier \citet[p.~450]{Milshteyn82a}.
 \item[B\DateWithWeekDay{1957-05-07} -- B\DateWithWeekDay{1957-05-19}]
 Tournée du pianiste canadien \GGould{} en Union soviétique~: trois concerts
 à Moskva le~7, le~8 et le~11~; trois concerts à Leningrad le~14, le~16 et
 le~18~; deux concerts/conférences à propos de la Seconde École de Vienne.
 Il y joue des œuvres de \JBach{}, \LBeethoven{}, Arnold Schoenberg, Alban
 Berg, Anton Webern, Ernst Krenek.
 Voir par exemple \citet[p.~396]{Shiryaeva}.
 \item[\DateWithWeekDay{1957-05-16}]
 Moskva~: musée \Scriabine{}.
 Concert enregistré en partie.
 Voir en particulier \citet[p.~450]{Milshteyn82a}.

 \textsc{\Scriabine{}}~: Sonate en \kF mineur, \Opus{6} (quatrième
 mouvement)~; Sonate, \Opus{68}~; Poème en \kC majeur, \Opus{52}
 \Number{1}~; Deux Morceaux, \Opus{57}~: \Number{1} Désir et \Number{2}
 Caresse dansée~; Danse languide, \Opus{51} \Number{4}~; Poème satanique,
 \Opus{36}~; Sonate, \Opus{62}~; Deux Danses, \Opus{73}~: \Number{1}
 Guirlandes et \Number{2} Flammes sombres~; Masque, \Opus{63} \Number{1}~;
 Trois Études, \Opus{65}.
 \emph{Bis} -- \textsc{\Scriabine{}}~: Poème en \kF \Sharp majeur, \Opus{32}
 \Number{1}~; Prélude en \kE \Flat mineur, \Opus{16} \Number{4}~; Énigme,
 \Opus{52} \Number{2}~; Feuillet d'album en \kE \Flat majeur, \Opus{45}
 \Number{1}.
 \item[\DateWithWeekDay{1957-06-12}]
 Moskva~: musée \Scriabine{}.
 Voir en particulier \citet[p.~450]{Milshteyn82a}.

 \textsc{\Scriabine{}}~: Poème tragique, \Opus{34}~; Dix Poèmes~; Vers la
 flamme, \Opus{72}~; Sonate, \Opus{66}~; Sonate, \Opus{70}.
 \item[B1957-07]
 \VSofronitsky{} se repose à la maison de vacances de Voronovo, près de
 Moskva \citep[voir][p.~413]{Shiryaeva}.
 \item[\DateWithWeekDay{1957-08-09}]
 Moskva~: musée \Scriabine{}.
 Concert pour les participants du Sixième Festival mondial de la jeunesse et
 des étudiants.
 Voir en particulier \citet[p.~450]{Milshteyn82a}.

 \textsc{\Scriabine{}}~: Sonate en \kF \Sharp mineur, \Opus{23}~; Sonate,
 \Opus{68}~; Sonate, \Opus{53}.
 \item[\DateWithWeekDay{1957-09-05}]
 Moskva~: musée \Scriabine{}.

 \textsc{\JBach{}/\Busoni{}}~: Deux Préludes de chorals en \kA mineur et en
 \kD mineur.
 \textsc{\Beethoven{}}~: Sonate en \kC mineur, \Opus{111}.
 \textsc{\Medtner{}}~: Sonate en \kC majeur, \Opus{11} \Number{3}.
 \textsc{\Scriabine{}}~: Deux Poèmes, \Opus{44}~; Poème fantastique en \kC
 majeur, \Opus{45} \Number{2}~; Trois Morceaux, \Opus{49}~; Sonate,
 \Opus{66}.
 \item[\DateWithWeekDay{1957-09-15}]
 Moskva~: musée \Scriabine{}.

 \textsc{\Scriabine{}}~: Sonate en \kF \Sharp mineur, \Opus{23}~; Préludes,
 \Opus{74} \Number{2}, \Number{3} et \Number{4}~; Sonate, \Opus{53}~;
 Sonate, \Opus{66}.
 \item[\DateWithWeekDay{1957-09-21}]
 Moskva~: maison des scientifiques.
 Premier concert après une période de dix-sept ou dix-huit mois consacrée
 tout entière au public restreint du musée \Scriabine{}.
 Voir \citet[p.~417]{Shiryaeva} et \citet[p.~14-15]{White}, et note relative
 à l'année~1955 en page~\pageref{bio:1955}.

 \textsc{\Schumann{}}~: Sonate en \kF mineur, \Opus{14}~;
 \emph{Kreisleriana}, \Opus{16}~; Carnaval, \Opus{9}.
 \item[\DateWithWeekDay{1957-09-25}]
 Moskva~: maison des scientifiques.
 Reprise du programme \Schumann{} du concert du~21 septembre.
 \item[\DateWithWeekDay{1957-09-30}]
 Moskva~: maison des scientifiques.
 Il s'agit sans doute du concert évoqué en termes très élogieux par
 \citet[p.~374]{Panarine}.

 \textsc{\Scriabine{}}~: Sonate en \kF \Sharp mineur, \Opus{23}~; Sonate en
 \kF \Sharp majeur, \Opus{30}~; Sonate, \Opus{68}~; Sonate, \Opus{62}~;
 Sonate, \Opus{53}.
 \item[\DateWithWeekDay{1957-10-12}]
 Moskva~: musée \Scriabine{}.
 Concert enregistré en partie.

 \textsc{\Beethoven{}}~: Sonate en \kF mineur, \Opus{57}.
 \item[\DateWithWeekDay{1957-10-20}]
 Moskva~: musée \Scriabine{}.
 Concert avec des œuvres de \Prokofiev{}.
 Peut-être la Sonate \Number{7} en \kB \Flat majeur, \Opus{83}~: avant le
 concert, \VSofronitsky{} a joué chez lui, pour Varvara
 \citeauthor{Nekrasova08} et son mari, une Septième Sonate mémorable
 \citep[voir][p.~180]{Nekrasova08}.
 \item[\DateWithWeekDay{1957-10-27}]
 Moskva~: maison des scientifiques.

 \textsc{\Mendelssohn{}}~: Variations sérieuses en \kD mineur, \Opus{54}.
 \textsc{\Chopin{}}~: Nocturne en \kF majeur, \Opus{15} \Number{1}~;
 Vingt-quatre Préludes, \Opus{28}.
 \textsc{\Liszt{}}~: Sonate en \kB mineur, S~178.
 \item[\DateWithWeekDay{1957-10-30}]
 Moskva~: maison des scientifiques.
 Concert pour les membres de la maison des scientifiques.
 Reprise du programme du concert du~27 octobre.
 \item[\DateWithWeekDay{1957-11-17}]
 Moskva~: maison des scientifiques.

 \textsc{\Prokofiev{}}~: Marche~; \emph{Skazka}~; Sept Pièces extraites de
 l'\Opus{12} (Marche, Gavotte, Rigaudon, Légende, Allemande, Prélude et
 Scherzo humoristique)~; Une Étude extraite de l'\Opus{2}~; Cinq Sarcasmes,
 \Opus{17}~; Sonate \Number{3} en \kA mineur, \Opus{28}~; Vingt Visions
 fugitives, \Opus{22}~; Sonate \Number{7} en \kB \Flat majeur, \Opus{83}.
 \item[\DateWithWeekDay{1957-11-25}]
 Moskva~: maison des scientifiques.
 Reprise du programme \Prokofiev{} du concert du~17 novembre.
 \item[\DateWithWeekDay{1957-12-23}]
 Moskva~: maison des scientifiques.

 \textsc{\Schumann{}}~: Sonate en \kG mineur, \Opus{22}~; Fantaisie en \kC
 majeur, \Opus{17}.
 \textsc{\Chopin{}}~: Ballade en \kG mineur, \Opus{23}~; Barcarolle en \kF
 \Sharp majeur, \Opus{60}~; Deux Mazurkas~; Scherzo en \kB \Flat mineur,
 \Opus{31}.
 \textsc{\Debussy{}}~: \emph{Serenade of the Doll}, L~113 \Number{III}~; Six
 Préludes (\emph{Minstrels}, L~117 \Number{XII}~; Feuilles mortes, L~123
 \Number{II}~; La Fille aux cheveux de lin, L~117 \Number{VIII}~;
 \emph{General Lavine -- eccentric}, L~123 \Number{VI}~; Canope, L~123
 \Number{X}~; Feux d'artifice, L~123 \Number{XII}).
 \item[\DateWithWeekDay{1957-12-30}]
 Moskva~: musée \Scriabine{}.
 Concert enregistré en partie.

 \textsc{\Scriabine{}}~: Étude en \kC \Sharp mineur, \Opus{2} \Number{1}~;
 Deux Impromptus en \kB \Flat mineur, \Opus{12} \Number{2}, et en \kF \Sharp
 mineur, \Opus{14} \Number{2}~; Cinq Études extraites de l'\Opus{42}~:
 \Number{1} en \kD \Flat majeur, \Number{2} en \kF \Sharp mineur, \Number{3}
 en \kF \Sharp majeur, \Number{4} en \kF \Sharp majeur et \Number{5} en \kC
 \Sharp mineur~; Deux Études en \kD \Flat majeur, \Opus{8} \Number{10}, et
 en \kB \Flat mineur, \Opus{8} \Number{11}~; Sonate en \kF \Sharp majeur,
 \Opus{30}~; Poème tragique, \Opus{34}~; Trois Morceaux, \Opus{45}~; Deux
 Morceaux extraits de l'\Opus{56}~; Trois Morceaux, \Opus{49}~; Deux
 Morceaux, \Opus{57}~; Poème satanique, \Opus{36}.
\end{description}

\section{Année~1958}

\begin{description}
 \item[\DateWithWeekDay{1958-01-05}]
 Moskva~: maison des scientifiques.
 Reprise du programme du concert du~23 décembre~1957.
 \citet[p.~180]{Nekrasova08} indique plutôt la date du~1\ier{} janvier pour
 ce concert.
 \item[\DateWithWeekDay{1958-01-07}]
 Moskva~: musée \Scriabine{}.
 Concert avec des œuvres de \Scriabine{}.
 \item[\DateWithWeekDay{1958-01-27}]
 Moskva~: maison des scientifiques, salle de concert de l'hôtel Sovetskaja,
 musée \Scriabine{} ou petite salle du conservatoire.
 Concert.
 \citet[p.~180-181]{Nekrasova08} indique la petite salle du conservatoire et
 le programme suivant, confirmé de manière moins précise et complète par
 \citet[p.~385]{Shiryaeva}.

 \textsc{\Schubert{}}~: Un Impromptu extrait du recueil D~935~; Deux
 Impromptus extraits du recueil D~899.
 \textsc{\Schumann{}}~: Études symphoniques, \Opus{13}.
 \textsc{\Liszt{}}~: Chapelle de Guillaume Tell, S~160 \Number{1}~;
 \emph{Sposalizio}, S~161 \Number{1}~; \emph{Sonetto~104 del Petrarca},
 S~161 \Number{5}.
 \textsc{\Schubert{}/\Liszt{}}~: \emph{Morgenständchen}, S~558 \Number{9}~;
 \emph{Auf dem Wasser zu singen}, S~558 \Number{2}.
 \textsc{\Liszt{}}~: Méphisto-valse.
 \item[\DateWithWeekDay{1958-02-14}]
 Moskva~: maison des scientifiques, salle de concert de l'hôtel Sovetskaja,
 musée \Scriabine{} ou petite salle du conservatoire.
 Concert.
 \citet[p.~181]{Nekrasova08} indique la salle de concert de l'hôtel
 Sovetskaja et mentionne que le programme était une reprise de celui du
 concert du~27 janvier.
 \item[\DateWithWeekDay{1958-02-27}]
 Moskva~: maison des scientifiques, salle de concert de l'hôtel Sovetskaja,
 musée \Scriabine{} ou petite salle du conservatoire.
 Le programme ci-dessous indique la salle de concert de l'hôtel Sovetskaja.

 \textsc{\Scriabine{}}~: Sept Préludes extraits des \Opus{11, 13, 16
 et~17}~; Douze Études extraites des \Opus{42 et~8}~; Poème tragique,
 \Opus{34}~; Valse en \kA \Flat majeur, \Opus{38}~; Deux Danses, \Opus{73}~:
 Flammes sombres (\Number{2}) et Guirlandes (\Number{1})~; Deux Poèmes,
 \Opus{71}~; Poème, \Opus{63} \Number{1} (Masque)~; Poème satanique,
 \Opus{36}.
 \item[\DateWithWeekDay{1958-04-27}]
 Moskva~: musée \Scriabine{}.

 \textsc{\Scriabine{}}~: Deux Préludes, \Opus{67}~; Cinq Préludes,
 \Opus{74}~; Sonate, \Opus{68}~; Sonate, \Opus{70}~; Sonate, \Opus{66}.
 \item[\DateWithWeekDay{1958-05-05}]
 Moskva~: musée \Scriabine{}.
 Concert enregistré en partie.

 \textsc{\Scriabine{}}~: Sonate, \Opus{66}.
 \item[\DateWithWeekDay{1958-06-01}]
 Moskva~: maison des scientifiques, salle de concert de l'hôtel Sovetskaja,
 musée \Scriabine{} ou petite salle du conservatoire.
 Concert.
 \citet[p.~181]{Nekrasova08} indique la petite salle du conservatoire, mais
 le programme n'est pas précisé.
 \item[\DateWithWeekDay{1958-06-08}]
 Moskva~: petite salle du conservatoire.
 Concert enregistré au moins en partie.

 \textsc{\Scriabine{}}~: Sonate en \kF \Sharp mineur, \Opus{23}~; Sonate,
 \Opus{68}~; Poème, \Opus{59} \Number{1}~; Prélude, \Opus{74} \Number{1}~;
 Prélude, \Opus{74} \Number{3}~; Prélude, \Opus{74} \Number{4}~; Sonate,
 \Opus{53}~; Sonate, \Opus{66}~; Poème, \Opus{69} \Number{1}~; Poème en \kF
 \Sharp majeur, \Opus{32} \Number{1}~; Étude en \kF \Sharp majeur, \Opus{42}
 \Number{3}~; Étude en \kF \Sharp majeur, \Opus{42} \Number{4}~; Étude en
 \kC \Sharp mineur, \Opus{42} \Number{5}.
 \item[\DateWithWeekDay{1958-06-26}]
 Moskva~: musée \Scriabine{}.
 Concert enregistré en partie.

 \textsc{\Schumann{}}~: \emph{Novelette} en \kE majeur, \Opus{21}
 \Number{7}~; \emph{Novelette} en \kF \Sharp mineur, \Opus{21} \Number{8}.
 \item[\DateWithWeekDay{1958-07-05}]
 Moskva~: musée \Scriabine{}.
 Concert enregistré en partie.

 \textsc{\Schumann{}}~: \emph{Novelette} en \kF \Sharp mineur, \Opus{21}
 \Number{8}~; Trois Romances, \Opus{28}~: \Number{1} en \kB \Flat mineur,
 \Number{2} en \kF \Sharp majeur et \Number{3} en \kB majeur~; Sonate en \kF
 \Sharp mineur, \Opus{11}.
 \textsc{\Rachmaninov{}}~: Étude-tableau en \kE \Flat mineur, \Opus{39}
 \Number{5}~; Moment musical en \kD \Flat majeur, \Opus{16} \Number{5}.
 \textsc{\Medtner{}}~: Deux \emph{Skazki} en \kB \Flat mineur, \Opus{20}
 \Number{1}, et en \kB mineur, \Opus{20} \Number{2}~; \emph{Novelette} en
 \kC mineur, \Opus{17} \Number{2}.
 \textsc{\Scriabine{}}~: Étude en \kF \Sharp majeur, \Opus{42} \Number{4}~;
 Étude en \kC \Sharp mineur, \Opus{42} \Number{5}~; Poème ailé, \Opus{51}
 \Number{3}~; Poème languide en \kB majeur, \Opus{52} \Number{3}~; Prélude
 en \kC \Sharp mineur, \Opus{11} \Number{10}~; Étude, \Opus{65} \Number{3}.
 \item[\DateWithWeekDay{1958-07-20}]
 Moskva~: musée \Scriabine{}.
 Concert enregistré en partie.

 \textsc{\Chopin{}}~: Ballade en \kF mineur, \Opus{52}~; Nocturne en \kG
 mineur, \Opus{15} \Number{3}~; Nocturne en \kG majeur, \Opus{37}
 \Number{2}.
 \item[\DateWithWeekDay{1958-07-26}]
 Moskva~: musée \Scriabine{}.
 Concert enregistré en partie.

 \textsc{\Schumann{}}~: \emph{Bunte Blätter}, \Opus{99} \Number{1} à
 \Number{8}~; Deux \emph{Novelettes}.
 \textsc{\Chopin{}}~: Deux Nocturnes en \kG mineur et en \kG majeur~;
 Ballade en \kF mineur, \Opus{52}.
 \textsc{\Rachmaninov{}}~: Étude-tableau en \kC majeur, \Opus{33}
 \Number{2}~; Étude-tableau.
 \textsc{\Scriabine{}}~: Quatre Préludes extraits de l'\Opus{11}~; Quatre
 Préludes, \Opus{37}~; Sonate en \kF \Sharp majeur, \Opus{30}.
 \item[B1958 (été)]
 \VSofronitsky{} en vacances près de Moskva.
 \item[\DateWithWeekDay{1958-09-15}]
 Moskva~: maison des scientifiques, salle de concert de l'hôtel Sovetskaja,
 musée \Scriabine{} ou petite salle du conservatoire.
 Le programme ci-dessous indique la maison des scientifiques.

 \textsc{\Scriabine{}}~: Sonate en \kF mineur, \Opus{6}~; Six Préludes
 extraits des \Opus{11, 16 et~17}~; Douze Études, \Opus{8}~; Douze Poèmes
 extraits des \Opus{44, 45 (\Number{2}~: Poème fantastique en \kC majeur),
 51 (\Number{3}~: Poème ailé), 52, 63, 69, 71 et~72 (Vers la flamme)}.
 \item[\DateWithWeekDay{1958-09-21}]
 Moskva~: maison des scientifiques, salle de concert de l'hôtel Sovetskaja,
 musée \Scriabine{} ou petite salle du conservatoire.
 Le programme ci-dessous indique la maison des scientifiques.

 \textsc{\Scriabine{}}~: Sonate en \kF mineur, \Opus{6}~; Six Préludes
 extraits des \Opus{11, 16 et~17}~; Douze Études, \Opus{8}~; Douze Poèmes
 extraits des \Opus{44, 45 (\Number{2}~: Poème fantastique en \kC majeur),
 51 (\Number{3}~: Poème ailé), 52, 63, 69, 71 et~72 (Vers la flamme)}.
 \item[\DateWithWeekDay{1958-09-28}]
 Moskva~: musée \Scriabine{}.
 Concert avec des œuvres de \Scriabine{}.
 \item[\DateWithWeekDay{1958-10-29}]
 \citet[p.~181]{Nekrasova08} mentionne cette date pour un concert, mais le
 lieu et le programme ne sont pas précisés.
 \item[\DateWithWeekDay{1958-11-25}]
 Moskva~: musée \Scriabine{}.
 Selon \citet[p.~181]{Nekrasova08}, la chronologie indique un concert à la
 petite salle du conservatoire de Moskva le~25 novembre, et les archives
 d'\AVizel{} mentionnent un concert au musée \Scriabine{} le~29 novembre.

 \textsc{\Schubert{}}~: Six Moments musicaux, D~780.
 \textsc{\Chopin{}}~: Une Polonaise extraite de l'\Opus{26}~; Prélude en \kD
 \Flat majeur, \Opus{28} \Number{15}~; Prélude en \kA \Flat majeur,
 \Opus{28} \Number{17}~; Prélude en \kE \Flat majeur, \Opus{28}
 \Number{19}~; Prélude en \kB \Flat majeur, \Opus{28} \Number{21}~; Ballade
 en \kA \Flat majeur, \Opus{47}.
 \textsc{\Scriabine{}}~: Poèmes~; Préludes~; Sonate, \Opus{68}.
 \item[\DateWithWeekDay{1958-12-21}]
 Moskva~: maison des scientifiques, salle de concert de l'hôtel Sovetskaja,
 musée \Scriabine{} ou petite salle du conservatoire.
 Concert.
 \citet[p.~181]{Nekrasova08} indique la maison des scientifiques et donne le
 programme suivant.

 \textsc{\Schubert{}}~: Six Moments musicaux, D~780.
 \textsc{\Schumann{}}~: Sonate en \kF \Sharp mineur, \Opus{11}.
 \textsc{\Chopin{}}~: Polonaise en \kC \Sharp mineur, \Opus{26} \Number{1}~;
 Barcarolle en \kF \Sharp majeur, \Opus{60}~; Tarentelle en \kA \Flat
 majeur, \Opus{43}~; Une Mazurka extraite de l'\Opus{50}~; Polonaise en \kF
 \Sharp mineur, \Opus{44}.
\end{description}

\section{Année~1959}

\begin{description}
 \item[\DateWithWeekDay{1959-01-07}]
 Moskva~: musée \Scriabine{}.
 Concert.
 Concert retrouvé par \citet{TADGO1960}.
 \item[\DateWithWeekDay{1959-01-13}]
 Moskva~: musée \Scriabine{}.
 Concert enregistré en partie, consacré à cinquante Préludes de \Scriabine{}
 selon les notes d'\ASofronitsky{}.

 \textsc{\Scriabine{}}~: Prélude en \kE \Flat majeur, \Opus{45} \Number{3}~;
 Prélude en \kF majeur, \Opus{49} \Number{2}~; Rêverie en \kC majeur,
 \Opus{49} \Number{3}~; Prélude en \kE \Flat majeur, \Opus{56} \Number{1}~;
 Dix Préludes extraits de l'\Opus{11}~: \Number{13} en \kG \Flat majeur,
 \Number{14} en \kE \Flat mineur, \Number{15} en \kD \Flat majeur,
 \Number{16} en \kB \Flat mineur, \Number{17} en \kA \Flat majeur,
 \Number{18} en \kF mineur, \Number{19} en \kE \Flat majeur, \Number{22} en
 \kG mineur, \Number{23} en \kF majeur et \Number{24} en \kD mineur~;
 Prélude en \kE \Flat mineur, \Opus{16} \Number{4}~; Prélude en \kF mineur,
 \Opus{17} \Number{5}~; Prélude en \kB \Flat majeur, \Opus{17} \Number{6}~;
 Prélude en \kG mineur, \Opus{27} \Number{1}~; Prélude en \kB majeur,
 \Opus{27} \Number{2}~; Prélude en \kD \Flat majeur/\kC majeur, \Opus{31}
 \Number{1}~; Prélude en \kF \Sharp mineur, \Opus{31} \Number{2}~; Prélude
 en \kE \Flat mineur, \Opus{31} \Number{3}~; Prélude en \kC majeur,
 \Opus{31} \Number{4}~; Prélude en \kE majeur, \Opus{33} \Number{1}~;
 Prélude en \kF \Sharp majeur, \Opus{33} \Number{2}~; Prélude en \kC majeur,
 \Opus{33} \Number{3}~; Prélude en \kD \Flat majeur, \Opus{35} \Number{1}~;
 Prélude en \kB \Flat majeur, \Opus{35} \Number{2}~; Prélude en \kB \Flat
 mineur, \Opus{37} \Number{1}~; Prélude en \kF \Sharp majeur, \Opus{37}
 \Number{2}~; Prélude en \kB majeur, \Opus{37} \Number{3}~; Prélude en \kG
 mineur, \Opus{37} \Number{4}~; Prélude en \kD majeur, \Opus{39}
 \Number{2}~; Prélude en \kG majeur, \Opus{39} \Number{3}~; Prélude en \kA
 \Flat majeur, \Opus{39} \Number{4}~; Prélude en \kF \Sharp majeur,
 \Opus{48} \Number{1}~; Prélude en \kC majeur, \Opus{48} \Number{2}~; Deux
 Préludes, \Opus{67} \Number{1} et \Number{2}~; Cinq Préludes, \Opus{74}
 \Number{1} à \Number{5}.
 \item[\DateWithWeekDay{1959-01-19}]
 Moskva~: musée \Scriabine{}.
 Concert enregistré en partie.

 \textsc{\Scriabine{}}~: Prélude en \kB majeur, \Opus{2} \Number{2}~; Trois
 Préludes extraits de l'\Opus{11}~: \Number{4} en \kE mineur, \Number{5} en
 \kD majeur et \Number{6} en \kB mineur.
 \item[\DateWithWeekDay{1959-03-14}]
 Moskva~: musée \Scriabine{}.

 \textsc{\Beethoven{}}~: Sonate \Number{5} en \kC mineur, \Opus{10}
 \Number{1}~; Sonate \Number{7} en \kD majeur, \Opus{10} \Number{3}~; Sonate
 \Number{25} en \kG majeur, \Opus{79}~; Sonate \Number{24} en \kF \Sharp
 majeur, \Opus{78}~; Sonate \Number{32} en \kC mineur, \Opus{111}.
 \item[B1959 (printemps)]
 Peu après ce concert du~14 mars, \VSofronitsky{} tombe gravement malade du
 cœur~; voir \citet[p.~379-380]{Shiryaeva} et \citet[p.~181]{Nekrasova08}.
 Nina Grigor'evna \citeauthor{Shiryaeva} et \AVizel{} restent presque deux
 mois à son chevet.
 Sa santé ne commence à s'améliorer que durant la seconde moitié du mois de
 mai, une lente amélioration jusqu'à l'été \citep[voir][p.~411]{Shiryaeva}.
 \item[\DateWithWeekDay{1959-08-17}]
 Moskva~: musée \Scriabine{}.
 Concert enregistré en partie.

 \textsc{\Liszt{}}~: \emph{Fantasie und Fuge über das Thema BACH}, S~529.
 \item[B1959-1960 (saison)]
 \VSofronitsky{}, en congé, ne se rend pas au conservatoire de Moskva~; il
 indique à \citet[p.~416]{Shiryaeva} qu'il se sent enfin \Quote{comme un
 artiste, heureux de ne pouvoir penser qu'à la musique}.
 \item[\DateWithWeekDay{1959-09-26}]
 Moskva~: maison des scientifiques.
 Voir \citet[p.~396]{Shiryaeva} à propos des œuvres de \JBach{} jouées.

 \textsc{\JBach{}}~: Deux Préludes et fugues en \kB \Flat majeur et en \kB
 \Flat mineur.
 \textsc{\Liszt{}}~: \emph{Fantasie und Fuge über das Thema BACH}, S~529~;
 Sonate en \kB mineur, S~178~; Méphisto-valse \Number{3}, S~216~;
 Méphisto-valse \Number{2}, S~515~; Méphisto-valse \Number{1}, S~514.
 \item[\DateWithWeekDay{1959-09-29}]
 Moskva~: maison des scientifiques.
 Reprise du programme du concert du~26 septembre, mais avec plusieurs
 modifications selon \citet[p.~182]{Nekrasova08}.

 \textsc{\Liszt{}}~: \emph{Sonetto~104 del Petrarca}, S~161 \Number{5}~;
 Valse oubliée~; Sonate en \kB mineur, S~178~; Première Méphisto-valse,
 S~514~; Troisième Méphisto-valse, S~216.
 \textsc{\Wagner{}/\Brassin{}}~: \emph{Feuerzauber} extrait de \emph{Die
 Walküre}.
 \textsc{\Scriabine{}}.
 \item[\DateWithWeekDay{1959-10-16}]
 Moskva~: maison des scientifiques.
 Voir en particulier \citet[p.~446]{Milshteyn82a}.
 \citet[p.~181]{Nekrasova08}, qui se fonde sur les archives d'\AVizel{},
 mentionne aussi un récital avec le programme ci-dessous, à la maison des
 scientifiques.

 \textsc{\Chopin{}}~: Polonaise en \kC \Sharp mineur, \Opus{26} \Number{1}~;
 Fantaisie en \kF mineur, \Opus{49}~; Deux Nocturnes, \Opus{27}~; Ballade en
 \kA \Flat majeur, \Opus{47}~; Ballade en \kF mineur, \Opus{52}~; Scherzo en
 \kB mineur, \Opus{20}~; Scherzo en \kB \Flat mineur, \Opus{31}~; Scherzo en
 \kC \Sharp mineur, \Opus{39}~; Trois Préludes extraits de l'\Opus{28}~:
 \Number{17} en \kA \Flat majeur, \Number{19} en \kE \Flat majeur et
 \Number{21} en \kB \Flat majeur~; Deux Mazurkas extraites de l'\Opus{30}~;
 Polonaise en \kF \Sharp mineur, \Opus{44}.
 \item[\DateWithWeekDay{1959-11-04}]
 Moskva~: petite salle du conservatoire.
 Concert avec des œuvres de \Schumann{}.

 \textsc{\Schumann{}}~: \emph{Bunte Blätter}, \Opus{99} (huit pièces)~;
 Trois \emph{Novelettes} extraites de l'\Opus{21}~; \emph{Kreisleriana},
 \Opus{16}~; Carnaval, \Opus{9}.
 \item[\DateWithWeekDay{1959-11-18}]
 Moskva~: petite salle du conservatoire.
 Concert avec des œuvres de \Schumann{}, enregistré en totalité.
 Voir en particulier \citet[p.~446]{Milshteyn82a}.

 \textsc{\Schumann{}}~: Arabesque en \kC majeur, \Opus{18}~; Fantaisie en
 \kC majeur, \Opus{17}~; Études symphoniques, \Opus{13}~; Carnaval,
 \Opus{9}~; \emph{Phantasiestücke} en \kD \Flat majeur, \Opus{12} \Number{1}
 (\emph{Des Abends})~; Romance en \kB \Flat mineur, \Opus{28} \Number{1}.
 \item[\DateWithWeekDay{1959-11-30}]
 Moskva~: maison des scientifiques.
 Voir en particulier \citet[p.~446-447]{Milshteyn82a}.

 \textsc{\Schubert{}}~: Un Impromptu extrait du recueil D~935~; Impromptu en
 \kE \Flat majeur, D~899 \Number{2}~; Impromptu en \kA \Flat majeur, D~899
 \Number{4}.
 \textsc{\Schumann{}}~: Sonate en \kF \Sharp mineur, \Opus{11}.
 \textsc{\Chopin{}}~: Ballade en \kG mineur, \Opus{23}~; Barcarolle en \kF
 \Sharp majeur, \Opus{60}~; Impromptu en \kG \Flat majeur, \Opus{51}~;
 Tarentelle en \kA \Flat majeur, \Opus{43}~; Deux Mazurkas extraites des
 \Opus{41 et~50}~; Polonaise en \kA \Flat majeur, \Opus{53}.
 \item[\DateWithWeekDay{1959-12-03}]
 Moskva~: maison des scientifiques.
 \citet[p.~181]{Nekrasova08} mentionne cette date pour un concert de
 \VSofronitsky{}, mais indique que cette date et celle du~30 novembre ne
 figurent que dans les archives d'\AVizel{}.
\end{description}

\section{Année~1960}

\begin{description}
 \item[\DateWithWeekDay{1960-01-06}]
 Moskva~: musée \Scriabine{}.
 Concert enregistré en grande partie.
 Il subsiste peut-être un doute à propos de la date exacte de ce récital
 \citep[voir][]{TADGO1960}.
 La plupart des références indiquent cette date du~6 janvier~1960~: voir
 \citet[p.~461]{Milshteyn82a}, \citet{Evans08}, \citet[p.~23]{Nikonovich11},
 \citet[p.~442]{Scriabine} et, en particulier, \citet[p.~182]{Nekrasova08}.
 Dans son article de~\citeyear{Nekrasova95}, \citeauthor{Nekrasova95} donne,
 en revanche, la date du lendemain, le~7 janvier~1960
 \citep[voir][]{Nekrasova95}.
 Dans la discographie et ci-dessous, je m'en suis tenu à la publication la
 plus récente de \VNekrasova{}, qui a reçu la correspondance entretenue par
 \VSofronitsky{} et \AVizel{}.
 La liste ci-dessous complète \citet[p.~442]{Scriabine} grâce au recensement
 des œuvres enregistrées \citep[voir][p.~23 et~30]{Nikonovich11}.
 Selon \citet[p.~442]{Scriabine}, la seconde partie du récital a commencé
 avec l'\Opus{52} \Number{1}.

 \textsc{\Scriabine{}}~: Sonate, \Opus{66} (enregistrée lors du concert
 du~24 décembre~1960 au musée \Scriabine{})~; Prélude en \kG \Sharp mineur,
 \Opus{22} \Number{1}~; Prélude en \kG \Sharp mineur, \Opus{11}
 \Number{12}~; Prélude en \kG \Flat majeur, \Opus{11} \Number{13}~; Prélude
 en \kB \Flat mineur, \Opus{37} \Number{1}~; Prélude en \kD \Flat majeur/\kC
 majeur, \Opus{31} \Number{1}~; Poème, \Opus{41}~; Poème-nocturne,
 \Opus{61}~; Deux Poèmes, \Opus{69} \Number{1} et \Number{2}~; Deux Danses,
 \Opus{73}~: \Number{1} Guirlandes et \Number{2} Flammes sombres~; Trois
 Préludes, \Opus{74} \Number{3} à \Number{5}~; Poème en \kC majeur,
 \Opus{52} \Number{1}~; Poème en \kC majeur, \Opus{44} \Number{2}~; Poème,
 \Opus{59} \Number{1}~; Poème ailé, \Opus{51} \Number{3}~; Poème languide en
 \kB majeur, \Opus{52} \Number{3}~; Deux Poèmes, \Opus{71}~; Masque,
 \Opus{63} \Number{1}~; Étude, \Opus{65} \Number{3}~; Vers la flamme,
 \Opus{72}~; Fragilité, \Opus{51} \Number{1}~; Quatre Préludes, \Opus{11}~:
 \Number{2} en \kA mineur, \Number{4} en \kE mineur, \Number{5} en \kD
 majeur et \Number{19} en \kE \Flat majeur~; Feuillet d'album en \kE \Flat
 majeur, \Opus{45} \Number{1}~; Poème en \kF \Sharp majeur, \Opus{32}
 \Number{1}~; Énigme, \Opus{52} \Number{2}~; Mazurka en \kF \Sharp majeur,
 \Opus{40} \Number{2}.
 \item[\DateWithWeekDay{1960-01-08}]
 Moskva~: petite salle du conservatoire.
 Concert avec des œuvres de \Scriabine{}, en partie enregistré et
 radiodiffusé~; \VSofronitsky{} lui-même a indiqué%
 \footnote{\foreignlanguage{russian}{\emph{Советская музыка}}, vol.~651,
 \Number{1} (1995), p.~107.}
 que seule la première partie du récital a été diffusée, et non la seconde.
 Le travail de \citeauthor{Malik} en~\citeyear{Malik} a permis de lever les
 incertitudes entre le concert du~8 janvier~1960 et celui du~2 février~1960,
 à partir de leurs premières éditions sur disques vinyles.
 Programme du concert retrouvé par \citet{TADGO1960}, avec des corrections
 apportées à la main sur le programme imprimé et qui contredisent en partie
 les indications de programme données par \citet[p.~182]{Nekrasova08}.
 Voir tableau~\ref{tab:600108_600202} en page~\pageref{tab:600108_600202}.

 \emph{Œuvres enregistrées}.--
 \textsc{\Scriabine{}}~: Deux Danses, \Opus{73}~; Feuillet d'album,
 \Opus{45} \Number{1}~; Fragilité, \Opus{51} \Number{1}~; Poème ailé,
 \Opus{51} \Number{3}~; Poème, \Opus{59} \Number{1}~; Deux Poèmes,
 \Opus{32}~; Masque, \Opus{63} \Number{1}~; Deux Poèmes, \Opus{69}~; Deux
 Poèmes, \Opus{71}~; Prélude en \kC \Sharp mineur pour la main gauche,
 \Opus{9} \Number{1}~; Douze Préludes extraits de l'\Opus{11}~: \Number{2}
 en \kA mineur, \Number{4} en \kE mineur, \Number{5} en \kD majeur,
 \Number{9} en \kE majeur, \Number{16} en \kB \Flat mineur, \Number{17} en
 \kA \Flat majeur, \Number{19} en \kE \Flat majeur, \Number{20} en \kC
 mineur, \Number{21} en \kB \Flat majeur, \Number{22} en \kG mineur,
 \Number{23} en \kF majeur et \Number{24} en \kD mineur~; Prélude en \kC
 majeur, \Opus{13} \Number{1}~; Prélude en \kB mineur, \Opus{13}
 \Number{6}~; Prélude en \kA majeur, \Opus{15} \Number{1}~; Prélude en \kF
 \Sharp majeur, \Opus{16} \Number{5}~; Prélude en \kC \Sharp mineur,
 \Opus{22} \Number{2}~; Sonate, \Opus{70}.

 \emph{Programme du concert}.--
 Première partie.
 \textsc{\Scriabine{}}~: Vingt-quatre Préludes extraits des \Opus{11, 13,
 15, 16 et~17}~; Six Poèmes extraits des \Opus{41, 51, 52 et~63}.
 Deuxième partie.
 \textsc{\Scriabine{}}~: Deux Poèmes, \Opus{32}~; Deux Poèmes, \Opus{69}~;
 Deux Poèmes, \Opus{71}~; Sonate \Number{10}, \Opus{70}~; Vers la flamme,
 \Opus{72}.
 Par rapport au programme mentionné par \citet[p.~182]{Nekrasova08}, il y a
 l'ajout de l'\Opus{51}, ainsi que le retrait du Poème-nocturne (\Opus{61})
 et de la Neuvième Sonate (\Opus{68}) et leur remplacement par les \Opus{32,
 69 et~71}.
 La liste exacte des Préludes et Poèmes de la première partie n'est précisée
 par aucune source.
 \item[B\DateWithWeekDay{1960-01-13}]
 Naissance de Viviana Sofronickaja-Dušinova.
 \item[\DateWithWeekDay{1960-01-13}]
 \citet[p.~183]{Nekrasova08}, citant les archives personnelles d'\AVizel{},
 mentionne cette date pour un concert de \VSofronitsky{}~; les archives n'en
 indiquent cependant ni le lieu, ni le programme.
 \item[\DateWithWeekDay{1960-01-14}]
 Moskva~: petite salle du conservatoire.
 Enregistrement pour la radio.
 \VSofronitsky{} enregistre des œuvres de \Chopin{}~: Polonaise en \kC
 \Sharp mineur, \Opus{26} \Number{1}~; Quatre Mazurkas~; Fantaisie~;
 Nocturne \Number{8} en \kD \Flat majeur, \Opus{27} \Number{2}~; Ballade
 \Number{3} en \kA \Flat majeur, \Opus{47}.
 \item[\DateWithWeekDay{1960-01-21}]
 Moskva~: grande salle du conservatoire.
 Enregistrement (de~20 à~24~heures) d'œuvres de \Chopin{}~: Polonaise en \kC
 \Sharp mineur, \Opus{26} \Number{1}~; Cinq Mazurkas.
 Durée totale enregistrée~: 27~minutes.
 \item[\DateWithWeekDay{1960-01-25}]
 Moskva~: grande salle du conservatoire.
 Reprise de l'enregistrement du~21 janvier.
 \item[\DateWithWeekDay{1960-01-28}]
 \citet[p.~183]{Nekrasova08}, citant les archives personnelles d'\AVizel{},
 mentionne cette date pour un concert de \VSofronitsky{}~; les archives n'en
 indiquent cependant ni le lieu, ni le programme.
 \item[\DateWithWeekDay{1960-02-02}]
 Moskva~: petite salle du conservatoire.
 Concert avec des œuvres de \Scriabine{}, enregistré au moins en partie.
 Le travail de \citeauthor{Malik} en~\citeyear{Malik} a permis de lever les
 incertitudes entre le concert du~8 janvier~1960 et celui du~2 février~1960,
 à partir de leurs premières éditions sur disques vinyles.
 Voir en particulier \citet[p.~447]{Milshteyn82a}.
 Presque tous les auteurs indiquent un programme consacré à \Scriabine{},
 mais \citet[p.~183]{Nekrasova08} affirme que la première partie du récital
 était constituée d'œuvres de \Chopin{} et la seconde de compositions de
 \Scriabine{} (il s'agit sans doute d'une erreur de
 \citeauthor{Nekrasova08}).
 Voir tableau~\ref{tab:600108_600202} en page~\pageref{tab:600108_600202}.

 \textsc{\Scriabine{}}~: Vingt Préludes~; Six Poèmes~; Sonate, \Opus{68}~;
 Deux Poèmes, \Opus{69}~; Deux Danses, \Opus{73}~; Sonate, \Opus{70}~;
 Fragilité, \Opus{51} \Number{1}~; Feuillet d'album en \kE \Flat majeur,
 \Opus{45} \Number{1}~; Étude en \kC \Sharp mineur, \Opus{42} \Number{5}~;
 Mazurka en \kF \Sharp majeur, \Opus{40} \Number{2}~; Mazurka en \kE mineur,
 \Opus{25} \Number{3}~; Étude en \kD \Sharp mineur, \Opus{8} \Number{12}.

 Selon \citeauthor{Bragina}, citée par \citet[p.~442-443]{Scriabine}, les
 Vingt Préludes et les Six Poèmes de la première partie du concert ont été
 joués dans l'ordre suivant.
 Pour les Préludes~: \Opus{13} \Number{1} en \kC majeur, \Opus{11}
 \Number{2} en \kA mineur, \Opus{13} \Number{3} en \kG majeur, \Opus{11}
 \Number{4} en \kE mineur, \Opus{11} \Number{5} en \kD majeur, \Opus{13}
 \Number{6} en \kB mineur, \Opus{15} \Number{1} en \kA majeur, \Opus{9}
 \Number{1} en \kC \Sharp mineur, \Opus{11} \Number{9} en \kE majeur,
 \Opus{11} \Number{10} en \kC \Sharp mineur, \Opus{22} \Number{2} en \kC
 \Sharp mineur, \Opus{16} \Number{2} en \kG \Sharp mineur, \Opus{16}
 \Number{5} en \kF \Sharp majeur, \Opus{16} \Number{4} en \kE \Flat mineur,
 \Opus{11} \Number{15} en \kD \Flat majeur, \Opus{11} \Number{16} en \kB
 \Flat mineur, \Opus{11} \Number{19} en \kE \Flat majeur, \Opus{11}
 \Number{21} en \kB \Flat majeur, \Opus{11} \Number{22} en \kG mineur et
 \Opus{11} \Number{24} en \kD mineur.
 L'enregistrement indique la présence de quelques Préludes supplémentaires.
 Pour les Poèmes~: \Opus{52} \Number{1} en \kC majeur, \Opus{59} \Number{1},
 \Opus{51} \Number{3} (Poème ailé), \Opus{52} \Number{3} en \kB majeur
 (Poème languide), \Opus{63} \Number{1} (Masque) et \Opus{36} (Poème
 satanique).
 \item[B1960-03]
 Dans une lettre datée du~18 avril \citep[voir][p.~184]{Nekrasova08},
 \VSofronitsky{} indique que des raisons de santé l'ont amené à annuler un
 séjour en Pologne, où il aurait dû se rendre dans le courant du mois de
 mars~1960 pour y jouer.
 Il est possible que ce déplacement prévu mais annulé soit celui de Warszawa
 évoqué par \citet{Voskobojnikov08} vers la fin de la deuxième émission de
 son cycle consacré au pianiste~: il s'agissait d'une célébration
 du~150\ieme{} anniversaire de la naissance de \Chopin{}, pour laquelle les
 autorités polonaises avaient sollicité la présence de \Sofronitsky{}.
 \item[\DateWithWeekDay{1960-03-12}]
 Moskva~: petite salle du conservatoire.
 Concert et programme indiqués par \citet[p.~183]{Nekrasova08} et retrouvés
 par \citet{TADGO1960}.

 \textsc{\Mozart{}}~: Sonate \Number{4} en \kE \Flat majeur, K~282.
 \textsc{\Schumann{}}~: \emph{Humoreske} en \kB \Flat majeur, \Opus{20}.
 \textsc{\Chopin{}}~: Ballade \Number{1} en \kG mineur, \Opus{23}.
 \textsc{\Schubert{}/\Liszt{}}~: Six lieder \citep[selon][]{TADGO1960} ou un
 seul \citep[selon][]{Nekrasova08}.
 \textsc{\Liszt{}}~: Feux follets, S~139 \Number{5}~; \emph{Tarantella}
 extraite de \emph{Venezia e Napoli}, S~162 \Number{3}.
 \item[\DateWithWeekDay{1960-03-14}]
 Moskva~: musée \Scriabine{}.
 \citet[p.~184]{Nekrasova08} mentionne ce récital, dont le programme ne
 figure pas dans les archives tenues par \AVizel{}.
 \item[B1960 (printemps)]
 \VSofronitsky{} contacté par l'équipe éditoriale du journal
 \foreignlanguage{russian}{Советский Союз} [Sovetskij Sojuz], afin d'écrire
 un article à propos de la musique de \RSchumann{}, à l'occasion
 du~150\ieme{} anniversaire de la naissance du compositeur~; un brouillon
 préliminaire de l'article est reproduit par \citet[p.~386-387]{Shiryaeva}.
 Le texte final de l'article de \Sofronitsky{} sur \Schumann{} dans la revue
 \foreignlanguage{russian}{Советский Союз}, publié en juin~1960, est
 retranscrit par \citet[p.~447-448]{Milshteyn82a}.
 \item[\DateWithWeekDay{1960-04-07}]
 Moskva~: musée \Scriabine{}.
 Concert avec des œuvres de \Scriabine{}, selon \citet[p.~443]{Scriabine}.
 Selon \citet[p.~184]{Nekrasova08}, en revanche, le programme était le
 suivant.

 \textsc{\Liszt{}}~: \emph{Sposalizio}, S~161 \Number{1}~; \emph{Il
 penseroso}, S~161 \Number{2}~; \emph{Canzonetta del Salvator Rosa}, S~161
 \Number{3}~; Trois \emph{Sonetto del Petrarca}, S~161 \Number{4, 5~et~6}.
 \textsc{\Beethoven{}}~: \emph{Sonata quasi una fantasia}, \Opus{27}
 \Number{2}.
 \textsc{\Scriabine{}}~: Fantaisie~; Trois Études extraites de l'\Opus{42}~;
 Douze Préludes~; Trois Poèmes.
 \item[\DateWithWeekDay{1960-04-18}]\phantomsection\label{bio:LateMCGH}
 Moskva~: grande salle du conservatoire.
 Concert mentionné par \VSofronitsky{}, dans une lettre datée du même jour
 \citep[voir][p.~184]{Nekrasova08}.
 Un concert de \Sofronitsky{} dans la grande salle du conservatoire, à cette
 époque de sa vie, est assez surprenant \citep[p.~59]{Juban}, mais la suite
 de la lettre évoque une répétition pour des examens, après le concert mais
 le même soir, à la petite salle qui avait sa préférence depuis~1955.
 \citet[p.~184]{Nekrasova08} ajoute que ce concert n'est mentionné ni dans
 la chronologie établie par \citet[p.~455-461]{Milshteyn82a}, ni dans les
 archives d'\AVizel{}~; il est encore absent de la chronologie établie par
 \citet[p.~443]{Scriabine}.

 \textsc{\Mendelssohn{}}~: Variations sérieuses en \kD mineur, \Opus{54}.
 \textsc{\Brahms{}}~: Ballade.
 \textsc{\Schumann{}}~: Fantaisie en \kC majeur, \Opus{17}.
 Nombreuses autres œuvres plus brèves (peut-être de \Schumann{}).
 \item[\DateWithWeekDay{1960-04-23}]
 Moskva~: petite salle du conservatoire.
 Concert avec des œuvres de \Liszt{}.

 \textsc{\Liszt{}}~: \emph{Sposalizio}, S~161 \Number{1}~; \emph{Il
 penseroso}, S~161 \Number{2}~; \emph{Canzonetta del Salvator Rosa}, S~161
 \Number{3}~; Sonate en \kB mineur, S~178~; \emph{Sonetto~123 del Petrarca},
 S~161 \Number{6}~; \emph{Sonetto~47 del Petrarca}, S~161 \Number{4}~;
 \emph{Sonetto~104 del Petrarca}, S~161 \Number{5}~; \emph{Gnomenreigen},
 S~145 \Number{2}~; Méphisto-valse.
 \emph{Bis} \citep[selon][p.~184]{Nekrasova08} -- \textsc{\Liszt{}}~: Valse
 oubliée~; Feux follets, S~139 \Number{5}.
 \item[\DateWithWeekDay{1960-04-27}]
 Moskva~: musée \Scriabine{}.
 Concert avec des œuvres de \Scriabine{}.
 Voir en particulier \citet[p.~372-376]{Panarine}.

 \textsc{\Scriabine{}}~: Étude en \kC \Sharp mineur, \Opus{2} \Number{1}~;
 Deux Nocturnes, \Opus{5}~; Huit Études extraites des \Opus{8 et~42}~;
 Fantaisie (sans numéro d'opus précisé)~; Deux Préludes, \Opus{27}~; Deux
 Préludes extraits de l'\Opus{48}~; Deux Morceaux, \Opus{57}~; Danse
 languide, \Opus{51} \Number{4}~; Sonate en \kF \Sharp majeur, \Opus{30}.
 \item[\DateWithWeekDay{1960-05-13}]
 Moskva~: petite salle du conservatoire.
 Concert enregistré en grande partie.
 En plus des œuvres enregistrées, il semble que l'\hbox{Étude} de
 \Scriabine{} en \kD \Sharp mineur, \Opus{8} \Number{12}, ait aussi été
 jouée lors de ce concert, à la fin des \emph{bis} \citep[p.~69]{White}.
 Voir en particulier \citet[p.~447]{Milshteyn82a}.

 \textsc{\Mozart{}}~: Fantaisie en \kC mineur, K~475.
 \textsc{\Schubert{}}~: Impromptu en \kG \Flat majeur, D~899 \Number{3}~;
 Impromptu en \kA \Flat majeur, D~899 \Number{4}.
 \textsc{\Schumann{}}~: Sonate en \kF \Sharp mineur, \Opus{11}.
 \textsc{\Chopin{}}~: Nocturne en \kF \Sharp majeur, \Opus{15} \Number{2}~;
 Nocturne en \kF majeur, \Opus{15} \Number{1}~; Scherzo en \kB mineur,
 \Opus{20}~; Scherzo en \kB \Flat mineur, \Opus{31}.
 \textsc{\Rachmaninov{}}~: Moment musical en \kD \Flat majeur, \Opus{16}
 \Number{5}~; Moment musical en \kE \Flat mineur, \Opus{16} \Number{2}.
 \textsc{\Scriabine{}}~: Sonate en \kF \Sharp majeur, \Opus{30}.
 \emph{Bis} -- \textsc{\Scriabine{}}~: Poème tragique, \Opus{34}~; Valse en
 \kA \Flat majeur, \Opus{38}~; Étude en \kB \Flat mineur, \Opus{8}
 \Number{11}~; Étude en \kD \Sharp mineur, \Opus{8} \Number{12}.
 \item[\DateWithWeekDay{1960-05-23}]
 Moskva~: petite salle du conservatoire.
 Concert avec des œuvres de \Chopin{}, pour commémorer le~150\ieme{}
 anniversaire de la naissance du compositeur.
 Programme indiqué par \citet[p.~185]{Nekrasova08}.

 \textsc{\Chopin{}}~: Vingt-quatre Préludes, \Opus{28}~; Ballade \Number{1}
 en \kG mineur, \Opus{23}~; Ballade \Number{3} en \kA \Flat majeur,
 \Opus{47}~; Mazurkas~; Valses~; Scherzo \Number{1} en \kB mineur,
 \Opus{20}~; Fantaisie en \kF mineur, \Opus{49}.
 \item[\DateWithWeekDay{1960-05-29}]
 Moskva~: petite salle du conservatoire.
 Concert avec des œuvres de \Scriabine{}, pour commémorer le~45\ieme{}
 anniversaire de la mort du compositeur.
 Concert et programme indiqués par \citet[p.~185]{Nekrasova08} et retrouvés
 par \citet{TADGO1960}.

 \textsc{\Scriabine{}}~: Six Préludes extraits de l'\Opus{11} (\Number{1} en
 \kC majeur, \Number{3} en \kG majeur, \Number{6} en \kB mineur, \Number{7}
 en \kA majeur, \Number{9} en \kE majeur et \Number{12} en \kG \Sharp
 mineur)~; Sonate \Number{3} en \kF \Sharp mineur, \Opus{23}~; Deux Préludes
 extraits des \Opus{35 et~48}~; Trois Études extraites de l'\Opus{42}
 (\Number{2} en \kF \Sharp mineur, \Number{4} en \kF \Sharp majeur et
 \Number{5} en \kC \Sharp mineur)~; Sonate \Number{4} en \kF \Sharp majeur,
 \Opus{30}~; Fantaisie en \kB mineur, \Opus{28}~; Sonate \Number{9},
 \Opus{68}~; Sonate \Number{5}, \Opus{53}.
 \emph{Bis} -- \textsc{\Scriabine{}}~: Un Poème extrait de l'\Opus{52}~;
 Ironies en \kC majeur, \Opus{56} \Number{2}~; Un Poème extrait de
 l'\Opus{32}.
 \item[\DateWithWeekDay{1960-06-18}]
 Moskva~: musée \Scriabine{}.
 Concert enregistré en partie.

 \textsc{\Scriabine{}}~: Prélude en \kE majeur, \Opus{11} \Number{9}~;
 Prélude en \kC \Sharp mineur, \Opus{11} \Number{10}~; Prélude en \kG \Sharp
 mineur, \Opus{16} \Number{2}.
 \item[\DateWithWeekDay{1960-06-19}]
 Moskva~: petite salle du conservatoire.
 Concert avec des œuvres de \Schumann{}, pour commémorer le~150\ieme{}
 anniversaire de la naissance du compositeur.
 Programme du concert indiqué par \citet[p.~185]{Nekrasova08} et retrouvé
 par \citet{TADGO1960}~; voir aussi \citet[p.~387]{Shiryaeva} pour les trois
 dernières œuvres mentionnées dans le programme ci-dessous.

 \textsc{\Schumann{}}~: Trois \emph{Phantasiestücke}, \Opus{111} (\Number{1}
 en \kC mineur, \Number{2} en \kA \Flat majeur et \Number{3} en \kC
 mineur)~; Deux \emph{Novelettes} extraites de l'\Opus{21} (\Number{1} en
 \kF majeur et \Number{8} en \kF \Sharp mineur)~; \emph{Kreisleriana},
 \Opus{16}~; \emph{Humoreske} en \kB \Flat majeur, \Opus{20}~; Romances,
 \Opus{28}~; Extraits des \emph{Bunte Blätter}, \Opus{99}.
 \textsc{\Schumann{}/\Liszt{}}~: \foreignlanguage{russian}{Посвящение}
 [Posvjaščenie, Dédicace].
 \item[\DateWithWeekDay{1960-06-28}]
 Moskva~: musée \Scriabine{}.
 \citet[p.~185]{Nekrasova08} mentionne cette date pour un concert de
 \VSofronitsky{}.
 Le programme n'a pas été préservé, mais la date est extraite des dossiers
 du musée.
 \item[\DateWithWeekDay{1960-07-12}]
 Moskva~: musée \Scriabine{}.
 Concert enregistré en partie.

 \textsc{\Scriabine{}}~: Deux Danses, \Opus{73}~; Masque, \Opus{63}
 \Number{1}~; Prélude, \Opus{67} \Number{1}~; Sonate-fantaisie en \kG \Sharp
 mineur, \Opus{19}.
 \item[\DateWithWeekDay{1960-07-18}]
 Moskva~: musée \Scriabine{}.
 Concert enregistré en partie.
 Les trois premières œuvres de \Scriabine{} sont mentionnées par
 \citet[p.~13]{Johansson}.
 \citet[p.~186]{Nekrasova08} indique un ordre permuté, avec la première
 partie consacrée à \Scriabine{} et la seconde à \Chopin{}~; elle mentionne
 aussi que l'interprétation du Nocturne de \Chopin{} en \kE \Flat majeur fut
 \Quote{une véritable révélation}.

 \textsc{\Scriabine{}}~: Allegro de concert en \kB \Flat mineur, \Opus{18}~;
 Prélude en \kG mineur, \Opus{27} \Number{1}~; Prélude en \kB majeur,
 \Opus{37} \Number{3}~; Prélude en \kF \Sharp majeur, \Opus{39} \Number{1}~;
 Prélude en \kD majeur, \Opus{39} \Number{2}~; Prélude en \kG majeur,
 \Opus{39} \Number{3}~; Prélude en \kA \Flat majeur, \Opus{39} \Number{4}~;
 Sonate en \kF \Sharp majeur, \Opus{30}.
 \textsc{\Chopin{}}~: Nocturne en \kE \Flat majeur, \Opus{9} \Number{2}~;
 Nocturne en \kF majeur, \Opus{15} \Number{1}~; Nocturne en \kF \Sharp
 majeur, \Opus{15} \Number{2}~; Sept Préludes extraits de l'\Opus{28}~:
 \Number{14} en \kE \Flat mineur, \Number{15} en \kD \Flat majeur,
 \Number{16} en \kB \Flat mineur, \Number{17} en \kA \Flat majeur,
 \Number{18} en \kF mineur, \Number{21} en \kB \Flat majeur et \Number{22}
 en \kG mineur.
 \item[\DateWithWeekDay{1960-07-20}]
 Moskva~: musée \Scriabine{}.
 \citet[p.~188]{Nekrasova08} mentionne une session d'enregistrement, au
 musée, avec le même programme que celui du concert précédent~; la session a
 duré trois heures, qui furent les dernières durant lesquelles
 \citeauthor{Nekrasova08} a pu écouter \Sofronitsky{} jouer en direct.
 \item[\DateWithWeekDay{1960-07-22}]
 Moskva~: musée \Scriabine{}.
 Concert enregistré en partie.
 Les deux pièces extraites de l'\Opus{56} sont mentionnées par
 \citet[p.~13]{Johansson}.

 \textsc{\Scriabine{}}~: Fantaisie en \kB mineur, \Opus{28}~; Poème en \kC
 majeur, \Opus{45} \Number{2}~; Étude en \kE \Flat majeur, \Opus{49}
 \Number{1}~; Prélude en \kF majeur, \Opus{49} \Number{2}~; Rêverie en \kC
 majeur, \Opus{49} \Number{3}~; Fragilité, \Opus{51} \Number{1}~; Ironies en
 \kC majeur, \Opus{56} \Number{2}~; Nuances, \Opus{56} \Number{3}~; Désir,
 \Opus{57} \Number{1}~; Caresse dansée, \Opus{57} \Number{2}.
 \item[B\DateWithWeekDay{1960-07-24}]
 Moskva~: musée \Scriabine{}.
 Rencontre de \VSofronitsky{} et de \VanCliburn{}%
 \footnote{\foreignlanguage{russian}{\emph{Советская музыка}}, vol.~264,
 \Number{11} (1960), p.~204-205.}
 \citep[voir par exemple][p.~399-403]{Shiryaeva}.
 \VanCliburn{} avait remporté en~1958 le premier prix du premier concours
 international \PTchaikovski{} à Moskva.
 Dans une lettre datée du~2 août~1960, \VSofronitsky{} évoque pour \AVizel{}
 sa rencontre avec \VanCliburn{} et indique que les œuvres qu'ils ont jouées
 l'un pour l'autre, à cette occasion, ont été enregistrées sur bande
 \citep[voir][p.~188]{Nekrasova08}.
 \item[\DateWithWeekDay{1960-09-19}]
 Moskva~: petite salle du conservatoire.
 Voir en particulier \citet[p.~451]{Milshteyn82a}.

 \textsc{\Mendelssohn{}}~: Variations sérieuses en \kD mineur, \Opus{54}.
 \textsc{\Chopin{}}~: Sonate en \kB mineur, \Opus{58}~; Nocturne en \kG
 majeur, \Opus{37} \Number{2}~; Nocturne en \kC mineur, \Opus{48}
 \Number{1}~; Ballade en \kF mineur, \Opus{52}~; Berceuse en \kD \Flat
 majeur, \Opus{57}~; Tarentelle en \kA \Flat majeur, \Opus{43}.
 \textsc{\Schubert{}/\Liszt{}}~: \emph{Frühlingsglaube}~; \emph{Auf dem
 Wasser zu singen}.
 \textsc{\SaintSaens{}/\Liszt{}}~: Danse macabre.
 \item[\DateWithWeekDay{1960-09-20}]
 Moskva~: musée \Scriabine{}.
 \citet[p.~188]{Nekrasova08} mentionne cette date pour un concert de
 \VSofronitsky{}.
 Le programme n'est pas connu, mais la date est extraite des archives
 d'\AVizel{} et confirmée par le musée.
 \item[\DateWithWeekDay{1960-09-21}]
 Moskva~: petite salle du conservatoire.
 Même programme que le~19 septembre~1960.
 \item[\DateWithWeekDay{1960-10-11}]
 Moskva~: petite salle du conservatoire.
 Concert enregistré en partie.
 Voir en particulier \citet[p.~451]{Milshteyn82a}.

 \textsc{\Schubert{}}~: Sonate en \kB \Flat majeur, D~960.
 \textsc{\Schubert{}/\Liszt{}}~: \emph{Litanei}.
 \textsc{\Liszt{}}~: Sonate en \kB mineur, S~178.
 \textsc{\Schubert{}/\Liszt{}}~: \emph{Der Müller und der Bach}~;
 \emph{Aufenthalt}~; \emph{Erlkönig}.
 \item[\DateWithWeekDay{1960-10-14}]
 Moskva~: petite salle du conservatoire.
 Concert enregistré en partie.
 Voir en particulier \citet[p.~451]{Milshteyn82a}.
 Même programme que le~11 octobre~1960.

 \emph{Bis} (le~14 octobre~1960) -- \textsc{\Schubert{}/\Liszt{}}~:
 \emph{Frühlingsglaube}~; \emph{Der Doppelgänger}.
 \item[\DateWithWeekDay{1960-10-22}]
 Moskva~: petite salle du conservatoire.
 Concert enregistré en partie.
 Voir en particulier \citet[p.~451-452]{Milshteyn82a}.
 Selon le programme de ce récital, retrouvé par \citet{TADGO1960}, la
 Ballade de \Chopin{}, \Opus{47}, n'a pas été jouée ce soir, et l'on trouve
 à la place six Études (trois de l'\Opus{10}, trois de l'\Opus{25}).

 \textsc{\Beethoven{}}~: \emph{Andante} favori en \kF majeur, WoO~57~;
 Sonate en \kF mineur, \Opus{57}.
 \textsc{\Chopin{}}~: Ballade en \kA \Flat majeur, \Opus{47}.
 \textsc{\Liszt{}}~: Funérailles, S~173 \Number{7}~; Valse oubliée
 \Number{1}, S~215 \Number{1}~; Méphisto-valse \Number{2}, S~515.
 \textsc{\Schubert{}/\Liszt{}}~: \emph{Frühlingsglaube}~; \emph{Litanei}~;
 \emph{Der Müller und der Bach}~; \emph{Der Doppelgänger}~;
 \emph{Aufenthalt}.

 Selon \citet[p.~70, note~13]{White}, les cinq œuvres de
 \Schubert{}/\Liszt{} ci-dessus ont été incluses dans le disque vinyle
 Melodija D010315/6, à partir des radiodiffusions des concerts du~22 et
 du~28 octobre~1960, mais elles n'ont pas été incluses dans le \Volume{8}
 des \Quote{Enregistrements complets}, à partir des radiodiffusions des
 mêmes concerts.
 Cela n'est toutefois pas confirmé dans la discographie de
 \citet[p.~69]{Malik}.
 En revanche, c'est confirmé dans celle de \citet[p.~2]{Nikonovich11}
 \citep[voir aussi][p.~377]{Scriabine}, raison pour laquelle l'information
 est maintenue ici.
 \item[\DateWithWeekDay{1960-10-28}]
 Moskva~: petite salle du conservatoire.
 Concert enregistré en partie.
 Voir en particulier \citet[p.~451-452]{Milshteyn82a}.
 Même programme que le~22 octobre~1960.
 \citet[p.~452]{Milshteyn82a} mentionne le \emph{Sonetto~123 del Petrarca},
 à la place du \emph{Sonetto~104 del Petrarca}, pour le récital de ce soir.

 \emph{Bis} (le~28 octobre~1960) -- \textsc{\Chopin{}}~: Étude en \kC \Sharp
 mineur, \Opus{10} \Number{4}.
 \textsc{\Liszt{}}~: \emph{Sonetto~104 del Petrarca}, S~161 \Number{5}.
 \textsc{\Rachmaninov{}}~: Prélude en \kG majeur, \Opus{32} \Number{5}~;
 Prélude en \kG \Sharp mineur, \Opus{32} \Number{12}.
 \textsc{\Chopin{}}~: Nocturne en \kF \Sharp majeur, \Opus{15} \Number{2}.
 Trois œuvres jouées en \emph{bis} lors d'au moins une de ces deux dates
 (le~22 et le~28 octobre~1960) -- \textsc{\Chopin{}}~: Étude en \kE majeur,
 \Opus{10} \Number{3}.
 \textsc{\Liszt{}}~: \emph{Gnomenreigen}, S~145 \Number{2}.
 \textsc{\Scriabine{}}~: Poème en \kF \Sharp majeur, \Opus{32} \Number{1}.

 Voir le disque vinyle Melodija D015025/6, le \Volume{8} des
 \Quote{Enregistrements complets} et \citet[p.~70, note~14]{White}.
 Voir aussi \citet[p.~4 et~6]{Nikonovich11}.
 \item[\DateWithWeekDay{1960-11-06}]
 Moskva~: petite salle du conservatoire.
 Voir en particulier \citet[p.~452]{Milshteyn82a}.
 Selon le programme de ce récital, retrouvé par \citet{TADGO1960}, une œuvre
 supplémentaire de \Liszt{}, la \emph{Canzonetta del Salvator Rosa}, S~161
 \Number{3}, aurait aussi été jouée.

 \textsc{\Beethoven{}}~: Sonate en \kD majeur, \Opus{28}~; Sonate en \kC
 mineur, \Opus{111}.
 \textsc{\Chopin{}}~: Scherzo en \kC \Sharp mineur, \Opus{39}.
 \textsc{\Liszt{}}~: \emph{Sposalizio}, S~161 \Number{1}~; \emph{Il
 penseroso}, S~161 \Number{2}~; Méphisto-valse \Number{1}, S~514.
 \emph{Bis} -- \textsc{\Chopin{}}~: Mazurka.
 \textsc{\Debussy{}}~: \emph{Serenade of the Doll}, L~113 \Number{III}~;
 \emph{General Lavine -- eccentric}, L~123 \Number{VI}~; Feux d'artifice,
 L~123 \Number{XII}.
 \item[\DateWithWeekDay{1960-11-14}]
 Moskva~: maison des scientifiques.
 Concert et programme retrouvés par \citet{TADGO1960}.

 \textsc{\Haendel{}}~: Variations en \kE majeur.
 \textsc{\Schumann{}}~: Études symphoniques, \Opus{13}.
 \textsc{\Chopin{}}~: Deux Nocturnes, \Opus{27} (\Number{1} en \kC \Sharp
 mineur et \Number{2} en \kD \Flat majeur)~; Impromptu \Number{3} en \kG
 \Flat majeur, \Opus{51}~; Barcarolle en \kF \Sharp majeur, \Opus{60}~;
 Scherzo \Number{3} en \kC \Sharp mineur, \Opus{39}~; Scherzo \Number{2} en
 \kB \Flat mineur, \Opus{31}.
 \item[\DateWithWeekDay{1960-11-22}]
 Moskva~: petite salle du conservatoire.
 Voir en particulier \citet[p.~440 et p.~452]{Milshteyn82a}.
 Récital évoqué et critiqué par \citet[p.~138-139]{Rabinovich61}, sous le
 pseudonyme de Florestan.
 \citet[p.~189]{Nekrasova08} ne mentionne que les \Opus{27 et~43} pour les
 œuvres de \Chopin{}.

 \textsc{\Haendel{}}~: Air et variations en \kE majeur.
 \textsc{\Schumann{}}~: Études symphoniques, \Opus{13}.
 \textsc{\Chopin{}}~: Nocturnes en \kC \Sharp mineur et en \kD \Flat majeur,
 \Opus{27} \Number{1} et \Number{2}~; Impromptu en \kG \Flat majeur,
 \Opus{51}~; Barcarolle en \kF \Sharp majeur, \Opus{60}~; Tarentelle en \kA
 \Flat majeur, \Opus{43}.
 \textsc{\Debussy{}}~: \emph{Serenade of the Doll}, L~113 \Number{III}~;
 \emph{Minstrels}, L~117 \Number{XII}~; La Fille aux cheveux de lin, L~117
 \Number{VIII}~; \emph{General Lavine -- eccentric}, L~123 \Number{VI}~;
 Canope, L~123 \Number{X}~; Feux d'artifice, L~123 \Number{XII}.
 \item[\DateWithWeekDay{1960-12-05}]
 Moskva~: petite salle du conservatoire.
 Ce concert aurait dû avoir lieu le~20 novembre, mais il a été reprogrammé
 \citep{TADGO1960}.
 Dernier concert en soliste au conservatoire de Moskva.
 Voir en particulier \citet[p.~441 et p.~452]{Milshteyn82a}.
 Par la suite, le travail de \VSofronitsky{} reprend au conservatoire de
 Moskva, et il indique donc ignorer quand aura lieu son prochain concert
 \citep[voir][p.~416]{Shiryaeva}.

 \textsc{\Haendel{}}~: Air et variations en \kE majeur.
 \textsc{\Schumann{}}~: Études symphoniques, \Opus{13}.
 \textsc{\Chopin{}}~: Nocturnes en \kC \Sharp mineur et en \kD \Flat majeur,
 \Opus{27} \Number{1} et \Number{2}~; Barcarolle en \kF \Sharp majeur,
 \Opus{60}~; Tarentelle en \kA \Flat majeur, \Opus{43}.
 \textsc{\Debussy{}}~: \emph{Doctor Gradus ad Parnassum}, L~113 \Number{I}~;
 \emph{Minstrels}, L~117 \Number{XII}~; Feuilles mortes, L~123 \Number{II}~;
 La Fille aux cheveux de lin, L~117 \Number{VIII}~; \emph{General Lavine --
 eccentric}, L~123 \Number{VI}~; Canope, L~123 \Number{X}~; Feux d'artifice,
 L~123 \Number{XII}.
 \textsc{\Prokofiev{}}~: Marche pour piano extraite de l'opéra L'Amour des
 trois oranges, \Opus{33ter} \Number{1}.
 \emph{Bis} -- \textsc{\Rachmaninov{}}~: Étude-tableau.
 \textsc{\Chopin{}}~: Mazurka.
 \item[\DateWithWeekDay{1960-12-11}]
 Moskva.
 Dernière session d'enregistrements en studio.

 \textsc{\Liszt{}}~: \emph{Sposalizio}, S~161 \Number{1}.
 \textsc{\Schubert{}/\Liszt{}}~: \emph{Litanei}, S~562 \Number{1}.
 \textsc{\Scriabine{}}~: Sonate-fantaisie \Number{2} en \kG \Sharp mineur,
 \Opus{19} (premier mouvement).
 \item[\DateWithWeekDay{1960-12-24}]
 Moskva~: musée \Scriabine{}.
 Concert enregistré en partie.

 \textsc{\Scriabine{}}~: Sonate, \Opus{66}.
 \item[\DateWithWeekDay{1960-12-27}]
 Moskva~: musée \Scriabine{}.

 \textsc{\Scriabine{}}~: Prélude en \kB \Flat majeur, \Opus{35} \Number{2}~;
 Prélude en \kD \Flat majeur, \Opus{35} \Number{1}~; Prélude en \kB \Flat
 mineur, \Opus{37} \Number{1}~; Prélude en \kD \Flat majeur/\kC majeur,
 \Opus{31} \Number{1}~; Sonate, \Opus{68}~; Sonate, \Opus{66}~; Vers la
 flamme, \Opus{72}.
\end{description}

\section{Année~1961}

\begin{description}
 \item[\DateWithWeekDay{1961-01-07}]
 Moskva~: musée \Scriabine{}.
 Dernier concert au musée \Scriabine{}, enregistré en partie.
 Voir en particulier \citet[p.~354]{Badeyan08}.
 \citet[p.~105]{Nikonovich08a} indique que \HNeuhaus{} était présent, au
 musée \Scriabine{}, pour ce concert de \VSofronitsky{}.

 \textsc{\Scriabine{}}~: Prélude en \kB \Flat majeur, \Opus{35} \Number{2}~;
 Prélude en \kD \Flat majeur, \Opus{35} \Number{1}~; Prélude en \kB \Flat
 mineur, \Opus{37} \Number{1}~; Sonate, \Opus{62}~; Mazurka en \kE mineur,
 \Opus{25} \Number{3}~; Sonate, \Opus{70}~; Danse, \Opus{73} \Number{2}~;
 Prélude en \kF \Sharp majeur, \Opus{37} \Number{2}~; Prélude, \Opus{74}
 \Number{2}~; Sonate, \Opus{66}.
 \item[\DateWithWeekDay{1961-01-09}]
 Moskva~: petite salle du conservatoire \citep[voir][p.~61]{Juban}.
 Dernière apparition en public, lors d'une soirée avec d'autres musiciens,
 pour les Brigades de travail communistes, des groupes récompensés
 d'ouvriers d'usine \citep[p.~71]{White}.

 \textsc{\Chopin{}}~: Fantaisie en \kF mineur, \Opus{49}~; Mazurka en \kC
 \Sharp mineur, \Opus{50} \Number{3}~; Ballade en \kG mineur, \Opus{23}.
 \textsc{\Rachmaninov{}}~: Prélude en \kG \Sharp mineur, \Opus{32}
 \Number{12}.
 \textsc{\Scriabine{}}~: Étude en \kD \Sharp mineur, \Opus{8} \Number{12}.
 \item[B1961-02]
 \citet[p.~190]{Nekrasova08} mentionne que des concerts de \VSofronitsky{}
 ont été annoncés pour les~19 et~21 février~1961, mais qu'ils n'ont pas eu
 lieu, en raison de la maladie qui progressait.
 \item[B\DateWithWeekDay{1961-03-07}]
 Pour le huitième anniversaire de la mort du compositeur, \VSofronitsky{}
 avait prévu un concert dédié à la musique de \SProkofiev{}~; le concert a
 été annulé en raison de la maladie \citep[voir][p.~393]{Shiryaeva}.
 Le pianiste travaillait en particulier sur des pièces de la Suite
 \emph{Roméo et Juliette}, \Opus{75}.
 \item[B\DateWithWeekDay{1961-04-18}]
 Dernière lettre de \VSofronitsky{} à \AVizel{}, où il lui indique en
 particulier que son concert prévu le~23 avril est en danger d'annulation.
 \item[B\DateWithWeekDay{1961-05-27}]
 Publication d'un article de \HNeuhaus{}%
 \footnote{\foreignlanguage{russian}{\emph{Советская культура}},
 27~mai~1961.}
 consacré à \VSofronitsky{}, s'achevant sur ces mots~: \emph{Слава ему,
 бесподобному поэту фортепиано~!}
 \Quote{Gloire à lui, poète incomparable du piano~!}
 \item[B\DateWithWeekDay{1961-06-23}]
 Dernière rencontre d'\INikonovich{} avec \VSofronitsky{}
 \citep[voir][p.~113]{Nikonovich08a}.
 \item[B\DateWithWeekDay{1961-08-05}]
 Annonce officielle de soutien de la requête du conservatoire de Moskva
 d'attribution du titre honorifique \Quote{Artiste du peuple de la~RSFSR}
 (République socialiste fédérative soviétique de Rossija) pour
 \VSofronitsky{}.
 \item[B\DateWithWeekDay{1961-08-29}]
 Décès de \VSofronitsky{} à Moskva, vers~3\up{h}~35\up{m} du matin
 \citep[voir][p.~65]{Sofronitsky82b}.
 \item[B\DateWithWeekDay{1961-08-31}]
 Inhumation au cimetière Novodevič'e à Moskva (secteur~8~[13]~155), après
 une cérémonie d'hommage à la petite salle du conservatoire
 \citep[voir][p.~325]{Zolotov08}.
\end{description}
