\chapter[%
Colophon][%
Colophon]{%
Colophon}
\label{chap:Colophon}

\section{Écoutes comparatives}

Afin d'établir une discographie, il est utile voire nécessaire de disposer
d'outils informatiques qui permettent de comparer des fichiers audio lors de
leur écoute.
Le logiciel~Audacity développé par \citet{Audacity} permet d'écouter deux ou
plusieurs fichiers audio et de visualiser en parallèle leurs spectrogrammes,
en superposant les fenêtres correspondantes.
On peut ensuite normaliser les durées ou tempos des fichiers audio comparés
-- en préservant ou non le diapason (hauteur ou \emph{pitch}) -- et enfin
établir une corrélation entre un nouveau fichier audio et une référence déjà
connue.
L'application~Sonic Visualiser développée par \citet{Cannam} produit le même
type de résultat mais d'une manière plus systématique avec le greffon~MATCH
dû à \citet{Dixon05a} et \citet{Dixon05b}.

\section{Composition typographique}

La compilation de ce document à partir des sources requiert le système de
composition typographique~\Latex développé par l'équipe du projet~\Latex{}3
\citep{LaTeX}.
La distribution complète du document comporte le présent fichier~PDF,
\cad~\fileurl{dvs.pdf}, et les fichiers de source~\Latex, \cad la classe de
document~\fileurl{cls/dvs.cls}, le fichier maître~\fileurl{dvs.tex} et les
sources proprement dites~\fileurl{tex/*.tex}.
L'outil~\Xindy de \citet{Xindy} permet de préparer les index~; les fichiers
de style sont~\fileurl{xdy/dvs\_*.xdy}.
(Voir ci-dessous quelques détails d'utilisation des index.)
Les outils~\Biber et~\Biblatex de \citet{Lehman} permettent de préparer la
bibliographie~; la base de données est~\fileurl{bib/dvs.bib}.
Le script de \emph{shell}~Bash~\fileurl{dvs.sh} automatise la production du
document \citep{Bash}.
Les sources sont rédigées en plusieurs langues -- français, anglais, russe,
portugais, espagnol, italien, allemand et néerlandais -- et utilisent les
alphabets latin et cyrillique~: l'emploi d'un moteur de mise en pages qui
utilise de manière native la représentation Unicode des caractères et les
polices de caractères au format~OpenType semble la meilleure solution pour
un projet de ce type.
La version~\engversion du moteur~\engname de \citet{LuaTeX} génère donc ici
la sortie~PDF du document, \via la version~\fmtversion{} du format~\Latex.
Le fichier~\fileurl{dvs.ver} renseigne les méta\-données du projet.
Les sommes de contrôle~SHA224 \citep{SHA} de tous les fichiers distribués se
trouvent dans le fichier~\fileurl{dvs.sha}.
Les autres fichiers sont des fichiers temporaires engendrés par~\engname,
\Biber, \Biblatex ou~\Xindy~: on peut les supprimer sans problème et laisser
le système les reconstituer à partir des sources brutes.
Le programme~QPDF de \citet{Berkenbilt} assure \emph{a~posteriori} la
linéarisation ou optimisation du fichier~PDF pour une visualisation en ligne
plus efficiente.

Une particularité de cette discographie est que la production automatique de
plusieurs de ses chapitres ou sections est déléguée au système d'indexation,
donc au programme~\Xindy, qui permet de maintenir l'homogénéité d'une
présentation par nature systématique.
Les éléments du document organisés comme des index sont~:
\begin{itemize}
 \item
 la liste des compositeurs et œuvres par ordre de numéros d'opus, en
 page~\pageref{chap:Oeuvres} (étiquette~: \fileurl{ndxworks}).
 Fichier de style pour~\Xindy~: \fileurl{xdy/dvs\_w.xdy}.
 Fichiers temporaires~: \fileurl{dvs.i*w}~;
 \item
 la chronologie des enregistrements en studio et en direct, en
 page~\pageref{chap:Chronologie} (étiquette~: \fileurl{ndxperfs}).
 Fichier de style pour~\Xindy~: \fileurl{xdy/dvs\_p.xdy}.
 Fichiers temporaires~: \fileurl{dvs.i*p}~;
 \item
 le contenu des disques vinyles et des disques compacts, en
 page~\pageref{chap:Contenu} (étiquette~: \fileurl{ndxdiscs}).
 Fichier de style pour~\Xindy~: \fileurl{xdy/dvs\_d.xdy}.
 Fichiers temporaires~: \fileurl{dvs.i*d}~;
 \item
 la liste des enregistrements indisponibles ou non publiés, en
 page~\pageref{sec:IndisponibleNonpublie} (étiquette~: \fileurl{ndxunpub}).
 Fichier de style pour~\Xindy~: \fileurl{xdy/dvs\_u.xdy}.
 Fichiers temporaires~: \fileurl{dvs.i*u}~;
 \item
 la liste des citations, associée au chapitre de bibliographie, en
 page~\pageref{chap:Listedescitations} (étiquette~: \fileurl{ndxauths}).
 Fichier de style pour~\Xindy~: \fileurl{xdy/dvs\_a.xdy}.
 Fichiers temporaires~: \fileurl{dvs.i*a}~;
 \item
 l'index onomastique, qui reprend les noms mentionnés dans le texte, en
 page~\pageref{chap:Indexonomastique} (étiquette~: \fileurl{ndxnames}).
 Fichier de style pour~\Xindy~: \fileurl{xdy/dvs\_n.xdy}.
 Fichiers temporaires~: \fileurl{dvs.i*n}.
\end{itemize}

Afin de résoudre les références croisées du document -- table des matières,
bibliographie, index,~etc. --, il est nécessaire d'en compiler les sources à
cinq reprises successives~; je recommande néanmoins de faire précéder les
quatre dernières compilations d'exécutions de~\Biber et de~\Xindy sur leurs
fichiers~\fileurl{dvs.bcf} et~\fileurl{dvs.idx*} respectifs.
Outre le moteur~\Luatex et les programmes~\Biber et~\Xindy, le document
emploie la classe~\fileurl{memoir} et les styles~\fileurl{ifthen},
\fileurl{showframe} (selon option), \fileurl{xcolor}, \fileurl{fontspec},
\fileurl{babel}, \fileurl{csquotes}, \fileurl{biblatex}, \fileurl{multicol},
\fileurl{calc}, \fileurl{tikz}, \fileurl{pgfcalendar}, \fileurl{soul},
\fileurl{xstring}, \fileurl{hyperref}, \fileurl{bookmark}, \fileurl{xspace},
\fileurl{hologo}, \fileurl{longtable}, \fileurl{microtype},
\fileurl{enumitem} et~\fileurl{nowidow}.
La compilation est réalisée, sous le système d'exploitation \citet{Debian},
avec la distribution~\Tex{}~Live supervisée par \citet{TeXLive}.

\section{Polices de caractères}

En dehors de quelques symboles, les polices de caractères utilisées sont
Libertinus Serif (police principale à empattements pour le texte), dessinée
par Philipp H.~\citet{Poll} sous le nom \emph{Linux Libertine} et reprise
ensuite par Khaled \citet{Hosny} puis par Caleb \citet{Maclennan}, Source
Sans Pro (police sans empattements pour les inter\-titres), dessinée par
Paul D.~\citet{HuntSans}, et Source Code Pro (police à chasse fixe), aussi
dessinée par Paul D.~\citet{HuntCode}~; les versions employées sont au
format~OpenType.
Les polices de caractères Asana Math, due à Apostolos \citet{Syropoulos15},
et Creative Commons Icons, due à Michael \citet{Ummels}, deux polices au
format~OpenType, fournissent les symboles nécessaires.
Le style~\fileurl{fontspec} de \citet{Fontspec} permet de sélectionner les
polices de caractères~OpenType sous les moteurs~\Xetex et~\Luatex dans les
documents~\Latex.
