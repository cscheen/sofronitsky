\chapter[%
Lexique des contemporains de Vladimir Sofronickij][%
Lexique des contemporains de Vladimir Sofronickij]{%
Lexique des contemporains de \VSofronitsky{}}
\label{chap:Lexique}

Ce lexique présente, pour chaque personnalité russe ou soviétique qui a été
en relation avec \VSofronitsky{} au cours de sa vie ou qui a écrit à propos
de lui, et qui est mentionnée dans le texte, une brève notice biographique.
Le lexique mentionne en particulier les importantes dynasties \Scriabine{},
\Sofronitsky{} et \Vizel{}.
Les fragments biographiques proviennent, pour la plupart, des trois ouvrages
collectifs placés sous la direction scientifique de \citet{Milshteyn82a}, de
\citet{Nikonovich08}, et de \citet{Scriabine}.
Les articles de \citet{Voskobojnikov09a, Voskobojnikov09b} fournissent dans
certains cas de plus amples informations biographiques.

\vspace{\baselineskip}

\begin{description}
 \item[Adžemov, Konstantin Christoforovič (\Dates{1911}{1985})]%
 \index[ndxnames]{Adžemov, Konstantin Christoforovič}
 Pianiste, enseignant et critique musical.
 Professeur au conservatoire de Moskva.
 \item[Alekseev, Aleksandr Sergeevič (\Dates{1926}{1991})]%
 \index[ndxnames]{Alekseev, Aleksandr Sergeevič}
 Pianiste et enseignant.
 Professeur au conservatoire d'\hbox{Odessa}.
 Élève de \HNeuhaus{}.
 \item[Archangel'skij, Vladimir Aleksandrovič (\Dates{1895}{1958})]%
 \index[ndxnames]{Archangel'skij, Vladimir Aleksandrovič}
 Pianiste et enseignant.
 Assistant de \VSofronitsky{} au conservatoire de Moskva.
 \item[Argamakov, Vasilij Nikolaevič (\Dates{1883}{1965})]%
 \index[ndxnames]{Argamakov, Vasilij Nikolaevič}
 Pianiste et enseignant.
 Compositeur et professeur au conservatoire de Moskva.
 \item[Askol'dova, Galina Davydovna (1927\dvsborn{})]%
 \index[ndxnames]{Askol'dova, Galina Davydovna}
 Pianiste et enseignante.
 Artiste honoré de Rossija.
 \item[Badejan, Tamara Georgievna (1935\dvsborn{})]%
 \index[ndxnames]{Badejan, Tamara Georgievna}
 Ingénieur du son.
 Restauratrice d'enregistrements d'archives.
 \item[Barinova, Galina Vsevolodovna (\Dates{1910}{2006})]%
 \index[ndxnames]{Barinova, Galina Vsevolodovna}
 Violoniste et enseignante.
 Professeur au conservatoire de Moskva.
 Artiste du peuple de l'\hbox{Union} soviétique.
 À partir de~1925, soliste de la philharmonie de Leningrad.
 À partir de~1934, vit et travaille à Moskva.
 \item[Baškirov, Dmitrij Aleksandrovič (\Dates{1931}{2021})]%
 \index[ndxnames]{Baškirov, Dmitrij Aleksandrovič}
 Pianiste et enseignant.
 Professeur au conservatoire de Moskva.
 Élève d'\hbox{Aleksandr} Gol'denvejzer.
 \item[Bendickij, Semen Solomonovič (\Dates{1908}{1993})]%
 \index[ndxnames]{Bendickij, Semen Solomonovič}
 Pianiste et enseignant.
 Élève de \HNeuhaus{}.
 \item[Berkovskaja, Elena Nikolaevna (\Dates{1923}{1998})]%
 \index[ndxnames]{Berkovskaja, Elena Nikolaevna}
 Diplômée de la faculté d'histoire de l'université d'\hbox{État} Lomonosov
 de Moskva.
 Pendant de nombreuses années, bibliothécaire en chef du département de
 littérature étrangère.
 \item[Berman, Lazar' Naumovič (\Dates{1930}{2005})]%
 \index[ndxnames]{Berman, Lazar' Naumovič}
 Pianiste.
 Élève d'\hbox{Aleksandr} Gol'denvejzer.
 \item[Bernbljum, Tina Michajlovna (1922\dvsborn{})]%
 \index[ndxnames]{Bernbljum, Tina Michajlovna}
 Pianiste.
 Élève de \VSofronitsky{}.
 \item[Blok, Aleksandr Aleksandrovič (\Dates{1880}{1921})]%
 \index[ndxnames]{Blok, Aleksandr Aleksandrovič}
 Poète, dramaturge, critique littéraire et traducteur.
 \item[Blumenfel'd, Feliks Michajlovič (\Dates{1863}{1931})]%
 \index[ndxnames]{Blumenfel'd, Feliks Michajlovič}
 Pianiste, compositeur, chef d'orchestre et enseignant au conservatoire de
 Sankt-Peterburg.
 Professeur au conservatoire de Moskva.
 \item[Bogdanov-Berezovskij, Valerian Michajlovič (\Dates{1903}{1971})]%
 \index[ndxnames]{Bogdanov-Berezovskij, Valerian Michajlovič}
 Compositeur et musicologue.
 Élève de \LNikolaiev{} pour le piano.
 \item[Borisov, Jurij Al'bertovič (\Dates{1956}{2007})]%
 \index[ndxnames]{Borisov, Jurij Al'bertovič}
 Directeur musical, régisseur de théâtre dramatique, directeur de cinéma et
 scénariste.
 \item[Bošnjakovič, Oleg Dragomirovič (\Dates{1920}{2006})]%
 \index[ndxnames]{Bošnjakovič, Oleg Dragomirovič}
 Pianiste et enseignant.
 Élève de \KIgumnov{} et de \HNeuhaus{}.
 \item[Bragina, Ol'ga Fedorovna (1934\dvsborn{})]%
 \index[ndxnames]{Bragina, Ol'ga Fedorovna}
 Pianiste et enseignante.
 Élève de \VSofronitsky{}.
 \item[Bujukli, Vsevolod Ivanovič (\Dates{1873}{1920})]%
 \index[ndxnames]{Bujukli, Vsevolod Ivanovič}
 Pianiste.
 Fils d'\hbox{Anna} Lebedeva-Gecevič.
 \item[Čerkasov, Gennadij Konstantinovič (\Dates{1930}{2002})]%
 \index[ndxnames]{Čerkasov, Gennadij Konstantinovič}
 Chef d'orchestre et pianiste.
 Professeur au conservatoire de Moskva.
 A travaillé comme chef d'orchestre au théâtre Bol'šoj de~1962 à~1964 et
 comme chef d'orchestre principal du théâtre d'opérette de Moskva de~1964
 à~1972.
 Directeur de l'orchestre de chambre du conservatoire.
 À partir de~1972, rédacteur en chef de la diffusion musicale de la radio
 \Quote{\emph{All-Union}}.
 À partir du milieu des années~1990 et jusqu'en~2002, directeur du Centre
 d'\hbox{État} russe pour la télévision et la radiodiffusion.
 Artiste du peuple de Rossija.
 \item[Činaev, Vladimir Petrovič (1950\dvsborn{})]%
 \index[ndxnames]{Činaev, Vladimir Petrovič}
 Pianiste, enseignant et musicologue.
 \item[Daugovet, Ekaterina Francevna (\Dates{1882}{1942})]%
 \index[ndxnames]{Daugovet, Ekaterina Francevna}
 Pianiste et professeur.
 Amie de \VSofronitsky{}, décédée durant le siège de Leningrad.
 \item[Del'son, Viktor Jul'evič (\Dates{1907}{1970})]%
 \index[ndxnames]{Del'son, Viktor Jul'evič}
 Musicologue et auteur d'un livre à propos de \VSofronitsky{}~; voir
 \citet{Delson59, Delson70}.
 \item[Dukel'skij, Vladimir Aleksandrovič (\Dates{1903}{1969})]%
 \index[ndxnames]{Dukel'skij, Vladimir Aleksandrovič}
 Compositeur russe et américain -- connu aussi sous le nom de Vernon Duke.
 Poète et mémoiriste.
 Ami de \SProkofiev{}.
 Rencontre \VSofronitsky{} lors du séjour de celui-ci à Paris.
 \item[Dušinova, Valentina Nikolaevna (\Dates{1921}{1964})]%
 \index[ndxnames]{Dušinova, Valentina Nikolaevna}
 Pianiste.
 Élève et deuxième épouse de \VSofronitsky{} (voir Sofronickaja
 \emph{infra}).
 \item[Èšpaj, Andrej Jakovlevič (\Dates{1925}{2015})]%
 \index[ndxnames]{Èšpaj, Andrej Jakovlevič}
 Compositeur et pianiste.
 Artiste du peuple de l'\hbox{Union} soviétique.
 Élève de \VSofronitsky{} pour le piano.
 \item[Fichtengol'c, Lidija Izrailevna (1924\dvsborn{})]%
 \index[ndxnames]{Fichtengol'c, Lidija Izrailevna}
 Pianiste et enseignante.
 Élève de \HNeuhaus{}.
 \item[Gakkel', Leonid Evgen'evič (1936\dvsborn{})]%
 \index[ndxnames]{Gakkel', Leonid Evgen'evič}
 Musicologue, critique musical, pianiste et enseignant.
 \item[Geronimus, Aleksandr Efimovič (\Dates{1908}{1980})]%
 \index[ndxnames]{Geronimus, Aleksandr Efimovič}
 Pianiste et enseignant.
 \item[Glazunov, Aleksandr Konstantinovič (\Dates{1865}{1936})]%
 \index[ndxnames]{Glazunov, Aleksandr Konstantinovič}
 Compositeur, chef d'orchestre et enseignant.
 Recteur du conservatoire de Sankt-Peterburg.
 Parrain du fils aîné de \VSofronitsky{}, Aleksandr.
 \item[Gol'c, Boris Grigor'evič (\Dates{1913}{1942})]%
 \index[ndxnames]{Gol'c, Boris Grigor'evič}
 Compositeur et pianiste.
 Meurt à vingt-huit ans en défendant Leningrad assiégée.
 Élève de \LNikolaiev{} pour le piano.
 \item[Gol'denvejzer, Aleksandr Borisovič (\Dates{1875}{1961})]%
 \index[ndxnames]{Gol'denvejzer, Aleksandr Borisovič}
 Pianiste, enseignant, auteur d'ouvrages sur la musique et compositeur.
 Professeur et recteur du conservatoire de Moskva.
 Membre de l'\hbox{Union} des compositeurs soviétiques.
 Élève d'\AZiloti{} et de \PPabst{}.
 \item[Golubovskaja, Nadežda Iosifovna (\Dates{1891}{1975})]%
 \index[ndxnames]{Golubovskaja, Nadežda Iosifovna}
 Pianiste et claveciniste.
 Professeur au conservatoire de Leningrad.
 \item[Gornostaeva, Vera Vasil'evna (\Dates{1936}{2015})]%
 \index[ndxnames]{Gornostaeva, Vera Vasil'evna}
 Pianiste et enseignante.
 Professeur et titulaire de la chaire de piano au conservatoire de Moskva.
 Présidente de l'association des musiciens de Moskva.
 Artiste du peuple de la~RSFSR (République socialiste fédérative soviétique
 de Rossija).
 Élève de \HNeuhaus{}.
 \item[Gorochovskij, Jurij Nikolaevič (\Dates{1907}{1973})]%
 \index[ndxnames]{Gorochovskij, Jurij Nikolaevič}
 Chercheur chimiste.
 Second cousin de \VSofronitsky{}.
 \item[Grigorjan, Džul'etta Armenovna (1927\dvsborn{})]%
 \index[ndxnames]{Grigorjan, Džul'etta Armenovna}
 Pianiste.
 Élève de \VSofronitsky{}.
 \item[Igumnov, Konstantin Nikolaevič (\Dates{1873}{1948})]%
 \index[ndxnames]{Igumnov, Konstantin Nikolaevič}
 Pianiste et enseignant aux conservatoires de Moskva et de Tbilissi.
 Professeur et recteur du conservatoire de Moskva.
 Élève d'\AZiloti{}, de Nikolaj Zverev et de \PPabst{}.
 \item[Ivanov-Djatlov, Vladimir Ivanovič (1943\dvsborn{})]%
 \index[ndxnames]{Ivanov-Djatlov, Vladimir Ivanovič}
 Ingénieur mécanicien.
 Candidat en sciences techniques.
 Professeur à l'université nationale de l'automobile et de la route de
 Moskva.
 Descendant de \STaneiev{}.
 \item[Judina, Marija Veniaminovna (\Dates{1899}{1970})]%
 \index[ndxnames]{Judina, Marija Veniaminovna}
 Pianiste et enseignante.
 Enseignante ou professeur aux conservatoires de Leningrad, de~1923 à~1931,
 de Tbilissi, de~1933 à~1935, et de Moskva, à partir de~1936, ainsi qu'à
 l'institut Gnesin à Moskva, de~1943 à~1960.
 Élève de \FBlumenfeld{} et de \LNikolaiev{}.
 \item[Jurenev, Rostislav Nikolaevič (\Dates{1912}{2002})]%
 \index[ndxnames]{Jurenev, Rostislav Nikolaevič}
 Critique de cinéma.
 \item[Kalinenko, Nina Maksimovna (\Dates{1923}{2006})]%
 \index[ndxnames]{Kalinenko, Nina Maksimovna}
 Pianiste.
 Élève de \VSofronitsky{}, née Fejgina.
 Voir Prometheus Editions EDITION003.
 \item[Kočkina, Valentina Ivanovna (\Dates{1923}{2005})]%
 \index[ndxnames]{Kočkina, Valentina Ivanovna}
 Pianiste.
 Élève de \VSofronitsky{}.
 \item[Kogan, Roksana Vladimirovna (1937\dvsborn{})]%
 \index[ndxnames]{Kogan, Roksana Vladimirovna}
 Mathématicienne.
 Fille cadette de \VSofronitsky{} et \EScriabina{} (voir Sofronickaja
 \emph{infra}).
 Vit aux États-Unis d'\hbox{Amérique} depuis~1974.
 \item[Končalovskij, Maksim Vladimirovič (1940\dvsborn{})]%
 \index[ndxnames]{Končalovskij, Maksim Vladimirovič}
 Pianiste et enseignant.
 Élève d'\EGuilels{}.
 \item[Končalovskij, Petr Petrovič (\Dates{1876}{1956})]%
 \index[ndxnames]{Končalovskij, Petr Petrovič}
 Peintre.
 Un des fondateurs du mouvement pictural \foreignlanguage{russian}{Бубновый
 валет} (Valet de carreau) à Moskva au début des années~1910.
 \item[Kondrat'ev, Andrej Konstantinovič (\Dates{1927}{2001})]%
 \index[ndxnames]{Kondrat'ev, Andrej Konstantinovič}
 Pianiste et enseignant.
 Diplômé du conservatoire de Leningrad.
 Enseigne pendant de nombreuses années dans des instituts de musique, en
 particulier à Sankt-Peterburg.
 \item[Korsakova, Nina Nikolaevna (\Dates{1924}{2007})]%
 \index[ndxnames]{Korsakova, Nina Nikolaevna}
 Pianiste et enseignante.
 Élève de \VSofronitsky{} à partir de~1951.
 \item[Kuznecov, Anatolij Michajlovič (\Dates{1935}{2010})]%
 \index[ndxnames]{Kuznecov, Anatolij Michajlovič}
 Biographe, chercheur à propos de \MYudina{}.
 Auteur de livres et d'articles sur \MYudina{}.
 \item[Lebedeva-Gecevič, Anna Vasil'evna]%
 \index[ndxnames]{Lebedeva-Gecevič, Anna Vasil'evna}
 Pianiste et enseignante.
 Mère de Vsevolod Bujukli.
 Professeur de \VSofronitsky{} en Pologne.
 Élève de Nikolaj Rubinštejn (\Dates{1835}{1881}).
 \item[Lobanov, Pavel Vasil'evič (\Dates{1923}{2016})]%
 \index[ndxnames]{Lobanov, Pavel Vasil'evič}
 Pianiste et enseignant.
 Élève de \VSofronitsky{}.
 Voir Prometheus Editions EDITION003.
 \item[Lobaševa, Irina Faekovna]%
 \index[ndxnames]{Lobaševa, Irina Faekovna}
 Diplômée de l'université de Kazan' (Institut de peinture, de sculpture et
 d'architecture -- Faculté de théorie et d'histoire de l'art).
 Candidate en art.
 Professeur associé à l'institut d'\hbox{État} académique des beaux-arts
 V.I.~Surikov à Kazan'.
 Professeur d'histoire de l'art.
 \item[Mejerchol'd, Vsevolod Èmil'evič (\Dates{1874}{1940})]%
 \index[ndxnames]{Mejerchol'd, Vsevolod Èmil'evič}
 Dramaturge et metteur en scène.
 Dédie à \VSofronitsky{} la première représentation de sa mise en scène de
 \emph{La Dame de pique} en~1934.
 Assassiné par la police politique d'\JStaline{}.
 \item[Merkulov, Aleksandr Michajlovič (1951\dvsborn{})]%
 \index[ndxnames]{Merkulov, Aleksandr Michajlovič}
 Pianiste, enseignant et musicologue.
 Candidat en histoire de l'art.
 Professeur au conservatoire de Moskva.
 \item[Meržanov, Viktor Karpovič (\Dates{1919}{2012})]%
 \index[ndxnames]{Meržanov, Viktor Karpovič}
 Pianiste et enseignant.
 Élève de \SFeinberg{}.
 \item[Metner, Nikolaj Karlovič (\Dates{1880}{1951})]%
 \index[ndxnames]{Metner, Nikolaj Karlovič}
 Compositeur et pianiste.
 Vit à l'étranger à partir de~1921, hormis un bref retour en Rossija
 en~1927.
 \VSofronitsky{} est son \Quote{élève informel} à Paris en~1928-1929.
 \item[Michajlovskij, Aleksandr Konstantinovič (\Dates{1851}{1938})]%
 \index[ndxnames]{Michajlovskij, Aleksandr Konstantinovič}
 Pianiste, compositeur et enseignant.
 Professeur de \VSofronitsky{} en Pologne.
 Élève d'\IMoscheles{}, de Carl Reinecke, de Theodor Coccius, de Karol
 Mikuli et de la Princesse Marcelina Czartoryska (Radziwiłł)~: un élève des
 disciples de \LBeethoven{} et de \FChopin{}.
 \item[Miklaševskaja, Aleksandra Nikolaevna]%
 \index[ndxnames]{Miklaševskaja, Aleksandra Nikolaevna}
 Pianiste.
 \item[Mil'štejn, Jakov Isaakovič (\Dates{1911}{1981})]%
 \index[ndxnames]{Mil'štejn, Jakov Isaakovič}
 Pianiste, musicologue et enseignant.
 Directeur d'un ouvrage consacré à \VSofronitsky{}~: voir
 \citet{Milshteyn70, Milshteyn82a}.
 Élève de \KIgumnov{}.
 \item[Model', Viktor Petrovič (\Dates{1910}{1981})]%
 \index[ndxnames]{Model', Viktor Petrovič}
 Pianiste, enseignant et musicologue.
 \item[Moroškina, Natalija Georgievna (1933\dvsborn{})]%
 \index[ndxnames]{Moroškina, Natalija Georgievna}
 Pianiste et enseignante.
 Élève de \VSofronitsky{}.
 \item[Možanskaja, Vera Borisovna (1916\dvsborn{})]%
 \index[ndxnames]{Možanskaja, Vera Borisovna}
 Critique d'art.
 Arrière-petite-fille d'\ARubinstein{}.
 \item[Muravlëv, Jurij Alekseevič (1927\dvsborn{})]%
 \index[ndxnames]{Muravlëv, Jurij Alekseevič}
 Pianiste.
 Professeur au conservatoire de Moskva.
 Artiste du peuple de Rossija.
 \item[Naumov, Lev Nikolaevič (\Dates{1925}{2005})]%
 \index[ndxnames]{Naumov, Lev Nikolaevič}
 Pianiste, enseignant et compositeur.
 Élève de \HNeuhaus{}.
 \item[Nejgauz, Genrich Gustavovič (\Dates{1888}{1964})]%
 \index[ndxnames]{Nejgauz, Genrich Gustavovič}
 Pianiste, enseignant et auteur d'ouvrages sur la musique et l'art du piano.
 Professeur et recteur du conservatoire de Moskva.
 Neveu de \FBlumenfeld{}.
 \item[Nejgauz, Stanislav Genrichovič (\Dates{1927}{1980})]%
 \index[ndxnames]{Nejgauz, Stanislav Genrichovič}
 Pianiste et enseignant.
 Professeur au conservatoire de Moskva.
 Fils de \HNeuhaus{}.
 \item[Nekrasova, Varvara Borisovna (\Dates{1909}{1997})]%
 \index[ndxnames]{Nekrasova, Varvara Borisovna}
 Pianiste et enseignante.
 À la mort d'\hbox{Aleksandra} (Ada) Èmil'evna Vizel', elle a reçu la
 correspondance entretenue par Ada Vizel' et \VSofronitsky{}.
 \item[Nikolaev, Leonid Vladimirovič (\Dates{1878}{1942})]%
 \index[ndxnames]{Nikolaev, Leonid Vladimirovič}
 Pianiste, compositeur et enseignant.
 Artiste du peuple de la~RSFSR (République socialiste fédérative soviétique
 de Rossija).
 Docteur en arts (1941).
 Un des plus importants représentants de l'école du piano en Rossija durant
 la première moitié du~\textsc{xx}\ieme{} siècle.
 Durant ses années d'enseignement au conservatoire de Sankt-Peterburg
 (Petrograd, Leningrad), il a formé, en particulier, \VSofronitsky{},
 \DChostakovitch{}, \MYudina{}, Pavel Serebrjakov, Natan Perel'man,
 Aleksandr Krejn, \VBogdanovBerezovsky{} et Vladimir Deševov.
 \item[Nikonovič, Igor' Vladimirovič (\Dates{1935}{2012})]%
 \index[ndxnames]{Nikonovič, Igor' Vladimirovič}
 Pianiste et enseignant.
 Professeur à l'institut Gnesin à Moskva.
 Artiste honoré de la Fédération de Rossija.
 Consacre toute sa vie au travail créatif de \VSofronitsky{}.
 Participe à la publication de tous les enregistrements de ce dernier.
 Élève de \HNeuhaus{} et de \VSofronitsky{}.
 \item[Nudel'man, Dina Èmmanuilovna (1919\dvsborn{})]%
 \index[ndxnames]{Nudel'man, Dina Èmmanuilovna}
 Pianiste.
 Élève de Berta Rejngbal'd et de \VSofronitsky{}.
 \item[Oborin, Lev Nikolaevič (\Dates{1907}{1974})]%
 \index[ndxnames]{Oborin, Lev Nikolaevič}
 Pianiste et enseignant.
 Professeur au conservatoire de Moskva.
 Élève de \KIgumnov{}.
 \item[Orlovskij, Vladimir Vladimirovič (1934\dvsborn{})]%
 \index[ndxnames]{Orlovskij, Vladimir Vladimirovič}
 Candidat en histoire de l'art.
 Professeur au conservatoire de Rostov-na-Donu.
 Engagé en~1953-1958 au conservatoire de Moskva dans la classe de
 \VSofronitsky{}.
 Défend en~1991 sa thèse sur le thème \Quote{\Sofronitsky{} -- professeur}~;
 pour un résumé, voir \citet{Orlovsky91}.
 \item[Panarin, Vladimir Fedorovič (1922\dvsborn{})]%
 \index[ndxnames]{Panarin, Vladimir Fedorovič}
 Pianiste-\emph{\foreignlanguage{german}{konzertmeister}}.
 \item[Paperno, Dmitrij Aleksandrovič (1929\dvsborn{})]%
 \index[ndxnames]{Paperno, Dmitrij Aleksandrovič}
 Pianiste, diplômé du conservatoire de Moskva en~1951 dans la classe du
 professeur Aleksandr Gol'denvejzer.
 Enseignant à l'institut Gnesin à Moskva de~1967 à~1973.
 Vit de nos jours aux États-Unis d'\hbox{Amérique}.
 Professeur de plusieurs universités américaines dans la classe de piano.
 Auteur d'ouvrages en russe et en anglais~; voir \citet{Paperno03}.
 \item[Pasternak, Boris Leonidovič (\Dates{1890}{1960})]%
 \index[ndxnames]{Pasternak, Boris Leonidovič}
 Poète, romancier et traducteur.
 \item[Pasternak, Evgenij Borisovič (\Dates{1923}{2012})]%
 \index[ndxnames]{Pasternak, Evgenij Borisovič}
 Ingénieur militaire.
 Fils de \BPasternak{} (\Dates{1890}{1960}).
 Durant ses dernières années, biographe et éditeur des œuvres de son père.
 \item[Podol'skaja, Vera Vasil'evna (\Dates{1914}{1970})]%
 \index[ndxnames]{Podol'skaja, Vera Vasil'evna}
 Pianiste et enseignante.
 \item[Pokrovskij, Boris Aleksandrovič (\Dates{1912}{2009})]%
 \index[ndxnames]{Pokrovskij, Boris Aleksandrovič}
 Directeur d'opéra.
 Artiste du peuple de l'\hbox{Union} soviétique.
 Lauréat du Prix \Lenin{} et du Prix d'\hbox{État}.
 Directeur général du théâtre Bol'šoj de~1952 à~1963 et de~1970 à~1982.
 Directeur artistique et directeur fondateur du théâtre de musique de
 chambre de Moskva.
 \item[Prokof'ev, Sergej Sergeevič (\Dates{1891}{1953})]%
 \index[ndxnames]{Prokof'ev, Sergej Sergeevič}
 Compositeur, pianiste et chef d'orchestre.
 \item[Rabinovič, David Abramovič (\Dates{1900}{1978})]%
 \index[ndxnames]{Rabinovič, David Abramovič}
 Critique musical.
 Écrivait sous le pseudonyme de \Quote{Florestan}.
 \item[Richter, Svjatoslav Teofilovič (\Dates{1915}{1997})]%
 \index[ndxnames]{Richter, Svjatoslav Teofilovič}
 Pianiste.
 \item[Rumjancev, Vasilij Vasil'evič]%
 \index[ndxnames]{Rumjancev, Vasilij Vasil'evič}
 Technicien de scène de la petite salle et de la grande salle du
 conservatoire de Moskva.
 Pendant quarante ans, de~1920 à~1960, il a servi avec honnêteté et soin les
 concertistes qui se sont produits dans ces deux salles.
 On peut dire qu'il était, en quelque sorte, le \Quote{compagnon constant}
 des concerts.
 \item[Šaborkina, Tat'jana Grigor'evna (\Dates{1906}{1986})]%
 \index[ndxnames]{Šaborkina, Tat'jana Grigor'evna}
 Pianiste, musicologue et directrice du musée \Scriabine{} à Moskva.
 \item[Safonov, Il'ja Kirillovič (\Dates{1937}{2004})]%
 \index[ndxnames]{Safonov, Il'ja Kirillovič}
 Ingénieur.
 Candidat en sciences techniques.
 Auteur de publications dans des revues de musique et de discours lors de
 conférences au musée central de la culture musicale \MGlinka{}.
 Petit-fils de Vasilij Safonov (\Dates{1852}{1918}).
 \item[Sal'nikov, Georgij Ivanovič (1923\dvsborn{})]%
 \index[ndxnames]{Sal'nikov, Georgij Ivanovič}
 Compositeur et professeur au conservatoire de Moskva.
 Candidat en histoire de l'art.
 Artiste honoré de la~RSFSR (République socialiste fédérative soviétique de
 Rossija).
 Élève de \VSofronitsky{} en~1943-1944 (première année) et en~1947-1948
 (cinquième année) pour le piano.
 \item[Savkevič, Kira Pantelejmonovna (1922\dvsborn{})]%
 \index[ndxnames]{Savkevič, Kira Pantelejmonovna}
 Peintre et enseignante.
 Amie intime de la famille \Sofronitsky{}.
 \item[Savostjuk, Oleg Michajlovič (1927\dvsborn{})]%
 \index[ndxnames]{Savostjuk, Oleg Michajlovič}
 Peintre et artiste graphique.
 Artiste du peuple de Rossija.
 Académicien, professeur, travailleur honoré des arts de Pologne.
 \item[Savšinskij, Samarij Il'ič (\Dates{1891}{1968})]%
 \index[ndxnames]{Savšinskij, Samarij Il'ič}
 Pianiste et enseignant.
 Professeur au conservatoire de Leningrad.
 \item[Šerševskij, Grigorij Il'ič (1911\dvsborn{})]%
 \index[ndxnames]{Šerševskij, Grigorij Il'ič}
 Pianiste et enseignant.
 Professeur au conservatoire du Belarus'.
 \item[Širjaeva, Nina Grigor'evna (1924\dvsborn{})]%
 \index[ndxnames]{Širjaeva, Nina Grigor'evna}
 Ingénieur électricien.
 Traductrice.
 \item[Skrjabin, Aleksandr Nikolaevič (\Dates{1872}{1915})]%
 \index[ndxnames]{Skrjabin, Aleksandr Nikolaevič}
 Compositeur et pianiste.
 \item[Skrjabin, Aleksandr Serafimovič (1947\dvsborn{})]%
 \index[ndxnames]{Skrjabin, Aleksandr Serafimovič}
 Président de la Fondation publique régionale pour la promotion du
 patrimoine créatif \AScriabine{}.
 Petit-neveu d'\AScriabine{}.
 Directeur ou codirecteur d'ouvrages consacrés à \VSofronitsky{}~: voir
 \citet{Nikonovich08, Scriabine}.
 \item[Skrjabina, Ariadna Aleksandrovna (\Dates{1905}{1944})]%
 \index[ndxnames]{Skrjabina, Ariadna Aleksandrovna}
 Poète.
 Fille d'\AScriabine{} et de Tat'jana Fedorovna Šlëcer.
 Assassinée par la Milice française avant la fin du régime fasciste de Vichy
 en France.
 \item[Skrjabina, Elena Aleksandrovna (\Dates{1900}{1990})]%
 \index[ndxnames]{Skrjabina, Elena Aleksandrovna}
 Fille d'\AScriabine{}, première épouse de \VSofronitsky{} (voir
 Sofronickaja \emph{infra}).
 \item[Skrjabina, Marija Aleksandrovna (\Dates{1901}{1989})]%
 \index[ndxnames]{Skrjabina, Marija Aleksandrovna}
 Fille d'\AScriabine{} et de Vera Ivanovna Skrjabina (Isakovič).
 \item[Skrjabina, Tat'jana Fedorovna (née Šlëcer) (\Dates{1883}{1922})]%
 \index[ndxnames]{Skrjabina, Tat'jana Fedorovna}
 Deuxième épouse d'\AScriabine{} (voir Šlëcer \emph{infra}).
 \item[Skrjabina, Vera Ivanovna (née Isakovič) (\Dates{1875}{1920})]%
 \index[ndxnames]{Skrjabina, Vera Ivanovna}
 Pianiste.
 Première épouse d'\AScriabine{}.
 \item[Šlëcer, Tat'jana Fedorovna (\Dates{1883}{1922})]%
 \index[ndxnames]{Šlëcer, Tat'jana Fedorovna}
 Deuxième épouse d'\AScriabine{} (voir Skrjabina \emph{supra}).
 \item[Smirnov, Mstislav Anatol'evič (\Dates{1924}{2000})]%
 \index[ndxnames]{Smirnov, Mstislav Anatol'evič}
 Pianiste, enseignant et musicologue.
 Professeur au conservatoire de Moskva.
 Artiste honoré de la~RSFSR (République socialiste fédérative soviétique de
 Rossija).
 \item[Sofronickaja, Elena Aleksandrovna (\Dates{1900}{1990})]%
 \index[ndxnames]{Sofronickaja, Elena Aleksandrovna}
 Fille d'\AScriabine{}, première épouse de \VSofronitsky{} (voir Skrjabina
 \emph{supra}).
 \item[Sofronickaja, Elena Vladimirovna (\Dates{1898}{1986})]%
 \index[ndxnames]{Sofronickaja, Elena Vladimirovna}
 Sœur de \VSofronitsky{}.
 \item[Sofronickaja, Irina Ivanovna (\Dates{1920}{2020})]%
 \index[ndxnames]{Sofronickaja, Irina Ivanovna}
 Conservatrice de musée et chercheur au musée \Scriabine{} à Moskva, où elle
 a travaillé comme assistante de recherches pendant presque trente ans.
 Belle-fille de \VSofronitsky{}, épouse d'\ASofronitsky{}.
 \item[Sofronickaja, Natal'ja Vladimirovna (\Dates{1897}{1979})]%
 \index[ndxnames]{Sofronickaja, Natal'ja Vladimirovna}
 Sœur de \VSofronitsky{}.
 \item[Sofronickaja, Ol'ga Vladimirovna (\Dates{1893}{1933})]%
 \index[ndxnames]{Sofronickaja, Ol'ga Vladimirovna}
 Sœur de \VSofronitsky{}.
 \item[Sofronickaja, Roksana Vladimirovna (1937\dvsborn{})]%
 \index[ndxnames]{Sofronickaja, Roksana Vladimirovna}
 Fille cadette de \VSofronitsky{} et \EScriabina{} (voir Kogan
 \emph{supra}).
 \item[Sofronickaja, Valentina Nikolaevna (\Dates{1921}{1964})]%
 \index[ndxnames]{Sofronickaja, Valentina Nikolaevna}
 Deuxième épouse de \VSofronitsky{} (voir Dušinova \emph{supra}).
 \item[Sofronickaja, Vera Aleksandrovna (\Dates{1875}{1935})]%
 \index[ndxnames]{Sofronickaja, Vera Aleksandrovna}
 Mère de \VSofronitsky{}.
 \item[Sofronickaja, Vera Vladimirovna (\Dates{1901}{1948})]%
 \index[ndxnames]{Sofronickaja, Vera Vladimirovna}
 Sœur jumelle de \VSofronitsky{}.
 \item[Sofronickij, Aleksandr Vladimirovič (\Dates{1921}{1995})]%
 \index[ndxnames]{Sofronickij, Aleksandr Vladimirovič}
 Astro\-physicien et mathématicien.
 Enseignant.
 Fils aîné de \VSofronitsky{} et \EScriabina{}.
 \item[Sofronickij, Nikolaj Vladimirovič]%
 \index[ndxnames]{Sofronickij, Nikolaj Vladimirovič}
 Frère de \VSofronitsky{}.
 \item[Sofronickij, Vladimir Nikolaevič (\Dates{1869}{1942})]%
 \index[ndxnames]{Sofronickij, Vladimir Nikolaevič}
 Professeur de physique et de mathématiques.
 Père de \VSofronitsky{}.
 \item[Sokolov, Nikolaj Nikolaevič (\Dates{1938}{2004})]%
 \index[ndxnames]{Sokolov, Nikolaj Nikolaevič}
 Musicologue, diplômé en~1964 du conservatoire de Moskva.
 En~1969-1974, secrétaire académique du musée central de la culture musicale
 \MGlinka{}.
 En~1975-1987, chercheur scientifique senior de ce même musée.
 Auteur de publications et de travaux scientifiques sur l'histoire de la
 musique russe des~\textsc{xix}\ieme{} et~\textsc{xx}\ieme{} siècles, ainsi
 que sur les arts de la scène contemporains.
 \item[Sosina, Ljudmila Aleksandrovna (1918\dvsborn{})]%
 \index[ndxnames]{Sosina, Ljudmila Aleksandrovna}
 Pianiste et enseignante.
 Élève de Berta Rejngbal'd et d'\hbox{Aleksandr} Gol'denvejzer.
 \item[Šostakovič, Dmitrij Dmitrievič (\Dates{1906}{1975})]%
 \index[ndxnames]{Šostakovič, Dmitrij Dmitrievič}
 Compositeur.
 Élève de \LNikolaiev{} pour le piano.
 \item[Spiridonova, Vera Michajlovna]%
 \index[ndxnames]{Spiridonova, Vera Michajlovna}
 Artiste honoré de la République du Tatarstan.
 Professeur au conservatoire de Kazan'.
 \item[Štejnberg, Maksimilian Oseevič (\Dates{1883}{1946})]%
 \index[ndxnames]{Štejnberg, Maksimilian Oseevič}
 Compositeur et enseignant.
 Professeur de \VSofronitsky{} pour la composition à Petrograd.
 Élève et beau-fils de \NRimskiKorsakov{}.
 \item[Sturcel', Valerija Jul'evna (1921\dvsborn{})]%
 \index[ndxnames]{Sturcel', Valerija Jul'evna}
 Pianiste et enseignante.
 \item[Tolstoj, Dmitrij Alekseevič (\Dates{1923}{2003})]%
 \index[ndxnames]{Tolstoj, Dmitrij Alekseevič}
 Compositeur et pianiste.
 En~1943-1944, étudie au conservatoire de Moskva comme élève de
 \VSofronitsky{} pour le piano et de Vissarion Šebalin (\Dates{1902}{1963})
 pour la composition.
 Fils de l'écrivain Aleksej Nikolaevič Tolstoj (\Dates{1883}{1945}).
 \item[Tracevskaja, Ol'ga Michajlovna (1923\dvsborn{})]%
 \index[ndxnames]{Tracevskaja, Ol'ga Michajlovna}
 Pianiste.
 Élève de \VSofronitsky{}.
 \item[Van~Cliburn (Harvey Lavan Jr.) (\Dates{1934}{2013})]%
 \index[ndxnames]{Van~Cliburn (Harvey Lavan Jr.)}
 Pianiste.
 Lauréat de la première édition du concours international \Tchaikovski{} à
 Moskva en~1958.
 Rencontre \VSofronitsky{} au musée \Scriabine{} en~1960.
 \item[Vedernikov, Nikolaj Anatol'evič (archiprêtre) (1928\dvsborn{})]%
 \index[ndxnames]{Vedernikov, Nikolaj Anatol'evič}
 Diplômé de deux facultés du conservatoire d'\hbox{État} de Moskva et de
 l'académie théologique de Moskva.
 \item[Vicinskij, Aleksandr Vladimirovič (\Dates{1904}{1984})]%
 \index[ndxnames]{Vicinskij, Aleksandr Vladimirovič}
 Pianiste et enseignant.
 Auteur d'un entretien publié avec \VSofronitsky{}~: voir
 \citet{Vitsinsky04, Vitsinsky}.
 \item[Vizel', Aleksandra (Ada) Èmil'evna (\Dates{1899}{1974})]%
 \index[ndxnames]{Vizel', Aleksandra (Ada) Èmil'evna}
 Peintre et architecte.
 Fille d'\EVizel{} (\Dates{1866}{1943}).
 \item[Vizel', Aleksandra Èmil'evna (née Strauss) (\Dates{1866}{1939})]%
 \index[ndxnames]{Vizel', Aleksandra Èmil'evna (née Strauss)}
 Peintre.
 Épouse d'\EVizel{} (\Dates{1866}{1943}).
 \item[Vizel', Andrej Oskarovič (1930\dvsborn{})]%
 \index[ndxnames]{Vizel', Andrej Oskarovič}
 Peintre.
 Fils d'\hbox{Oskar} Èmil'evič Vizel' (\Dates{1895}{1939}).
 \item[Vizel', Èmil' Èmil'evič (\Dates{1897}{1930})]%
 \index[ndxnames]{Vizel', Èmil' Èmil'evič}
 Philologue.
 Fils d'\EVizel{} (\Dates{1866}{1943}).
 \item[Vizel', Èmil' Oskarovič (\Dates{1866}{1943})]%
 \index[ndxnames]{Vizel', Èmil' Oskarovič}
 Peintre.
 \item[Vizel', Oskar Èmil'evič (\Dates{1895}{1939})]%
 \index[ndxnames]{Vizel', Oskar Èmil'evič}
 Ethnographe, orientaliste et enseignant.
 Fils d'\EVizel{} (\Dates{1866}{1943}).
 \item[Vizel', Tat'jana Èmil'evna (\Dates{1904}{1976})]%
 \index[ndxnames]{Vizel', Tat'jana Èmil'evna}
 Peintre, spécialiste du domaine des arts de la décoration théâtrale.
 Fille d'\EVizel{} (\Dates{1866}{1943}).
 \item[Zak, Jakov Izrailevič (\Dates{1913}{1976})]%
 \index[ndxnames]{Zak, Jakov Izrailevič}
 Pianiste et enseignant.
 Professeur au conservatoire de Moskva.
 \item[Zinger, Evsej Michajlovič (\Dates{1911}{1977})]%
 \index[ndxnames]{Zinger, Evsej Michajlovič}
 Pianiste et professeur au conservatoire de Novosibirsk.
 Candidat en histoire de l'art.
 A travaillé aux conservatoires de Saratov, de Leningrad et d'\hbox{Almaty}.
 Auteur à Moskva, en~1976, d'un livre sur l'histoire de l'art du piano en
 France jusqu'au milieu du~\textsc{xix}\ieme{} siècle.
 \item[Zolotov, Andrej Andreevič (1937\dvsborn{})]%
 \index[ndxnames]{Zolotov, Andrej Andreevič}
 Journaliste, scénariste et critique musical.
 \item[Žukov, Igor' Michajlovič (\Dates{1936}{2018})]%
 \index[ndxnames]{Žukov, Igor' Michajlovič}
 Pianiste et chef d'orchestre.
 Diplômé du conservatoire de Moskva dans les classes d'\EGuilels{} et de
 \HNeuhaus{}.
 Lauréat du concours international de piano Marguerite Long et Jacques
 Thibaud en~1957.
 Artiste honoré de Rossija.
 \item[Žukova, Ol'ga Michajlovna (1925\dvsborn{})]%
 \index[ndxnames]{Žukova, Ol'ga Michajlovna}
 Pianiste et enseignante.
 Candidate en histoire de l'art.
 Professeur au conservatoire de Moskva.
 Artiste honoré de la Fédération de Rossija.
 Élève d'\hbox{Aleksandr} Gol'denvejzer et de \VSofronitsky{}.
\end{description}
