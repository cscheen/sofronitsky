\chapter[%
Documents audio-visuels][%
Documents audio-visuels]{%
Documents audio-visuels}
\label{chap:Documents}

\section*{Leçons}

Il s'agit de l'enregistrement de leçons reconstituées entre \VSofronitsky{}
et certains de ses anciens étudiants.

\section[%
Skrjabin~:
Poème en \kF \Sharp majeur, \Opus{32} \Number{1}]{%
\Scriabine{}~:
Poème en \kF \Sharp majeur, \Opus{32} \Number{1}}

\begin{workitemize}
 \item\Performance{1954-10-02}{16}{13}{\MVSA}{\Live}
 \Comment{Enregistrement d'une leçon donnée par \VSofronitsky{} à
 \PLobanov{}, dans l'appartement du maître, rue Novopesčanaja à \Moscow.}
 \begin{perfitemize}
  \item\EditionOnCD{%
  \onhand{Prometheus Editions EDITION003}}
 \end{perfitemize}
\end{workitemize}

\section[%
Skrjabin~:
Prélude en \kE mineur, \Opus{11} \Number{4}]{%
\Scriabine{}~:
Prélude en \kE mineur, \Opus{11} \Number{4}}

\begin{workitemize}
 \item\Performance{1954-10-02}{7}{34}{\MVSA}{\Live}
 \Comment{Enregistrement d'une leçon donnée par \VSofronitsky{} à
 \NFeigina{}, dans l'appartement du maître, rue Novopesčanaja à \Moscow.}
 \begin{perfitemize}
  \item\EditionOnCD{%
  \onhand{Prometheus Editions EDITION003}}
 \end{perfitemize}
\end{workitemize}

\section[%
Skrjabin~:
Prélude en \kD majeur, \Opus{11} \Number{5}]{%
\Scriabine{}~:
Prélude en \kD majeur, \Opus{11} \Number{5}}

\begin{workitemize}
 \item\Performance{1954-10-02}{5}{38}{\MVSA}{\Live}
 \Comment{Enregistrement d'une leçon donnée par \VSofronitsky{} à
 \NFeigina{}, dans l'appartement du maître, rue Novopesčanaja à \Moscow.}
 \begin{perfitemize}
  \item\EditionOnCD{%
  \onhand{Prometheus Editions EDITION003}}
 \end{perfitemize}
\end{workitemize}

\section*{Entretiens}

Il s'agit de l'enregistrement d'un entretien privé dans le cadre domestique.

\section[%
Entretien avec Pavel Lobanov]{%
Entretien avec \PLobanov{}}

Le disque compact supplémentaire de l'édition à tirage limité du boîtier
Prometheus Editions EDITION003 propose un entretien de \VSofronitsky{} avec
\PLobanov{} daté 1954-09-21, même date que les cinq enregistrements privés
qui ont été réalisés au domicile de \VSofronitsky{}~; voir
\S{}~\ref{sec:PR_1}, \S{}~\ref{sec:PR_2}, \S{}~\ref{sec:PR_3},
\S{}~\ref{sec:PR_4}, \S{}~\ref{sec:PR_5}.
La durée de l'entretien est de trois minutes.

\section*{Documentaires}

Il s'agit d'une liste de documentaires radiophoniques et audio-visuels qui
ont été consacrés à \VSofronitsky{}.

\section[%
Documentaire d'Andrej Končalovskij]{%
Documentaire d'\AKonchalovsky{}}

En~2007, \AKonchalovsky{} a réalisé un film documentaire pour la télévision
moscovite \citep[voir][]{Konchalovsky07}, consacré à \VSofronitsky{}.
Il a été édité en~2015 sous la forme d'un~DVD joint au boîtier Melodija MEL
CD~10~02312.
La durée de ce film documentaire est de quarante-quatre minutes.

\section[%
Émissions radiophoniques de Valerij Voskobojnikov]{%
Émissions radiophoniques de \VVoskobojnikov{}}

En~2008, \VVoskobojnikov{} a réalisé un cycle de neuf émissions
radiophoniques de vingt-neuf minutes chacune, consacrées à \VSofronitsky{}.
Ces émissions ont été diffusées par Radio Vatican
\citep[voir][]{Voskobojnikov08}.
Par autorisation de Radio Vatican, qui demeure propriétaire de tous les
droits de reproduction, l'utilisation des fichiers~MP3, correspondant aux
émissions diffusées, est libre et gratuite pour une écoute privée et
personnelle.
Pour toute autre utilisation, il est nécessaire de contacter Radio Vatican.

\section[%
Images filmées]{%
Images filmées}

En~2013, on a diffusé, sur le site de partage de vidéos YouTube, un bref
documentaire d'une douzaine de minutes à propos de \VSofronitsky{}~; ce
documentaire contient quelques images filmées du pianiste qui sont,
\hbox{semble-t-il}, les premières rendues publiques.
Le documentaire comporte en outre une iconographie parfois inédite~:
\begin{itemize}
 \item \url{https://www.youtube.com/watch?v=9PdSd2M9i1I}
 (documentaire original en russe)~;
 \item \url{https://www.youtube.com/watch?v=xuF7tX3gsNc}
 (documentaire original en russe, avec des sous-titres en anglais)~;
 \item \url{https://www.youtube.com/watch?v=kDPFkOfjXQw}
 (extrait constitué des images filmées du pianiste).
\end{itemize}

\section[%
Documentaire en russe~(1976)]{%
Documentaire en russe~(1976)}

En~2016, on a diffusé, sur le site de partage de vidéos YouTube, un
documentaire de vingt-sept minutes en russe à propos de \VSofronitsky{} et
daté de~1976~; il comporte une iconographie rare et parfois unique~:
\begin{itemize}
 \item \url{https://www.youtube.com/watch?v=WQuweOurF8A}
 (documentaire original en russe).
\end{itemize}
